\documentclass[math]{cours}
\title{Introduction à la Robotique}
\author{Justin Carpentier et Stéphane Caron}

\begin{document}
\bettertitle
\section*{Introduction}
L'objectif est de créer un framework général permettant de préparer à l'automatisation de tous les types de robots:
\begin{itemize}
	\item Manipulation
	\item Locomotion
	\item Hybrides
\end{itemize}
Un robot humanoïde étant beaucoup plus complexe qu'un quadripède, la construction de robots humanoïdes et leur perfectionnement font avancer la robotique.

On peut voir la robotique comme la fabrique artificielle du mouvement (par opposition au vivant).
La robotique s'itnéresse au lien entre des choses qui bougent et le monde réel, et à son contrôle par ordinateur.
C'est aussi le miroir de l'humain et des manières dont nous bougeons naturellement.
On a une analogie entre les robots et le vivant.
\begin{center}\begin{tabular}{cc}
		Outil Inerte           & Pierre  \\
		\midrule
		Action sans Locomotion & Végétal \\
		\midrule
		Locomotion             & Animal
	\end{tabular}\end{center}
L'objectif est de construire des algorithmes pour des robots capables d'agir de manière autonome, notamment, qui tournent sur des ordinateurs portables.

\section*{Historique de la Robotique}
La robotique a seulement 60 ans~!
\begin{itemize}
	\item UNIMATE est le premier bras robotique industriel, vendu à partir de 1961.
	      Il a une logique pneumatique.
	\item Les premières téléopérations assistées par ordinateur à la CEA (Commission de l'Énergie Atomique) ont commencées dans les années 1970
	\item Du côté de la recherche, le LAAS-CNRS a construit le robot HILARE dans les années 70 sous l'impulsion de Jean Lagasse.
	\item Les robots industriels se sont énormément développés entre les années 80 et 90.
	\item Les premiers robots à jambes sont nés dans les années 80 sous la direction de Mark Hebert (fondateur de Boston Dynamics).
	      Les premiers bipèdes de l'Inria sont nés à partir des années 90.
\end{itemize}

\subsection{Introduction au Contrôle du Mouvement}
L'objectif est de se mouvoir pour agir.
Le mouvement a lieu dans l'espace physique, espace commun entre l'objet et le robot.
On n'a accès qu'à une projection à travers les \textit{sens} du robot, \emph{son espace sensoriel} et on envoie ensuite des commandes à \emph{l'espace moteur}.
Il faut donc prendre en compte une boucle de contrôle:
\begin{category}[]
	& Physique\arrow[dl, bend right, "Percevoir"] & \\
	Sensoriel \arrow[rr, bend right] & & Moteur\arrow[ul, bend right]
\end{category}

On crée un boucle de contrôle:
\begin{itemize}
	\item On définit une action
	\item On représente les éléments nécessaires à l'action et on contrôle les paramètres
	\item On calcule les réactions nécessaires
\end{itemize}
Une partie de l'intelligence du mouvement est dans le design.
Un bon design de robot facilite le mouvement.

Le principe fondamental de la locomotion:
On construit un espace de référence avec les positions des objets alentours.
On calcule le déplacement de notre robot et on calculer ainsi les déplacements des autres objets:
on compose des mouvements, ce qu'on fait donc mathématiquement en utilisant des crochets de Lie.

On va faire des bilans de force qui agissent physiquement sur le centre de masse.
Les robots humanoïdes sont, par le calcul, mus par des forces.
On va donc devoir résoudre, efficacement, les équations de la mécanique Lagrangienne.

Le roboticien a pour but de construire une fonction de contrôle pour passer de l'espace sensoriel à l'espace moteur.
Il y a deux manières principales:
\begin{description}
	\item[Le Contrôle Optimal] On essaie de calculer les commandes moteurs en passant pas une estimation optimale et un contrôle optimal:
	      on a deux étapes de modélisation, le calcul de $x$ un état intermédiaire, et le calcul de $u$ l'ensemble des commandes moteurs.
	      On résout alors le problème de minimisation:
	      \begin{align*}
		      \min_{x(.), u(.)} \int_{0}^{T}l_{t}\left(x(t), u(t)\right)\d t \\
		      \dot{x}(t) = f_{t}\left(x(t), u(t)\right)                      \\
		      x(0) = x_{0}
	      \end{align*}
	\item[L'apprentissage par Politique] On veut calculer les commandes moteurs en passant par une simulation du résultat des consignes.
	      Ici, $x$ est un état caché de simulation qu'on crée par intégration puis génération, et on regénère des données de mesures.
	      On procède donc par apprentissage, et on cherche à résoudre:
	      \begin{align*}
		      \min_{\color{vulm}\theta}\mathbb{E}_{\xi}\left[\int_{0}^{T}l_{t}\left(x(t), {\color{vulm}\pi_{\theta}\left(x(t), \xi(t)\right)}\right)\d t\right] \\
		      \dot{x}(t) = f_{t}\left(x(t), {\color{vulm}\pi_{\theta}(x_{t}, \dot{\xi}(t))}\right)                                                              \\
		      x(0) \sim \xi_{0}
	      \end{align*}
\end{description}

\section{Les Mathématiques du Mouvement}
\subsection{Cinématique et Rotations}
\begin{definition}
	La cinematique est l'étude du mouvement sans considérer les forces qui le génère.
	\label{def:kinematics}
\end{definition}
On va essayer d'obtenir une chaîne cinématique du robot (ensemble de joints/pivots attachés par des solides rigides) pour le modéliser.
Étant donné les angles des joints, on va chercher à calculer la position du robot.
Réciproquement, étant donné une position et une orientation, on va chercher à calculer les angles des joints à chercher.
Plus généralement, on va chercher à résoudre les équations cinématiques sur une trajectoire, i.e. calculer les vitesses des joints.

\begin{definition}
	Une fois qu'on a défini un repère de reférence dans l'espace, on peut localiser n'importe quel point de l'espace en spécifiant un vecteur de position.
	Pour établir l'orientation d'un système on y attache un système de coordonnées.
\end{definition}

Il y a d'autres manières de représenter les rotations:
\begin{itemize}
	\item Matrices Orthonormales $3\times 3$ - $9$ items
	\item Les Angles d'Euler - $3$ items
	\item L'Axe de Rotation - $3$ items
	\item Les Quaternions Unitaires - $4$ items
\end{itemize}

\begin{definition}
	Pour les matrices orthonormales, l'orientation est définie par la matrice de rotation, i.e. l'expression du nouveau système de coordonnées dans le repère de référence.
	\label{def:rotationmatrix}
\end{definition}

\begin{proposition}
	Chaque élément de la matrice de rotation correspond au produit scalaire entre les deux systèmes de coordonnées.
\end{proposition}
\begin{proposition}
	Le groupe des rotations en dimension $3$ correspond au groupe Spécial Orthogonal de dimension $3$.
	En particulier, si $R \in SO(3)$:
	\begin{itemize}
		\item $R$ est orthonormale
		\item $R\transpose{R} = \Id_{3}$
		\item $\det(R) = 1$.
	\end{itemize}
	On a $9$ degrés de liberté mais $6$ contraintes liées à l'orthogonalité, $SO(3)$ est donc un groupe de dimension $3$.
\end{proposition}

\begin{definition}
	Les angles d'Euler sont une suite de $3$ rotations successives autour des axes fixés du repère de référence.
	C'est une méthode équivalente aux matrices orthonormales, bien que coûteuse à calculer.
	\label{def:eulerangles}
\end{definition}

\begin{proof}

\end{proof}

Il y a parfois des singularités, lorsqu'obtenir un mouvement continu implique d'avoir une vitesse infinie.

\begin{definition}
	Les quaternions sont la donnée d'un vecteur $\vec{k}$ et d'un angle $\theta$ permettant d'obtenir $4$ paramètres dont les carrés somment à $1$.
	C'est une méthode équivalent aux matrices orthonormales mais qui n'est pas différentiable car génère une double couverture de la sphère unité.
	\label{def:quaternions}
\end{definition}

On a également besoin de définir les vitesses angulaires.
\begin{definition}
	Les vitesses angulaires correspondent à la dérivée temporelle d'une rotation.
	Elles sont dans l'espace tangent de $SO(3)$ noté $\mathfrak{so}(3)$.
	\label{}
\end{definition}

\begin{thm}
	$\so(3)$ est le groupe des matrices antisymétriques.
	\label{prop:so3}
\end{thm}

\begin{proof}
	On a, en dérivant la condition d'orthogonalité:
	\begin{align*}
		R(t)\transpose{R(t)} = \Id_{3}                       \\
		\dot{R}\transpose{R} + R\transpose{\cdot{R}} = 0_{3} \\
		\cdot{R}\transpose{R} = -\transpose{\left(\cdot{R}\transpose{R}\right)}
	\end{align*}
	On a: $W = -\transpose{W}$ si on pose $W = -\transpose{\left(R\transpose{\cdot{R}}\right)}$.
	Donc:
	\begin{equation*}
		W \in \left\{ \begin{bmatrix}
			0     & -w_{1} & -w_{2} \\
			w_{1} & 0      & -w_{3} \\
			w_{2} & w_{3}  & 0
		\end{bmatrix}\ \middle|\ w \in \R^{3}\right\} = \A_{3}
	\end{equation*}
	La réciproque est triviale.
\end{proof}


\begin{proposition}
	Si $W$ reste constante dans le temps, on peut paramétriser les rotations par un vecteur de $\R^{3}$.
	\label{prop:exponentialmap}
\end{proposition}
\begin{proof}
	On a: $W = -\transpose{W}$ si on pose $W = -\transpose{\left(R\transpose{\cdot{R}}\right)}$.
	On a donc:
	\begin{equation*}
		\cdot{R}(t) = WR(t) \text{ avec } R(0) = R_{0} \in SO(3)
	\end{equation*}
	Autrement dit:
	\begin{equation*}
		R(t) = R(0) \exp\left(tW\right) \text{ avec } \exp(tW) = \sum_{n = 0}^{+\infty} \frac{\left(tW\right)^{n}}{n!}
	\end{equation*}
\end{proof}
Puisqu'on discrétise le temps, on peut toujours considérer $W$ constante et appliquer cette formule.

Pour les matrices antisymétriques, il est aisé de calculer cette exponentielle:
\begin{proposition}
	On a:
	\begin{equation*}
		\exp(tW) = \Id_{3} + \sin(t)W + \frac{1 - \cos(t)}{t}W^{2}
	\end{equation*}
	\label{prop:antisymmetricexponential}
\end{proposition}
\begin{proof}
	Étant donné $W \in \A_{3}$, on a $W^{2} = W\transpose{W} - \Id_{3}$ et $W^{3} = -W$.
\end{proof}

\begin{proposition}
	L'exponentielle correspond à:
	\begin{equation*}
		\exp: \applic{\so(3)}{SO(3)}{W}{\Id_{3} + \sin(t)W + \frac{1 - \cos(t)}{t}W^{2}}
	\end{equation*}
	qui est surjective et $2\pi$ périodique.
	\label{prop:surjectiveexponential}
\end{proposition}
Autrement dit, si on n'échantillonne pas assez, on va perdre en informations sur le nombre de tour faits.

\begin{thm}
	On peut définir un opérateur réciproque à l'exponentielle qu'on notera $\log: R \in SO(3) \to \so(3)$.
	On a:
	\begin{equation*}
		W = \frac{\norm{W}}{2\sin(\norm{W})}\begin{bmatrix}
			r_{32} - r_{23} \\
			r_{13} - r_{31} \\
			r_{21} - r_{12}
		\end{bmatrix}
	\end{equation*}
	Ceci nous permet de calculer la distance entre deux rotations, appelée géodésique:
	\begin{equation*}
		d(R_{1}, R_{2}) = \norm{\log\left(\transpose{R_{1}}R_{2}\right)}
	\end{equation*}
	\label{prop:definitionlogarithm}
\end{thm}

\subsection{Transformations Rigides}
\begin{definition}
	Le groupe des transformations rigides en dimension $3$ est le groupe spécial euclidien dénoté $SE(3)$:
	\begin{equation*}
		SE(3) = \left\{g: \E^{3} \to \E^{3}, \forall X, Y\in \E^{3}, \norm{g(X) - g(y)}^{2} = \norm{X - Y}^{2}\right\}
	\end{equation*}
	\label{def:SE3}
\end{definition}

\begin{proposition}
	Une transformation rigide agit sur des points $p \in \R^{3}$ et est isomorphe à une rotation et une translation:
	\begin{equation*}
		M(p) = R(p) + t\in E^{3}
	\end{equation*}
	\label{prop:decompos}
\end{proposition}

De la même manière qu'on a pu plonger $SO(3)$ dans le groupe des matrices orthonormales, on peut plonger $SE(3)$ dans le groupe des matrices inversibles $GL(4)$:
\begin{equation*}
	M(p) = \begin{bmatrix}
		R     & t \\
		0_{3} & 1
	\end{bmatrix}
	\begin{bmatrix}
		p \\
		1
	\end{bmatrix}
\end{equation*}

On a une loi de composition:
\begin{equation*}
	M_{1} \circ M_{2} = \begin{bmatrix}
		R_{1}R_{2} & t_{1} + R_{1}t_{2} \\
		0_{3}      & 1
	\end{bmatrix}
\end{equation*}
Ceci nous permet de définir un espace des vitesses linéaires et angulaires:
\begin{equation*}
	\dot{M}(s)M(s)^{-1} = \begin{bmatrix}
		\dot{R}(s)\transpose{R}(s) & \dot{t}(s) - \dot{R}(s)\transpose{R}(s)t(s) \\
		0_{3}                      & 1
	\end{bmatrix}
\end{equation*}
On reconnaît la vitesse angulaire et la vitesse linéaire:
la paire $(v, w) \in se(3) \simeq \R^{3} \times \so(3)$ correspond au vecteur tangent.

\begin{definition}
	De la relation entre $M(s)$ et sa dérivée $\dot{M}(s)$, on peut construire une carte exponentielle:
	\begin{equation*}
		\exp_{\se(3)}(v, w) = \begin{bmatrix}
			\exp_{\so(3)}(w) & \frac{1}{\norm{w}}\left(w\transpose{w}v + \left(I_{3} - \exp_{\so(3)}(w)\right)w \times v \right) \\
			0_{3}            & 1
		\end{bmatrix}
	\end{equation*}
	de sorte que:
	\begin{equation*}
		\exp_{\se(3)}: \applic{\se(3)}{SO(3)}{(v, w)}{M}
	\end{equation*}
	\label{def:expse3}
\end{definition}

\begin{definition}
	On peut de même définir une distance par un logarithme:
	\begin{equation*}
		\log_{SE(3)}: M \longmapsto
	\end{equation*}
	\label{def:logSE3}
\end{definition}

\subsection{Cinématique directe}
L'idée est, à partir des différents angles au niveau des articulations du robot, comment calculer la position et l'orientation terminale.

\begin{definition}
	Une chaîne cinématique est composée de corps rigides interconnectés par des articulations. Les articulations définissent les degrés de liberté possible le long de la chaîne cinématique.
	La topologie de l'articulation définit les degrés de liberté, i.e., le mouvement relatif permis entre les deux corps.
	\label{def:chainecinematique}
\end{definition}

Pour une liaison pivot, on a par exemple:
	\begin{equation*}
			      K_{i}: q\in \S^{1} \simeq \R \mapsto \begin{bmatrix}
				      \cos q  & \sin q & 0 & 0 \\
				      -\sin q & \cos q & 0 & 0 \\
				      0       & 0      & 1 & 0 \\
				      0       & 0      & 0 & 1
			      \end{bmatrix}
		      \end{equation*}
	\label{def:articulations}

\begin{proposition}
	On peut calculer le placement de l'organe terminal en connaissant la mesure des angles articulaires:
	\begin{enumerate}
		\item On calcule la position relative du repère d'une articulation par rapport au repère de l'articulation précédente
			\begin{equation*}
				M_{i}^{i - 1}(q_{i}) = P_{i}^{i- 1}K_{i}(q_{i})
			\end{equation*}
		\item On récupère en composant la position relative du repère $n$ par rapport au repère de la première base.
			\begin{equation*}
				K_{n}^{0} : q = (q_{1}, \ldots, q_{n}) \in Q_{1}^{n} = \times_{i = 1}^{n} Q_{i} \mapsto \times_{i = 1}^{n}P_{i}^{i - 1}K_{i}(q_{i}) \in \mathrm{SE}(3)
			\end{equation*}
		\item On peut alors lire l'espace des configurations:
		      \begin{equation*}
			      Q = \times_{k = 1}^{N} Q_{i} \simeq \R^{\sum_{k = 1}^{N} n_{i}}
		      \end{equation*}
	\end{enumerate}
	\label{prop:positiondirecte}
\end{proposition}

\begin{definition}
	Le changement de configuration d'une articulation provient de la mise en mouvement de cette articulation:
	\begin{enumerate}
		\item Position relative de l'articulation entre les deux repères adjacents:
			\begin{equation*}
				K_{i}: q_{i} \in Q_{i} \to M_{i}(q_{i}) \in \mathrm{SE}(3)
			\end{equation*}
		\item Vitesse relative engendrée par l'articulation:
			\begin{equation*}
				k_{i}: (q_{i}, \dot{q_{i}}) \in T_{q_{i}}Q_{i} \to v_{i}(q_{i}) = S_{i}(q_{i})\dot{q_{i}} \in \mathrm{se(3)}
			\end{equation*}
	\end{enumerate}
	Pour quelques liaisons:
	\begin{multicols}{2}
	\begin{itemize}
		\item Liaison Pivot:
			\begin{equation*}
				S_{i}(q_{i}) = \begin{bmatrix}
					0 \\ 0 \\ 0 \\ 0 \\ 0 \\ 1
				\end{bmatrix}\in \R^{6 \times 1}
			\end{equation*}
		\item Prismatique \begin{equation*}
			      S_{i}(q_{i}) = \begin{bmatrix}
				      0 \\ 0\\ 1\\ 0\\ 0\\ 0
			      \end{bmatrix}
		      \end{equation*}
		\item Cylindrique
		\item Planaire
		\item Vis
		\item Sphérique
	\end{itemize}
	\end{multicols}{2}
	\label{def:vitessearticulaire}
\end{definition}

\begin{proposition}
	On peut calculer la vitesse de l'organe terminal en connaissant la mesure des angles et des vitesses articulaires:
	\begin{enumerate}
		\item On calcule la position relative de l'organe terminal:
			\begin{equation*}
				K_{n}^{0} : q = (q_{1}, \ldots, q_{n}) \in Q_{1}^{n} = \times_{i = 1}^{n} Q_{i} \to \times_{i = 1}^{n}P_{i}^{i - 1}K_{i}(q_{i}) \in \mathrm{SE}(3)
			\end{equation*}
		\item On calcule la vitesse relative de l'organe terminal:
		      \begin{equation*}
			      k_{n}^{0}: (q, \dot{q}) \mapsto \sum_{i = 1}^{n} k_{i}(q_{i}, \dot{q_{i}}
		      \end{equation*}
	\end{enumerate}
	\label{prop:vitesseterminale}
\end{proposition}

\begin{thm}
	On relie la vitesse articulaire à la vitesse spatiale des corps en mouvements par:
	\begin{equation*}
		k_{n}(q, \dot{q}) = J(q)\begin{bmatrix}
			\dot{q_{1}} \\
			\vdots      \\
			\dot{q_{n}}
		\end{bmatrix}
	\end{equation*}
\end{thm}
\begin{proof}
	Pour chaque articulation on a:
	\begin{equation*}
		v_{i}(q_{i}) = S_{i}(q_{i})\cdot{q}_{i}
	\end{equation*}
	Ceci correspond à la dérivée temporelle de la géométrie articulaire:
	\begin{equation*}
		\dot{M_{i}}(q_{i})M_{i}^{-1}(q_{i}) = S_{i}(q_{i})\dot{q}_{i}
	\end{equation*}
	Finalement:
	\begin{equation*}
		k_{n}(q, \dot{q}) = \sum_{i = 1}^{n}k_{i}(q_{i}, \dot{q}_{i}) =
		\begin{bmatrix}
			S_{i}(q_{1}), \cdots, S_{n}(q_{n})
		\end{bmatrix}
		\begin{bmatrix}
			\dot{q}_{1} \\
			\vdots      \\
			\dot{q_{n}}
		\end{bmatrix}
	\end{equation*}
	On reconnaît bien la jacobienne.
\end{proof}

\subsection{Cinématique Inverse}
\begin{definition}
	La cinématique inverse consiste à résoudre le problème suivant: étant donné la position et l'orientation d'un corps, à calculer la valeur des angles articulaires.
	Autrement dit, il s'agit de résoudre le problème suivant:
	\begin{equation*}
		\min_{q\in Q} \norm{\log \left(K_{n}^{0}(q)^{-1}M^{*}\right)}
	\end{equation*}
	\label{def:cinematiqueinverse}
\end{definition}
On part du fait que si $q\in Q$ et $\delta q \in T_{q} Q$, alors $q + \delta q = q^{+} \in Q$.
Le problème initial peut se mettre sous la forme d'un problème d'optimisation:
\begin{equation*}
	\min_{x \in \R^{n}}\frac{1}{2}\norm{f(x)}^{2}
\end{equation*}
On résout ce problème en linéarisant et en créant une suite d'itérés:
\begin{equation*}
	\min_{p\in \R^{n}} \frac{1}{2}\norm{f(x + p)}^{2}
\end{equation*}
i.e.:
\begin{align*}
	\min_{p\in \R^{n}}\frac{1}{2}\norm{f(x) + J(x)p}^{2}                                   \\
	\nabla_{p}\frac{1}{2}\norm{\cdot}^{2} = \transpose{J(x)}\left(f(x) + J(x)p \right) = 0 \\
	p^{*} = -\left(\transpose{J(x)}J(x) \right)^{-1}J(x)f(x)                               \\
	p^{*} = -J(x)^{+}f(x)
\end{align*}
On définit alors une suite $x_{k}$ vérifiant $f(x_{k + 1}) < f(x_{k})$ à une marge près:
\begin{equation*}
	x_{k + 1} = x_{k} + \alpha_{k}p_{k}
\end{equation*}
vérifiant:
\begin{equation*}
	f(x_{k} + \alpha_{k}p_{k}) < f(x_{k}) + \alpha_{k} Df(x_{k})p_{k}
\end{equation*}
et on itère jusqu'à obtenir $\norm{\transpose{J(x)}f(x)} < \epsilon^{*}$.

Toutefois, il peut y avoir plusieurs solutions différentes pour une même position finale.

\subsection{Suivi de Trajectoire}
Pour le suivi de trajectoire, on cherche à minimiser l'écart entre la position + vitesse de l'organe terminal vis-à-vis de la trajectoire de référence.

\begin{proposition}
	Valeurs de l'erreur:
	\begin{equation*}
		e(t, q(t)) = M(q(t)) \ominus M^{*}(t) = \log_{\se(3)}\left(M^{*}(t)^{-1}M(q(t)) \right)
	\end{equation*}
	Vitesse de l'erreur:
	\begin{equation*}
		\dot{e}(t, q(t), \dot{q}(t)) = v(q(t), \dot{q}(t)) - v*(t) = J(q(t))\dot{q}(t) - v^{*}(t)
	\end{equation*}
	Profil de correction d'erreur:
	\begin{equation*}
	\dot{e}(t, q(t), \dot{q}(t)) = -\lambda e(t, q(t)), \lambda > 0
	\end{equation*}
	\label{prop:erreurs}
\end{proposition}

On retrouve à nouveau un problème linéaire:
\begin{equation*}
	\min_{\dot{q}(t)}\frac{1}{2}\norm{\dot{e} + \lambda e}^{2}
\end{equation*}
qui se résout en:
\begin{equation*}
	\min_{\dot{q}}\frac{1}{2}\norm{J(q)\dot{q} - v^{*} + \lambda\left( M(q)\ominus M^{*} \right)}
\end{equation*}
i.e. $\min_{x} \frac{1}{2}\norm{Ax - b}^{2}$ qui est atteint en:
\begin{equation*}
	x^{*} = A^{+}b = \left(\transpose{A}A\right)^{-1}\transpose{A}
\end{equation*}

Pour résoudre correctement ceci, on va exploiter la redondance en solutions, on se donne deux taches, dont une strictement prioritaire:
\begin{equation*}
	(A_{1}, b_{1}) \gg (A_{2}, b_{2})
\end{equation*}
On peut avoir, ou une approche relachée:
\begin{equation*}
	\min_{x} \alpha_{1}\norm{A_{1} x -b_{1}}^{2} + \alpha_{2}\norm{A_{2}x - b_{2}}^{2} \text{ avec } \alpha_{1} \gg \alpha_{2} > 0
\end{equation*}
ou une approche stricte:
\begin{equation*}
	\min_{x}\frac{1}{2}\norm{A_{2}x - b_{2}}^{2} \text{ avec } A_{1}x = b_{1}
\end{equation*}

Ceci se résout:
\begin{equation*}
	\min_{x}\frac{1}{2}\norm{A_{1}x - b_{1}}^{2} \Rightarrow x^{*} = A_{1}^{+}b_{1}
\end{equation*}
Par redondance:
\begin{equation*}
	\forall y_{2} \in \R^{n}, x^{*} = A_{1}^{+}b_{1} + \left(I_{n} - A_{1}^{+}A_{1}\right)y_{2}
\end{equation*}


\section{Planification de Mouvement}
\subsection{Mathématisation}
\begin{definition}
	On appelle espace de travail l'espace ambient $\E^{3}$.
	Un système physique est alors un ensemble $\{K_{1}, \cdots, K_{N}\} \in \mS \subseteq \mathcal{K}\left(\E^{3}\right)^{N}$ de compacts.
	L'espace $\mS$ de tous les ensembles possibles est appelé espace d'états.
	\label{def:motionplanningspace}
\end{definition}


\begin{definition}
	On va paramétrer l'espace d'états par des transformations $q = \phi_{1}, \ldots, \phi_{M} \in \mC$ de sorte que:
	\begin{equation*}
		S_{q} = \underbrace{\phi_{1}(K_{1}^{0}), \ldots, \phi_{M}(K_{M}^{0})}_{\text{Robot}}, \underbrace{K_{M+1}^{0}, \ldots, K_{N}^{0}}_{\text{Obstacles}}
	\end{equation*}
	où les $K_{i}^{0}$ sont fixés.
	On appelle l'ensemble des transformations générant $\mS$ l'espace des configurations $\mC$.
	En général, $\mC$ est une variété. La dimension de $\mC$ est le degré de liberté du système.
\end{definition}
La dimension numérique n'est pas nécessairement la dimension minimale de la représentation.
Les corps de référence $K_{i}^{0}$ peuvent être représentés ou bien par des triangulations (meshes) ou bien par une décomposition polynomial ($f(x, y) \leq 0$ avec $f$ une fonction polynomiale bien choisie).

\begin{definition}
Pour gérer les collisions on définit:
\begin{align*}
	\mC_{obs} = \{q \in \mS \\
			\exists i\neq j \in \left(1, M\right)^{2}, \phi_{i}\left(K_{i}^{0}\right)\cap \phi_{j}\left(K_{j}^{0}\right) \neq 0 \\
			\exists i \in \left(1, M\right), j\in \left(M + 1, N\right), \phi_{i}(K_{i}^{0}) \cap K_{j}^{0} \neq \emptyset\}
	\end{align*}
et on pose $\mC_{free} = \mC \setminus\mC_{obs}$.
Un espace physique $S_{q} = \phi_{1}\left(K_{1}^{0}\right), \ldots, \phi_{M}(K_{M}^{0}), K_{M + 1}^{0}, \ldots, K_{N}$ est possible si et seulement si $q \in \mC_{free}$.
\end{definition}

Avec l'approche par espace de configurations, l'objectif de la planification de mouvement de trouver un mouvement continu entre des états du système qui ne cause pas de collision devient un problème de trajectoire ponctuelle.
Étant donnés $q_{i}, q_{g} \in \mC_{free}$, trouver:
\begin{equation*}
	\gamma \in \cont^{0}\left(\left[0, 1\right], \cont_{free}\right), \gamma(0) = q_{i}, \gamma(1) = q_{g}
\end{equation*}

\subsection{Paradigmes et algorithmes}
Il y a trois paradigmes principaux qui sont en compétition:
\begin{itemize}
	\item La planification combinatoire (planification exacte), morale mais peu réaliste.
	\item L'optimisation de trajectoire (planification locale), moderne mais locale.
	\item La planification de mouvement basée sur l'échantillonnage (planification probabiliste), peu morale mais répandue.
\end{itemize}

Les algorithmes de plannification de mouvement peuvent être:
\begin{itemize}
	\item Complets: Si une solution existe elle est trouvée, sinon on renvoie une erreur.
	\item Semi-Complets: Si une solution existe elle est trouvée, sinon, on peut tourner pendant un temps infini.
	\item Probablement Complets: Si une solution existe, la probabilité de la trouver tend vers $1$ quand le nombre d'itérations tend vers l'infini.
\end{itemize}

\subsubsection{Planification Combinatoire et Optimisante}
Pour cet algorithme on construit un nombre fini de routes, des graphes dont les sommets sont des configurations et les arêtes sont des chemins.
Une route est un graphe topologique de l'espace des configurations accessibles $\mC_{free}$ si elle est accessible depuis $\mC_{free}$ et préserve l'homotopie.

Dans le cas où $\mC_{free}$ est linéaire par morceau, on peut utiliser le principe de balayage du plan pour déterminer l'accessibilité en $\O(n\log(n))$.
On calcule alors une triangulation de Delaunay et son graphe d'adjacence en $\O(n^{2})$.

Dans le cas général, on utilise des décompositions spécialisées et l'algorithme de Canny, puis la procédure générale de décomposition algébrique en cylindres de Schwartz et Sharir.

Pour l'algorithme d'optimisation, on n'a pas de garanties d'explorer toutes les classes d'homotopie et on a besoin de méthodes de contrôle optimal avec un coût terminal fort.

\subsubsection{Plannification par Échantillonnage}
Ici, on va commencer par utiliser un détecteur de collision pour séparer la phase de planification de la géométrie initiale.
Ensuite, on va échantillonner systématiquement pour limiter le non-déterminisme.
Pour générer des routes approximatives, on va échantillonner des configurations, voir si elles entrent en collision, construire une route de celles qui sont libres et ajouter comme arêtes celles qui sont navigables sans collisions.
La navigation se fait en utilisant une interpolation linéaires avec la carte exponentielle d'un groupe de Lie.
Cette méthode génère de nombreux sommets inutiles, ce qui aumgente le temps de calcul quadratiquement.
On utilise alors le théorème de Chvátal pour la résolution du problème de la gallerie d'art.
On obtient alors l'algorithme Visibility-PRM, décrit par Laumond en 2000.

\subsection{Contrainte Dynamique}
On pose $\dot{x} = f(x, u)$.
On dit qu'un système est non-holonome si
\begin{equation*}
	\left\{f(x, u) \in T_{x}\cont, u \in \mathcal{U} \right\}
\end{equation*}
ne couvre par l'espace tangent $T_{x}\cont$.
On dit que le système est localement contrôlable si pour tout intervalle de temps, la région atteignable contient un voisinage de l'état.
Dans le cas général, on doit travailler dans $T\cont$ pour l'espace position/vitesse.
Ceci double la dimension et ajoute un grand nombre de contraintes, ce qui limite l'efficacité des méthodes d'échantillonnage.

\end{document}
