\documentclass[math]{cours}
\title{Lambda-Calcul et Catégories}
\author{Paul-André Mellies}

\begin{document}
\bettertitle

\section{Introduction}
\subsection{Introduction Historique}
Le $\lambda$-calcul a été introduit dans les années 1930 par Church.
Il est en lien avec des questions de linguistique, de logique et de calculabilité.
\begin{definition}
	Le $\lambda$-calcul est un langage de preuves pour une logique intuitionniste minimale (ou pour la théorie simple des types).
	\label{def:lambdacalcul}
\end{definition}
Le $\lambda$-calcul non typé a la puissance des machines de Turing.

\begin{definition}
	Les catégories sont des structures algébriques (parfois appelées monoïdes à plusieurs objets)
	\label{def:categoryyyyyyyy}
\end{definition}
Historiquement, les catégories ont été introduites pour la topologie algébriques dans les années 1940 avec les travaux de Eilenberg et Maclane.
Leur objectif était de comprendre les propriétés fondamentales des espaces en s'intéressant aux morphismes entre espaces (les fonctions continues).

Il y a une connexion forte au niveau de la théorie des preuves entre $\lambda$-calcul et théorie, qui est très similaire à ce qui s'était passé lors de la définition des algèbres de Boole.
Dans ce deuxième cas, Boole montre qu'on peut mettre un ordre partiel sur les formules de la logique classique:
\begin{equation*}
	\phi \leq \Psi \text{ si et seulement si } \phi \Rightarrow \Psi
\end{equation*}
Une algèbre de Boole $(A, \leq, \land, \lor, \lnot, \top, \bot)$ est un ensemble ordonnée $A, \leq$ muni de fonctions préservant l'ordre $\land, \lor: A\times A \to A$:
\begin{equation*}
	\phi_{1} \leq \Psi_{1} \land \phi_{2} \leq \Psi_{2} \Longrightarrow \phi_{1} \land \phi_{2} \leq \Psi_{1}\land \Psi_{2}, \phi_{1} \lor \phi_{2} \leq \Psi_{1}\lor \Psi_{2}
\end{equation*}
et d'une fonction inversant l'ordre $\lnot: A \times A$:
\begin{equation*}
	\phi \leq \Psi \Rightarrow \lnot \Psi \leq \lnot \phi
\end{equation*}
vérifiant un certain nombre d'axiomes:
\begin{multicols}{2}
	\begin{itemize}
		\item Associativité:
				\begin{align*}
					\left(\phi_{1}\land\phi_{2}\right) \land \phi_{3} = \phi_{1}\land\left(\phi_{2}\land \phi_{3}\right)\\
					\left(\phi_{1}\lor\phi_{2}\right) \lor \phi_{3} = \phi_{1}\lor\left(\phi_{2}\lor \phi_{3}\right)
				\end{align*}
		\item Neutralité:
				\begin{align*}
					\phi \land \top = \top \land \phi = \phi\\
					\phi \lor \bot = \bot \lor \phi = \phi
				\end{align*}
		\item Commutativité:
			\begin{equation*}
				\phi \land \Psi = \Psi \land \phi, \phi \lor \Psi = \Psi \lor \phi
			\end{equation*}
		\item Distributivité:
			\begin{align*}
				\phi \land \left(\Psi_{1} \lor \Psi_{2}\right) = \left(\phi \land \Psi_{1}\right) \lor \left(\phi \land \Psi_{2}\right)\\
				\phi \lor \left(\Psi_{1} \land \Psi_{2}\right) = \left(\phi \lor \Psi_{1}\right) \land \left(\phi \lor \Psi_{2}\right)\\
				\lnot\left(\phi \land \Psi\right) = \lnot\phi \lor \lnot \Psi, \lnot\top = \bot
			\end{align*}
		\item Idempotence:
			\begin{align*}
				\phi = \lnot\lnot \phi\\
				\phi \land \phi = \phi, \phi \lor \phi = \phi
			\end{align*}
	\end{itemize}
\end{multicols}
Dans une algèbre de Boole, $\phi \land \psi$ est le plus grand minorant de $\phi$ et de $\psi$ et $\phi \lor \psi$ est le plus petit majorant de $\phi$ et $\psi$
\begin{remarque}
	On peut définir l'implication dans les algèbres de Boole comme $\phi \implies \psi = \lnot\phi \lor \psi$
	\label{rem:implicationboole}
\end{remarque}
On va passer du système des algèbres de Boole ($\phi \leq \psi$ s'il existe une preuve que $\phi$ implique $\psi$)
au système de catégories comme proposé par Lambek.
\begin{definition}
	On peut voir une catégorie comme un graphe dont les noeuds sont appelés objets et les arêtes sont appelées morphismes, maps ou flèches.
	On peut composer les arêtes d'une catégorie, comme pour se déplacer sur le graphe.
\end{definition}
Ici on considère une catégorie dont les objets sont des formules logiques, et les morphismes sont des preuves d'implication.
Il y a donc des liens très forts entre les catégories obtenues avec des formules et des preuves et celles obtenues par des types et des programmes fonctionnels entre les types.
On va ici étudier les catégories à travers leurs représentations: on peut mieux comprendre une catégorie en la représentant comme une famille d'actions au moyen d'un foncteur.

\subsection{Notion de Catégorie: Premiers Exemples}
\begin{definition}[Catégorie]
	Une \emph{catégorie} est décrite par les données suivantes:
	\begin{itemize}
		\item[0] Une classe\footnote{propriété non incarnée par un ensemble} d'objets (les noeuds d'un graphe).
			On appelle les catégories dont les objets définissent un ensemble des \emph{petites catégories}.
		\item[1] Pour toute paire d'objets $A, B$, un ensemble $\Hom(A, B)$ de fonctions de $A$ vers $B$ appelées \emph{morphismes} ou \emph{maps}.
			On note ceci: $f: A \to B$ ou $A \xrightarrow{f} B$.
		\item[2] Pour tous objets $A, B, C$, une loi de composition $\circ_{A, B, C}$:
			\begin{equation*}
				\begin{array}{rcl}
				\Hom(B, C) \times \Hom(A, B) &\rightarrow & \Hom(A, C)\\
				(g, f) &\mapsto & g\circ f
				\end{array}
			\end{equation*}
		\item[2] Pour tout objet $A$, une fonction identité $\id_{A} \in \Hom(A, A)$
		\item[3] Associativité:
			\begin{center}
				\begin{category}[]
				& B \arrow{r}{g} & C\arrow{rd}{h} & \\
				A\arrow{ur}{f}\arrow{urr}{g\circ f}\arrow{rrr}{h\circ\left(g\circ f\right)} & & & D
				\end{category}
			\end{center}
			On peut aussi voir la composition comme la couverture de l'aire entre les noeuds du graphe:
			\begin{category}[]
				& B\arrow[r, no head]\arrow[drr, no head] & C \arrow[no head]{dr}{} & \\
				A\arrow[ur, no head]\arrow[urr, crossing over, no head]\arrow[rrr, no head] & & & D
			\end{category}
			ou encore:
			\begin{category}[]
				& B\arrow[d, no head]\arrow[no head]{ddr}{}\arrow[no head]{ddl}{} & \\
				& C\arrow[no head]{dr}{}\arrow[no head]{dl}{} & \\
				A\arrow[rr, no head] & & D
			\end{category}
		\item[3] Neutralité: $f\circ \id_{A} = f = \id_{B} \circ f$.
	\end{itemize}
	\label{def:categorie}
\end{definition}
\begin{remarque}
	Voir \emph{nerf d'une catégorie} pour voir la notion d'ensemble simplicial.
\end{remarque}

\begin{exemple}
	\begin{description}
	\item[PoSet] On considère d'abord les ensembles partiellement ordonnés comme des catégories:
		\begin{proposition}
	Chaque ensemble partiellement ordonné $(A, \leq)$ définit une catégorie dont les objets sont des éléments $a, b, c$ de $A$
		avec une map $a\to b$ si et seulement si $a \leq b$ et $\Hom(a, b)$ un singleton si $a\leq b$ et $\emptyset$ sinon.
	\label{prop:posetcat}
	\end{proposition}
\begin{proof}
	On doit montrer l'existence d'une identité, d'une loi de composition, et les propriétés d'associativité et de neutralité:
		\begin{itemize}
				\item Par réflexivité de l'ordre: $a \leq a$ et donc $a \xrightarrow{\id_{a}} a$ existe.
					\item Par transitivité: si $a \leq b$ et $b \leq c$ alors $a\leq c$ et on peut donc voir la transitivité comme une composition:
						\item L'associativité et la neutralité découlent immédiatement du fait que chaque $\Hom(a, b)$ contient au plus un élément.
						\end{itemize}
			\end{proof}

	Réciproquement, une catégorie $\phi$ telle que chaque ensemble d'homomorphismes contienne au plus un élément est la même chose qu'un préordre:
\begin{category}[]
	a\arrow[loop left]{l}{\id = g \circ f} \arrow{r}{f} &
		b\arrow{l}{g} \arrow[loop right]{l}{f\circ g = \id}
\end{category}

	\item[Monoïde] On considère maintenant les Monoïdes comme des catégories.
\begin{proposition}
	Chaque monoïde (M, m, e) définit une catégorie notée $\B M$ appelée sa suspension avec un seul objet $*$ tel que:
		$\Hom(*, *) = M$ et $\circ: m, n \mapsto n\cdot m$.
	\label{prop:moncat}
	\end{proposition}
	L'associativité et la neutralité de la catégorie $\B M$ sont des conséquences directes de l'associativité et de la neutralité du monoïde.

	\begin{remarque}
En prenant $M = \left(\N, +, 0\right)$, la représentation ainsi obtenue des entiers a un lien direct avec la théorie de l'homotopie: c'est le groupe de Poincaré (ou groupe fondamental) d'un espace topologique pointé.
Tout espace topologique définit une catégorie dont les objets sont les éléments de l'espace topologique et les flèches sont les chemins, à homotopie près.
\end{remarque}

	\item[Types] On considère la catégorie cartésienne fermée des types simples comme objets et des $\lambda$-termes simplement typés (module $\beta\eta$-équivalence) comme morphismes:
		\begin{category}[column sep=large]
			A\arrow[r, "x:A\models t:B"]\dbarrow[rr, "{x:A\models a\left[t/y\right]: C}"] & B\arrow[r, "y:B\models u:C"] & C
 \end{category}
$A \xrightarrow{x:A \models t:B} B \xrightarrow{y:B\models u:C} C$ flèche dessous ($x:A \models u[t/y]: C$)
Cette catégorie jouera le rôle en théorie de la démonstration de l'algèbre de Boole des formules
\end{description}
\end{exemple}

\section{Catégories Cartésiennes}
\begin{definition}
	Un produit cartésien de deux objets $A$ et $B$ dans un catégorie $\phi$ est la donnée d'un triplet
	\begin{equation*}
		\left(A \times B, \pi_{1}: A\times B \to A, \pi_{2}: A \times B \to B\right)
	\end{equation*}
	\begin{category}[]
		A & & B \\
		& A\times B \arrow{ul}{\pi_{1}}\arrow{ur}{\pi_{2}} &
	\end{category}
	tel que pour toute paire de flèches: $X\xrightarrow{f} A$ et $X\xrightarrow{g} B$, il existe un et une seule flèche: $h: X \to A \times B$ telle que
	$f = \pi_{1} \circ h, g = \pi_{2} \circ h$.
	Pour $\phi = Set$, par exemple, $A \times B = \left\{(a, b)\mid a \in A, b\in B\right\}$ et $\pi_{1}: (a, b) \mapsto a$ et $h: x \mapsto (fx, gx)$.
	\label{def:prodcart}
\end{definition}

\begin{exemple}
	Dans une catégorie définie par une relation d'ordre sur $A, \leq$, le produit cartésien de $a, b\in A$ c'est la même chose que la borne inférieure $a\land b$ de $a$ et $b$ définie comme le plus grand des minorants de $a$ et $b$.
	\begin{category}[]
		a & & b\\
		& a\land b\arrow{ul}{\leq}\arrow{ur}{\leq} &\\
		& x\arrow[bend right]{uur}\arrow[bend left]{uul}\arrow[dotted]{u}{h} &
	\end{category}
\end{exemple}

\begin{definition}[Objet Terminal]
	Un objet terminal $\mathds{1}$ dans une catégorie $\cont$ est un objet tel que pour tout objet $A$ de $\cont$, $\Hom(A, \mathds{1})$ est une singleton.
	\label{def:objterm}
\end{definition}
\begin{remarque}
	Un objet initial est un objet terminal dans la catégorie duale (catégorie ou on renverse les flèches).
\end{remarque}

\begin{definition}
	Une catégorie cartésienne est une catégorie $\cont$ munie d'un produit cartésien $\left(A\times B, \pi_{1}, \pi_{2}\right)$ et munie d'un objet terminal.
	\label{def:cartcat}
\end{definition}

\begin{definition}
	Une paire $\mathds{I} \xrightarrow{f} \mathds{J}$ et $\mathds{J} \xrightarrow{g} \mathds{I}$ telle que $f\circ g= \id_{\mathds{J}}$ et $g\circ f = \id_{\mathds{I}}$ est appelée \emph{isomorphisme}
	\label{def:isomorphisme}
\end{definition}

\begin{proposition}
	Deux objets terminaux sont isomorphes. Deux produits cartésiens d'une même paire d'objets sont isomorphes.
	\label{prop:isomorphismeterminaux}
\end{proposition}
\begin{proof}
	Soit $\mathds{I}, \mathds{J}$ deux objets terminaux d'une même catégorie. Il existe un unique morphisme $f$ (resp. $g$) de $\mathds{I}$ (resp. $\mathds{J}$) vers $\mathds{J}$ (resp. $\mathds{I}$).
	De même, il existe un unique morphisme $\id_{\mathds{J}}$ de $\mathds{J}$ vers lui-même.
	Le diagramme ci-dessous commute donc:
	\begin{category}[]
		& \mathds{J} & \\
		\mathds{J}\arrow{ur}{\id_{\mathds{J}}}\arrow{rr}{g} & & \mathds{I}\arrow{ul}{f}
	\end{category}
	En particulier, on a bien $f \circ g = \id_{\mathds{J}}$.
	Puisque la situation est symétrique, $g\circ f= \id_{\mathds{I}}$ et donc deux objets terminaux sont isomorphes.
	\medskip
	Si on a deux produits cartésiens $A\times B, A\otimes B$ de deux objets $A, B$, alors:
	\begin{category}[]
		A\times B\arrow[bend right]{ddr}{\pi_{A, 1}}\arrow{ddrr}{\pi_{B, 1}} & & & A\otimes B\arrow{ddll}{\pi_{A, 2}}\arrow[bend left]{ddl}{\pi_{B, 2}}\\
		& & &\\
		& A & B &
	\end{category}
	En particulier, par définition du produit cartésien, puisqu'il existe deux applications de $A\times B$ vers $A, B$, il existe une unique application $h_{1, 2}$ de $A\times B$ vers $A\otimes B$ telle que $\pi_{A, 2}\circ h_{1, 2} = \pi_{A, 1}$:
	\begin{category}[]
		& A\times B\arrow[bend left]{ddr}{\pi_{B, 1}}\arrow[bend right]{ddl}{\pi_{A, 1}}\arrow[bend right, dotted]{d}{f}& \\
				& A\otimes B\arrow[dotted, bend right]{u}{g}\arrow{dr}{\pi_{B, 2}}\arrow{dl}{\pi_{A, 2}} & \\
		A & & B
	\end{category}
	Il suffit donc de montrer que l'identité fait commuter le diagramme pour que avoir $f \circ g = \id_{A\times B}$ et donc le résultat:
	\begin{category}[]
		& & A\\
		& A & \\
		A\times B\arrow{r}{f}\arrow[bend left]{ur}{\pi_{A, 1}}\arrow[bend left]{uurr}{\pi_{A, 1}}\arrow[bend right]{drr}{\pi_{B, 1}} & A \otimes B\arrow{u}{\pi_{A, 2}}\arrow{r}{g} & A\times B\arrow{uu}{\pi_{A, 1}}\arrow{d}{\pi_{B, 2}} \\
		& & B\\
	\end{category}

	On aurait aussi pu construire une catégorie $Span(A, B)$:
	\begin{itemize}
		\item Les objets sont des triplets $\scalar{f, X, g}$:
			\begin{category}[]
				A &  & B\\
				& X\arrow{ul}{f}\arrow{ur}{g} &
			\end{category}
		\item Les flèches sont des $\scalar{f, X, g} \xrightarrow{h} \scalar{f', Y, g'}$:
			\begin{category}[]
				& Y\arrow{dl}\arrow{dr} & \\
				A & & B\\
				& X\arrow{ur}\arrow{ul}\arrow{uu}{h} &
			\end{category}
	\end{itemize}
	Alors, $A \times B, \pi_{1}, \pi_{2}$ est un produit cartésien dans $\cont$ si et seulement si $\scalar{\pi_{1}, A\times B, \pi_{2}}$ est un objet terminal dans $Span(A, B)$.
\end{proof}

\section{La 2-catégorie des catégories, foncteurs et transformations naturelles}
\subsection{Foncteurs entre Catégories}
\begin{definition}
	Soient $\A, \B$ deux catégories.
	Un foncteur $\mF : \A\to \B$ est la donnée de:
	\begin{enumerate}
		\item[0] Un objet $F(A) \in \B$ pour tout objet $A$ de $\A$.
		\item[1] Pour toute paire d'objets $A_{1}, A_{2} \in \A$, une fonction:
			\begin{equation*}
				F_{A_{1}, A_{2}}: \applic{\Hom_{\A}(A_{1}, A_{2})}{\Hom_{\B}(FA_{1}, FA_{2})}{f}{F(f)}
			\end{equation*}
		\item[2] On demande que les équations suivantes soient satisfaites:
			\begin{itemize}
				\item $F(g \circ f) = F(g) \circ F(f)$ où
					\begin{category}[]
						& A_{2}\arrow[dr, "g"] &\\
						A_{1}\arrow[ur, "f"]\arrow[rr, "g\circ f"] & & A_{3}
					\end{category}
					C'est à dire:
					\begin{category}[]
						A_{1}\arrow[d, "f"]\arrow[to path={[pos=0.25] -- ([xshift=.5cm]\tikztostart.east) |- (\tikztotarget) \tikztonodes}, dd, "g \circ f"] & & FA_{1}\arrow[d, "Ff"]\arrow[to path={[pos=0.25] -- ([xshift=.5cm]\tikztostart.east) |- (\tikztotarget) \tikztonodes}, dd, "Fg \circ Ff"] & \\
						A_{2}\arrow[d, "g"] & & FA_{2}\arrow[d, "Fg"] & \\
						A_{3} & & FA_{3} &
					\end{category}
				\item Si $A \xrightarrow{\id_{A}} A$, $F(\id_{A}) = \id_{F(A)}$
			\end{itemize}
			Autrement dit, l'image de la composée est égale à la composée des images.
	\end{enumerate}
	\label{def:foncteur}
\end{definition}

\begin{exemple}
	\begin{enumerate}
		\item Un foncteur $F : \A \to \B$ entre catégories de préordre est la même chose qu'une fonction croissante (order preserving).
		\item Un foncteur $F : \A \to \B$ entre catégories à un objet est la même chose qu'un homomorphisme $M \to N$ si $\A = \Sigma M$ et $\B = \Sigma N$.
		\item Si $M$ est un monoïde, $\A = \Sigma M$ la catégorie à un objet associée, un foncteur $F: \A \to \mathrm{Set}$ la catégorie des ensembles et fonctions est la donnée d'un ensemble $X$ (l'image de $M$) et d'une action à gauche de $M$ sur $X$.
			En effet, puisque chaque élément de $M$ est une flèche de $\Sigma M$ de l'objet dans lui même, pour tous $m, n \in M$, on a une flèche de $F(*)\to F(*)$ telles que $F(m\cdot n) = F(m) \circ F(n)$.
			On vérifie alors bien les propriétés d'une action à gauche.

			Similairement, si $M$ est le monoïde libre engendré par un alphabet $A$, l'action à droite $X \times A^{*} \to X$ étant une famille de fonctions $\delta_{a} : X \to X$ pour $a \in A$, i.e. un automate déterministe et total dont l'ensemble des états est $X$.
		\item Soit $\mathbb{G}$ la catégorie à deux objets et quatre morphismes:
			\begin{category}[]
				1\arrow[loop right, "\id_{1}"]\arrow[d, shift left,"s"]\arrow[d, shift right,"t" swap]\\
				0\arrow[loop right, "\id_{0}"]
			\end{category}
			Un foncteur $F:\mathbb{G} \to \mathrm{Set}$ est une paire d'ensembles $E= F(1)$, $V = F(0)$, et de deux fonctions $F(s), F(t): E \to V$.
			En voyant $E$ comme un ensemble d'arêtes et $V$ comme un ensemble de sommets, $F(s)$ peut être vue comme une fonction $\partial_{0}$ qui à une arête $(x, y)$ associe $x$.
			Rajouter un élément $2$ avec deux morphismes vers $1$ permettrait de définir des graphes avec des $2$-arêtes entre arêtes.
			En prenant la catégorie des faces d'un triangle on obtiendrait la catégorie des ensembles simpliciaux.
	\end{enumerate}
\end{exemple}

\subsection{Transformations entre Foncteurs}
\subsubsection{Transformation}
On va essayer de suivre l'intuition selon laquelle la théorie des catégories préserve l'ordre.
\begin{category}[]
	\left(A,\leq_{A}\right)\arrow[bend left, "g" name=U, rr]\arrow[bend right, rr, "f" {swap, name=D}] & & \left(B, \leq_{B}\right)\arrow[to path={([yshift=.1cm]D.north) -- ([yshift=-.1cm]U.south)\tikztonodes}, "\rotatebox{90}{$\leq$}" description, phantom]
\end{category}
On a alors $f\leq g \Leftrightarrow \forall a \in A, f a \leq g a$.

On va essayer de généraliser cette définition. On se donne deux foncteurs $F, G$ de $\A$ dans $\B$ et on va définir une transformation point à point de $F$ vers $G$:
\begin{definition}
	Une transformation $\theta: F \Rightarrow G$ est une famille $\left(\theta_{A}: FA \to GA\right)_{A \in \mathrm{Obj}\A}$ de flèches de $\B$ indicée par les objets de $\A$.
	On note ceci:
	\begin{category}[]
		\A\arrow[bend left, "G" name=U, rr]\arrow[bend right, "F" {swap, name=D}, rr] & & \B\arrow[Rightarrow, to path={([yshift=.1cm]D.north) -- ([yshift=-.1cm]U.south)\tikztonodes}, "\theta"]
	\end{category}
	\label{def:transformation}
\end{definition}

\begin{definition}
	La catégorie $\mathrm{Trans}(\A, \B)$ a pour objet les foncteurs $\A \xrightarrow{F} \B$ et pour flèches les transformations $\theta: F \Rightarrow G$.
	\begin{itemize}
		\item La transformation $\id_{F}: F\Rightarrow F$ est définie par $\id_{F} = \left(F A \xrightarrow{\id_{FA}} FA\right)_{A\in \mathrm{Obj}\A}$
		\item La transformation $\psi \cdot \phi: F\Rightarrow H$ composée de $\phi: F\Rightarrow G$ et $\psi: G\Rightarrow H$ telle que: $\left(\psi\cdot\phi\right)_{A} = \psi_{A}\circ_{\B}\phi_{A}$.
	\end{itemize}
	\label{def:transcat}
\end{definition}

\subsubsection{Action à Gauche de Post-Composition}
Supposons qu'on ait la situation suivante:
\begin{category}[]
	\A\arrow[bend left, "G" name=U, rr]\arrow[bend right, "F" {swap, name=D}, rr] & & \B\arrow[Rightarrow, to path={([yshift=.1cm]D.north) -- ([yshift=-.1cm]U.south)\tikztonodes}, "\theta"]\arrow[r, "H"] &\cont
\end{category}
où $\A, \B, \cont$ sont des catégories, $F, G: \A \to \B$ et $H: \B \to \cont$ sont des foncteurs et $\theta: F\rightarrow G$ est une transformation.

\begin{definition}
	La transformation $H \circ_{L} \theta: H\circ F \Rightarrow H\circ G$ est définie par $\left(H \circ \theta\right)_{A} = H\left(\theta_{A}\right): HFA \to HGA$
	\label{def:actiontransfo}
\end{definition}
Autrement dit, une transformation est la donnée pour tout objet de la catégorie de départ d'une flèche dans la catégorie d'arrivée.

\begin{proposition}
	On a alors une série d'équations:
	\begin{enumerate}
		\item On a:
			\begin{equation*}
				H\circ_{L} \left(\psi \cdot \phi\right) = \left(H\circ_{L}\psi\right)\cdot \left(H\circ_{L}\psi\right)
			\end{equation*}
			et de même:
			\begin{equation*}
				H\circ_{L} \Id_{F} = \Id_{H\circ F}
			\end{equation*}
			Autrement dit:
			\begin{equation*}
				H\circ_{L} :
				\begin{array}{rcl}
					\mathrm{Trans}\left(\A, \B\right) & \longrightarrow & \mathrm{Trans}\left(\A, \cont\right)\\
					F: \A \to \B & \longmapsto & H\circ F: \A\to \cont\\
					\begin{tikzcd}
						\A\arrow[bend left, "G" name=U, rr]\arrow[bend right, "F" {swap, name=D}, rr] & & \B\arrow[Rightarrow, to path={([yshift=.1cm]D.north) -- ([yshift=-.1cm]U.south)\tikztonodes}, "\theta"]
					\end{tikzcd} & \longmapsto & \begin{tikzcd}[column sep=large]
						\A\arrow[bend left, "H\circ G" name=U, rr]\arrow[bend right, "H\circ F" {swap, name=D}, rr] & & \cont\arrow[Rightarrow, to path={([yshift=.1cm]D.north) -- ([yshift=-.1cm]U.south)\tikztonodes}, "H\circ_{L} \theta"]
					\end{tikzcd}
				\end{array}
			\end{equation*}
			$H \circ_{L} -$ est un foncteur. On dit que l'action est fonctorielle.
		\item On a:
			\begin{equation*}
				\left(H'\circ H\right)\circ_{L}\theta = H'\circ_{L}\left(H\circ_{L}\theta\right)
			\end{equation*}
			et de même:
			\begin{equation*}
				\Id_{\B}\circ_{L}\theta = \theta
			\end{equation*}
	\end{enumerate}
	\label{prop:transfocomp}
\end{proposition}
\begin{proof}
	\begin{enumerate}
		\item La première propriété est immédiate par la composition des foncteurs.
		\item On a:
			\begin{equation*}
				\begin{aligned}
					\left( \left( H'\circ H \right) \circ_{L} \theta \right)_{A\in \Obj\A} =& H'\circ H\left(\theta_{A}: FA\to GA\right)\\
					=& H'\left(H\theta_{A}\right)\\
					=& H'\circ \left(H\circ \theta_{A}\right)\\
					=& \left( H'\circ_{L}\left( H \circ_{L} \theta \right) \right)_{A\in \Obj\A}
				\end{aligned}
			\end{equation*}
			D'où la deuxième propriété.
	\end{enumerate}
\end{proof}

\subsubsection{Action à droite de Pré-Composition}
On suppose qu'on à:
\begin{category}
	\A\arrow[r, "H"] & \B\arrow[bend left, rr, "G" name=U]\arrow[bend right, rr, "F"{swap, name=D}] & & \cont\uuarrow["\theta"]{D}{U}
\end{category}<++>


\section{TD 1}
\subsection{Catégories et Foncteurs}
\begin{enumerate}
	\item On a le tableau suivant:
		\begin{tabular}{>{\sc}ccc}
			\bf Catégorie & \bf Objets &\bf Flèches\\
			\midrule
			Set & Ensembles & Fonctions\\
			Top & Espaces Topologiques & Fonctions Continues\\
			Vect & Espaces Vectoriels & Applications Linéaires\\
			Grp & Groupes & Morphismes
		\end{tabular}
	\item Un foncteur est un morphisme entre catégories.
	\item La catégorie \textsc{Cat} est la catégorie dont les objets sont des catégories et les flèches sont des foncteurs
\end{enumerate}

\subsection{Catégories Cartésiennes}
\begin{description}
	\item[Question 3, 4, 5] Voir la preuve de \ref{prop:isomorphismeterminaux}.
	\item[Question 6] On a le diagramme commutatif suivant:
		\begin{category}[]
			& A\arrow[bend right]{ddl}\arrow[bend left]{ddr}{\id_{A}}\arrow[dotted]{d}{h} & \\
			& \term \times A\arrow{dl}\arrow{dr}{\pi_{A}}\arrow{dd}{\pi_{A}} & \\
			\term & & A\arrow[bend left]{dl}{\id_{A}}\\
			& A &
		\end{category}
		Donc $\pi_{A} \circ h = \id_{A}$.
		\begin{category}[]
			& \term\times A\arrow{d}{\pi_{A}}\arrow[bend left]{ddr}{\pi_{1}}\arrow[bend right]{ddl}{\pi_{A}} & \\
			& A\arrow{dl}{\id_{A}}\arrow{d}{h} & \\
			A & \term \times A\arrow{l}{\pi_{A}}\arrow{r}{\pi_{1}} & \term
		\end{category}

		De même pour $A \times \term$, par symétrie.
		On peut par ailleurs procéder de même que pour les produits cartésiens pour montrer que $A\times \term \simeq \term\times A$.
	\item[Question 7] On montre que $B\times A$ vérifie les propriétés de produit cartésien pour $A$ et $B$:
		\begin{category}[]
			& C\arrow{dr}{f}\arrow{dl}{g}\arrow[dotted]{d}{h} & \\
			B & B \times A\arrow{l}{\pi_{B}}\arrow{r}{\pi_{A}} & A
		\end{category}
	\item[Question 8]
		On a le diagramme suivant:
		\begin{category}[]
			& & \left(A\times B\right)\times C\arrow[color=black, dl, "\pi_{A\times B}"]\arrow[color=vulm, ddrr, "\pi_{C}"]\arrow[color=blue, dashed, dddr, "\pi_{B\times C}"] & & \\
			& A\times B\arrow[color=vulm, dl, "\pi_{A}"]\arrow[color=vulm, dr, "\pi_{B}"] & & & \\
		A & & B & & C \\
		& & & B\times C\arrow[color=yulm, ul, "\pi_{B}"]\arrow[color=yulm, ur, "\pi_{C}"]\\
		& & A\times \left(B\times C\right)\arrow[color=yulm, uull, "\pi_{A}"]\arrow[color=black, ur, "\pi_{B\times C}"]\arrow[color=blue, dashed, uuul, "\pi_{A\times B}"] & &
		\end{category}
		On a donc un morphisme de $(A\times B) \times C \to B$ et un morphisme $(A\times B)\times C \to C$.
		Il existe donc un (unique) morphisme $(A\times B)\times C\to B\times C$ (faisant commuter le diagramme idoine).
		On a maintenant deux morphismes depuis $(A\times B)\times C$:
		\begin{itemize}
			\item un vers $A$
			\item un vers $B \times C$
		\end{itemize}
		On peut donc trouver un (unique) morphisme $h$:$(A\times B)\times C \to A\times (B\times C)$ (faisant toujours commuter le diagramme idoine).
		On construit de la même façon $\tilde{h}: A\times (B\times C) \to (A\times B) \times C$.
		On vérifie de façon similaire à la question précédente que $h\circ \tilde{h} = \id$ et $\tilde{h} \circ h = id$.
		Donc $(A\times B)\times C \simeq A\times (B\times C)$.
	\item[Question 9] On notera $A + B$ le coproduit de $A$ et $B$.
		Il doit faire le diagramme commuter:
		\begin{category}[]
			& C &\\
			A\arrow[ur, "f"]\arrow[r, "i_{A}" swap] & A\bigsqcup B\arrow[dashed, "h", u] & B\arrow[ul, "g" swap]\arrow[l, "i_{2}"]
		\end{category}
		On remarque notamment que $A + B$ est un coproduit de $A$ et $B$ dans $\cont$ si et seulement si $A + B$ est un produit de $A$ et $B$ dans $\cont^{op}$.
		En prenant pour $A + B$ (si $A, B \in \mathrm{Set}$) l'ensemble abstrait défini à isomorphisme près par:
		\begin{equation*}
			\left\{ (A, a) \mid a\in A \right\} \cup \left\{ (B, b) \mid b \in B \right\}
		\end{equation*}
		avec $i_{A}$ et $i_{B}$ les inclusions.
		On a bien le résultat puisque si $f: A \to C$ et $g: B\to C$, la fonction $h: (A, x) \mapsto f(x); (B, y) \mapsto g(y)$ est unique car entièrement définie sur $A + B$.
		Formellement c'est l'union disjointe.
	\item[Question 10] L'objet terminal de $\mathrm{Rel}$ est l'ensemble vide. En effet, on a toujours une unique relation entre $X$ et l'ensemble vide: la relation vide.
		Il est clair que $\mathrm{Rel}^{op} = \mathrm{Rel}$.
		L'union disjointe est le produit cartésien.
		On prend comme projection $\pi_{A} = \left\{ \left( (A, e), e \right)\mid e \in A \right\}$
	\item[Question 11] On prend pour objet terminal l'espace vectoriel nul.
		On prend comme produit cartésien la somme directe disjointe sur les bases.
		Par le même raisonnement que précédemment, on a le résultat.
	\item[Question 12] L'objet terminal est la catégorie triviale avec un objet et un morphisme.
		On définit le produit de cartésien par des couples d'objet et dont les flèches sont des couples de fonctions.
		La projection est alors similaire à celle de $\mathrm{Set}$.
	\item[Question 13] De la même manière que précédemment, en considérant des couples de morphisme du produit cartésien de l'origine dans le produit cartésien de l'image.
		On a le résultat par propriété fondamentale.
	\item[Question 14]
\end{description}


\end{document}
