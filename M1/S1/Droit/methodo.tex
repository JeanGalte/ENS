\documentclass[math]{cours}
\author{Clara Pinsard}
\title{Méthodologies Juridiques}
\date{2024-2025}
\begin{document}
\bettertitle
\section*{Introduction}
Au cours des 12 séances, on se concentre sur 3 exercices principaux:
\begin{enumerate}
	\item La dissertation Juridique
	\item Le commentaire d'arrêts, de décisions, de textes (constitutions,).
		On s'attardera longuement dessus.
	\item Le cas pratique: l'objectif est ici d'apporter une solution juridique à une situation composée de fait.
\end{enumerate}

La validation prend en compte:
\begin{itemize}
	\item La participation au long du semestre,
	\item Un exercice intermédiaire facultatif à rendre à l'une des séances,
	\item Un exercice final obligatoire au choix parmi les trois à rendre à la dernière séance.
\end{itemize}

\section*{Les outils du juriste}
\subsection*{Ressources Primaires}
En ligne:
\begin{itemize}
	\item Légifrance regroupe les lois, décrets et jurisprudences.
	\item Les sites des juridictions internes: cour de cassation, conseil d'état, conseil constitutionnel
	\item \url{eur-lex.europa.eu} pour le droit de l'Union Européenne
\end{itemize}

Les codes édités par Lexi Nexis et Dalloz par exemple.
Un article du code civil se présente avec les informations suivantes:
\begin{itemize}
	\item Son numéro
	\item La loi qui l'a créé
	\item Ses alinéas
	\item Ses références textuelles, i.e. les articles proches dans d'autre codes
	\item Ses références doctrinales. La doctrine est l'ensemble des personnes qui font des recherches en droit.
		Les références doctrinales permettent donc de compléter le commentaire de l'article.
	\item Sous l'article se trouvent des jurisprudences qui viennent préciser le droit consacré.
\end{itemize}


\subsection*{Les Ressources Secondaires}
Comme ressources secondaires on trouve:
\begin{itemize}
	\item Les recueils de jurisprudence: GAJA, GAJC (grands arrêts de la justice administrative et civile)
	\item Les revues:
		\begin{itemize}
			\item Hebdomadaires: Lextenso, Lexisnexis (la semaine juridique: actualités juridiques: générale: JCPG, entreprise: JCPE, notariat: JCPN), Dalloz (recueil Dalloz: D.), Lamy (RCDC)
			\item Trimestrielles: revue trimestrielle de droit civil et commerciale
			\item Semestrielles: Titre VII (revue numérique du Conseil Constitutionnel).
		\end{itemize}
	\item Les manuels et encyclopédies
\end{itemize}

\section{Introduction au Raisonnement Juridique}
\subsection{Le Syllogisme Juridique}
On utilise une forme de base de la logique aristotélicienne: on a deux propositions, une majeure et une mineure qu'on rapproche pour tirer une conclusion.
Ici, on n'a pas vraiment des propositions mais des prescriptions.
Le syllogisme est composé de trois étapes:
\begin{enumerate}
	\item La majeure: L'affirmation d'une norme applicable,
	\item La mineure: La confrontation de la norme aux faits,
	\item La conclusion: quant à l'applicabilité de la norme.
\end{enumerate}

% TODO: insérer exemple consentement contrat
Par exemple, pour un contrat conclus sans consentement d'une des parties:


\subsection{Les difficultés de mise en oeuvre du syllogisme juridique}
Il est nécessaire pour appliquer le syllogisme de connaître la règle applicable, mais aussi des conditions de son application.
\begin{quote}
	La théorie générale du droit se distingue aussi de la méthodologie juridique qui envisage les moyens de résolution des problèmes pratiques rencontrés par les juristes, tels que les méthodes d'interprétation des énoncés normatifs ou les critères d'identification et de résolution des antinomies: des conflits entre règles de droits.
	\begin{flushright}
		Eric \textsc{Millard}, Théorie Générale du Droit aux éditions Dalloz
	\end{flushright}

\end{quote}

Il y a donc trois grands problèmes juridiques:
\begin{enumerate}
	\item L'obscurité de la règle: liée notamment aux termes techniques peu connus utilisés parfois en droit, et en même temps à l'utilisation de termes courants parfois mal définis.
		À propos, on pourraît poser la question de l'interprétation du mot \textit{vehicle} sur un panneau \textit{No vehicle in the park}, suivant Herbert Hart.
	\item Le nombre de règles applicables: on utilise une hiérarchie des normes pour limiter les conflits entre des sources de droit différentes.
	\item Le cas où aucune règle ne s'applique: Auquel cas on doit interpréter une autre règle pour trouver une solution.
		Ceci découle parfois de la distinction de vitesses d'évolution entre le droit et la société.
\end{enumerate}
Il y a donc notamment des problèmes liés à l'identification de la règle applicable.

\subsection{L'identification de la règle applicable par l'interprétation}
Les juristes ont proposés différentes solutions pour interpréter le droit.
Ces solutions ont deux grands objectifs communs:
\begin{enumerate}
	\item Garantir la compatibilité des normes: éviter des contradictions apparentes
	\item Garantir la complétude des normes: éviter les vides juridiques
\end{enumerate}
Pour arriver à leurs fins, les juristes proposent des méthodes générales:
\begin{enumerate}
	\item Des techniques d'interprétations: des procédés et maximes interprétatitves
	\item Des modèles d'interprétation
\end{enumerate}

On peut trouver trois principaux procédés d'interprétation:
\begin{description}
	\item[L'interprétation \textit{a pari}]
		Celle-ci se fait par analogie, ou à une situation, ou aux effets d'une autre décision.
		Par exemple: pour l'application des règles du divorce à l'annulation du mariage.
		L'analogie intervient alors au niveau des effets.
		Cette interprétation ne peut pas avoir lieu en droit pénal, et ce, parce que le droit pénal est fondé sur le fait qu'il n'y a pas d'infraction sans texte.

	\item[L'interprétation \textit{a fortiori}]
		Celle-ci est possible lorsque la raison d'être de la règle se retrouve avec plus de force encore dans le cas non prévu par le texte.
		Par exemple, si un hôtel interdit de séjourner dans son établissement avec un animal domestique, \textit{à fortiori}, il est interdit d'y séjourner en compagnie d'un animal sauvage.

	\item[L'interprétation \textit{a contrario}]
		Si un objet est inclu dans une règle de droit, en particulier, on considère que le contraire en est exclu.
		Par exemple, l'article 6 du code civil dit:
		\begin{quote}
			On ne peut déroger par des conventions particulières aux lois qui intéressent l'ordre public et les bonnes m\oe urs.
		\end{quote}
		On en déduit \textit{à contrario} un principe de liberté contractuelle: il est possible de déroger aux lois qui ne sont dictées ni par l'ordre public ni par les bonnes m\oe urs.
		De même: le droit pénal réprime certains comportements, à contrario, il est impossible de réprimer un comportement non prévu par le droit pénal.
\end{description}
Il y a parfois plusieurs possibilités d'interprétation.
Celle que l'on retiendra est la plus cohérente vis-à-vis de l'environnement.
Le juge tranche, possiblement en cherchant l'intention de l'auteur de la règle.

Les maximes d'interprétation sont des principes généraux qui ne sont pas écrits, mais qui sont souvent appliqués en droit français.
Elles sont retrouvées dans plusieurs domaines du droit.
Par exemple:
\begin{itemize}
	\item \textit{Les lois spéciales dérogent aux lois générales}: des régimes plus précis s'appliquent à la place de régimes plus généraux.
		Par exemple, les droits spéciaux des contrats (vente, pacs) dérogent au droit commun des contrats.
	\item \textit{Les exceptions doivent être strictement interprétées}:
		Par exemple, le principe en droit des contrats est la capacité de contracter.
		La liste des exceptions prévues à l'article 1146 du Code civil (mineurs non émancipés et majeurs protégés) ne doit pas être allongée.
	\item Il y a d'autres maximes inutiles mais souvent mentionnées par les juristes: \textit{Ce qui est clair ne s'interprète pas}.
\end{itemize}

Pour les cas plus généraux, on a des modèles d'interprétation:
\begin{description}
	\item[L'Exégèse] C'est un modèle français de droit civil, fondé sur l'interprétation de Portalis.
		Le modèle considère que le droit est un texte sacré.
		Il part du principe qu'il suffit de lire le droit pour connaître le droit, le code civil se suffisant à lui même, en se fondant sur l'interprétation littérale.
		Il a une forte importance sur l'enseignement du droit, centré sur le code civil.
		Il arrive que certains professeurs ne fassent que lire le code civil en le commentant légèrement.
		Dans ce modèle, la loi est complète, tout est dans la loi, et toute difficulté peut être résolue en se référant au code civil.
		Ce modèle s'affirme notamment politiquement neutre.
		L'affirmation de neutralité est une affirmation de légitimation.
		Les personnes qui soutiennent l'exégèse sont conservateurs, et l'idée de ne pas se référer à d'autres éléments est à l'origine du conservatisme juridique.
		L'idée est que ce qui est du droit a déjà été prévu.
		À partir de la fin du 19ème siècle on a commencé à se questionner sur une possible ouverture théorique.
	\item[La Libre Recherche Scientifique] C'est un modèle réactionnaire à l'exégèse, qui apparaît fin 19ème.
		Il se fonde sur l'idée que pour interpréter le droit, il faut le replacer dans le monde social et scientifique.
		Il a été porté notamment par François Geny (Doyen de la faculté de droit de Nancy, protagoniste du mouvement des \emph{Juristes Inquiets}, dont le but était de réformer le code civil tout en adhérant aux idées qu'il porte.
		Pour l'auteur, le texte ne peut être sollicité à l'infini, et il faut s'intéresser au contexte)
		et par Maurice Hauriou (grand arrêtiste travaillant sur le droit administratif, partie du droit publique construite sans code, contrairement au droit civil.
		Il rejette complètement l'exégèse et souhaite intégrer la sociologie au droit. Pour lui, la sociologie peut être utilisée pour interpréter le droit.).
		Pour cette école de pensée, le juge doit évaluer le droit en prenant en compte des données sociales, historiques.
		Le rôle du juge est bien plus important dans ce modèle.
		La puissance est donné au juge plus qu'au législateur.
	\item[L'interprétation Téléogique] C'est un modèle qui considère l'interprétation du droit en fonction de l'objectif de la règle.
	\item[L'interprétation Évolutive] C'est un modèle qui considère que le droit évolue en s'adaptant au contexte contemporain.
\end{description}

En Suisse, l'article premier dit que le juge a le pouvoir de faire acte de législateur, c'est une approbation claire de la méthode de la Libre Recherche Scientifique.
Ce n'est pas du tout ce que prévoit le code civil en France, où le juge n'a pas le droit d'agir comme législateur.

\begin{quote}
	\begin{center}
		Article 4 du Code Civil
	\end{center}
	{\it	Le juge qui refusera de juger, sous prétexte du silence, de l'obscurité ou de l'insuffisance de la loi, pourra être poursuivi comme coupable de déni de justice.}
	\begin{center}
		Article 5 du Code Civil
	\end{center}
	{\it	Il est défendu aux juges de prononcer par voie de disposition générale et réglementaire sur les causes qui leur sont soumises.}
\end{quote}

Ces deux articles sont les seuls permettant de mieux cadrer les pouvoirs des juges.
Le second prohibe les arrêts de réglements, arrêts suffisamment généraux et abstraits pour s'appliquer exactement dans d'autres cas.
Le juge ne doit pas faire d'interprétation générale de la loi, et ne peut juger que sur la situation qui lui est présentée.
Toutefois, les juges des hautes cours proposent très souvent des décisions assez générales pour qu'elles puissent s'appliquer plus tard (et parfois le disent même).
Il faut relativiser l'article 5, notamment dans des matières qui n'ont été que peu réformées, par exemple en responsabilité civile: il y a 6 articles dans le code civil au total.
Sur les 15 dernières années, on se retrouve face à une inflation législative et décrétale très importante, car le droit agit énormément plus sur des sujets techniques.
Le juge ne s'exprime pas, n'interprète pas de la même manière selon le domaine.
Plus la loi est précise, plus le juge est lié.

\subsection{Les différences de raisonnement selon les différentes professions juridiques}
Pour les différentes professions du milieu juridique, une fois que le problème de l'interprétation juridique est résolu, comment appliquer le syllogisme juridique ?
L'existence de multiples interprétations du droit sert l'application du droit.
Il s'agit toujours de questions d'argumentation.
Il y a donc différentes manières d'appliquer le syllogisme selon la profession de l'écosystème juridique:
\begin{description}
	\item[L'avocat] Il part de la solution qui est dans l'intérêt de son client (conclusion) pour trouver une règle applicable aux faits (majeure du syllogisme)
	\item[Le juge] En principe, il juge en suivant le syllogisme, et donc en partant d'une règle de droit pour l'appliquer aux faits.
		Dans l'article 12 du code de procédure civile, il est écrit que \textit{Le juge tranche le litige conformément aux règles de droit qui lui sont applicables}.
		Il faut qu'il y ait une permanence des décisions, pour que les droits qui sont garantis le restent dans le temps.
		Il est en réalité complètement illusoire de penser que le juge ne compte que sur la règle de droit.
		Il y a déjà des questions d'interprétation, mais également le fait que le juge, est un humain et a sa propre conception du juge.
		L'article 12 proscrit le jugement en équité (selon la conception qu'à le juge du juste), mais il arrive des exceptions, notamment en droit privé, où des articles (par exemple l'article 700 du code de procédure civile) prévoient cette possibilité pour le juge.
		\begin{quote}
			\begin{center}
				Article 700 du Code de Procédure Civile
			\end{center}
			{\it Le juge condamne la partie tenue aux dépens ou qui perd son procès à payer:
			\begin{enumerate}
				\item A l'autre partie la somme qu'il détermine, au titre des frais exposés et non compris dans les dépens;
				\item Et, le cas échéant, à l'avocat du bénéficiaire de l'aide juridictionnelle
					partielle ou totale une somme au titre des honoraires et frais,
					non compris dans les dépens,
					que le bénéficiaire de l'aide aurait exposés s'il n'avait pas eu cette aide.
					Dans ce cas, il est procédé comme il est dit aux alinéas 3 et 4 de l'article 37 de la loi n°01-647 du 10 juillet 1991.
		\end{enumerate}
		Dans tous les cas, le juge tient compte de l'équité ou de la situation économique de la partie condamnée.
		Il peut, même d'office, pour des raison tirées des mêmes considérations, dire qu'il n'y a pas lieu à ces condamnations.
		Néanmoins, s'il alloue une somme au titre du 2° du présent article, celle-ci ne peut être inférieure à la part contributive de l'État.
	}
		\end{quote}
		Le principe de base est que la partie qui perd sont procès à payer les frais de celui-ci (les dépens, définis à l'article 695 de ce même code ainsi que d'autres frais).
		Dans les frais exposés et non compris pour les dépens, on a par exemple le manque à gagner lié au fait pour une partie de participer personnellement aux opérations de procédure, en l'espèce une perte de salaire.
		Il y a (rarement) eu également la prise en compte du préjudice moral occasionné par les \textit{peines et tracas du procès}.
		Il peut y avoir des dérives: si le montant est trop élevé, il y a punition de la partie perdante et atteinte au droit à exercer la justice.
		Si au contraire le montant est trop faible, il ne couvrirait pas les frais de procédure de la partie gagnante.
		Une solution serait d'objectiver les frais de procédure, ce qui est difficile de part la nécessité de solvabilité de la partie perdante.
\end{description}

\end{document}
