\documentclass{cours}
\title{Cours de Turc Débutant}
\author{Marc Aymes}
\date{2123-2124}
\usepackage{longtable}
\usepackage{nicematrix}

\newcommand{\ch}{\c{s}}
\newcommand{\Ch}{\c{S}}
\newcommand{\I}{\.{I}}
\newcommand{\ug}{\u{g}}
\newcommand{\sci}{\textsc{i}}
\newcommand{\sca}{\textsc{a}}
\newcommand{\sce}{\textsc{e}}
\newcommand*{\thead}[1]{\multicolumn{1}{c}{\bfseries #1}}

\begin{document}
\section{Birinci Ders}
\subsection{Quelques Mots}
\begin{tabular}{>{\sl}p{.5\textwidth}|>{\sl}p{.5\textwidth}}
    \toprule
    taksi                                                                  & okul                                       \\
    gar (\text{précédé d'un morphème\newline qui indique le type de gare}) & tuvalet (\text{les toilettes/la toilette}) \\
    pazar (\text{marché/dimanche})                                         & tekstil                                    \\
    spor                                                                   & üniversite                                 \\
    telefon                                                                & bisküvi (\text{biscuit})                   \\
    fotokopi                                                               & kuaför                                     \\
    ferforje                                                               & asansör                                    \\
    mikrop                                                                 & klasör                                     \\
    deterjan                                                               & mösyö                                      \\
    banliyö                                                                & feribot                                    \\
    istatistik                                                             & fi\ch (\text{prise électrique / facture})  \\
    kontaklens                                                             & afi\ch                                     \\
    istasyon (\text{gare, presque pareil que gar})                         & ti\ch ört                                  \\
    tren                                                                   & \ch ömendöfer                              \\
    pantolon                                                               & \ch ike                                    \\
    lambda                                                                 & \ch oke                                    \\
    banyo (\text{salle de bain/bain})                                      & maç                                        \\
    klima (\text{climatisation})                                           & ofsayt (\text{hors-jeu})                   \\
    kürdan                                                                 & gofret                                     \\
    ekipman                                                                & foto\ug raf                                \\
    otomatikman                                                            & co\ug rafya (\text{géographie})            \\
    mayo                                                                   & tüyo (\text{information/tip})              \\
    \bottomrule
\end{tabular}

\subsection{\textsl{Sesli ve sessiz harfler} (Voyelles et consonnes)}
\begin{itemize}
    \item Accentuation: L'accent tonique porte généralement sur la dernière syllabe du mot, sauf suffixe ou particule \emph{enclitique} qui déplace l'accent sur la syllabe précédente.
    \item L'alphabet turc comporte 29 lettres dont 8 voyelles :
          \begin{center}
              abcçdefg\ug h\i ijklmnoöprs\ch tuüvyz
          \end{center}
    \item Lettres singulières : \textsl{ç, \ug, \i, i, ö, s, \ch, u, ü}
\end{itemize}

\subsection{Prononciation}
\begin{itemize}
    \item Le \textsl{l} est généralement \og plat\fg mais devient \og creux \fg devant ou après \textsl{a, \i, o, u}
    \item On prononce le doublement consonantique
    \item Il y a un assourdissement (dévoisement)/sonorisation (voisement) des consonnes finales :
          \begin{center}
              \textsl{b, d, g, c} $\leftrightarrow$ \textsl{p, t, k, ç}
          \end{center}
          Une consonne finale redevient sonore lorsqu'une voyelle lui est suffixée.
    \item \textsl{\ug} est \og doux \fg (\textsl{yumu\ch ak}) muet, allonge la voyelle qui le précède. Entre deux voyelles, il ne se prononce pas. Après des voyelles \og fine \fg, dites aussi antérieures ou palatales (\textsl{e, i, ö, ü}) il se prononce souvent en consonne \textsl{y}.
    \item \textsl{h} est toujours \og aspiré \fg, sauf dans \textsl{Mehmet}
    \item \textsl{\i} n'est ni \og i \fg ni \og ö \fg. Il se prononce en ramenant la langue en arrière
    \item \textsl{y} est toujours une consonne, jamais un \og i \fg.
\end{itemize}

\subsection{\textsl{Lütfen dikkat}}
\begin{itemize}
    \item Le verbe \og être \fg au présent n'a pas d'équivalent en turc. On recourt à un équivalent nominal utilisé comme prédicat. Auquel on adjoint \emph{facultativement} le suffixe \textsl{-d\i r}. On parle donc de {\bf prédication nominale}. La proposition formée est donc nominale et non verbale. \textsl{De\ug il} est la forme négative de cette prédication nominale.
    \item La particule interrogative \textsl{m\sci} :
          \begin{itemize}
              \item s'emploie en l'absence de pronom interrogatif
              \item se place aussitôt après le mot sur lequel porte l'interrogation
              \item est enclitique (accentue la syllage qui la précède)
          \end{itemize}
    \item Le comportement des voyelles est régie par la règle de l'hamonie vocalique (en \textsl{small caps} ci-après).
\end{itemize}

\section{Ikinci Ders}
\subsection{Morphologie Générale}
\begin{itemize}
    \item Il n'y a pas de genre en turc
    \item Il y a deux catégories de mots: noms et verbes
    \item Pour les noms, il y a une différence entre emploi comme substantif, adjectif et adverbe, selon l'emploi.
    \item L'adjectif est invariable, il n'y a pas d'accord morphologique des mots entre eux (en nombre et en cas. )
\end{itemize}

\subsection{Harmonie Vocalique}
La plupart des mots sont régis par l'harmonie vocalique. Les suffixes sont vocalisés suivant deux types :
\begin{center}
    \textsl{e/a} ou \textsl{i/\i/ü/u}
\end{center}
On a alors les changements suivants :
\begin{tabular}{>{\sl}c@{\ \ $\Rightarrow$\ \ }>{\sl}c}
    a/\i & a \text{ ou } \i \\
    e/i  & e \text{ ou } i  \\
    o/u  & a \text{ ou } u  \\
    ö/ü  & e \text{ ou } ü
\end{tabular}

\subsection{\textsl{Biti\ch kenlik}: Suffixes Dérivatifs}
Le turc est une langue agglutinante. Les racines lexicales et suffixes s'ajoutent les uns aux autres dans un ordre syntaxiquement réglé. La suffixation est le procédé unique de la morphologie turque. Il existe aussi des agglutinations non suffixales : \textsl{günay-d\i n}. Les suffixes peuvent être nominaux ou verbaux. \\
Les suffixes de dérivation forment des noms à partir de noms ou de verbes :
\begin{center}
    \begin{tabular}{>{\bf}cc@{\ \ $\longrightarrow$\ \ }>{\sl}cc}
        Suffixe  & Sens            & Exemple   & Traduction                    \\
        -l\sci   & être muni de    & renkli    & coloré                        \\
        -s\sci z & être privé de   & i\ch siz  & chômeur                       \\
        -c\sci   & métier/activité & ögrenci   & étudiant                      \\
        -da\ch   & compagnonnage   & arkada\ch & ami (litt., compagnon de dos) \\
        -l\sci k & diminutif       & deftercik & petit cahier                  \\
    \end{tabular}
\end{center}
\subsection{Pronoms Personnels}
\begin{center}
    \begin{tabular}{cccc}
                  & 1ère    & 2ème    & 3ème      \\
        Singulier & \sl Ben & \sl Sen & \sl O     \\
        Pluriel   & \sl Biz & \sl Siz & \sl Onlar
    \end{tabular}
\end{center}
On peut suffixer ces pronoms : \textsl{sensiz} veut dire sans toi, \textsl{benci} égocentrique et \textsl{benlik} égo.

\subsection{Nombres}
On compte en base dix en MSB :
\begin{center}
    \begin{tabular}{c>{\sl}c|c>{\sl}c}
        1   & bir   & 10   & on         \\
        2   & iki   & 20   & yirmi      \\
        3   & üç    & 30   & otuz       \\
        4   & dört  & 40   & k\i rk     \\
        5   & be\ch & 50   & elli       \\
        6   & alt\i & 60   & altm\i \ch \\
        7   & yedi  & 70   & yetmi\ch   \\
        8   & sekiz & 80   & seksen     \\
        9   & dokuz & 90   & doksan     \\
        100 & yüz   & 1000 & bin
    \end{tabular}
\end{center}

\section{\textsl{Üçünçü Ders}}
\subsection{\textsl{Ünlü uyumu - etrafl\i ca} : Catégorisation des voyelles}
\begin{center}
    \linespread{1.5}
    \begin{NiceTabular}{p{.2\linewidth}cccc}
                                                          & \multicolumn{2}{c}{\textsl{Düz} (plane) = non labiale} & \multicolumn{2}{c}{\textsl{Yuvarlak} (ronde) = labiale}                                   \\
                                                          & \textsl{Geni\ch} (ouverte)                             & \textsl{Dar} (étroite)                                  & \textsl{Geni\ch} & \textsl{Dar} \\
        \textsl{Kal\i n} (épaisse) = postérieure, vélaire & \sl \bf a                                              & \sl \bf \i                                              & \sl \bf o        & \sl \bf u    \\
        \textsl{\I nce} (fine) = antérieure, palatale     & \sl \bf e                                              & \sl \bf i                                               & \sl \bf ö        & \sl \bf ü

        \CodeAfter
        \begin{tikzpicture}
            \draw [gray] (1-|1) -- (1-| 2) ;
            \draw [black] (1-|2) -- (1-| 6) ;
            \draw [gray] (2-|1) -- (2-| 2) ;
            \draw [black] (2-|2) -- (2-| 6) ;
            \draw [black] (3-|1) -- (3-| 6) ;
            \draw [black] (4-|1) -- (4-| 6) ;
            \draw [black] (5-|1) -- (5-| 6) ;
            \draw [gray] (1-|1) --(3-|1);
            \draw [gray] (1-|2) --(3-|2);
            \draw [black] (3-|1) -- (5-|1);
            \draw [black] (3-|2) -- (5-| 2);
            \draw [black] (2-|3) -- (5-| 3);
            \draw [black] (1-|4) -- (5-| 4);
            \draw [black] (2-|5) -- (5-| 5);
            \draw [black] (1-|6) -- (5-|6);
        \end{tikzpicture}
    \end{NiceTabular}
\end{center}
S'ensuivent, suivant \og La tendance humaine naturelle au moindre effort musculaire \fg :
\begin{itemize}
    \item si le mot commence par une voyelle antérieure, celles qui suivent le sont également ; idem pour les postérieures.
    \item si la première voyelle est non labiale, les suivantes le sont également. Ex. {\sl i\ch-siz}
    \item si la première voyelle est labiale, les suivantes sont soit labiales fermées (ex. {\sl yol-cu}), soit non labiales ouvertes (ex. {\sl yol-da}, en chemin).
\end{itemize}

Le vocable est riche en termes qui sont en eux-mêmes des exceptions à ces règles :
\begin{itemize}
    \item \textsl{anne, dahi, elma, hangi, hani, inanmak, karde\ch, selam, \ch i\ch man, tiyatro, viraj, ziyaret}
    \item les mots composés : \textsl{aç\i kgöz, bilgisayar, çekyat, han\i meli}
    \item certains suffixes sont invariables {\sl -da\ch, -ki} : \textsl{din-da\ch, gönül-da\ch, meslek-ta\ch, ülk£u-da\ch, ak\ch amki, yar\i nki, duvardaki, yoldaki}
\end{itemize}
Dans la plupart des cas, les suffixes se règlent sur la dernière voyelle du mot, sauf pour certains mots dits \og d'emprunt \fg ou exceptions fameuses gouvernées par une forme d'harmonie consonantique ({\sl saat-te}).

\subsection{\textsl{Biti\ch kenlik} - Suffixes Nominaux}
Sur la base nominale se greffent principalement (outre les suffixes de dérivation dont l'apprentissage relève davantage du vocabulaire que de la grammaire) trois types de suffixes dont l'étude grammaticale est nécessaire :
\begin{enumerate}
    \item Suffixe de Nombre : Le Pluriel\\
          \begin{itemize}
              \item Emploi Nominal : \textsl{iyi gün-ler, iyi ak\ch am-lar} ; ou usage idiomatique avec un prénom : \textsl{Marc'lar} = Marc et ses proches
              \item Emploi Verbal : \textsl{gidiyor-lar} = ils partent
          \end{itemize}
    \item Suffixe de Personne : Le Possessif
          \begin{center}
              \begin{tabular}{cccc}
                            & 1ère           & 2ème           & 3ème         \\
                  Singulier & -(\sci)m       & -(\sci)n       & -(s)\sci     \\
                  Pluriel   & -(\sci)m\sci z & -(\sci)n\sci z & -l\sca r\sci
              \end{tabular}
          \end{center}
          Exemples : \textsl{okulum, kitab\i n, kalemi, yüzü\ug ümüz, defteriniz, apartmanlar\i, dairem, hocan, lisesi, Fransa'm\i z, bölgeniz, üyeleri}
    \item Suffixes de Cas :
          \begin{center}
              \begin{NiceTabular}{c>{\bf}ccc>{\sl}cc}
                  \multirow{4}{.15\linewidth}{Cas Spatiaux}     & Locatif   & -d\sca     & sur/en/dans         & ben-de         & sur moi         \\
                                                                & Complété  & -ki        & \og qui est \fg     & ben-de-ki      & qui est sur moi \\
                                                                & Ablatif   & -d\sca n   & venir de, approcher & \I stanbul'dan & Stanbouliote    \\
                                                                & Directif  & -(y)\sca   & aller à, s'éloigner & ev-e           & à la maison     \\
                  \multirow{2}{.15\linewidth}{Cas Grammaticaux} & Génitif   & -(n)\sci n & ComDéfini du Nom    &                &                 \\
                                                                & Accusatif & -(y)\sci   & ComDéfini du Verbe  &                &
                  \CodeAfter
                  \begin{tikzpicture}
                      \draw [dotted] (1 -| 2) -- (7 -| 2);
                      \draw [black] (2 -| 2) -- (2 -| 7);
                      \draw [black] (3 -| 2) -- (3 -| 7);
                      \draw [black] (4 -| 2) -- (4 -| 7);
                      \draw [black] (5 -| 2) -- (5 -| 7);
                      \draw [black] (6 -| 2) -- (6 -| 7);
                  \end{tikzpicture}
              \end{NiceTabular}
          \end{center}
\end{enumerate}
La déclinaison des pronoms personnels est irrégulière à la première personne au directif et au génitif. \\
Au cas absolu, tout \emph{nom} turcc est employé sans aucun suffixe de cas, sans que cela l'empêche néanmoins de remplir les fonctions grammaticales les plus variées:
\begin{itemize}
    \item Sujet : \textsl{çocuk gülüyor} = l'enfant rit
    \item Complément du Nom : \textsl{çocuk arabas\i} = voiture d'enfant
    \item COD : \textsl{bu kad\i n üç çocuk yeti\ch iyor} = cette femme élève trois enfants
    \item Complément Post-Positionnel : \textsl{çocuk için} = pour les enfants ; \textsl{çocuk gibi} = comme un enfant, \textsl{çocuk ile} = avec l'enfant
\end{itemize}
Les suffixes de cas ne se cumulent pas, ils prennent place après les suffixes de nombre et de personne (dans cet ordre). Quand le suffixe de cas s'ajoute à un suffixe de troisième personne, une \emph{consonne de liaison} \og n \fg s'intercale entre les deux :
\begin{itemize}
    \item \textsl{ev-im-de} = dans ma maison; \textsl{ev-ler-de} = dans les maisons; \textsl{ev-ler-im-de} = dans mes maison.
    \item \textsl{ev-i-n-de} = dans sa maison; \textsl{ev-leri-n-de} = dans leur(s) maison(s) \emph{ou} dans ses maisons (car le cumul \textsl{ev-ler-leri} n'est pas possible).
    \item Au directif, la consonne de liaison \og y \fg cède la place à \og n \fg: on dit \textsl{ev-i-n-e} et non pas \textsl{ev-i-y-e}.
\end{itemize}
Dans le cas d'un mot se terminant par une consonne sourde, la consonne initiale de certains suffixes s'assourdit : {\sl d} devient {\sl t}, {\sl c} devient {\sl ç}, etc...
Par exemple : \textsl{bu hedefte} = dans cette cible; \textsl{bu amaçta} = dans ce but; \textsl{bu ahbapta} = chez ce pote; \textsl{bu arkada\ch ta} = chez ce camarade; \textsl{bu saatte} = à cette heure-ci (exception à l'harmonie vocalique).\\
De même pour le suffixe d'activité \textsl{-c\sci} : \textsl{sütçü} = le laitier; \textsl{topçu} = l'artilleur; \textsl{tarihçi} = l'historien.

\section{Dördüncü Ders}
\subsection{Suffixes Nominaux de Personne}
\begin{center}
    \begin{tabular}{c>{\sl}c>{\sl}c>{\sl}c}
                  & 1ère       & 2ème       & 3ème    \\
        Singulier & ad\i m     & ad\i n     & ad\i    \\
        Pluriel   & ad\i m\i z & ad\i n\i z & adlar\i
    \end{tabular}
\end{center}
Ce type de suffixes est dit \og possessif \fg, car il est notamment utilisé pour traduire l'adjectif possessif français. Cependant sa fonction plus générale est d'exprimer la relation entre un nom et une personne. Ce suffixe joue un rôle essentiel dans la formation générique du rapport d'annexion, qui s'effectue par ajout d'un suffixe de 3e personne au nom complété. Le nom complément, quant à lui, est soit au génitif soit au cas absolu:
\begin{itemize}
    \item Le génitif marque le complément défini du nom : \textsl{çocu\ug un arabas\i} = la voiture de l'enfant $\neq$ \textsl{çocuk arabas\i} = la voiture d'enfant
    \item Le complément indéfini du nom demeure au cas absolu : \textsl{Türkiye Cumhuriyeti} (république); lieux géographiques : \textsl{Van gölü}, variétés botaniques ou zoologiques (\textsl{semizotu} = pourpier, \textsl{hamamböce\ug i} = cafard, \textsl{\ch am f\i st\i \ug \i } = \textsl{antep f\i st\i \ug \i} = pistache, \textsl{çamf\i st\i \ug \i} = pignon de pin.)
    \item Les Épithètes de nationalité : \textsl{Frans\i z ö\ug rencisi} mais \textsl{Amerika'l\i ö\ug renci}. D'une manière générale, les noms désignant des groupes nationaux, sociaux ou ethniques forment des complément du nom.
\end{itemize}
Il est impossible de superposer plusieurs suffixes possessifs; le cas échéant, le suffixe de troisième personne cède la place aux suffixes de première ou de deuxième personne. \\
Au titre des combinaisons à possessifs multiples, on applique le principe d'économie : \textsl{su tesis hizmetleri} = les services d'aménagement hydraulique.

A la différence du français, le complément de matière n'est pas (sauf en cas d'association inattendue) construit comme un complément du nom, mais comme épithète au cas absolu. On dit \textsl{deri ceket} et non \textsl{deri ceketi} pour dire \og la veste en cuir \fg. Il en va de même pour l'expression de la quantité : \textsl{üç kilo ekmek} = trois kilos de pain. Dans certaines combinaisons, le premier nom prend valeur adjective, il n'y a alors pas de suffixe de personne: \textsl{ara sokak} = rue de traverse; \textsl{ana fikir} = idée-force.

\subsection{\textsl{Rakamlar} - Numération}
Ordinaux :
\begin{center}
    \begin{tabular}{>{\sl}c@{$\longrightarrow$}>{\sl}c}
        Bir   & Birinci  \\
        \I ki & \I kinci \\
        üç    & Üçünçü   \\
        \dots & \dots
    \end{tabular}
\end{center}
Distributifs :
\begin{center}
    \begin{itemize}
        \item Suffixe \textsl{-(\ch)\sce r}: équivalent de \og tant par tête \fg : \textsl{iki\ch er elmam\i z var} = nous avons chacun deux pommes
        \item Et redoublé : \textsl{alt\i \ch ar, alt\i \ch ar} = par groupes de six, six par six.
    \end{itemize}
\end{center}
Fractions :
\begin{center}
    \begin{itemize}
        \item En général, $\frac{x}{y} = $ \textsl{X-d\sca Y}. Par exemple, \textsl{ikide bir} = un sur deux.
        \item En particulier, \textsl{yar\i m} = un demi (adjectif); \textsl{yar\i} = moitié (substantif) ; \textsl{buçuk} = ... et demi
    \end{itemize}
\end{center}

\subsection{\textsl{Çok mu, az m\i ?}}
Après un qualificatif de nombre ou de quantité, on n'ajoute pas de suffixe de pluriel \textsl{-l\sce r} : \textsl{çok po\ug aça istiyorum} ou \textsl{be\ch\ litre kahve}. Il en va de même pour tout ensemble collectif d'items non individualisés (quoiqu'éventuellement dénombrables): \textsl{kitap al\i yorum} = je prends des livres; \textsl{elma veriyorum} = je donne des pommes. Exception à valeur emphatique : \textsl{çok te\ch ekkürler} = merci beaucoup.\\
\textsl{ne kadar}, littéralement \og autant que quoi ?\fg, peut être distingué de \textsl{kaç} qui interroge sur le nombre plutôt que sur la quantité. Ainsi, \textsl{ne kadar paran var?} appelle une réponse du type \textsl{az/çok} tandis que \textsl{kaç paran var?} suppose de répondre numériquement (même si {\sl kaç} peut également prendre une valeur indéfinie et \textsl{kaça?} être synonyme de \textsl{ne kadar?} pour demander le prix.)

\section{Be\ch inci Ders}
\subsection{Suffixe Nominal Prédicatif}
\begin{center}
    \begin{tabular}{>{\sl\bf}c>{\sl}c>{\sl}c>{\sl}c>{\sl}c}
        -(y)\sci m          & iyi-yim        & Frans\i z-\i m         & Türk-üm       & o\ug ul-um       \\
        -s\sci n            & iyi-sin        & Frans\i z-s\i n        & Türk-sün      & o\ug ul-sun      \\
        -(d\sci r)          & iyi-(dir)      & Frans\i z-(d\i r)      & Türk-(dür)    & o\ug ul-(dur)    \\
        -(y)\sci z          & iyi-yiz        & Frans\i z-\i z         & Türk-üz       & o\ug ul-uz       \\
        -s\sci n\sci z      & iyi-siniz      & Frans\i z-s\i n\i z    & Türk-sünüz    & o\ug ul-sunuz    \\
        -(d\sci r)(l\sce r) & iyi-(dir)(ler) & Frans\i z-(d\i r)(lar) & Türk-(dür)ler & o\ug ul-(dur)lar \\
    \end{tabular}
\end{center}

\subsection{Verbes et Suffixes Nominaux}
Le verbe est formé d'une \emph{base}. Dans les dictionnaires elle est donnée augmentée du suffixe \textsl{-m\sce k} de l'infinitif.\\
La conjugaison régulière s'effectue par l'ajout du suffixe nominal prédicatif. Dans la majorité des cas donc, on peut parler de conjugaison prédicative de type nominal.

\subsection{Conjugaison Prédicative de Type Nominal}
Il faut bien marquer la différence entre suffixe nominal prédicatif et suffixe nominal de personne. L'accentuation permet de trancher en cas d'ambiguïté : les prédicatifs sont enclitiques, les suffixes de personne sont accentués. A l'interrogatif, le suffixe de conjugaison prédicative nominale va sur la particule interrogative.

\subsection{Une conjugaison verbale: l'impératif}
L'impératif fait exception à l'unicité de la conjugaison turque: il adopte en effet des suffixes de personne d'un type particulier, et leur distribution est spécifique:
\begin{itemize}
    \item Contrairement aux autres formes verbales la 3e personne du singulier est marquée par un suffixe: \textsl{gelsin} (qu'il vienne), \textsl{kals\i n} (qu'elle reste), \textsl{olsun} (qu'elle soit/devienne), \textsl{ölsün} (qu'il meure)
    \item A l'inverse, la base verbale reste nue à la 2e personne du singulier.
\end{itemize}
\begin{center}
    \begin{tabular}{c>{\sl}c>{\sl}c>{\sl}c}
                  & 1ère        & 2ème       & 3ème          \\
        Singulier & $\emptyset$ & -          & -sin          \\
        Pluriel   & $\emptyset$ & -(y)in(iz) & -sin(l\sce r)
    \end{tabular}
\end{center}

\section{Alt\i nc\i\ Ders}
\subsection{Déclinaison des Pronoms Personnels}
\begin{center}
    \begin{tabular}{>{\sl}c>{\sl}c>{\sl}c>{\sl}c>{\sl}c>{\sl}c}
        \thead{Cas Abs.} & \thead{Acc.} & \thead{Gén.} & \thead{Dir.} & \thead{AbL.} & \thead{Loc.} \\
        ben              & beni         & benim        & bana         & benden       & bende        \\
        sen              & seni         & senin        & sana         & senden       & sende        \\
        o                & onu          & onun         & ona          & ondan        & onda         \\
        biz              & bizi         & bizim        & bize         & bizden       & bizde        \\
        siz              & sizi         & sizin        & size         & sizden       & sizde        \\
        onlar            & onlar\i      & onlar\i n    & onlara       & onlardan     & onlarda
    \end{tabular}
\end{center}

\subsection{Locutions Pronominales}
L'ajout du suffixe nominal de personne transforme en pronoms certains noms-adjectifs :
\begin{center}
    \begin{tabular}{cp{.6\linewidth}}
        \toprule
        \textsl{bir}, un                    & \textsl{biri}, l'un, quelqu'un (parfois avec redoublement {\sl birisi})\newline {\sl birbiri}, l'un l'autre | {\sl birbirleri}, les uns les autres                                                               \\
        \midrule
        \textsl{birinci}                    & \textsl{birincisi}, le premier \newline ({\sl birinciniz}, le premier d'entre vous)                                                                                                                              \\
        \midrule
        \textsl{dörtte üç}                  & \textsl{dörtte üçümüz}, les trois quarts d'entre nous.                                                                                                                                                           \\
        \midrule
        \textsl{ba\ch ka, di\ug er}, autre  & \textsl{ba\ch as\i \ di\ug eri}, in autre, l'autre \newline à distinguer de {\sl öbür} qui doit être réservé aux alternatives.                                                                                   \\
        \midrule
        \textsl{çok}, beaucoup              & {\sl ço\ug u}, la plupart de...\newline çoklar\i, la plupart d'entre eux                                                                                                                                         \\
        \midrule
        \textsl{baz\i}, certains            & {\sl baz\i lar\i }, certains d'entre eux                                                                                                                                                                         \\
        \midrule
        \textsl{az}, peu                    & \textsl{az\i}, peu de\dots                                                                                                                                                                                       \\
        \midrule
        \textsl{daha}, plus, davantage      & \textsl{dahas\i}, le supplément\newline \textsl{dahas\i \ var}, il y en a encore                                                                                                                                 \\
        \midrule
        \textsl{daha büyük}                 & \textsl{daha büyü\ug ü}, le plus grand \newline \textsl{daha bÜyü\ug ünüz}, le plus grand d'entre vous                                                                                                           \\
        \midrule
        \textsl{hangi}, quel ?              & {\sl hangisi}, lequel (d'entre eux) ? \newline {\sl hangimiz}, lequel d'entre nous ?                                                                                                                             \\
        \midrule
        \textsl{hep}, toujours, entièrement & \textsl{hepsi}, la totalité (suffixation irrégulière) \newline \textsl{hepimiz}, nous tous \newline (pour l'emploi adjectival, on utilisera \textsl{bütün} ou {\sl tüm}, qui peuvent eux même être substantivés) \\
        \midrule
        \textsl{kaç}, combien               & \textsl{kaç\i m\i z}, combien parmi nous ?\newline {\sl kaç\i nc\i s\i}, le quantième                                                                                                                            \\
        \midrule
        \textsl{kendi}, propre, à soi       & \textsl{kendisi}, lui-même \newline \textsl{kendim}, moi-même                                                                                                                                                    \\
        \bottomrule
    \end{tabular}
\end{center}

\subsection{La Proposition Nominale Épithète}
Considérons une proposition nominale, par exemple \textsl{dam k\i rm\i z\i}, le toit est rouge. Placée devant un nom, cette proposition nominale peut devenir une épithète complexe. \\
Cela permet notamment d'exprimer l'équivalent d'une relative française de type \og \textsc{Dont + Verbe être} \fg : \\
$\rightarrow$ \textsl{dam\i k\i rm\i z\i ev}, la maison dont le toit est rouge (= \textsl{k\i rm\i z\i daml\i ev})\\
\textsl{Dam} est le sujet logique de la quasi-proposition ainsi constituée. Il est affecté d'un suffixe de 3ème personne qui renvoie \textsl{ev}, lui même sujet grammatical du groupe nominal dans son entier.

\subsection{Post-Positions}
Les post-positions turques assurent la même fonction qu'une préposition en français, tout en étant placées après ce à quoi elles se rapportent. On distingue les post-positions dites \og primaires \fg (qui se construisent avec un cas) des \og secondaires \fg (dérivées de noms ; infra. leçon 7). Les post-positions primaires se construisent respectivement avec :
\begin{description}
    \item[Le Cas Absolu:]\
        \begin{center}
            \begin{tabular}{l>{\sl}l}
                gibi                                                         & comme                    \\
                ile \textnormal{ou} -(y)l\sce \textnormal{par agglutination} & avec, par le moyen de    \\
                için                                                         & pour                     \\
                kadar, denli                                                 & autant que               \\
                üzere                                                        & sur le point de, afin de
            \end{tabular}
        \end{center}
        Ces post-positions exigent le génitif avec les pronoms personnels ou démonstratifs (ainsi qu'avec l'interrogatif {\sl kim}) — à l'exception des pronoms portant le suffixe de nombre.
        Ex.: on dit {\sl benim ile (=benimle), onun ile (=onunla), senin gibi, bizim için}, mais {\sl bizler için} et {\sl onlar için}; au téléphone: {\sl kiminle görüşüyorum?} (à qui ai-je l'honneur de parler?)
    \item[Le Directif:] \
        \begin{center}
            \begin{tabular}{l>{\sl}l}
                ait                    & appartenant à, relevant de            \\
                dair                   & relatif à, concernant                 \\
                go\ug ru               & vers, en direction de                 \\
                göre, nazaren          & selon, considérant, eu égard à        \\
                kadar, dek, de\ug in   & jusqu'à                               \\
                kar\ch \i              & contre, opposé\textperiodcentered e à \\
                kar\ch \i n, ra\ug men & en dépit de, malgré                   \\
                nispeten               & proportionnellement à, relativement à
            \end{tabular}
        \end{center}
    \item[L'Ablatif:] \
        \begin{center}
            \begin{tabular}{>{\sl}ll>{\sl}ll}
                asa\ug \i       & au-dessous de                      & $\neq$ \ yukar\i   & au-dessus de                  \\
                önce, evvel     & avant                              & $\neq$ sonra       & après                         \\
                beri, bu yana   & depuis, à compter de               & $\neq$ öte         & pas d'emploi post-positionnel \\
                geri            & à l'arrière de, en retard sur      & $\neq$ ileri       & à l'avant de, en avance sur   \\
                içeri           & à l'intérieur de                   & $\neq$ d\i\ch ar\i & à l'extérieur de              \\
                itibaren        & à parti de                         &                    &                               \\
                ba\ch ka, gayri & autre que, outre que               &                    &                               \\
                dolay\i, ötürü  & en raison de, à cause de           &                    &                               \\
                fazla           & en plus de, en excès par rapport à &                    &                               \\
                yana            & du côté de, favorable à            &                    &                               \\
            \end{tabular}
        \end{center}
\end{description}
Dans certaines locutions, des post-positions primaires peuvent également être substantivées d'une façon semblable :
\begin{center}
    \begin{tabular}{>{\sl}ll}
        \toprule
        bu gibiler(i)                & les ... de ce genre        \\
        \midrule
        bunun gibisi, bu adam gibisi & le pareil, le semblable à  \\
        \midrule
        öyle gibime geliyor          & ça me semble être ainsi    \\
        \midrule
        bunun kadar\i                & semblable quantité         \\
        \midrule
        yemek sonralar\i             & les \og après-repas\fg     \\
        \midrule
        fazlas\i yala                & trop (adv.), excessivement \\
        \midrule
        do\ug rusu                   & à vrai dire, en fait       \\
        \bottomrule
    \end{tabular}
\end{center}
\section{\textsl{Yedinci Ders}}
\subsection{Post-Positions Secondaires}
À  la différence des post-positions \og primaires\fg, celles-ci sont dérivées de noms, donc de mots dont l'usage couvre par ailleurs dans la langue le spectre de toutes les déclinaisons nominales possibles.\\
Ces noms font office de post-positions dans le cas où ils font simultanément l'objet:
\begin{enumerate}
    \item d'un rapport d'annexion: le nom est marqué par un suffixe de personne, le complément du nom est au génitif ou au cas absolu
    \item d'une suffixation par un cas spatial.
\end{enumerate}
On distingue trois cas de figure :
\begin{enumerate}
    \item \begin{itemize}
              \item L'emploi post-positionnel ne modifie pas la signification du nom
              \item il est possible avec plusieurs cas spatiaux différents
              \item la déclinaison de l'antécédent n'altère pas le sens de la phrase: l'emploi du cas absolu est possible, celui du génitif étant plus courant
          \end{itemize}
          \begin{center}
              \begin{tabular}{>{\sl}ll|>{\sl}ll}
                  \toprule
                  üst, üzer-       & sur               & alt                      & sous           \\
                  \midrule
                  ön               & devant            & arka, art, pe\ch         & derrière       \\
                  \midrule
                  iç, içeri, dahil & dans              & di\ch, di\ch ar\i, hariç & hors           \\
                  \midrule
                  ara              & intervalle, entre & yan, taraf               & côté           \\
                  \midrule
                  etraf, çevre     & pourtour, autour  & orta                     & milieu, centre \\
                  \midrule
                  ba\ch            & début             & kar\ch \i                & vis-à-vis      \\
                  \midrule
                  uç               & bout, extrémité                                               \\
                  \bottomrule
              \end{tabular}
          \end{center}
    \item \begin{itemize}
              \item L'emploi post-positionnel ne modifie que marginalement la signification du nom
              \item il n'est possible qu'à un seul cas spatial (ou avec le suffixe adverbial -(n)c\sce)
              \item la déclinaison de l'antécédent n'altère pas le sens de la phrase: emploi du cas absolu aussi bien que du génitif
          \end{itemize}
          \begin{center}
              \begin{tabular}{>{\sl}ll|>{\sl}ll}
                  \toprule
                  boy     & longueur              & boyunca                 & au long de              \\
                  \midrule
                  esna    & instant               & esnas\i nda             & au moment où            \\
                  \midrule
                  saye    & ombre, égide          & sayesinde               & grâce à                 \\
                  \midrule
                  s\i ra  & rang                  & s\i ra s\i nda          & au cours de             \\
                  \midrule
                  u\ug ur & chance, bonne fortune & u\ug runa ou u\ug runda & pour l'amour, au nom de \\
                  \midrule
                  yer     & place                 & yerine                  & au lieu de              \\
                  \midrule
                  zarf    & enveloppe             & zarf\i nda              & pendant, durant         \\
                  \bottomrule
              \end{tabular}
          \end{center}
    \item \begin{itemize}
              \item L'emploi post-positionnel modifie la signification du nom
              \item Il n'est possible qu'à un seul cas spatial
              \item La déclinaison de l'antécédent altère le sens de la phrase: l'emploi post-positionnel requier le cas absolu, l'usage du génitif revient à restaurer le sens initial du nom (la seule exception concerne les pronoms personnels, qui demeurent au génitif avec la post-position)
          \end{itemize}
          \begin{center}
              \begin{tabular}{>{\sl}ll|>{\sl}ll}
                  \toprule
                  hak     & droit, dû       & hakk\i nda     & à propos de, concernant        \\
                  \midrule
                  taraf   & côté            & taraf\i ndan   & de la part de                  \\
                  \midrule
                  yüz     & face            & yüzünden       & à cause de                     \\
                  \midrule
                  bak\i m & soin, attention & bak\i m\i ndan & du point de vue de, eu égard à \\
                  \midrule
                  nam     & réputation      & nam\i na       & au nom de                      \\
                  \bottomrule
              \end{tabular}
          \end{center}

\end{enumerate}

\section{\textsl{Sekizinci ders} - Suffixes Verbaux}
\subsection{Réfléchi}
Le sujet subit les effets de l'action qu'il exerce. Le sens est celui d'un réfléchi français, de \og faire quelque chose pour soi\fg, ou d'un verbe de sentiment.\\
Suffixe {\sl -(\sci)n}: {\sl gezin} (se promener), {\sl giyin} (s'habiller), {\sl para edin} (se faire de l'argent), {\sl sevin} (aimer pour soi, se réjouir de… [complément au directif]), {\sl y\ kan} (se laver), {\sl söylen} (se dire à soi-même, râler, marmonner)
\subsection{Contributif}
Le sujet agit en même temps que d'autres; \og faire ensemble\fg\\
Suffixe \textsl{\sci\ch}: \textsl{görü\ch} (se voir, s'entretenir avec), \textsl{yeti\ch} (rejoindre, parvenir à), \textsl{yar\i\ch} (être en compétition avec), \textsl{yap\i\ch} (coller, adhérer)

\subsection{Factitif}
Le sujet fait effectuer une action, ou la laisse effectuer.\\
Suffixe :
\begin{enumerate}
    \item[a)] \textsl{-t} après une base verbale de plus d'une syllabe terminée par une voyelle, un {\sl l} ou un {\sl r}.
    \item[b)] \textsl{d\sci r} après une base verbale monosyllabique (sauf cas particuliers) ou polysyllabique terminée par une consonne autre que {\sl l} et \textsl{r}.
\end{enumerate}
Cas particuliers :
\begin{center}
    \begin{NiceTabular}{>{\sl}ll>{\sl}ll}
        \toprule
        \multicolumn{4}{l}{Suffixe \textsl{-(\sci)r}:}                                                                     \\
        \midrule
        bat\i r                    & couler, envoyer par le fond & göçür                    & faire migrer                 \\
        bitir                      & achever, finir              & kaç\i r                  & faire fuir, laisser échapper \\
        do\ug ur                   & donner naissance            & içir                     & faire boire                  \\
        doyur                      & rassasier                   & pi\ch ir                 & cuire, cuisiner              \\
        duyur                      & annoncer                    & \ch i\ch ir              & faire enfler                 \\
        dü\ch ür                   & faire tomber                & ta\ch \i r               & faire déborder               \\
        geçir                      & faire passer                & uçur                     & faire voler                  \\
        \midrule
        \multicolumn{4}{l}{Suffixe \textsl{-\sce r}:}                                                                      \\
        \midrule
        gider                      & faire partir                & ç\i kar                  & extraire                     \\
        kopar                      & casser, arracher            &                          &                              \\
        \midrule
        \multicolumn{4}{l}{Suffixe \textsl{-\sci t}:}                                                                      \\
        \midrule
        ak\i t                     & faire s'écouler             & \ch sap\i t              & faire dévier                 \\
        kokut                      & faire sentir, faire puer    & sark\i t                 & laisser prendre              \\
        korkut                     & faire craindre              & ürküt                    & effrayer                     \\
        \midrule
        \multicolumn{4}{l}{Irréguliers:}                                                                                   \\
        \midrule
        gel $\rightarrow$ getir    & faire venir, apporter       & em $\rightarrow$ emzir   & allaiter                     \\
        kal $\rightarrow$ kald\i r & lever, enlever              & gör $\rightarrow$ göster & montrer                      \\
        \bottomrule
        \CodeAfter
        \begin{tikzpicture}
            \draw (2-|3) -- (9-|3);
            \draw (10-|3) -- (12-|3);
            \draw (13-|3) -- (16-|3);
            \draw (17-|3) -- (19-|3);
        \end{tikzpicture}
    \end{NiceTabular}
\end{center}
En cas de double factitif, la suffixation suit un principe d'alternance:
\begin{center}
    \textsl{t} est ajouté après \textsl{-d\sci r}, \textsl{-\sci r} ou \textsl{-\sce r}; \textsl{-d\sci r} vient après \textsl{-t} ou \textsl{-\sci t}
\end{center}
La combinaison de \textsl{-(\sci)\ch} peut signifier la répétition intensive d'une action: \textsl{ara} $\rightarrow$ \textsl{ara\ch t\i r}, \textsl{at} $\rightarrow$ \textsl{at\i \ch t\i r}.

\subsection{Passif}
Suffixe \textsl{-(\sci)l}: \textsl{edil} (être fait), \textsl{görül} (être vu), \textsl{sevil} (être aimé).\\
Après une base verbale terminée par une voyelle ou un \textsl{l}, la forme régulière du suffixe est remplacée par le suffixe réfléchi: \textsl{al\i n} (être pris).\\
Le passif turc est fréquemment employé avec valeur d'impersonnel, même avec des verbes intransitifs: \textsl{gidiliyor} (on s'en va). C'est plus courant encore avec les verbes transitifs: \textsl{\ch ehir geziliyor} (on se promène en ville), \textsl{görülüyorsun} (on te voit).
\subsection{Négation et Impossibilité}
Suffixe \textsl{-m\sce} (enclitique) et \textsl{(y)\sce m\sce} (accentué sur la première syllabe), non cumulables.\\
Il s'agit de la seule manière possible de nier un prédicat verbal: c'est léquivalent de la négation nominale {\sl de\ug il}\\
\textsl{-(y)\sce m\sce} a pour antonyme le suffixe de possibilité \textsl{-(y)\sce bil} qui peut lui-même se cumuler aux suffixes de négation et d'impossibilité: \textsl{gelmeyebiliyorum} = il se pourrait que je ne vienne pas; \textsl{gelemeyebiliyorum} = il se pourrait que je ne puisse pas venir.

\subsection{Exemples:}
\begin{center}
    \begin{NiceTabular}{>{\sl}lp{.12\linewidth}p{.13\linewidth}p{.12\linewidth}p{.12\linewidth}p{.12\linewidth}p{.10\linewidth}}
        \CodeBefore
        \begin{tikzpicture}
            \draw[rectangle, draw = gray, fill = gray!30] (1-|2) -- (1-|8) -- (2-|8) -- (2-|2);
        \end{tikzpicture}
        \Body
        \RowStyle{\bf\vspace{.1pt}}
              & Réfléchi                               & Contributif                                                                       & Factitif                                                                       & [c + f]                                              & Passif                                            & Négatif \\
        al    & al\i n : s'offenser, se scandaliser de & al\i \ch: s'habituer à, se familiariser avec                                      & ald\i r: faire prendre, faire acheter; se soucier de, donner de l'importance à & al\i \ch t\i r: habituer, accoutumer, rôder, exercer & al\i n: être pris, reçu, acheté                   & alma    \\
        anla  & -                                      & anla\ch: s'accorder, s'entendre, se concerter                                     & anlat: raconter; expliquer                                                     & anla\ch t\i r                                        & anla\ch \i l                                      & anlama  \\
        çek   & çekin: se gêner, se garder de          & çeki\ch: se disputer, se chamailler                                               & çektir: faire tirer; faire souffrir, faire endurer                             & çeki\ch tir: tirailler; dénigrer, médire             & çekim                                             & çekme   \\
        ç\i k & -                                      & ç\i k\i\ch: invectiver, vitupérer; suffire à qqch (se dit d'une somme d'argent)   & ç\i kar                                                                        & ç\i k\i \ch t\i r: réunir (une somme d'argent)       & ç\i k\i l: sortir, émerger; obtenir un diplôme de & ç\i kma \\
        gel   & gelin: parvenir                        & geli\ch: se développer, grandir, prospérer                                        & getir                                                                          & geli\ch tir                                          & gelin: en arriver à                               & gelme   \\
        git   & -                                      & gidi\ch: démander, chatouiller                                                    & gider                                                                          & gidi\ch tir                                          & gidil                                             & gitme   \\
        gül   & -                                      & gülü\ch: rire les uns avec les autres, les uns des autres; se distraire, folâtrer & güldür: faire rire,distraire, amuser                                           & gülü\ch tür                                          & gülün: se moquer de                               & gülme   \\
        ol    & -                                      & olu\ch: prendre forme, être constitué de                                          & oldur: faire advenir, mûrir                                                    & olu\ch tur: former, constituer; engendrer            & olun: devenir                                     & olma    \\
        öl    & -                                      & -                                                                                 & öldür: assassiner, tuer                                                        & -                                                    & -                                                 & ölme    \\
        sev   & sevin: se réjouir de                   & sevi\ch: s'aimer, se faire l'amour                                                & sevdir: faire aimer, attirer les faveurs; laisser caresser                     & sevi\ch tir                                          & sevil: être aimé, caressé                         & sevme
        \CodeAfter
        \begin{tikzpicture}
            \foreach \i in {2,..., 11}
                {\draw (\i-|1) -- (\i-|8);}
            \foreach \i in {2,...,7}{\draw[dotted,thin] (1-|\i) -- (12-|\i);}
        \end{tikzpicture}
    \end{NiceTabular}
\end{center}

\section{\textsl{Dokuzuncu Ders}}
\subsection{Orhan Gencebay | \textsl{Hatas\i z kul olmaz}}
\textsl{a\ch k\i na} = Pour l'amour de
\subsection{Les neuf caractères de la déclinaison verbale}
Il faut souligner l'irréductibilité du système de conjugaison à la division entre temps, aspects et modes. En turc certaines formes temporelles peuvent, selon leur emploi, relever de plusieurs modes ou aspects.

\begin{description}
    \item[Progressif:] {\sl -(i)yor}
        \begin{itemize}
            \item exprime k'actuib dans son développement concret, avec une nuance d'actualisation descriptive.
            \item à l'origine, dérivé de \textsl{yor\i r}, aoriste de l'ancien \textsl{yor\i mak}, \og aller, marcher\fg.
            \item prêter attention aux modifications vocaliques :
                  \begin{itemize}[label = \textbullet]
                      \item dans le cas d'une base verbale consonantique, la voyelle insérée avec \textsl{-yor} suit l'harmonie vocalique: \textsl{geliyor, görüyor, al\i yor, ko\ch uyor}
                      \item base verbale se terminant en \textsl{e/a}: \textsl{bekle} $\to$ \textsl{bekliyor}, \textsl{anla} $\to$ \textsl{anl\i yor}. De même avec le négatif : \textsl{al- + -ma- + -yor} $\to$ \textsl{alm\i yor}, etc...
                      \item mais si la voyelle initiale du suffixe est \og ronde \fg (= labiale, \textsl{o, u, ö, ü}) \emph{et} si la première voyelle de la base verbale l'est également, le \textsl{e} ou {\sl a} en question le sera aussi: \textsl{yolla- + -yor} $\to$ \textsl{yolluyor} (envoyer); \textsl{türe} devient \textsl{türüyor} (dériver); \textsl{gözle} $\to$ \textsl{gözlüyor} (observer); \textsl{topla} $\to$ \textsl{topluyor} (amasser)
                  \end{itemize}
        \end{itemize}
    \item [Duratif:] {\sl - m\sce kt\sce}
          \begin{itemize}
              \item décrit une action en cours
          \end{itemize}
    \item [Intentif:] {\sl -(y)\sce c\sce k}
          \begin{itemize}
              \item exprime l'intention; employé pour les faits dont l'accomplissement futur est tenu pour décidé.
              \item différence avec l'éventuel ou \og temps large\fg (infra) qui exprime l'éventualité sur un plan plus objectif; l'intentif exprime une probabilité plus élevée que l'aoriste à l'affirmatif; mais plus faible au négatif.
          \end{itemize}
    \item [L'éventuel ou \og temps large\fg (geni\ch\ zaman) (aoriste):] \textsl{-(\sce)r, -(i)t}
          \begin{itemize}
              \item exprime :
                    \begin{enumerate}
                        \item l'action dans sa généralité, sans actualisation ni limitation de durée;
                        \item une éventualité;
                        \item une formule de requête polie: \textsl{afferdersiniz, outrur musunuz ?; olur mu ?}
                    \end{enumerate}
              \item \textsl{-r} après un radical verbal terminé par une voyelle: \textsl{anlar, benzer, der, korur, yer}
              \item \textsl{-\sce r} après un radical verbal monosyllabique terminé par une consonne: \textsl{yapar, eder, kaçar, geçer, doyar} (être rassasié)
              \item \textsl{-\sci r} après une base verbale polysyllabique, ou monosyllabique dérivée: \textsl{b\i rak\i r, doyurur};
              \item exceptions ! Il y a 13 verbes monosyllabiques qui, bien que leur base soit non dérivée et terminée par une consonne forment leur aoriste en \textsl{-(\sci r)} : \textsl{al\i r, bilir, bulur, durur, gelir, görür, kal\i r, olur, ölür, san\i r, var\i r, verir, vurur} [Ne pas confondre avec les formes irrégulières du factitif.]
              \item le négatif aoriste:
                    \begin{enumerate}
                        \item est irrégulier: \textsl{-m\sce z} (réduit à \textsl{-m\sce} aux 1ères personnes) accentué, suivi d'une déclinaison personnelle quasi régulière (\textsl{gelmem, gelmezsin, gelmez, gelmeyiz, gelmezsiniz, gelmezler}).
                        \item avec l'interrogatif, permet de former (en langue parlée) un présent immédiat: \textsl{O s\i rada villan\i z gözüme çarpmaz m\i\dots}
                    \end{enumerate}
          \end{itemize}
    \item [Constatif:] \textsl{-d\sci}
          \begin{itemize}
              \item exprimer le résultat d'une action ou d'un processus constatés par le locuteur. \textsl{geldim} = j'arrive tout de suite, je suis là
              \item à la différence de l'aoriste, du progressif et du médiatif, il ne peut pas être projeté dans le future, et n'entre donc jamais dans la formation d'expressions du futur antérieur français.
              \item le suffixe est accentué: on distingue ainsi \textsl{vard\`i} (\textsl{var} + suffixe verbal constatif = \og il est arrivé \fg) et \textsl{v\`ardi} (\textsl{var} + suffixe prédicatif constatatif = \og il y en avait\fg).
          \end{itemize}
    \item [Médiatif (non-constatation)/inférentiel (dubitatif):] \textsl{-m\sci\ch}
          \begin{itemize}
              \item exprime le résultat acquis d'une action ou d'un processus non constatés, ainsi qu'un résultat inattendu.
              \item combiné avec \textsl{olacak} il permet d'exprimer le futur antérieur français: \textsl{yar\i m saat sonra bu dersi bitimi\ch \ olaca\ug \i z} aussi bien qu'un futur simple d'état: \textsl{ders yar\i m saat sonra bitmi\ch \ olacak}
              \item conjugé avec \textsl{-d\sci r} ou d'autres formes du verbe être, cette forme perd sa connotation inférentielle et s'assimile à un passé défini: \textsl{gelmi\ch tir}, il est venu.
          \end{itemize}
    \item [Déontique:] \textsl{-m\sce l\sci}
          \begin{itemize}
              \item exprime l'obligation physique, logique, technique, sociale, morale...
              \item peut s'employer pour marquer une conjecture déduite logiquement de la situation ou du contexte \textsl{y\i ld\i r\i m pek uza\ug a dü\ch memi\ch \ olmal\i}.
              \item peut enfin s'exprimer, bien qu'à la voie active, la valeur impersonnelle habituellement associée au passif: \textsl{jandarmaya haber vermeli} ($\cong$ \textsl{verimeli})
          \end{itemize}
    \item [Hypothétique:] \textsl{-s\sce}
          \begin{itemize}
              \item exprime une hypothèse
              \item certaines phrases formées uniquement d'une proposition hypothétique ont une valeur d'optatif: \textsl{bir çay daha rica etsem} (encore un thé, je vous prie)
          \end{itemize}
    \item [Subjonctif:] \textsl{-(y)\sce}
          \begin{itemize}
              \item exprime:
                    \begin{enumerate}
                        \item le souhait, le désir et la nécessité
                        \item dans certaines formules, le voeu, la bénédiction, la malédiction: \textsl{lanet ola!}
                        \item aux 1ères personnes, l'emploi courant est comparable à celui de \textsl{let's} en anglais: \textsl{yar\i n gelelim, bir bakay\i m}
                    \end{enumerate}
              \item tend à s'assimiler à l'impératif:
                    \begin{enumerate}
                        \item dans l'usage courant, \textsl{lanet ola!} devient \textsl{lanet olsun!}
                        \item excepté aux 2èmes personnes: ainsi la différence demeure marquée entre \textsl{gel!} (viens) et \textsl{gelesin!} (puisses-tu venir!) ou entre \textsl{sa \ug ol} (merci, usage courant) et \textsl{sa\ug olas\i n} (usage non moins courant mais plus révérencieux).
                    \end{enumerate}
          \end{itemize}
\end{description}
\section{Onuncu Ders}
\subsection{La Conjugaison: Types de suffixation Personnelle}
Il est possible d'analyser la déclinaison du verbe comme un système de \og conjugaison personnelle\fg  (Bazin 1987 : 85 et suiv.) : c'est-à-dire qu'aux suffixes modo-temporels de la \og caractéristique\fg  verbale (leçon 9) s'ajoutent des suffixes de prédication nominale, ceux-là même qui servent à décliner l'équivalent turc du verbe \og être\fg.\\
La conjugaison régulière procède donc par adjonction de ces suffxes prédicatifs personnels aux radicaux (dérivés ou non) du verbe.
\begin{center}
    \begin{tabular}{>{\sl}rl>{\sl}rl}
        geliyor-um & je viens & doktor-um & je suis médecin\\
        \midrule
        bakar-\i m & je vois, je verrai(s) & Frans\i z-\i m& Je suis Français(e)
    \end{tabular}
\end{center}

Une conjugaison spécifique s'applique cependant aux formes du passé constatif et de l'hypotétique (type {\sc ii}) à quoi s'ajoutent un type (type {\sc iii}) réservé au subjonctif et un dernier (type {\sc iv}) à l'impératif : 
\begin{center}
    \begin{NiceTabular}{>{\sl}l>{\sl}l>{\sl}l>{\sl}l}
        \CodeBefore
        \begin{tikzpicture}
            \draw[fill = gray!30, draw = gray, rectangle] (1-|1) -- (1-|5) --(2-|5) -- (2-|1);
        \end{tikzpicture}
        \Body
        \sc Type I & \sc Type II & \sc Type III & \sc Type IV\\
        -(y)\sci m & -m & -(y)\sce y\sci m & - \\
        -s\sci n & -n & -(y)\sce s\sci n & -\\
        -(dir) & - & -(y)\sce & -s\sci n\\
        -(y)\sci z & -k & -(y)\sce l\sci m & -(y)\sce l\sci m\\
        -s\sci n \sci z & -n\sci z & -(y)\sce s\sci n\sci z & -(y)\sci n(\sci z)\\
        -(d\sci r)l\sce r & -l\sce r & -(y) \sce l \sce r & -s\sci nl\sce r\\
        \CodeAfter
        \begin{tikzpicture}
            \draw (2-|1) -- (2-|5);
            \foreach \i in {3,...,8} 
                {\draw (\i-|1) -- (\i-|5);};
            \foreach \i in {2,...,4}
                {\draw (1-|\i) -- (8-|\i);};
        \end{tikzpicture}
    \end{NiceTabular}
\end{center}

\subsection{Auxiliares \og Caractériels\fg Prédicatifs à la Conjugaison}
Les caractères verbaux peuvent aussi, sauf exception, faire l'objet d'une combinaison avec l'un des trois suffixes suivants : 
\begin{center}
    {\sl -(y)d\sci} ou {\sl idi}\\
    \textsl{-(y)m\sci \ch} ou \textsl{im\sci \ch}\\
    \textsl{-(y)s\sce} ou \textsl{ise}
\end{center}
Ces \og verbes-suffixes \fg ou \og copular markers\fg sont des formes du verbe \og être\fg. Il s'agit donc de suffixes prédicatifs. Cela signifie qu'ils peuvent être appliqués à des phrases nominales aussi bien qu'à des formes verbales.\\
Ces suffixes prédicatifs doivent être soigneusement distingués des suffixes non prédicatifs de la caractéristique verbale auxquels ils sont apparentés, et dont ils suivent d'ailleurs aussi bien la déclinaison que (le plus souvent) la sémantique. 
\begin{itemize}
    \item \textsl{-d\sci} est prédicatif dans \textsl{gelirdim} (= {\sl gelir idim}) ou \textsl{oturuyorduk} (= {\sl oturuyor idik}); il est non prédicatif dans \textsl{geldim} ou \textsl{oturduk}.
    \item \textsl{-m\sci \ch} est prédicatif dans \textsl{gelecekmi\ch iz} (= {\sl gelecek imi\ch iz}) ou \textsl{konu\ch uyormu\ch sunuz} (= {\sl konu\ch uyor imi\ch siniz}); non prédicatif dans \textsl{gelmi\ch iz} ou \textsl{konu\ch mu\ch sunuz}
    \item \textsl{-s\sce} est prédicatif dans \textsl{yap\i yorsan} (= {\sl yap\i yor isen}) ou \textsl{söyleyeceksek} (= {\sl söleyecek isek}); et non prédicatif \textsl{yapsan} ou \textsl{söylesek}.
\end{itemize}
En cas d'ajout d'une désinence personnelle, celle-ci est déplacée de la forme verbale vers ce nouveau suffixe. \\


Cette combinatoire permet de nombreuses modulations de la signification, dont voici un aperçu: 
\begin{itemize}
    \item Progressif + \textsl{-(y)d\sci} ou \textsl{-(y)m\sci \ch} = imparfait descriptif :\\
    \textsl{Bir önceki hafta bu sokaktan geçilemiyordu} = Une semaine avant, il était impossible d'emprunter cette rue.
    \item Intentif + {\sl -(y)d\sci} ou {\sl -(y)m\sci \ch} = imparfait-intentif qui peut exprimer:
    \begin{itemize}
        \item Un irréel du passé/du présent: \textsl{Zaten bunu yapacakt\i m} = C'est d'ailleurs ce que je m'apprêtais à faire.
        \item Le souhait: \textsl{Bir çay daha isteyecektim} = J'aurais voulu un autre thé s'il vous plaît \\
        \textsl{Bir çay daha isteyecekmi\ch siniz} = J'ai comme l'impression que vous souhaiteriez un autre thé.
    \end{itemize}
    \item \textsl{Geni\ch Zaman} + {-(y)d\sci} ou {\sl -(y)m\sci \ch}: 
    \begin{itemize}
        \item Un imparfait d'habitude, un passé de narration : \textsl{Eskiden herkes göç edermi\ch} = Autrefois tout le monde était nomade dit-on.
        \item Un irréel du passé ou du présent\footnote{La nuance avec l'imparfait-intentif ci-dessus tient à ce qu'ici on attribue les faits non réalisés à une simple éventualité plutôt qu'à une intention ou à une tendance.} : \textsl{Tek ba\ch\i ma gelemezdim} = Je n'aurais pas pu venir tout seul.
    \end{itemize}
    \item Médiatif + {\sl -(y)d\sci} = plus-que-parfait (noter qu'en l'espèce la forme \textsl{-m\sci \ch} perd sa connotation inférentielle et s'assimile à un passé défini):\textsl{Gelmi\ch tim} = j'étais venu. 
    \item Subjonctif + {\sl -(y)d\sci} :\textsl{Geleydi!} = Si seulement il était venu!
    \item Hypothétique + {\sl -(y)d\sci} = cette combinaison a le même sens que la précédente, et est d'emploi plus fréquent : \textsl{Gelseydi!} = Si seulement il était venu!
\end{itemize}


\appendix
\newpage
\begin{longtable}{>{\sl}p{.21\textwidth}p{.21\textwidth}|>{\sl}p{.21\textwidth}p{.21\textwidth}}
    \multicolumn{4}{c}{\bf \large Petit Dictionnaire Turc $\longrightarrow$ Français}                                            \\
    \toprule
    Mot en Turc      & Traduction                                           & Mot en Turc     & Traduction                       \\
    \midrule \midrule
    \endfirsthead
    \toprule
    Mot en Turc      & Traduction                                           & Mot en Turc     & Traduction                       \\
    \midrule \midrule
    \endhead
    \bottomrule
    \endfoot

    kitap            & livre                                                & a\ug aç         & arbre                            \\
    \midrule
    sa\ug            & sain                                                 & yuvarlak        & rond                             \\
    \midrule
    da\ug            & montagne                                             & kibrit          & allumette                        \\
    \midrule
    i\ug ne          & aiguille                                             & alçak           & bas (adj.)                       \\
    \midrule
    ya\ug mur        & pluie                                                & güzel           & beau                             \\
    \midrule
    ince             & fin                                                  & iyi             & bon                              \\
    \midrule
    e\ug lence       & amusement                                            & çirkin          & laid                             \\
    \midrule
    y\i ld\i r\i m   & foudre                                               & \ch eker        & sucre                            \\
    \midrule
    ay               & lune                                                 & kad\i n         & femme                            \\
    \midrule
    çay              & thé                                                  & sand\i k        & caisse/coffre                    \\
    \midrule
    ay\i             & ourse                                                & koltuk          & fauteuil                         \\
    \midrule
    day\i            & oncle maternel                                       & tavan           & plafond                          \\
    \midrule
    bu               & ce, cette, ces (ceci, proche)                        & duvar           & mur                              \\
    \midrule
    da               & aussi                                                & ev              & maison                           \\
    \midrule
    de\ug il         & n'est pas                                            & el              & main                             \\
    \midrule
    demek            & dire                                                 & gül             & rose                             \\
    \midrule
    Frans\i zca      & français (langue)                                    & yüzük           & anneau/bague                     \\
    \midrule
    hay\i r          & non                                                  & etek            & jupe/flancs géographiques        \\
    \midrule
    kap\i            & porte                                                & gömlek          & chemise                          \\
    \midrule
    masa             & porte                                                & erkek           & garçon                           \\
    \midrule
    m\sc{i}          & (particule interrogative)                            & ceket           & veste                            \\
    \midrule
    ne               & quoi, que                                            & çizgi           & trait/tracé                      \\
    \midrule
    pencere          & fenêtre                                              & ayak            & pied                             \\
    \midrule
    sandalye         & chaise                                               & kapal\i         & fermé                            \\
    \midrule
    \ch u            & ce, cette, ces (cela, moins proche)                  & aç\i k          & ouvert                           \\
    \midrule
    beyaz            & blanc                                                & bo\ch           & vide                             \\
    \midrule
    k\i rm\i z\i     & rouge                                                & dolu            & complet                          \\
    \midrule
    sar\i            & jaune                                                & kirli           & sale                             \\
    \midrule
    pembe            & rose                                                 & temiz           & propre                           \\
    \midrule
    ye\ch il         & vert                                                 & pahal\i         & cher                             \\
    \midrule
    siyah            & noir                                                 & ucuz            & bon marché                       \\
    \midrule
    yeni             & nouveau                                              & eski            & vieux                            \\
    \midrule
    a\ug \i r        & lourd                                                & hafif           & léger                            \\
    \midrule
    uzun             & long                                                 & k\i sa          & court                            \\
    \midrule
    ad               & prénom                                               & ders            & leçon                            \\
    \midrule
    ho\ch            & gai/heureux                                          & ho\ch geldiniz  & soyez le bienvenu                \\
    \midrule
    memnum oldum     & enchanté                                             & renk            & couleur                          \\
    \midrule
    i\ch             & travail                                              & e\ch            & pair                             \\
    \midrule
    ögrenmek         & étudier                                              & arka            & dos                              \\
    \midrule
    din              & religion                                             & adam            & homme                            \\
    \midrule
    defter           & cahier                                               & kalem           & crayon                           \\
    \midrule
    var              & existant/il y a                                      & yok             & absent/il n'y a pas              \\
    \midrule
    anne             & mère                                                 & dahi            & de plus                          \\
    \midrule
    elma             & pomme                                                & hangi           & quel                             \\
    \midrule
    hani             & ah ça mais                                           & inanmak         & croire                           \\
    \midrule
    karde\ch         & frère                                                & selam           & salut                            \\
    \midrule
    \ch i\ch man     & enflé                                                & ziyaret         & visite                           \\
    \midrule
    aç\i kgöz        & dégourdi                                             & bilgisayar      & ordinateur                       \\
    \midrule
    han\i meli       & chèvrefeuille                                        & çekyat          & canapé-lit                       \\
    \midrule
    dinda\ch         & coreligionnaire                                      & gönülda\ch      & ami intime                       \\
    \midrule
    meslekta\ch      & collègue                                             & ülküda\ch       & compagnon de cause, d'idéal      \\
    \midrule
    ak\ch amki       & du soi                                               & yar\i nki       & de demain                        \\
    \midrule
    süt              & lait                                                 & top             & artillerie                       \\
    \midrule
    tarih            & histoire                                             & saat            & heure                            \\
    \midrule
    yar\i m          & un demi (adjectif)                                   & yar\i           & moitié (substantif)              \\
    \midrule
    buçuk            & ... et demi                                          & usta            & artisan                          \\
    \midrule
    sa\ug            & sain                                                 & peynir          & fromage                          \\
    \midrule
    zeytin           & olive                                                & sade            & nature (plain/regular)           \\
    \midrule
    tane             & exemplaire                                           & üstüne          & au-dessus                        \\
    \midrule
    az               & peu                                                  & iyi o zaman     & dans ce cas                      \\
    \midrule
    al\i yorum       & prendre                                              & ba\ch ka        & autre                            \\
    \midrule
    arzu             & souhait                                              & yan\i nda       & avec/à côté                      \\
    \midrule
    \ch eker         & sucre                                                & kahve           & café                             \\
    \midrule
    gelmek           & venir                                                & gitmek          & partir                           \\
    \midrule
    afiyet           & appétit                                              & olmak           & être/se produire                 \\
    \midrule
    Borcum ne kadar? & Combien vous dois-je ?                               & para            & monnaie/argent                   \\
    \midrule
    gün              & jour                                                 & haber           & nouvelle(s)                      \\
    \midrule
    nas\i ls\i n     & comment vas-tu                                       & sa\ug ol        & merci                            \\
    \midrule
    ara              & intervalle                                           & s\i ra          & rang/tour de passage             \\
    \midrule
    ara s\i ra       & de temps à autres                                    & konu\ch mak     & parler                           \\
    \midrule
    gerçek           & réalité                                              & ögretmek        & enseigner                        \\
    \midrule
    i\ch te          & ben voilà quoi                                       & u\ug ra\ch mak  & faire des efforts                \\
    \midrule
    art\i k          & désormais, à parti de là                             & yollamak        & envoyer                          \\
    \midrule
    türemek          & dériver                                              & gözlemek        & observer                         \\
    \midrule
    toplamak         & amasser                                              & almak           & prendre                          \\
    \midrule
    beklemek         & attendre                                             & affetmek        & pardonner                        \\
    \midrule
    doymak           & être rassasié                                        & b\i rakmak      & abandonner                       \\
    \midrule
    yenmek           & vaincre                                              & yemek           & manger                           \\
    \midrule
    fener            & phare                                                & sanmak          & penser                           \\
    \midrule
    a\ch k           & amour, passion                                       & kaybetmek       & perdre                           \\
    \midrule
    bende            & esclave, serviteur                                   & kul             & esclave                          \\
    \midrule
    can              & âme, être en vie                                     & olmaz           & n'est pas (ne sauraît être)      \\
    \midrule
    derman           & remède, force                                        & ne olur         & je t'en prie, je t'en supplie    \\
    \midrule
    dert             & peine, maladie, douleur, tracas                      & olsa            & si c'est                         \\
    \midrule
    dil              & langue                                               & ölmek           & mourir                           \\
    \midrule
    duymak           & entendre, ressentir                                  & raz\i           & prêt, consentant, content        \\
    \midrule
    elde olmak       & être disponible, accesible, conquis                  & salmak          & lâcher, libérer, dégager         \\
    \midrule
    feryat           & cri, lamentation                                     & seven           & (cellui) qui aime                \\
    \midrule
    gönül            & coeur, désir, inclination                            & sitem           & reproche                         \\
    \midrule
    güç              & force, difficile                                     & söz             & mot, parole                      \\
    \midrule
    haber            & nouvelle                                             & tez             & prompt, sous peu                 \\
    \midrule
    hal              & condition, état                                      & uzak            & loin                             \\
    \midrule
    hasret           & désir, envie, nostalgie                              & ümit            & loin                             \\
    \midrule
    hata             & faute                                                & ya\ch amak      & vivre                            \\
    \midrule
    kendim           & soi-même                                             & kalp            & coeur                            \\
    \midrule
    yormak           & fatiguer                                             & yorulmak        & être fatigué                     \\
    \midrule
    abla             & soeur aïnée                                          & merkez          & centre                           \\
    \midrule
    adam             & homme                                                & nazik           & délicat, gentil                  \\
    \midrule
    ak\ch am         & soir                                                 & okumak          & lire, étudier                    \\
    \midrule
    alçak            & bas                                                  & ögrenmek        & apprendre, étudier               \\
    \midrule
    apartman         & immeuble                                             & parti           & fête                             \\
    \midrule
    ayd\i n          & (adj.) éclairé, cultivé | (n.) intellectuel          & sabah           & matin                            \\
    \midrule
    bahçe            & jardin                                               & ses             & voix, son                        \\
    \midrule
    bugün            & aujourd'hui                                          & s\i n\i f       & classe                           \\
    \midrule
    çal\i \ch mak    & travailler                                           & sokak           & rue                              \\
    \midrule
    da\ug            & montage                                              & son             & (n.) fin \newline (adj.) dernier \\
    \midrule
    ev               & maison                                               & sürmek          & durer, se poursuivre             \\
    \midrule
    Eylül            & septembre                                            & \ch ans         & chance                           \\
    \midrule
    fikir            & idée, avis                                           & \ch ehir        & ville                            \\
    \midrule
    gelmek           & venir                                                & toplant\i       & réunion                          \\
    \midrule
    gitmek           & aller, partir                                        & ülke            & pays                             \\
    \midrule
    gürültü          & bruit                                                & vapur           & bateau à vapeur                  \\
    \midrule
    güzel            & beau                                                 & yemek           & repas                            \\
    \midrule
    hafta            & semaine                                              & yer             & lieu, place, endroit             \\
    \midrule
    halk             & peuple                                               & yeralmak        & avoir lieu, se tenir             \\
    \midrule
    hava             & air, temps (météo)                                   & yüksek          & haut                             \\
    \midrule
    karde\ch         & frère/soeur                                          & yürümek         & marcher                          \\
    \midrule
    ki\ch i          & personne                                             & zaman           & temps                            \\
    \midrule
    zor              & difficile, ardu \newline (n.) contrainte, coercition & konu            & sujet (topic)                    \\
    \midrule
    konu\ch mak      & parler                                               & köy             & village                          \\
    \midrule
    memleket         & pays                                                 & pek             & adverbe de renforcement          \\
    \midrule
    soru             & question                                             & sormak          & poser une question               \\
    \midrule
    kebap            & viande rôtie                                         & a\ug a          & maître, grand propriétaire       \\
    \midrule
    as\i lzade       & noble, aristocrate                                   & ba\ug \i rmak   & brailler, crier, clamer          \\
    \midrule
    bakmak           & regarder                                             & bey             & seigneur, gentilhomme            \\
    \midrule
    bile             & même (adv.)                                          & boy             & taille (dimension)               \\
    \midrule
    budala           & idiot, niais                                         & çat\i r çat\i r & en craquant, de force, aisément  \\
    \midrule
    çatlamak         & crever, éclater, se fendre                           & dolu            & plein, rempli                    \\
    \midrule
    dönmek           & tourner, être en rotation                            & dünya           & monde, univers                   \\
    \midrule
    gezmek           & se promener                                          & ne gezer!       & pas du tout!                     \\
    \midrule
    görmek           & voir                                                 & göz             & oeil                             \\
    \midrule
    hâlâ             & encore, toujours                                     & hayran          & admiratif, admirateur            \\
    \midrule
    hayranolmak      & admier, s'émerveiller                                & \i k\i nmak     & se contracter, s'efforcer        \\
    \midrule
    imrenmek         & aspirer à, désirer, envier                           & kalkmak         & se lever, se dresser             \\
    \midrule
    k\i skanmak      & envier, jalouser                                     & kimi            & certain, quelque                 \\
    \midrule
    kurba\ug a       & grenouille                                           & nice            & nombreux/un grande nombre de     \\
    \midrule
    öküz             & boeuf                                                & ötmek           & chanter, siffler (oiseau)        \\
    \midrule
    peki             & d'accord, soit, eh bien                              & \ch ehzade      & prince                           \\
    \midrule
    \ch imdi         & maitnenant, à présent                                & \ch i\ch mek    & (s')enfler, (se) gonfler         \\
    \midrule
    vakit            & temps, moment                                        & yakla\ch mak    & s'approcher                      \\
    \midrule
    yetmek           & suffir                                               & yumurta         & oeuf                             \\
    \midrule
    henüz            & encore, toujours                                     & evet            & oui
\end{longtable}


\end{document}
