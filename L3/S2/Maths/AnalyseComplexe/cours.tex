\documentclass{cours}

\title{Le cours d'Ariane Mézard}
\author{Ariane Mézard}

\begin{document}
\part{Fonctions Holomorphes}
\section{Fonctions Analytiques}
\subsection{Séries Entières}
\begin{définition}{Série Entière}{}
    Une série entière est une série de la forme $\sum_{n \in \N}a_{n}z^{n}$ où $z \in \C$ et $a_{n} \in \C$.\\
    Le domaine de convergence de la série entière est l'ensemble $\Delta$ des nombres complexes $z \in \C$ pour lesquels la série converge.  
\end{définition}
\begin{propositionfr}{Critère de Cauchy}{}
    Soient $a_{n}$ une suite complexe et $0 < r < r_{0}$. S'il existe $M > 0$ tel que 
    \[
        \abs{a_{n}}r_{0}^{n} \leq M, n \geq 0
    \]
    alors $a_{n}z^{n}$ converge normalement sur $\overline{D}(0, r)$.
\end{propositionfr}
\begin{proof}
    Pour tout $n \in \N$ et $z \in \overline{D}(0, r)$ on a : 
    \[
        \abs{a_{n}z^{n}} \leq \abs{a_{n}}r^{n} \leq M\left(\frac{r}{r_{0}}\right)^{n}
    \]
    Comme $0 < r < r_{0}$, $M \left(\frac{r}{r_{0}}\right)^{n}$ est le terme d'une série géométrique convergente.
\end{proof}

\begin{corollaire}{Rayon de Convergence}{}
    Soit $\sum_{n \in \N} a_{n}z^{n}$ une série entière et $R \in \R_{+} \cup \{+ \infty\}$ défini par 
    \[
        R = \sup \left\{r \geq 0 \text{ tel que la suite } \left(\abs{a_{n}}r^{n}\right)_{n \in \N} \text{ soit bornée}\right\}
    \]
    Alors le domaine de convergence $\Delta$ de la série vérifie : 
    \[
        D(0, R) \subseteq \Delta \subseteq \overline{D}(0, R)
    \]
\end{corollaire}

\begin{définition}{Rayon de Convergence}{}
    On appelle le nombre $R$ défini ci-dessus rayon de convergence.
\end{définition}

\begin{propositionfr}{Rayon d'Hadamard}{}
    Le rayon de convergence est donné par 
    \[
        R = \liminf_{n \to \infty}\frac{1}{\abs{a_{n}}^{1/n}}
    \]
    Avec la convention $1/0 = \infty$
\end{propositionfr}

\begin{lemme}{Lemme d'Abel}{}
    Soit $u_{n}$ une suite réelle décroissante vers $0$ et $v_{n}$ une suite complexe telle que les sommes partielles $s_{n} = \sum_{k = 0}^{n} v_{k}$ soient bornées. Alors la série $\sum u_{n}v_{n}$ converge.
\end{lemme}

\begin{propositionfr}{Principe des Zéros Isolés}{}
    Soit $f(z) = \sum a_{n}z^{n}$ la somme d'une série entière de rayon de convergence $R > 0$. Si au moins un des coefficients $a_{n}$ n'est pas nul, il existe $r \in \left] 0, +\infty\right[$ tel que $f$ ne s'annule pas pour $\abs{z} \in \left]0, r\right[$.
\end{propositionfr}
\begin{proof}
    Soit $l = \min \{n \in \N, a_{n} \neq 0\}$, on a : 
    \[
        f(z) = \sum_{n \geq l} a_{n}z^{n} = z^{l}g(z)
    \]
    avec $g(z) = a_{l} + a_{l + 1}z + \ldots$ et $g(0) \neq 0$. 
\end{proof}

\begin{définition}{Dérivée Complexe}{}
    Une fonction $f : U \to \C$ admet une dérivée par rapport à la variable complexe au point $z_{0}$ si
    \[
        \lim_{z \to u} \frac{f(z_{0} + u)- f(z_{0})}{u}
    \]
    existe. Cette limite est alors appelée dérivée de $f$ en $z_{0}$.
\end{définition}

\begin{propositionfr}{Dérivée d'une Série Entière}{}
    Soit $f(z) = \sum a_{n}z^{n}$ une série entière de rayon de convergence $R > 0$. Alors, pour tout $l \in \N^{\star}$, les dérivées $l$-ièmes de $f$ ont pour rayon de convergence $R$ et pour expression : 
    \[
        f^{(l)}(z) = \sum_{n \in \N}\frac{(n + l)!}{n!}a_{n + l}z^{n}  
    \]
\end{propositionfr}

\begin{corollaire}{Primitive}{}
    Une série entière $f(z) = \sum a_{n}z^{n}$ de rayon de convergence $R > 0$ admet sur $D(0, R)$ une primitive complexe
    \[
        F(z) = \sum \frac{a_{n}}{n + 1}z^{n + 1}
    \]
\end{corollaire}

\begin{propositionfr}
    Soit $f(z) = \sum a_{n}z^{n}$ une série entière de rayon de convergence $R > 0$. Soit $z_{0} \in D(0, R)$. La série entière 
    \[
        \sum_{n \in \N}\frac{1}{n!}f^{(n)}(z_{0})\omega^{n}
    \]
    a un rayon de convergence supérieur à $R - \abs{z_{0}}$ et pour tout $z \in D(z_{0}, R - \abs{z_{0}})$,
    \[
        f(z) = \sum_{n \geq 0} \frac{1}{n !}f^{(n)}(z_{0})(z - z_{0})^{n}
    \]
\end{propositionfr}

\subsection{Fonctions Analytiques}
\begin{définition}{Fonction Analytique}{}
    Une fonction $f : U \to \C$ est dite analytique si elle est DSE au voisinage de chaque point de $U$. 
\end{définition}

\begin{propositionfr}{Dérivabilité}{}
    Une fonction analytique sur un ouvert $U$ de $\C$ admet des dérivées de tous ordres qui sont des fonctions analytiques sur $U$. De plus, pour tout $z_{0} \in U$, $f$ est somme de sa série de Taylor en $z_{0}$ sur un voisinage de $z_{0}$.
\end{propositionfr}

\begin{corollaire}{Unicité du DSE}{}
    Une fonction analytique sur $U$ admet un unique développement en série entière au voisinage de chaque point de $U$.
\end{corollaire}

\begin{lemme}{Nullité}{}
    Si $U$ est connexe et $f$ est analytique sur $U$, nulle sur un ouvert non-vide de $U$, alors $f$ est identiquement nulle sur $U$. 
\end{lemme}

\begin{propositionfr}{Zéros Isolés}
    Soit $f$ une fonction analytique sur un ouvert connexe $U$. Si $f$ n'est pas identiquement nulle, ses zéros sont isolés, i.e. si $z_{0} \in U$ avec $f(z_{0}) = 0$, alors il existe $r > 0$ tel que $z_{0}$ soit le seul $z_{0}$ de $f$ sur $D(z_{0}, r)$
\end{propositionfr}

\begin{théorème}{Prolongement Analytique}{}
    Soit $U$ un ouvert connexe de $\C$, $f, g$ des fonctions analytiques sur $U$. Si $f, g$ coincident sur une partie $\Sigma$ de $U$ qui a un point d'accumulation dans $U$, alors elles coincident sur $U$. 
\end{théorème}

\begin{définition}{Primitive}{}
    Etant donnée une fonction analytique $f$ sur $U$, une fonction analytique $F$ de $U$ dans $\C$ est dite primitive de $f$ si $F'(z) = f(z)$ sur $U$. 
\end{définition}

\subsection{Détermination du Logarithme}
\begin{définition}{Détermination de l'Argument}{}
    Soit $U \subseteq \C^{\star}$ ouvert. Une fonction continue $\arg : U \to \R$ est dite détermination continue de l'argument sur $U$ si pour tout $z \in U$, $\exp(i\arg(z))=\frac{z}{\abs{z}}$
\end{définition}
\begin{définition}{Détermination Principale}{}
    La détermination continue de l'argument 
    \[
        \begin{array}{rl}
            \C - \R_{-} \longrightarrow & \left] - \pi, \pi \right[\\
            z \mapsto & 2\arctan\left(\frac{y}{x + \sqrt{x^{2} + y^{2}}}\right)
        \end{array}
    \]
    en prenant la racine carrée de $z$ appartenant au demi-plan $\Re z > 0$ est appelée détermination principale de l'argument.
\end{définition}
\begin{définition}{Logarithme}{}
    Soit $U \subseteq \C^{\star}$ ouvert. Une fonction continue $f : U \to \C$ est dite détermination du logarithme sur $U$ si 
    \[
        \forall z \in U, \exp(f(w)) = w
    \]
\end{définition}

\begin{définition}{Détermination Principale du Log}{}
    On définit pour $\theta \in \R$ la fonction
    \[
        \log_{\theta} : \C \to \R_{-}e^{i\theta}, \log_{\theta}(w)=\log \abs{w} + i\arg_{\theta}(w)
    \]
    La fonction $\log_{0}$ est appelée détermination principale du logarithme et notée $\log$. 
\end{définition}

\begin{propositionfr}{DSE du Logarithme}{}
    $\log$ est DSE sur $D(1, 1)$ et sur $D(0, 1)$ on a
    \[
        \log(1 + z) = \sum \frac{(-1)^{n + 1}}{n}z^{n}
    \]
    Par conséquent, sur $D(z_{0}, \abs{z_{0}})$, 
    \[
        g(z) = \log z_{0} + i\theta_{0} + \sum_{n \geq 1}\frac{(-1)^{n - 1}}{n}\left(\frac{z - z_{0}}{z_{0}}\right)^{n}
    \]
    est une détermination analytique du logarithme. 
\end{propositionfr}

\begin{propositionfr}{Analycité des Déterminations}{}
    Il y a équivalence sur un ouvert connexe $U$ de $\C^{\star}$ pour une application continue $l$ entre : 
    \begin{itemize}
        \item $l$ est une détermination du logarithme à l'addition d'une constante près
        \item $l$ est une primitive analytique de $\frac{1}{z}$ sur $U$. 
    \end{itemize}
\end{propositionfr}

\begin{définition}{Détermination}{}
    Soit $U \subseteq \C^{\star}$ et $\alpha \in \C$. Une détermination continue de $z^{\alpha}$ est une application continue $g$ de $U$ dans $\C$ telle qu'il existe une détermination du logarithme $l(z)$ de $z$ telle que $g(z) = \exp^{\alpha l(z)}$.
\end{définition}

\section{Théorie de Cauchy}
\subsection{Homotopie et Simple Connexité}
\begin{définition}{Chemin}{}
    Soit $\left[a, b\right]$ un intervalle de $\R$. Un chemin $\gamma: \left[a, b\right] \to \C$ est une application continue. Le point $\gamma(a)$ est appelé origine et le point $\gamma(b)$ est dit extrémité. On orientera par défaut un chemin dans le sens des paramètres croissants. Si $\gamma(a) = \gamma(b)$, le chemin est dit lacet d'origine $\gamma(a)$.
\end{définition}

\begin{définition}{Opérations}{}
    \begin{enumerate}
        \item Si $\gamma$ est constant, son image est réduite à un point. Il est alors appelé chemin (ou lacet) constant.
        \item Soit $\alpha \in \R^{\star}$, $\gamma : t \in [0, 1] \mapsto e^{2i\pi \alpha t}$ est un chemin dont l'image est une partie du cercle unité $\partial D(0, 1)$. Si $\alpha = n \in \Z^{\star}$, $\gamma\left([0, 1]\right)$ est le cercle tout entier parcouru $n$ fois. 
        \item Si $\gamma : [a, b] \to \C$ est un chemin, le chemin opposé 
        \[
            \gamma^{0} : t \in [a, b] \mapsto \gamma(a + b - t)
        \]
        est $\gamma$ parcouru en sens inverse. 
        \item La juxtaposition de $\gamma_{1}, \gamma_{2}$ tels que $\gamma_{1}(b) = \gamma_{2}(c)$ est le chemin $\gamma = \gamma_{1} \land \gamma_{2} : \left[a, d + b - c\right] \to \C$
        \[
            \gamma(t) = \begin{cases}
                \gamma_{1}(t) & \text{ pour } a\leq t \leq b\\
            \gamma_{2}(t - b + c ) & \text{ pour } b \leq t \leq d + b - c
            \end{cases}
        \]
    \end{enumerate}
\end{définition}

\begin{définition}{Homotopie}{}
    Soit $U$ un ouvert de $\C$, $\gamma_{i} : I \to U$, $i \in \{1, 2\}$ deux chemins. Une homotopie de $\gamma_{1}$ à $\gamma_{2}$ dans $U$ est une application continue $\phi$ de $I \times J$ dans $U$ où $I = [a, b]$ et $J = [c, d]$ sont deux intervalles de $\R$ telle que : 
    \[
        \phi(t, c) = \gamma_{1}(t) \text{ et } \phi(t, d) = \gamma_{2}(t), t\in I
    \]
\end{définition}

\begin{définition}{Simple Connexité}{}
    Un espace topologique $X$ connexe par arcs est dit simplement connexe si tout lacet dans $X$ est homotope à un point dans $X$. 
\end{définition}

\begin{propositionfr}{}{}
    \begin{itemize}
        \item Un espace topologique est simplement connexe si et seulement si tous les chemins de même extrémités sont homotopes.
        \item Un ouvert étoilé par rapport à un point est simplement connexe. En particulier, dans $\C$, le plan, un demi-plan, un disque ouvert, l'intérieur d'un rectangle ou d'un triangle sont simplement connexes. 
        \item Le demi-plan ouvert $\Im z > 0$ auquel nous ôtons un nombre fini de demi-droites fermées $z = t +i\beta_{k}$, $t\in \left]-\infty, \alpha_{k}\right]$ est simplement connexe non étoilé. 
        \item $\C^{\star}$ n'est pas simplement connexe car le cercle unité n'est pas homotope à un chemin constant. 
    \end{itemize}
\end{propositionfr}

\subsection{Intégrales sur un Chemin}
Dorénavant, les chemins sont supposés $\cont^{1}$ par morceaux. 

\begin{définition}{Equivalence de Chemins}{}
    Deux chemins $\gamma_{i} : I_{i} \to \C$ sont dits équivalents s'il existe une bijection croissante $\phi : I_{2} \to I_{1}$ continue de réciproque continue et $\cont^{1}$ par morceaux telle que : 
    \[
        \gamma_{2}(t) = \gamma_{1}(\phi(t)), t\in I_{2}
    \]
\end{définition}

\begin{définition}{Intégrale le long d'un Chemin}{}
    Soit $f : U \to \C$ continue et $\gamma : I = [a, b] \to \C$ un chemin avec $\gamma(I) \subseteq U$. Alors, la fonction $t : f(\gamma(t))\gamma'(t)$ est continue par morceaux dans $[a, b]$. On appelle intégrale de $f$ le long du chemin $\gamma$ : 
    \[
        \int_{\gamma}f(z)\d z = \int_{a}^{b}f(\gamma(t))\gamma'(t)\d t
    \]
\end{définition}

\begin{définition}{Longueur}{}
    La longueur d'un chemin est le réel : 
    \[
        long(\gamma) = \int_{a}^{b}\abs{\gamma^{'}(t)}\d t
    \]
\end{définition}

\begin{propositionfr}{Propriétés}{}
    \begin{itemize}
        \item Si $F$ est une primitive de $f$, pour tout chemin $\gamma$ : 
        \[
            \int_{\gamma}f(z)\d z = F(\gamma(b)) - F(\gamma(a))
        \]
        \item Si $\gamma_{1} \sim \gamma_{2}$ alors 
        \[
            \int_{\gamma_{1}} f = \int_{\gamma_{2}} f
        \]
        \item Si $[Z_{0}, z_{1}] \subseteq U$, nous notons $\int_{[z_{0}, z_{1}]} f(z)\d z = \int_{\gamma}f(z)\d z$ où $\gamma : t\in [0, 1] \mapsto (1 - t)z_{0} + tz_{1}$.
        \item Si $\partial D(z_{0}, r) \subseteq U$, soit le lacet $\gamma : \theta \in [0, 2\pi] \mapsto z_{0} + re^{i\theta}$. On a : 
        \[
            \int_{\gamma}f(z) \d z = \int_{\partial D(z_{0}, r)}f(z)\d z = \int_{0}^{2\pi}f(z_{0} + re^{i\theta})ire^{i\theta}\d\theta
        \]
        \item En séparant parties réelles et imaginaires, $f = P + iQ$ et $\gamma = u + iv$, on a :
        \[
            \begin{aligned}
                \int_{\gamma}f(z) \d z = & \int_{a}^{b}\left(\left(P \circ \gamma\right)u' - \left(Q \circ \gamma\right)v'\right)\d t + i \int_{a}^{b}\left(\left(Q \circ \gamma\right)u' + \left(P \circ \gamma\right)u'\right)\d t\\
                = &\int_{\gamma}\left(P\d x - Q\d y\right) + i \int_{\gamma}\left(P\d y + Q \d x\right)
            \end{aligned}
        \]
        \item On a :
        \[
            \int_{\gamma} f(z) \d z = - \int_{\gamma^{0}}f(z) \d z
        \]
        \item On a :
        \[
            \abs{\int_{\gamma}f(z)\d z} \leq long(\gamma)\max_{\gamma}\abs{f}
        \] 
    \end{itemize}
\end{propositionfr}

\subsection{Théorème de Cauchy}
\begin{théorème}{de Cauchy}{}
    Soit $U \subseteq \C$ un ouvert connexe et $f$ une fonction analytique dans $U$. Si $\gamma_{1}, \gamma_{2}$ sont deux lacets homotopes dans $U$, alors
    \[
        \int_{\gamma_{1}}f(z) \d z = \int_{\gamma_{2}}f(z)\d z
    \]
    En particulier, si $U$ est simplement connexe, l'intégrale sur un lacet de $f$ est nulle. 
\end{théorème}

\begin{théorème}{}{}
    Soit $U \subseteq \C$ un ouvert simplement connexe. 
    \begin{enumerate}
        \item Toute fonction analytique dans $U$ admet une primitive.
        \item Si $f : U \to \C^{\star}$ est analytique, alors il existe $g : U \to \C$ analytique tel que $\exp(g) = f$ sur $U$. 
    \end{enumerate}
\end{théorème}

\subsection{Formule de Cauchy}
\begin{lemme}{Intégrité de l'Indice}{}
    Soit $\gamma : I = [c, d] \to \C$ un lacet et $a\notin \gamma(I)$. Alors
    \[
        j(a, \gamma) = \frac{1}{2i\pi}\int_{\gamma}\frac{\d z}{z - a} \in \Z
    \]
\end{lemme}
\begin{proof}
    Pour $t \in [c, d]$ on pose 
    \[
        h(t) = \int_{c}^{t}\frac{\gamma'(s)\d s}{\gamma(s) - a}
    \]
    On a $h'(t) = \frac{\gamma'(t)}{\gamma(t)-a}$, sauf en un nombre fini de points de $I$. \\
    Remarquons que $g(t) = e^{-h(t)}\left(\gamma(t) - a\right)$ a pour dérivée
    \[
        g'(t)= - h'(t)e^{-h(t)}\left(\gamma(t)- a\right) + \gamma'(t)e^{-h(t)} = 0
    \]
    sauf en un nombre fini de points de $I$. Comme $g$ est continue, elle est constante et $g(c) = g(d)$. \\
    Or, $h(c) = 0$ donc $g(c) = \gamma(c) - a = g(d) = e^{-h(d)}(\gamma(d) - a)$. Mais $\gamma$ est un lacet, donc $\gamma(c) = \gamma(d)$. Donc $h(d) = 2in\pi$. Donc $j(a, \gamma) = n \in \Z$. 
\end{proof}
\begin{définition}{Indice}{}
    L'entier $j(a, \gamma)$ est appelé indice de $a$ par rapport au lacet $\gamma$ et s'interprète comme le nombre de fois que le lacet tourne autour de $a$ lorsque $a$ est intérieur au lacet.
\end{définition}

\begin{propositionfr}{Propriétés}{}
    \begin{enumerate}
        \item Soit $\gamma, \gamma_{1}, \gamma_{2}$ des lacets de même origine dont les lacets ne contiennent pas $a$. Alors,
        \[
            j(a, \gamma^{0}) = -j(a, \gamma) \text{ et } j(a, \gamma_{1} \land \gamma_{2}) = j(a, \gamma_{1}) + j(a, \gamma_{2})
        \]
        \item En appliquant le théorème de Cauchy à la fonction analytique $1/(z - a)$ dans $\C - \{a\}$, nous obtenons $j(a, \gamma_{1}) = j(a, \gamma_{2})$ si $\gamma_{1}, \gamma_{2}$ sont homotopes dans $\C - \{a\}$.
        \item Soit $U \subset \C$ un ouvert simplement connexe et $\gamma \subset U$. Si $a \notin U$, alors $j(a, \gamma) = 0$.
        \item Si $\gamma$ set un lacet dans $\C$, pour tout ouvert connexe $U$ de $\C - \gamma(I)$, la fonction $z \mapsto j(z, \gamma)$ est constante dans $U$.
        \item Soit $\gamma_{n} : t \mapsto e^{int}$, on a : 
        \[
            j(z_{0}, \gamma_{n}) = \begin{cases}
                n & si \abs{z_{0}} < 1\\
                0 & si \abs{z_{0}} > 1
            \end{cases}
        \]
    \end{enumerate}
\end{propositionfr}
\begin{proof}[Démonstration du point iv.]
    Soit $z \in D(z_{0}, r) \subseteq U$, 
    \[
        j(z, \gamma) = \frac{1}{2i\pi}\int_{\gamma}\frac{\d u}{u - z} = \frac{1}{2i\pi}\int_{\gamma_{1}}\frac{\d u}{u - z} = \frac{1}{2i\pi}\int_{\gamma}\frac{\d u}{u - z_{0}} = j(z_{0}, \gamma)
    \]
    pour $\gamma_{1} : t \mapsto \gamma(t) + (z - z_{0})$ qui est homotopie à $\gamma$ via 
    \[
        \phi(t, s) = \gamma(t) + s(z - z_{0}), 0 \leq s \leq 1
    \]
    Donc $j(\cdot, \gamma)$ est localement constante donc constante sur $U$ connexe. 
\end{proof}

\begin{théorème}{Formule de Cauchy}{}
    Soit $U \subseteq \C$ un ouvert simplement connexe, $\gamma : I \to U$ un lacet dans $U$. Soit $f$ analytique sur $U$. Pour tout $w \in U \setminus \gamma(I)$
    \[
        j(w, \gamma)f(w) = \frac{1}{2i\pi}\int_{\gamma}\frac{f(z)}{z - w}\d z
    \]
\end{théorème}
\begin{proof}
    La fonction
    \[
        g : z \in U \mapsto \begin{cases}
            \frac{f(z)-f(w)}{z-w} & \text{ si } z \neq w\\
            f'(w) & \text{ si } z = w
        \end{cases}
    \]
    est analytique sur $U$. En effet pour $r > 0$ assez petit, $f$ admet un développement de Taylor sur $D(w, r) \subseteq U$ et donc pour $z \in D(w, r)$ :
    \[
        g(z) = f'(w) + \frac{f''(w)}{2!}(z - w) + \cdot + \frac{f^{(n)}(w)}{n!}\left(z - w\right)^{n - 1} + \cdot
    \]
    Comme $U$ est simplement connexe, le théorème de Cauchy donne $\int_{\gamma} g = 0$ et comme $w \notin \gamma(I)$, $\int_{\gamma}\frac{f(z) - f(w)}{z - w}\d z = 0$ c'est à dire : 
    \[
        \int_{\gamma}\frac{f(z)\d z}{z - w} = f(w)\int_{\gamma}\frac{\d z}{z - w} = 2i\pi j(w, \gamma)f(w)
    \]
\end{proof}

\begin{corollaire}{Valeur en un point}{}
    On a : 
    \[
        f(w) = \frac{1}{2i\pi}\int_{\partial D(z_{0}, r)}\frac{f(z)}{z-w}\d z, w \in D(z_{0}, r)
    \]
\end{corollaire}

\begin{propositionfr}{Continuité sur un Lacet}{cont_lacet}
    Soit $\gamma : I = [c, d] \to \C$ un lacet et $g : \gamma(I) \to \C$ une fonction définie et continue sur $\gamma(I)$. Alors : 
    \[
        f(z) = \int_{\gamma}\frac{g(u)\d u}{u - z}
    \]
    est définie et analytique dans $\C \setminus \gamma(I)$.\\
    Précisément, pour tout $w \in \C \setminus \gamma(I)$ pour tout $n \in \N$ et 
    \[
        c_{n} = \int_{\gamma} \frac{g(u)\d u}{(u - w)^{n + 1}}
    \]
    nous avons un développement en série entière convergente 
    \[
        f(z) = \sum_{n \geq 0}c_{n}\left(z - w\right)^{n}
    \]
    dans tout disque ouvert de centre $w$ et de rayon $r = d(w, \gamma(I))$ et 
    \[
        f^{(n)}(w) = n!c_{n} = n!\int_{\gamma}\frac{g(u)\d u}{\left(u - w\right)^{n + 1}}
    \]
\end{propositionfr}
\begin{proof}
    Pour tout $u \in \gamma(I), z \in D(w, qr), q \in [0, 1]$, la série 
    \[
        \frac{1}{u - z} = \frac{1}{u - w} \frac{1}{1 - \frac{z-w}{u-w}} = \sum_{n = 0}^{+\infty}\frac{\left(z - w\right)^{n}}{\left(u - w\right)^{n + 1}}
    \]
    est convergente. Comme $\left(g \circ \gamma\right)\gamma'$ est continue par morceaux sur $[c, d]$ il existe $M$ tel que 
    \[
        \abs{g(\gamma(t))\gamma'(t)} \leq M
    \]
    Donc :
    \[
        \abs{g\left(\gamma(t)\right)\gamma'(t)\frac{(z - w)^{n}}{\left(\gamma(t) - w\right)^{n + 1}}} \leq M \frac{q^{n}}{r}, t\in [c, d]
    \]
    Finalement, la série sous l'intégrale est normalement convergente et : 
    \[
        f(z) = \int_{c}^{d}\frac{g(\gamma(t))\gamma'(t)\d t}{\gamma(t) - z} = \int_{c}^{d}g(\gamma(t))\gamma'(t)\left(\sum_{n = 0}^{+\infty}\frac{(z - w)^{n}}{\left(\gamma(t) - w\right)^{n + 1}} \right)\d t
    \]
    et donc $f(z) = \sum_{n = 0}^{+ \infty} c_{n}(z - w)^{n}$
\end{proof}

\begin{propositionfr}{Dérivée $n$-ième}{}
    Soit $f$ analytique sur $U$ et $\gamma$ le bord de $\overline{D}(w, r) \subseteq U$. D'après la formule de Cauchy : 
    \[
        f^{(n)}(w) = \frac{n!}{2\pi r^{n}}\int_{0}^{2\pi}\frac{f(w + re^{it})}{e^{nit}}\d t
    \]
\end{propositionfr}

\begin{corollaire}{}{}\label{taylor_ser}
    \begin{enumerate}
        \item Soit $f$ analytique sur $U$. Pour tout $a \in U$, la série de Taylor de $f$ au voisinage de $a$ est convergente et a pour somme $f(z)$ dans le plus grand disque ouvert de centre $a$ contenu dans $U$
        \item Si $f$ est analytique sur $\C$, sa série de Taylor en tout point de $\C$ est convergente sur $\C$. 
    \end{enumerate}
\end{corollaire}
\begin{proof}
    On applique la formule de Cauchy sur le contour $\gamma$ d'un disque $D(a, r)$ contenu dans $U$. Pour $z \in D(a, r), j(z, \gamma) = 1$ et 
    \[
        f(z) = \frac{1}{2i\pi}\int_{\gamma}^{}\frac{f(w)}{w -z}\d z
    \]
    La proposition \ref{propo:cont_lacet} donne un développement en série entière de $f$ en $z - a$ convergeant sur $D(a, r)$. Par unicité du développement, il s'agit de la série de Taylor. En faisant tendre $r$ vers $d(a, \C - U)$, nous obtenons le résultat annoncé. 
\end{proof}

\begin{corollaire}{Constance Locale}{}\label{prop:constance_locale}
    Supposons $U$ connexe, $a \in U$ et $f : U \to \C$ analytique. Si pour tout $k > 0, f^{(k)}(a) = 0$, alors $f$ est constante sur $U$. 
\end{corollaire}
\begin{proof}
    D'après le corollaire \ref{prop:taylor_ser}, $f$ est localement somme de sa série de Taylor. Donc $f$ est constante sur un ouvert contenant $a$. Soit $\Omega = \{w \in U, \forall k > 0, f^{(k)}(w) = 0\}$. Cet ensemble est ouvert, non vide, et fermé. Par connexité de $U$, $\Omega = U$, $f' = 0$ sur $U$ et $f$ est constante sur $U$.
\end{proof}

\begin{théorème}{Multiplicité}{multiplicité}
    Soit $f : U \to \C$ analytique non constante au voisinage de $a \in U$. Si $f(a) = 0$, il existe un unique entier $m \geq 1$ et $g : V \to \C$ analytique sur un voisinage $V$ de $a$ tels que 
    \[
        f(z) = \left(z - a\right)^{m}g(z), g(a) \neq 0, z\in V
    \]
    En particulier, le point $a$ possède un voisinage dans lequel il est l'unique zéro de $f$.
\end{théorème}
\begin{proof}
    D'après le corollaire \ref{prop:constance_locale}, si $f$ n'est pas constante dans un voisinage de $a$, il existe $m \geq 1$ tel $f^{(m)}(a) \neq 0$ et $f'(a) = \ldots = f^{(m-1)}(a) = 0$.\\
    Comme $f(a) = 0$, on peut alors factoriser $(z -a)^{m}$ dans le développement en série de Taylor de $f$ en $a$.
\end{proof}

\begin{définition}{Ordre}{}
    L'entier $m$ du théorème précédent est dit ordre de $f$ en $a$, noté $ord(f, a)$.
\end{définition}

\subsection{Inégalités de Cauchy, Premières Applications}
\begin{propositionfr}{Inégalités de Cauchy}{}
    Soit $f : U \to \C$ analytique, $\overline{D}(w, r) \subset U, r > 0$. On a, pour $n \in \N$ : 
    \[
        \abs{f^{(n)}(w)}\leq \frac{n!}{r^{n}}\sup_{z \in \partial D(w, r)}\abs{f(z)}
    \]
\end{propositionfr}
\begin{proof}
    On a :
    \[
        f^{(n)}(w) = \frac{n!}{2\pi r^{n}}\int_{0}^{2\pi}\frac{f(w + re^{it})}{e^{nit}} \d t
    \]
    On en déduit immédiatement le résultat.
\end{proof}
\begin{lemme}{Bornitude et Polynomialité}{}
    Soit $f$ analytique sur $\C$. Supposons qu'il existe $A, B \geq 0$ tels que 
    \[
        \forall z \in \C, \abs{f(z)}\leq A\left(1 + \abs{z}\right)^{B}
    \] 
    Alors $f$ est un polynôme de degré $\leq B$. 
\end{lemme}
\begin{proof}
    Soit $n \geq \lfloor B \rfloor + 1 > B$. Par les inégalités de Cauchy, puisque 
    \[
        \sup_{\partial D(z, r)} \abs{f(z)} \leq A\left(1 + \abs{z} + r\right)^{B}
    \]
    on a : 
    \[
        \abs{f^{(n)}(z)} \leq \frac{n!}{r^{n}}A(1 + \abs{z} + r)^{B}
    \]
    En faisant tendre $r$ vers $+\infty$, par croissance comparée, $f^{(n)}(w) = 0$ pour $n \geq B$. Localement, $f$ étant somme de sa série de Taylor, c'est localement un polynôme de degré au plus $B$, ce qui est donc le résultat.
\end{proof}

\begin{théorème}{Liouville}{}
    Une fonction analytique bornée sur $\C$ est constante.
\end{théorème}

\begin{théorème}{d'Alembert-Gauss}{}
    Tout polynôme $P \in \C[z]$ de degré $\geq 1$ admet une racine dans $\C$.
\end{théorème}
\begin{proof}
    Par l'absurde, si $P(z) = \sum_{i = 0}^{d}a_{i}z^{i}$ ne s'annule pas, $f = 1/P$ est analytique sur $\C$ et $\abs{f(z)} \sim \frac{1}{\abs{a_{d}}\abs{z}^{d}}$ tend vers $0$ quand $\abs{z}$ tend vers $+\infty$. En particulier, $f$ est bornée sur $\C$ donc constante d'après le théorème de Liouville. Ainsi, $P = 1/f$ est constant, ce uqui est absurde.
\end{proof}

\begin{théorème}{Topologie}{}
    Les ouverts $\C$ et $D(0, 1)$ sont homéomorphes mais pas isomorphes. 
\end{théorème}


\section{Fonctions Holomorphes}
\subsection{Définitions}
\begin{définition}{Holomorphie}{}
    Une fonction $f : U \to \C$ est dite holomorphe en $z_{0} \in U$ si la limite \[\lim_{h \in \C \to 0} \frac{f(z_{0} + h) - f(z_{0})}{h}\] existe. On la note $f'(z_{0})$.\\
    On définit $\O(U)$ l'ensemble des fonctions holomorphes.
\end{définition}

\begin{propositionfr}{Exemples Holomorphe}{}
    \begin{itemize}
        \item Si $f$ est constante, $f$ est holomorphe et $f' = 0$
        \item Si $f$ est un polynôme, $f$ est holomorphe
        \item Si $f$ est analytique, $f$ est holomorphe
        \item $\sin, \cos, \exp, \tan$ sont holomorphes sur $\C$.
        \item $z \mapsto \bar{z}$ n'est pas holomorphes en aucun point : 
        \[
            \frac{f(z + h) - f(z)}{h} = \frac{\bar{h}}{h}
        \]
        n'a pas de limite en $0$. 
        \item $f(z) = \abs{z}^{2}$ n'est holomorphe que pour $z = 0$ : 
        \[
            \frac{\left(z + h\right)\left(\bar{z} + \bar{h}\right) - z\bar{z}}{h} = \frac{h\bar{z} + \bar{h}z + h\bar{h}}{h}
        \]
        n'a une limite que si $z = 0$. 
    \end{itemize}
\end{propositionfr}

\subsection{$\R$-différentiabilité}
\begin{définition}{Forme Différentielle}{}
    Une $1$-forme différentielle sur $\Omega$ est une application $\alpha : \Omega \to Hom_{\R}\left(\R^{n}, \C\right)$.\\
    En particulier, les $\d x_{i} \in Hom_{\R}\left(\R^{n}, \C\right)$ qui à $a \mapsto \d x_{i}(a) = a_{i}$ permettent d'écrire : 
    \[
        \alpha(x) = \sum_{i = 1}^{n} \alpha_{i}(x)\d x_{i}
    \]
    où $\alpha_{i} : \Omega \to \C$. On a alors : 
    \[
        \alpha(x)(a) = \sum_{i = 1}^{n}\alpha_{i}(x)\d x_{i}(a)
    \]
    On dit que $\alpha$ est de classe $\cont^{k}$ si et seulement si tous les $\alpha_{i}$ sont de classe $\cont^{k}$.
\end{définition}

\begin{définition}{$\R$ Différentiabilité}{}
	Une fonction $f$ d'un ouvert connexe $\Omega$ de $\R^{n}$ est dite $\R$-différentiable sur $ \Omega$ si et seulement si il existe une $1$-forme différentielle $\d f : \Omega \to Hom_{\R}\left(\R^{n}, \C\right)$ telle que 
	\[
		f(z + h) = f(z) + \d f(z)(h) + o(h)
	\]
	On pose $\d_{x}f = \sum_{i = 1}^{n}\frac{\partial f}{\partial x_{i}}(x)\d x_{i}$.
\end{définition}

Dans la suite on travaille dans $\C \simeq \R^{2}$ et pour $h \in \C$, on note $h = k + il = (k, l)$ et pour $z \in U = \Omega, z = x+ iy = (x, y)$.

\begin{propositionfr}{Différentielle dans une base}{}
	Soit $f : U \to \C$ différentiable de différentielle $\d f = \frac{\partial f}{\partial x}\d x + \frac{\partial f}{\partial y} \d y$. On a $\forall z \in U$ : 
	\[	
		\d_{z}f = \partialderiv{f}{x}(z)\d x + \partialderiv{f}{y}(z) \d y
	\]
	En $h = k + il$ : 
	\[
		\d_{z}f(h) = \partialderiv{f}{x}(z)k + \partialderiv{f}{y}(z)l
	\]
	On définit 
	\[
		\d z = \d x + i\d y \text{ et } \d\bar{z} = \d x - i\d y
	\]
	On a alors : 
	\[	
		\d f = \frac{1}{2}\left(\partialderiv{f}{x} - i\partialderiv{f}{y}\right)\d z + \frac{1}{2}\left(\partialderiv{f}{x} + i\partialderiv{f}{y}\right)\d\bar{z}
	\]
	ce qu'on écrit aussi : 
	\[	
		\d f = \partialderiv{f}{z}\d z + \partialderiv{f}{\bar{z}}\d\bar{z}
	\]
	On a par ailleurs 
	\[
		\overline{\left(\partialderiv{f}{z}\right)} = \partialderiv{\bar{f}}{\bar{z}}
	\]
\end{propositionfr}

\begin{propositionfr}{Exemples}{}
	\begin{enumerate}
		\item Si $f(z) = z$, $\partialderiv{z}{z} = \partialderiv{f}{z} = 1$ et $\partialderiv{z}{\bar{z}} = 0$. A l'inverse, $\partialderiv{\bar{z}}{z} = 0$.
		\item Pour $P(x, y) = \sum_{0 \leq \alpha, \beta\leq d}c_{\alpha, \beta}x^{\alpha}y^{\beta}$. En notant $x = \frac{z + \bar{z}}{2}$ et $y = \frac{z - \bar{z}}{2i}$, on a : 
			\[
				P(z) = \sum_{\alpha, \beta}a_{\alpha, \beta}z^{\alpha}\bar{z}^{\beta}
			\]
			où on a 
			\[	
				a_{\alpha, \beta} = \frac{1}{\alpha!\beta!}\frac{\partial^{\alpha + \beta}}{\partial z^{\alpha}\partial z^{\beta}}P(0)
			\]
			On retrouve que $P$ est holomorphe si on a $a_{\alpha, \beta} = 0$ pour $\beta \geq 1$. 
	\end{enumerate}
\end{propositionfr}

\begin{théorème}{Lien $\C$-dérivabilité et $\R$-différentiabilité}{}
	Soit $f : U \to \C$. On a équivalence entre : 
	\begin{enumerate}
		\item $f \in \O(U)$
		\item $f$ est $\R$-différentiable sur $U$ et $\d_{z} f$ est $\C$-linéaire pour tout $z \in U$
		\item $f$ est $\R$-différentiable sur $U$ et $\partialderiv{f}{\bar{z}} = 0$ pour tout $z \in U$
	\end{enumerate}
\end{théorème}

\begin{proof}
	\begin{description}
		\item[$i \Rightarrow ii$] $f(z + h) = f(z) + hf'(z) + o(h) \Longrightarrow f$ est $\R$-différentiable en $z$ et $\d_{z}f : \R^{2}\to \C$ qui à $h \mapsto hf'(z)$ est $\C$-linéaire.
		\item[$ii \Rightarrow iii$] On a : 
			\[
				\begin{aligned}
					\d_{z}f(h) = & \partialderiv{f}{z}(z)\d z(h) + \partialderiv{f}{\bar{z}}(z)\d \bar{z}(h)\\
					= & \partialderiv{f}{z}(z)h + \partialderiv{f}{\bar{z}}(z)\bar{h}\\
					\d_{z}f(h) =& \partialderiv{f}{z}h + \partialderiv{f}{\bar{z}}\bar{h}
				\end{aligned}
			\]
			On a alors : $\d_{z}f(ih) = \partialderiv{f}{z}ih -i\partialderiv{f}{\bar{z}}\bar{h}$. Mais $\d_{z}f$ est $\C$-linéaire par hypothèse. Donc : 
			\[
				\d_{z}f(ih) = i\d_{z}f(h) = i\partialderiv{f}{z}h + i\partialderiv{f}{\bar{z}}\bar{h}
			\]
			Ainsi : $\partialderiv{f}{\bar{z}} = - \partialderiv{f}{\bar{z}} = 0$.
		\item[$iii \Rightarrow i$] On a : 
			\[
				\d_{z}f(h) = \partialderiv{f}{z}h
			\]
			D'où 
			\[
				f(z + h) = f(z) + \partialderiv{f}{z}h + o(h)
			\]
			Ainsi : 
			\[	
				\lim_{h \to 0}\frac{f(z + h) - f(z)}{h} = \partialderiv{f}{z}
			\]
			et $f$ est holomorphe en $z$. 
	\end{description}
\end{proof}

\begin{propositionfr}{Équations de Cauchy-Riemann}{}
	On note $f(x +iy) = P(x, y) + iQ(x, y)$ où $P, Q : \R^{2} \to \R$. Si $f$ est holomorphe, on a :  
	\[	
		\partialderiv{f}{x} + i\partialderiv{f}{y} = 0
	\]
	i.e. 
	\[
		\begin{cases}
			\partialderiv{P}{x} - \partialderiv{Q}{y} =& 0\\
			\partialderiv{Q}{x} + \partialderiv{P}{y} =& 0
		\end{cases}
	\]
	Ce sont les équations de Cauchy-Riemann.
\end{propositionfr}
\begin{proof}
	On a : 
	\[
		\d_{z}f(h) = \partialderiv{f}{x}(z)\d x(h) + \partialderiv{f}{y}(z)\d y(h) = f'(z)h = f'(z)(k + il)
	\]
	On obtient 
	\[	
		f'(z)(k + il) = \partialderiv{f}{x}(z)k + \partialderiv{f}{y}(z)l
	\]
	et donc : 
	\[
		\begin{cases}
			f'(z) = \partialderiv{f}{x}(z)&\\
			if'(z) = \partialderiv{f}{y}(z)
		\end{cases}
	\]
	On obtient ainsi la première égalité en identifiant.\\
	On réécrit ceci avec $\partialderiv{f}{\bar{z}} = 0$ : 
	\[
		\begin{cases}
			\partialderiv{P}{x} - \partialderiv{Q}{y} =& 0\\
			\partialderiv{Q}{x} + \partialderiv{P}{y} =& 0
		\end{cases}
	\]
\end{proof}

\begin{propositionfr}{Constance sur un Connexe}{}
	Si $f$ est holomorphe sur $U$ connexe on a équivalence entre : 
	\begin{itemize}
		\item $f$ est constante sur $U$
		\item $\Re f$ l'est
		\item $\Im f$ l'est
		\item $\abs{f}$ l'est
		\item $\bar{f}$ est holomorphe
	\end{itemize}
\end{propositionfr}

\subsection{Intégrale sur le bord d'un Compact}
\begin{définition}{Classe du Bord d'un Compact}{}
	Soit $K$ un compact de $\R^{2}$. $K$ est dit à bord compact de classe $\cont^{1}$ par morceaux si pour tout élément $z_{0} \in \partial K$, il existe des coordonnées $(u, v)$ associées à un repère affine de $\R^{2}$ d'origine $z_{0}$ orienté positivement par rapport à l'orientation canonique de $\R^{2}$ et un rectangle ouvert $R = \left\{ -\delta < u < \delta\right\} \times \left\{-\eta < v < \eta\right\}$ tel que : 
	\[	
		K \cap R = \left\{(u, v) \in R, v \geq h(u)\right\}
		\]
	où $h$ est une fonction réelle $\cont^{1}$ par morceaux sur $\left[-\delta, \delta \right]$ avec $h(0) = 0$ et $\sup \abs{h} < \eta$.
\end{définition}

\begin{définition}{Orientation du Bord}{}
	Soit $K$ un compact à bord de classe $\cont^{1}$ par morceaux. On appelle orientation canonique du bord l'orientation donnée par les arcs $u \mapsto (u, h(u))$ avec $u$ croissant.
\end{définition}

\begin{lemme}{Existence de l'Orientation}{lemme:exist-orient}
	La définition a du sens.
\end{lemme}
\begin{lemme}{Recoupement de Rectangles}{}
	Soit $R, R'$ des rectangles ouverts tels que $\partial K \cap R \cap R' \neq \emptyset$.\\
	On définit 
	\[
		K \cap R = \left\{(u, v) \in R, v\geq h(u)\right\}
	\]
	et 
	\[
		K \cap R' ) \left\{(u', v')\in R', v' \geq l(u')\right\}
	\]
	Alors, les orientations sur $\partial K \cap R \cap R'$ coïncident.
\end{lemme}
\begin{proof}
	Soit $z_{0}  \in \partial K \cap R \cap R'$. $h$ et $l$ sont $\cont^{1}$ par morceaux. 
	En évitant un nombre fini de points de $\partial K \cap R\cap R'$ on peut supposer $h$ et $l$ $\cont^{1}$ en $z_{0}$. 
	Autrement dit, le bord admet une tangente en $z_{0}$. On a deux repères affines orientés $(z_{0}, e_{1}, e_{2})$ et $(z_{0}, e_{1}', e_{2}')$ qui génèrent des coordonnées $(u, v)$ et $(u', v')$.
	Quitte à remplacer $h$ par $h(u) - h'(0)u$ on peut supposer que $h'(0) = l'(0) = 0$. \\
	Ainsi, $e_{1}$ et $e_{1}'$ sont colinéaires. Puisqu'on a supposé que $(e_{1}, e_{2})$ et $(e_{1}', e_{2}')$ sont orientés positivement par rapport à l'orientation canonique de $\R^{2}$ et puisque $e_{2}$ et $e_{2}'$ doivent être dans le même sens (i.e. à l'intérieur du compact), on a bien le fait que $e_{1}$ et $e_{1}'$ sont dans le même sens.\\
	Finalement, les orientations sur $\partial K \cap R \cap R'$ coïncident. 
\end{proof}

\subsection{Formule de Green-Riemann}
Soit $p, n \in \N^{*}$. On note $\Lambda_{\R}^{p}\left(\R^{n}\right)$ le $\R$-ev des formes $p$-linéaires alternées sur $\R^{n}$. Toute forme $S \in \Lambda_{\R}^{p}(\R^{n})$ s'écrit de manière unique 
\[
    S = \sum_{1 \leq i_{1} < \cdots <i_{p} \leq n}c_{i_{1},\ldots, i_{p}}\d x_{i_{1}} \land \cdots \land \d x_{i_{p}}
\]
\begin{définition}{Produit Extérieur}{}
    Pour $S \in \Lambda_{\R}^{p}(\R^{n}), T\in \Lambda_{\R}^{q}\left(\R^{n}\right)$ on définit le produit extérieur de $S$ et $T$ noté $S \land T \in \Lambda_{\R}^{p + q}\left(\R^{n}\right)$ comme : 
    \[
        S\land T\left(v_{1}, \ldots, v_{p + q}\right) = \frac{1}{p!q!}\sum_{\sigma}\sgn(\sigma)S(v_{\sigma(1)},\ldots, v_{\sigma(p)})T(v_{\sigma(p + 1)}, \ldots, v_{\sigma(p + q)})
    \]
\end{définition}

\begin{propositionfr}{Exemples}{}
   La paire $(\d x, \d y)$ forme une base de $\Lambda_{\R}^{1}(\C)$. Les seuls produits extérieurs à considérer sont : 
        \[
            \d x \land \d x = \d y \land \d y = 0, \ \ \d x \land \d y = -\d y \land \d x
        \]
    De plus, $\d x \land \d y$ est la forme bilinéaire alternée déterminant dans la base canonique. 
\end{propositionfr}

\begin{définition}{$2$-forme différentielle}{}
    Une $2$-forme différentielle $\beta$ sur un ouvert $U$ de $\C$ est une application continue de $U$ dans $\Lambda_{\R}^{2}(\C)$ : $\beta = w(x, y)\d x \land \d y$ pour $w$ continue. 
\end{définition}

\begin{définition}{Intégrale d'une $2$-forme}{}
    Soit $\beta = w(x, y)\d x \land \d y$ une $2$-forme différentielle sur $U$. On définit : 
    \[
        \int_{U}\beta = \int_{U}w(x, y)\d x\d y
    \]
\end{définition}
\begin{définition}{Différentielle d'une Différentielle}{}
    Soit $\alpha = u(x, y)\d x + v(x, y)\d y$ une $1$-forme différentielle $\cont^{1}$ sur $U$. La différentielle $\d \alpha$ de $\alpha$ est la $2$-forme différentielle 
    \[
        \d \alpha = \d u\land \d x + \d v \land \d y = \left(\partialderiv{v}{x} - \partialderiv{u}{y}\right)\d x \land \d y 
    \]
\end{définition}

\begin{propositionfr}{Exemples}{}
    \begin{enumerate}
        \item Soit $\alpha$ une $1$-forme différentielle sur $U$ et $f : U\to \C$ de classe $\cont^{1}$. Alors 
        \[
            \d(f\alpha) = \d f \land \alpha + f\d \alpha
        \]
        \item Si on écrit la $1$-forme différentielle $\cont^{1}$ $\alpha = f\d z + g\d \bar{z}$ on a : 
        \[
            \d(f\d z + g \d\bar{z}) = \left(\partial_{z}g - \partial_{\bar{z}}f\right) \d z \land \d\bar{z} = -2i\left(\partial_{z}-\partial_{\bar{z}}f\right)\d x \land \d y
        \]
        En particulier, si $\alpha = f\d z$ avec $f$ holomorphe, alors $\partial_{\bar{z}}f = 0$ et $\d \alpha = 0$. 
    \end{enumerate}
\end{propositionfr}

\begin{lemme}{Formule de Green-Riemann sur un rectangle}{Green-Riemann}
    Soit $K$ un compact de $\C$ à bord de classe $\cont^{1}$ par morceaux orienté canoniquement. Soit $\alpha = u(x, y)\d x + v(x, y)\d y$ une forme $1$-différentielle de classe $\cont^{1}$ sur un ouvert de $K$ à support dans un rectangle $R = [- \delta, \delta] \times [-\eta, \eta] \subseteq K$. Alors 
    \[
        \int_{\partial K} \alpha = \int_{K}\d \alpha \text{ i.e. } \int_{\partial K}u(x, y)\d x + v(x, y) \d y = \int_{K}\left(\partialderiv{v}{x} - \partialderiv{u}{y}\right)\d x\d y
    \]
\end{lemme}
\begin{proof}
    Comme $K$ est compact, $R\subseteq \mathring{K}$ ou bien $\partial{K \cap R}$ est le graphe d'une fonction $h$ et $K \cap R$ est la partie située à l'intérieur du graphe.\\
    Supposons $R \subseteq \mathring{K}$, alors $u = v = 0$ sur $\partial R$ et 
    \[
        \int_{-\delta}^{\delta}\partialderiv{v}{x}(x, y)\d x = v(\delta, y) - v(-\delta, y) = 0
    \]
    \[
        \int_{-\eta}^{\eta}\partialderiv{u}{y}(x, y)\d y = u(x, \eta) - u(x, -\eta) = 0  
    \]
    et 
    \[
        \int_{K}\d \alpha = \int_{R}\d \alpha = \int_{-\delta \leq x \leq \delta, -\eta \leq y \leq \eta} \left(\partialderiv{v}{x} - \partialderiv{u}{y}\right)\d x\d y = 0
    \]
    Puisque le support de $\alpha$ ne rencontre pas $\partial K$ on a $\int_{\partial K} \alpha = 0 = \int_{K}\d \alpha$. \\
    Si $K \cap R = \{(x, y) \in R \mid y \leq h(x)\}$, alors 
    \begin{gather*}
        \int_{K}\d \alpha = \int_{K \cap R}\d \alpha = \int_{- \delta}^{\delta}\d x \int_{h(x)}^{\eta}\left(\partialderiv{v}{x} - \partialderiv{u}{y}\right)\d y\\
        = \int_{- \delta}^{\delta}\left(\frac{\partial}{\partial x}\left(\int_{h(x)}^{\eta}v(x, y)\d x\right)+ v(x, h(x))h'(x) - \left(u(x, \eta) - u(x, h(x))\right)\right) \d x\\
        = \int_{h(\delta), \eta}v(\delta, y)\d y - \int_{h(-\delta)}^{\eta} v(-\delta, y)\d y + \int_{-\delta}^{\delta} \left(v(x, h(x))h'(x) + u(x, h(x))\right)\d x\\
        = \int_{-\delta}^{\delta} \left(v(x, h(x))h'(x) + u(x, h(x))\right)\d x
    \end{gather*}
    car $u(x, \eta) = v(\delta, y) = v(-\delta, y) = 0$. Par ailleurs, comme $\partial K \cap R$ est paramétré par $y = h(x)$,
    \[
        \int_{\partial{K}\cap R}\alpha = \int_{\partial K \cap R} u(x, y)\d x + v(x, y)\d y = \int_{- \delta}^{\delta} \left(v(x, h(x))h'(x) + u(x, h(x))\right)\d x
    \]
    Le support de $\alpha$ est inclus dans $R$. Nous concluons donc :
    \[
        \int_{K}\d \alpha = \int_{\partial K} \alpha 
    \]
\end{proof}

\begin{définition}{Partition de l'unité}{}
    Soit $K$ un compact de $\C$ recouvert par un nombre fini d'ouverts $U_{i}$. Une partition de l'unité de classe $\cont^{1}$ subordonnée au recouvrement $U_{i}$ est une famille $\phi_{i}$ de fonctions de $K$ dans $[0, 1]$ de classe $\cont^{1}$ à support dans $U_{i}$ telles que $\sum \phi_{i}(x) = 1$ pour tout $x \in K$. 
\end{définition}

\begin{lemme}{Unité sur un Voisinage}{unit_voisin}
    Soit $z \in U$. Il existe $V$ un voisinage de $z$ avec $\overline{V}\subseteq U$ et une fonction $\cont^{1}$ $\phi_{U}$ à support dans $U$ valant $1$ sur $V$. 
\end{lemme}
\begin{proof}
    Soit $r > r'>0$ tel que $D(z, r') \subset D(z, r) \subset U$. On définit les fonctions $\cont^{\infty}$ 
    \[
        f_{r} : \R \to \R, f_{r}(t) = \begin{cases}
            e^{\frac{1}{t^{2}-r^{2}}} & \text{ si } \abs{t} < r\\
            0 & \text{sinon}
        \end{cases}
    \]
    et 
    \[
        g_{r} : \R \to \R, g_{r}(s) = \frac{\int_{-\infty}^{s}f_{r}(t)\d t}{\int_{- \infty}^{\infty}f_{r}(t)\d t}
    \]
    En particulier : 
    \[
        g_{r}(s) = \begin{cases}
            0 & \text{si } s \leq -r\\
            1 & \text{si } s \geq r
        \end{cases}
    \]
    Alors, $V = D(z, r')$ et $\phi_{U}(w) = f_{r}\left(r + \frac{2r}{r-r'}\left(r'-\abs{w - z}\right)\right)$ conviennent. 
\end{proof}

\begin{lemme}{Existence d'une Partition}{}
    Soit $K \subseteq \C$ un compact et $\left(U_{i}\right)$ un recouvrement fini par des ouverts de $K$. Il existe une partition de l'unité $\cont^{1}$ subordonnée au recouvrement $U_{i}$
\end{lemme}
\begin{proof}
    Pour tout $z \in K \setminus U_{j}$ il existe $i$ tel que $z \in U_{i}$. Par le lemme \ref{lemme:unit_voisin} on constuit $\psi_{z}^{j}$ de classe $\cont^{1}$ qui vaut $1$ sur un voisinage ouvert $W_{z}^{j}$ de $z$ et dont le support est dans l'ouvert $U_{i} \cap \left(K \setminus U_{j}\right)$. Le support de $\psi_{z}^{j}$ est un fermé de $K$ donc est compact. \\
    On obtient donc un recouvrement ouvert $W_{z}^{j}$ du compact $K \setminus U_{j}$ donc on extrait un sous-recouvrement fini $\left\{W_{z_{1}}^{j}, \ldots, W_{z_{j_{l}}}^{j}\right\}$. \\
    On procède de même pour tout $j \leq n$. En réindexant on obtient une famille finie $\left(\psi_{l}\right)_{1 \leq l \leq N}$ de fonctions dont l'union des supports recouvre $K$, i.e. pour tout $z \in K$, il existe $l$ tel que $\psi_{l}(z) > 0$. On pose alors 
    \[
        \psi = \sum \psi_{l} \text{ et pour } l\leq N, \rho_{l} = \frac{\psi_{l}}{\psi}
    \]
    Ainsi, $\rho_{l}$ est une partition de l'unité de classe $\cont^{1}$ de $K$ telle que pour tout $l$ il existe $1 \leq i \leq n$ tel que le support de $\rho_{l}$ soit inclus dans $U_{i}$.
\end{proof}

\begin{théorème}{Formule de Green-Riemann}{Green-Riemann}
    Soit $K$ un compact de $\C$ à bord de classe $\cont^{1}$ par morceaux orienté canoniquement. Soit $\alpha$ une $1$-forme différentielle de classe $\cont^{1}$ sur un ouvert de $K$. On a alors 
    \[
        \int_{\partial K}\alpha = \int_{K}\d \alpha
    \]
\end{théorème}
\begin{proof}
    Comme $K$ est compact, il est recouvert par un nombre fini de rectangles ouverts $R_{j}$ qui vérifient $R_{j} \subseteq \mathring{K}$ ou $\partial\left(K \cap R_{j}\right)$ est le graphe d'une fonction $h_{j}$ et $K\cap R_{j}$ est la partie située à l'intérieur du graphe. Soit $\left(\chi_{j}\right)$ une partition de l'unité subordonnée au recouvrement $R_{j}$. Écrivons $\alpha = \sum \alpha_{j}$ où les $1$-formes différentielles $\alpha_{j} = \chi_{j}\alpha$ sont de classes $\cont^{1}$ à support dans $R_{j}$. On se ramène alors au cas du lemme \ref{lemme:Green-Riemann}
\end{proof}

\begin{théorème}{Cauchy}{}
    Soit $U$ un ouvert de $\C$, $K$ un compact à bord de classe $\cont^{1}$ par morceaux inclus dans $U$, avec l'orientation canonique du bord. Alors pour toute fonction holomorphe de classe $\cont^{1}$ sur $K$ nous avons 
    \[
        \int_{\partial K}f(z)\d z = 0
    \]
\end{théorème}
\begin{proof}
    On applique la formule de Green-Riemann \ref{thm:Green-Riemann} à $\alpha = f(z)\d z$, $1$-forme différentielle de classe $\cont^{1}$. On a $\d \alpha = -\partial_{\bar{z}}f\d z \land \d \bar{z} = 0$.
\end{proof}
\begin{corollaire}{Analycité Holomorphe $\cont^{1}$}{}
    Soit $f$ holomorphe de classe $\cont^{1}$ sur un ouvert $U$. Alors $f$ est analytique sur $U$. 
\end{corollaire}
\begin{proof}
    Soit $\overline{D}(w, r) \subseteq U$ et $\gamma$ le lacet $t \mapsto w + re^{it}$. Pour $\lambda \leq 1$, on pose
    \[
        g(\lambda) = \frac{1}{2i\pi}\int_{\gamma}\frac{f(z + \lambda(u - z))}{u - z} \d u = \frac{r}{2\pi}\int_{0}^{2\pi}\frac{f\left(z +\lambda\left(w + re^{it} - z\right)\right)}{w + re^{it} - z}e^{it}\d t
    \]
    Ainsi, $g$ est continue sur $[0, 1]$, dérivable sur $]0, 1[$ de dérivée 
    \[
        g'(\lambda) = \frac{r}{2\pi}\int_{0}^{2\pi}f'\left(z + \lambda\left(w + re^{it} - z\right)\right)e^{it}\d t = \left[\frac{1}{2i\pi\lambda}f\left(z + \lambda\left(w + re^{it} - z\right)\right)\right]_{t = 0}^{2\pi} = 0
    \]
    Donc $g$ est constante avec 
    \[
        g(1) = \frac{1}{2i\pi}\int_{\gamma}^{}\frac{f(u)\d u}{u - z} \text{ et } g(0) = \frac{1}{2i\pi}\int_{\gamma}\frac{f(z)\d u}{u - z} = f(z)
    \]
    D'où, par la proposition \ref{propo:cont_lacet}, $f$ est analytique 
\end{proof}
\subsection{Analycité des Fonctions Holomorphes}
\begin{lemme}{Goursat}{Goursat}
    Soient $U \subseteq \C$ un ouvert et $T$ un triangle inclus dans $U$. Pour tout fonction holomorphe sur $U$
    \[
        \int_{\partial T}f(z)\d z = 0
    \]
\end{lemme}
\begin{proof}
    Nous décopons $T$ en quatre triangles $T_{i}$ dont les sommets sont ceux de $T$ et les milieux des côtés de $T$. Nous orientons les arêtes opposées des triangles $T_{k}$ de telle façon que 
    \[
        I = \int_{\partial T}f(z)\d z = \sum_{k = 1}^{4}\int_{\partial T_{k}}f(z)\d z
    \]
    Il existe donc un indice $k$ avec $\abs{\int_{\partial T_{k}}f(z)\d z} \geq \abs{I}/4$. De cette façon, nous construisons une suite de triangles emboîtés $T_{0}' = T, T_{1}' = T_{k}$ avec $diam T_{n}' = diam T/2^{n}$ et $\abs{\int_{\partial T_{n}'} f(z)\d z} \geq \abs{I}/4^{n}$.
	L'intersection des triangles emboîtés $T_{n}'$ est réduite à un point $z_{0}$. Comme $f$ est holomorphe en $z_{0}$ : 
	\[
		f(z) = f(z_{0}) + (z - z_{0})f'(z_{0}) + (z - z_{0})\epsilon(z)
	\]
	avec $\epsilon(z)$ qui tend vers $0$ quand $z$ tend vers $z_{0}$.  
	On a ainsi :
	\[
		\abs{\int_{\partial T_{n}'} f(z)\d z} = \abs{\int_{\partial T_{n}'} (z - z_{0}) \epsilon(z) \d z} \leq {\rm long}(\partial T_{n}') \sup_{\partial T_{n}'} \abs{z - z_{0}}\abs{\epsilon(z)}
	\]
	et donc 
	\[
		\abs{\int_{\partial T_{n}'} f(z)\d z}\leq 3 \left({\rm diam}T_{n}'\right)^{2}\sup_{\partial T_{n}'}\abs{\epsilon(z)}
	\]
	Donc $\abs{I} \leq 4^{n}\abs{\int_{\partial T_{n}'} f(z) \d z} \leq 3 \left({\rm diam}T_{n}\right)^{2}\sup_{\partial T_{n}'}\abs{\epsilon(z)}$ et donc $I = 0$.
\end{proof}

\begin{théorème}{Goursat}{Goursat}
	Soit $U \subseteq \C$ ouvert et $K$ un compact à bord de classe $\cont^{1}$ avec l'orientation canonique du bord. Pour toute fonction holomorphe sur $U$ on a : 
	\[
		\int_{\partial K}f(z)\d z = 0
	\]
\end{théorème}
\begin{proof}
    On approche $K$ par des compacts à bords polygonaux. Notons $\delta = d(K, \C\setminus U) > 0$. 
    Paramétrons $\delta K$ par un nombre fini d'arcs $\cont^{1}$ par morceaux. Pour chaque tel arc $\gamma : [a, b] \to U$, soit une subdivision $a = \tau_{0} < \tau_{1} < \ldots < \tau_{n} = b$ telle que $\abs{\gamma(\tau_{j + 1}) - \gamma(\tau_{j})} \leq \epsilon \leq \delta/2$. 
    Chaque segment $[\gamma(\tau_{j + 1}), \gamma(\tau_{j})] \subset U$. 
    Pour $\epsilon$ assez petit, la réunion de ces segments constitue le bord d'un compact $K_{\epsilon}$ à bord polygonal. 
    $K_{\epsilon} = \cup_{i} T_{i}$ est réunion de triangles adjacents et le lemme de Goursat \ref{lemme:Goursat} implique 
    \[
        \int_{\partial K_{\epsilon}}f(z)\d(z) = \sum_{i}\int_{\partial T_{i}} f(z) \d z = 0
    \]
    D'après la proposition , on a bien : 
    \[
        \lim_{\epsilon \to 0}\int_{\partial K_{\epsilon}} = \int_{\partial K}f(z)\d z
    \]
    D'où le résultat. 
\end{proof}

\begin{théorème}{Formule de Cauchy}{formule_cauchy}
    Soit $f$ holomorphe sur un ouvert $U \subseteq \C$ et $K$ un compact à bord orienté $\cont^{1}$ par morceaux inclus dans $U$. Alors, pour tout $z \in K$ 
    \[
        f(z) = \frac{1}{2i\pi}\int_{\partial K}\frac{f(\omega)}{\omega -  z}\d \omega
    \]
\end{théorème}
\begin{proof}
    Soit $r > 0$ tel que $\overline{D(z, r)} \subset \mathring{K}$. On note $K_{r} = K \setminus D(z, r)$. $K_{r}$ est un compact à bord orienté $\cont^{1}$ par morceaux dont le bord est $\partial K_{r} = \partial K \cup \partial D^{-}(z, r)$ où $\partial D^{-}$ signifie que ce cercle a l'orientation opposée à celle obtenue comme bord de $\overline{D(z,r)}$. La fonction $g(\omega) = f(\omega)/\left(\omega - z\right)$ et holomorphe sur $U \setminus \{z\}$. Le théorème de Goursat \ref{thm:Goursat} appliqué à $g$ sur le compact $K_{r} \subseteq U \setminus \{z\}$ donne 
    \[
        \int_{\partial K}\frac{f(\omega)}{\omega - z}\d \omega - \int_{\partial D(z, r)} \frac{f(\omega)}{\omega - z}\d \omega = 0
    \]
    En posant $\omega = z + re^{it}$ on a : 
    \[
        \int_{\partial D(z, r)}\frac{f(\omega)}{\omega - z}\d \omega = \int_{0}^{2\pi}\frac{f(z + re^{it})}{re^{it}}ire^{it}\d t = i \int_{0}^{2\pi}f(z + re^{it}) \d t
    \]
    et cette dernière intégrale tend vers $2i\pi f(z)$ lorsque $r$ tend vers $0$ par continuité de $f$ au point $z$. 
\end{proof}
\begin{théorème}{Équivalence Holomorphie-Analycité}{equiv_holo_analy}
    Soit $f : U \to \C$. $f$ est holomorphe sur $U$ si et seulement si elle est analytique.
\end{théorème}
\begin{proof}
    On a déjà l'implication analycité holomorphie. Supposons $f$ holomorphe sur $U$ et $\overline{D}(z_{0}, r) \subset U$. Pour $z \in D(z_{0}, r)$, la formule de Cauchy \ref{thm:formule_cauchy} donne 
    \[
        f(z) ) \frac{1}{2i\pi}\int_{\partial D(z_{0}, r)}\frac{f(\omega)}{\omega - z}\d \omega
    \]
    Or 
    \[
        \frac{1}{\omega - z} = \sum_{n = 0}^{+ \infty}\frac{\left(z - z_{0}\right)^{n}}{\left(\omega - z_{0}\right)^{n + 1}}
    \]
    De plus, pour $\omega = z_{0} + re^{it}$,
    \[
        \abs{\frac{\left(z - z_{0}\right)^{n}}{\left(\omega - z_{0}\right)^{n + 1}}} = \frac{1}{r}\left(\frac{\abs{z - z_{0}}}{r}\right)^{n}, \text{ avec } \abs{z - z_{0}}/r < 1
    \]
    Par convergence normale pour $t \in [0, 2\pi]$, on obtient :
    \[
        \begin{aligned}
            f(z) &= \frac{1}{2i\pi}\int_{\partial D(z_{0},r )}f(\omega)\sum_{n = 0}^{+ \infty}\frac{\left(z - z_{0}\right)^{n}}{\left(\omega - z_{0}\right)^{n + 1}}\d \omega \\ &= \sum_{n = 0}^{+ \infty}\int_{\partial D(z_{0}, r)}\frac{f(\omega)}{2i\pi\left(\omega - z_{0}\right)^{n + 1}}\d \omega\left(z - z_{0}\right)^{n} \\ &= \sum a_{n}(z - z_{0})^{n}
        \end{aligned}
    \]
    et la série entière ci-dessus converge normalement sur les compacts de $D(z_{0}, r)$.
\end{proof}
\begin{corollaire}{Classe des Dérivées}{}
    Soit $U$ un ouvert de $\C$. Toute fonction holomorphe sur $U$ est de classe $\cont^{\infty}$ sur $U$.\\
    Précisément, pour tout $K \subset U$ compact à bord de classe $\cont^{1}$ par morceaux et pour tout $z \in \mathring{K}$ nous avons : 
    \begin{enumerate}
        \item $\forall n \geq 0, \frac{\partial^{n}f}{\partial z^{n}}(z) = f^{(n)}(z) = \frac{n!}{2i\pi}\int_{\partial K}\frac{f(\omega)}{\left(\omega - z\right)^{n + 1}}\d \omega$
        \item $\forall n \geq 0, \forall m \geq 0, \frac{\partial^{n + m}f}{\partial z^{n}\partial \bar{z}^{m}}(z) = 0$.
    \end{enumerate}
    En particulier, une fonction holomorphe $f$ admet des dérivées complexes $f^{(n)}$ d'ordre $n$ arbitraire et les dérivées $f^{(n)}$ sont holomorphes.
\end{corollaire}

\begin{théorème}{Morera}{morera}
    Soit $f$ une fonction continue sur un ouvert $U$ de $\C$. Nous supposons que $\int_{\partial T}f(z) \d z = 0$ pour tout triangle $T$ inclus dans $U$. Alors $f$ est holomorphe sur $U$.
\end{théorème}
\begin{proof}
    Soit $z_{0} \in U$ et $r > 0$ tel que $\overline{D}(z_{0}, r) \subset U$. Pour $z \in D(z_{0}, r)$, on pose 
    \[
        F(z) = \int_{[z_{0}, z]} f(\omega) \d \omega
    \]
    Soit $z \in D(z_{0}, r)$ et $h \neq 0$ tel que $z + h \in D(z_{0}, r)$. Comme le triangle de sommets $z_{0}, z, z + h$ est inclus dans $D(z_{0}, r)$, nous avons 
    \[
        \frac{F(z + h)-F(z)}{h} = \frac{1}{h}\int_{[z, z + h]}f(\omega)\d \omega = \int_{0}^{1} f(z + th)\d t
    \]
    Comme $f$ est continue au point $z$, 
    \[
        \lim_{h \in \C^{*}, h \to 0} \frac{F(z + h) - F(z)}{h} = f(z)
    \]
    Ainsi $F$ est holomorphe sur $D(z_{0}, r)$ donc analytique d'après le théorème \ref{thm:equiv_holo_analy} et sa dérivée $f = F'$ l'est donc aussi.
\end{proof}
\begin{corollaire}{$\Gamma$}{gamma}
    La fonction $\Gamma$
    \[
        \Gamma(s) = \int_{0}^{+\infty}e^{-t}t^{s - 1}\d t
    \]
    est holomorphe pour $\Re s > 0$.
\end{corollaire}
\begin{proof}
    L'intégrale converge en $t = 0$ car $\abs{t^{s - 1}e^{-t}} \leq t^{\Re s  - 1}$. À $s$ fixé pour $t \in \R_{+}$ grand, 
    \[
        \abs{t^{s - 1}e^{-t}} = t^{\Re s - 1}e^{-t}\leq e^{t/2}e^{-t} = e^{-t/2}
    \]
    Donc $\Gamma(s)$ est bien définie pour $\Re s > 0$. Soit $\gamma : [0, 1] \to \left\{s, \Re s > 0\right\}$ la courbe décrivant un triangle. Alors, d'après le théorème de Fubini 
    \[
        \int_{\gamma}\Gamma(s) \d s = \int_{\gamma}\int_{0}^{+\infty}t^{s - 1}e^{-t}\d t\d s = \int_{0}^{+ \infty}\left(\int_{\gamma}t^{s-1}\d s\right)e^{-t}\d t = 0
    \]
    Ainsi, en appliquant le théorème de Morera \ref{thm:morera}, la fonction $\Gamma$ est holomorphe sur le demi-plan $\Re s > 0$.
\end{proof}


\section{Propriétés Éléméntaires des Fonctions Holomorphes}
\subsection{Théorème d'inversion locale}
\begin{théorème}{Inversion Locale}{inv_loc}
    Si $f\in \O(U)$, $a \in U, f'(a) \neq 0$, alors, $\exists V$ voisinage ouvert de $a$ inclus dans $U$ sur lequel $f$ est biholomorphe sur $f(V)$ ouvert. 
\end{théorème}
\begin{proof}
    Comme $f\in \O(U)$, $f$ est $\R$-différentiable. Donc il existe un voisinage $V$ ouvert de $U$ contenant $a$ sur lequel $f_{\mid V} : V \to f(V)$ est un difféomorphisme. Alors, $\d_{f(z)}(f^{-1}) = \left(d_{z} f\right)^{-1}$ et donc $f^{-1} \in \O(U)$.
\end{proof}
\begin{proof}[Idée des Séries Majorantes]
    \begin{itemize}
        \item On suppose d'abord $a = 0, f(a) = 0, f'(a) = 1$. On a 
        \[
            f(z) = z - \sum_{n \geq 2}a_{n}z^{n}, z\in D(0, r)
        \]
        On veut résoudre $f(z) = \omega = z - \sum_{n \geq 2} a_{n}z^{n}$ i.e. $z = \omega + \sum_{n \geq 2}a_{n}z^{n}$. Mais, $\sum_{n \geq 2}a_{n}z^{n} = \O(w^{2})$ : 
        \[
            z = \omega + \sum_{n \geq 2}a_{n}\left(\omega + \O(\omega^{2})\right)^{n} = \omega + a_{2}\omega^{2} + \O(\omega^{3})
        \]
        On peut alors réinjecter : 
        \[
            z = \omega + a_{2}\omega^{2} + \left(2a_{2}^{2} + a_{3}\right)\omega^{3} + \O(\omega^{4})
        \]
        et ainsi de suite : 
        \[
            z = \omega + \sum_{n = 2}^{N}P_{n}(a_{2}, \ldots, a_{n})\omega^{n} + \O(\omega^{N+1})
        \]
        où les $P_{n} \in \N[X_{2}, \ldots, X_{n}]$.
        \item Montrons maintenant que cette série converge lorsque $N \to \infty$. On sait que la série $\sum a_{n}z^{n}$ converge sur $D(0, r)$. Pour $r' < r$, $\abs{a_{n}r'^{n}} \to 0$. Donc il existe $M > 0$ tel que $\abs{a_{n}}\leq M^{n}$. Or, \[z = \omega + \sum_{n = 2}^{+ \infty}P_{n}\left(M^{2}, \ldots, M^{n}\right)\omega^{n}\] est solution de :
        \[
            \begin{aligned}
                \omega =& z - \sum_{n \geq 2}M^{n}z^{n}\\
                =& z - \left(\frac{1}{1 - Mz} - 1 - Mz\right)
            \end{aligned}
        \]
        Donc 
        \[
            \begin{aligned}
                \left(1 - Mz\right)\omega =& z(1 - Mz) - 1 + 1 - Mz + Mz(1 - Mz)
            \end{aligned}
        \]
        C'est à dire : 
        \[
            z^{2}\left(M + M^{2}\right) + z\left(-M\omega - 1\right) + \omega = 0
        \]
        ou 
        \[
            z = \frac{\left(M\omega + 1\right) - \sqrt{\left(1 + M\omega\right)^{2} - 4\omega\left(M + M^{2}\right)}}{2(M + M^{2})}
        \]
        On prend ici pour $\sqrt{\cdot}$ la détermination holomorphe de $\left(\right)^{1/2}$ qui existe sur $D(1, 1)$ et pour laquelle $\sqrt{1} = 1$ de sorte que pour $\omega = 0$, $z = 0$.\\
        La série définissant $\sqrt{\cdot}$ converge alors sur $D(0, R)$ où $R = \frac{1}{\left(1 + \sqrt{2}\right)M + 4M^{2}}$. En effet, alors, on a 
        \[
            \abs{M^{2}\omega^{2}} \leq M^{2}\abs{\omega}R \leq \frac{M^{2}\abs{\omega}}{\left(1 + \sqrt{2}\right)M} = \left(\sqrt{2}- 1\right)M\abs{\omega}
        \]
        et donc 
        \[
            \abs{\left(2M + 4M^{2}\right)\omega - M^{2}\omega^{2}} \leq \left(2M + 4M^{2}\right)\abs{\omega} + \abs{M^{2}\omega^{2}}\leq \left(\left(1 + \sqrt{2}\right)M + 4M^{2}\right)\abs{w} < 1
        \]
        D'où la convergence de $g(\omega) = \omega + \sum_{n \geq 2} P_{n}\left(a_{2}, \ldots, a_{n}\right)\omega^{n}$ sur $D(0, R)$. et $g(D(0, R)) \subset D(0, 1/M)$.
        \item Par identification de la série entière en zéro et principe du prolongement analytique, nous avons $f \circ g(\omega) = \omega$ pour $\omega \in D(0, R)$. De plus, par construction, $g$ est injective sur $W = D(0, R)$ et l'image $\omega = f(z)$ atteint surjectivement $W$ sur $g(W) \subseteq D(0, 1/M) \cap f^{-1}(W)$. Prenons $V$ la composante connexe de $0$ dans $D(0, 1/M) \cap f^{-1}(W)$. Alors $f(V) \subset W$ et $g(W) \subset V$. $V, W$ sont ouverts et $f_{\mid V} \circ g_{\mid W} = id_{W}$. Par connexité de $V$ et prolongement analytique, $g_{\mid W} \circ f_{\mid V} = id_{V}$.
    \end{itemize}
\end{proof}

\begin{théorème}{Pré-Application Ouverte}{pre_open_app}
    Soit $f \in \O(U)$ non constante au voisinage de $a \in U$, $f(a) = 0$ et \[m = \min\{k \in \N^{*}\mid f^{(k)}(a) \neq 0\}\]
    Il existe alors un voisinage ouvert $V$ de $a$, un voisinage ouvert $W$ de $0$ et un biholomorphisme $\phi : V\to W$ tel que $\phi$ envoie $a$ sur $0$ et $f(z) = f(a) + \phi(z)^{m}$.
\end{théorème}
\begin{proof}
    D'après le théorème \ref{thm:multiplicité} il existe $U'\subseteq U$ un voisinage de $a$ et $g\in \O(U')$ tels que pour tout $z \in U'$
    \[
        f(z) - f(a) = \alpha(z - a)^{m}g(z)
    \]
    avec $\alpha \in \C^{*}$ et $g(a) = 1$.\\
    Soit $ V = \left\{z \in U'\mid \abs{g(z) - 1} < 1\right\}$. C'est un voisinage de $a$ sur lequel $\exp \frac{1}{m}\log(g(z))$ existe.\\
    On a alors
    \[
        \forall z \in V', f(z) = f(a) + \left(\phi(z)\right)^{m}
    \]
    où 
    \[
        \phi(z) = \alpha_{m}(z - a)\exp\left(\frac{1}{m}\log(g(z))\right)
    \]
    où $\alpha_{m}^{m} = \alpha$. Alors, $\phi \in \O(V')$ avec $\phi(a) = 0$ et $\phi'(a) = 1$. Par théorème d'inversion locale \ref{thm:inv_loc}, on a un voisinage $V \subset V'$ de $a$ sur lequel $\phi$ est un biholomorphisme.
\end{proof}

\begin{corollaire}{Solutions d'une Équation}{nb_sol_eq}
    Soit $f\in \O(U)$ non constante au voisinage de $a \in U$ et \[m = \min\{k \in \N^{*}\mid f^{(k)}(a) \neq 0\}\]. Alors, $\exists r,\rho \in \R_{+}^{*}$ tels que $\forall \omega \in D(f(a), \rho) \setminus \{f(a)\}$ l'équation $f(z) = \omega$ a exactement $m$ solutions dans $D(a, r)$.
\end{corollaire}
\begin{proof}
    On écrit par le théorème \ref{thm:pre_open_app} précédent $f(z) = \omega = f(a) + \phi(z)^{m}$ où $\phi : V \to W$ est tel que $\phi(a) = 0$. On suppose $\phi(z) = \left(\omega - f(a)\right)^{1/m}$ pour une certaine détermination de l'exponentielle. 
    On prend $r$ tel que $D(a, r) \subset V$. $\phi(D(a, r))$ est un ouvert de $W$ voisinage de $0$. Il existe un $\rho'$ tel que $D(0, \rho')$ est inclus dans $\phi(D(a, r))$. 
    Alors, pour tout $\omega \in D(f(a), \rho'^{m})$, $\left(\omega - f(a)\right)^{1/m} \in D(0, \rho')$. Mézalor, $e^{2ik\pi/m}\left(w -f(a)\right)^{1/m}$ sont dans $D(0, \rho')$.
    On obtient alors \[z_{k} = \phi^{-1}\left(e^{2ik\pi/m}\left(\omega - f(a)\right)^{1/m}\right) \in D(a, r)\].\\
    Les $z_{k}$ sont solutions de $f(z) = \omega$ et donc il y en a bien exactement $m$. \\
    De même, l'équation $f(z) = f(a)$ n'a qu'une solution $z = a$ dans $D(a, r)$ de multiplicité $m$.
\end{proof}

\begin{théorème}{Application Ouverte}{open_app}
    Une fonction holomorphe non constante sur un ouvert $U$ connexe est une application ouverte.
\end{théorème}
\begin{proof}
    Par le corollaire \ref{cor:nb_sol_eq}, tout point $z_{0} \in U$ admet un voisinage $V_{z_{0}} \subset U$ tel que $f(V_{z_{0}}) = D(f(z_{0}), \rho(z_{0}))$. Ainsi, $f(U) = \cup D(f(z_{0}), \rho(z_{0}))$ est ouvert. 
\end{proof}
\begin{théorème}{Théorème d'Inversion Gloable}{global_inv}
    Soit $U$ un ouvert connexe et $f \in \O(U)$ injective. Alors : 
    \begin{enumerate}
        \item $f(U)$ est un ouvert de $\C$
        \item $f'$ ne s'annule pas sur $U$
        \item $f : U \to f(U)$ est un biholomorphisme
    \end{enumerate}
\end{théorème}
\begin{proof}
    \begin{enumerate}
        \item D'après le théorème de l'application ouverte \ref{thm:open_app}, $f$, injective donc non constante, est ouverte donc $f(U)$ est ouverte et $f$ est une bijection continue ouverte de $U$ dans $f(U)$, i.e., un homéomorphisme. 
        \item Supposons qu'il existe $z_{0}$ pour lequel $f'(z_{0}) = 0$. Dans le théorème \ref{cor:nb_sol_eq}, on a un entier $m \geq 2$ et donc $f$ n'est pas injective au voisinage de $z_{0}$ ce qui est absurde. Donc $f'$ ne s'annule pas sur $U$. 
        \item D'après les deux premiers points et le théorème \ref{thm:inv_loc} d'inversion locale, $f^{-1}$ est holomorphe sur $f(U)$ et $f: U \to f(U)$ est un biholomorphisme.
    \end{enumerate}
\end{proof}

\begin{théorème}{Goursat}{Goursat}
	Soit $U \subseteq \C$ ouvert et $K$ un compact à bord de classe $\cont^{1}$ avec l'orientation canonique du bord. Pour toute fonction holomorphe sur $U$ on a : 
	\[
		\int_{\partial K}f(z)\d z = 0
	\]
\end{théorème}
\begin{proof}
    On approche $K$ par des compacts à bords polygonaux. Notons $\delta = d(K, \C\setminus U) > 0$. 
    Paramétrons $\delta K$ par un nombre fini d'arcs $\cont^{1}$ par morceaux. Pour chaque tel arc $\gamma : [a, b] \to U$, soit une subdivision $a = \tau_{0} < \tau_{1} < \ldots < \tau_{n} = b$ telle que $\abs{\gamma(\tau_{j + 1}) - \gamma(\tau_{j})} \leq \epsilon \leq \delta/2$. 
    Chaque segment $[\gamma(\tau_{j + 1}), \gamma(\tau_{j})] \subset U$. 
    Pour $\epsilon$ assez petit, la réunion de ces segments constitue le bord d'un compact $K_{\epsilon}$ à bord polygonal. 
    $K_{\epsilon} = \cup_{i} T_{i}$ est réunion de triangles adjacents et le lemme de Goursat \ref{lemme:Goursat} implique 
    \[
        \int_{\partial K_{\epsilon}}f(z)\d(z) = \sum_{i}\int_{\partial T_{i}} f(z) \d z = 0
    \]
    D'après la proposition , on a bien : 
    \[
        \lim_{\epsilon \to 0}\int_{\partial K_{\epsilon}} = \int_{\partial K}f(z)\d z
    \]
    D'où le résultat. 
\end{proof}

\begin{théorème}{Formule de Cauchy}{formule_cauchy}
    Soit $f$ holomorphe sur un ouvert $U \subseteq \C$ et $K$ un compact à bord orienté $\cont^{1}$ par morceaux inclus dans $U$. Alors, pour tout $z \in K$ 
    \[
        f(z) = \frac{1}{2i\pi}\int_{\partial K}\frac{f(\omega)}{\omega -  z}\d \omega
    \]
\end{théorème}
\begin{proof}
    Soit $r > 0$ tel que $\overline{D(z, r)} \subset \mathring{K}$. On note $K_{r} = K \setminus D(z, r)$. $K_{r}$ est un compact à bord orienté $\cont^{1}$ par morceaux dont le bord est $\partial K_{r} = \partial K \cup \partial D^{-}(z, r)$ où $\partial D^{-}$ signifie que ce cercle a l'orientation opposée à celle obtenue comme bord de $\overline{D(z,r)}$. La fonction $g(\omega) = f(\omega)/\left(\omega - z\right)$ et holomorphe sur $U \setminus \{z\}$. Le théorème de Goursat \ref{thm:Goursat} appliqué à $g$ sur le compact $K_{r} \subseteq U \setminus \{z\}$ donne 
    \[
        \int_{\partial K}\frac{f(\omega)}{\omega - z}\d \omega - \int_{\partial D(z, r)} \frac{f(\omega)}{\omega - z}\d \omega = 0
    \]
    En posant $\omega = z + re^{it}$ on a : 
    \[
        \int_{\partial D(z, r)}\frac{f(\omega)}{\omega - z}\d \omega = \int_{0}^{2\pi}\frac{f(z + re^{it})}{re^{it}}ire^{it}\d t = i \int_{0}^{2\pi}f(z + re^{it}) \d t
    \]
    et cette dernière intégrale tend vers $2i\pi f(z)$ lorsque $r$ tend vers $0$ par continuité de $f$ au point $z$. 
\end{proof}
\begin{théorème}{Équivalence Holomorphie-Analycité}{equiv_holo_analy}
    Soit $f : U \to \C$. $f$ est holomorphe sur $U$ si et seulement si elle est analytique.
\end{théorème}
\begin{proof}
    On a déjà l'implication analycité holomorphie. Supposons $f$ holomorphe sur $U$ et $\overline{D}(z_{0}, r) \subset U$. Pour $z \in D(z_{0}, r)$, la formule de Cauchy \ref{thm:formule_cauchy} donne 
    \[
        f(z) = \frac{1}{2i\pi}\int_{\partial D(z_{0}, r)}\frac{f(\omega)}{\omega - z}\d \omega
    \]
    Or 
    \[
        \frac{1}{\omega - z} = \sum_{n = 0}^{+ \infty}\frac{\left(z - z_{0}\right)^{n}}{\left(\omega - z_{0}\right)^{n + 1}}
    \]
    De plus, pour $\omega = z_{0} + re^{it}$,
    \[
        \abs{\frac{\left(z - z_{0}\right)^{n}}{\left(\omega - z_{0}\right)^{n + 1}}} = \frac{1}{r}\left(\frac{\abs{z - z_{0}}}{r}\right)^{n}, \text{ avec } \abs{z - z_{0}}/r < 1
    \]
    Par convergence normale pour $t \in [0, 2\pi]$, on obtient :
    \[
        f(z) = \frac{1}{2i\pi}\int_{\partial D(z_{0},r )}f(\omega)\sum_{n = 0}^{+ \infty}\frac{\left(z - z_{0}\right)^{n}}{\left(\omega - z_{0}\right)^{n + 1}}\d \omega = \sum_{n = 0}^{+ \infty}\int_{\partial D(z_{0}, r)}\frac{f(\omega)}{2i\pi\left(\omega - z_{0}\right)^{n + 1}}\d \omega\left(z - z_{0}\right)^{n} = \sum a_{n}(z - z_{0})^{n}
    \]
    et la série entière ci-dessus converge normalement sur les compacts de $D(z_{0}, r)$.
\end{proof}
\begin{corollaire}{Classe des Dérivées}{}
    Soit $U$ un ouvert de $\C$. Toute fonction holomorphe sur $U$ est de classe $\cont^{\infty}$ sur $U$.\\
    Précisément, pour tout $K \subset U$ compact à bord de classe $\cont^{1}$ par morceaux et pour tout $z \in \mathring{K}$ nous avons : 
    \begin{enumerate}
        \item $\forall n \geq 0, \frac{\partial^{n}f}{\partial z^{n}}(z) = f^{(n)}(z) = \frac{n!}{2i\pi}\int_{\partial K}\frac{f(\omega)}{\left(\omega - z\right)^{n + 1}}\d \omega$
        \item $\forall n \geq 0, \forall m \geq 0, \frac{\partial^{n + m}f}{\partial z^{n}\partial \bar{z}^{m}}(z) = 0$.
    \end{enumerate}
    En particulier, une fonction holomorphe $f$ admet des dérivées complexes $f^{(n)}$ d'ordre $n$ arbitraire et les dérivées $f^{(n)}$ sont holomorphes.
\end{corollaire}

\begin{théorème}{Morera}{morera}
    Soit $f$ une fonction continue sur un ouvert $U$ de $\C$. Nous supposons que $\int_{\partial T}f(z) \d z = 0$ pour tout triangle $T$ inclus dans $U$. Alors $f$ est holomorphe sur $U$.
\end{théorème}
\begin{proof}
    Soit $z_{0} \in U$ et $r > 0$ tel que $\overline{D}(z_{0}, r) \subset U$. Pour $z \in D(z_{0}, r)$, on pose 
    \[
        F(z) = \int_{[z_{0}, z]} f(\omega) \d \omega
    \]
    Soit $z \in D(z_{0}, r)$ et $h \neq 0$ tel que $z + h \in D(z_{0}, r)$. Comme le triangle de sommets $z_{0}, z, z + h$ est inclus dans $D(z_{0}, r)$, nous avons 
    \[
        \frac{F(z + h)-F(z)}{h} = \frac{1}{h}\int_{[z, z + h]}f(\omega)\d \omega = \int_{0}^{1} f(z + th)\d t
    \]
    Comme $f$ est continue au point $z$, 
    \[
        \lim_{h \in \C^{*}, h \to 0} \frac{F(z + h) - F(z)}{h} = f(z)
    \]
    Ainsi $F$ est holomorphe sur $D(z_{0}, r)$ donc analytique d'après le théorème \ref{thm:equiv_holo_analy} et sa dérivée $f = F'$ l'est donc aussi.
\end{proof}
\begin{corollaire}{$\Gamma$}{gamma}
    La fonction $\Gamma$
    \[
        \Gamma(s) = \int_{0}^{+\infty}e^{-t}t^{s - 1}\d t
    \]
    est holomorphe pour $\Re s > 0$.
\end{corollaire}
\begin{proof}
    L'intégrale converge en $t = 0$ car $\abs{t^{s - 1}e^{-t}} \leq t^{\Re s  - 1}$. À $s$ fixé pour $t \in \R_{+}$ grand, 
    \[
        \abs{t^{s - 1}e^{-t}} = t^{\Re s - 1}e^{-t}\leq e^{t/2}e^{-t} = e^{-t/2}
    \]
    Donc $\Gamma(s)$ est bien définie pour $\Re s > 0$. Soit $\gamma : [0, 1] \to \left\{s, \Re s > 0\right\}$ la courbe décrivant un triangle. Alors, d'après le théorème de Fubini 
    \[
        \int_{\gamma}\Gamma(s) \d s = \int_{\gamma}\int_{0}^{+\infty}t^{s - 1}e^{-t}\d t\d s = \int_{0}^{+ \infty}\left(\int_{\gamma}t^{s-1}\d s\right)e^{-t}\d t = 0
    \]
    Ainsi, en appliquant le théorème de Morera \ref{thm:morera}, la fonction $\Gamma$ est holomorphe sur le demi-plan $\Re s > 0$.
\end{proof}


\section{Propriétés Éléméntaires des Fonctions Holomorphes}
\subsection{Théorème d'inversion locale}
\begin{théorème}{Inversion Locale}{inv_loc}
    Si $f\in \O(U)$, $a \in U, f'(a) \neq 0$, alors, $\exists V$ voisinage ouvert de $a$ inclus dans $U$ sur lequel $f$ est biholomorphe sur $f(V)$ ouvert. 
\end{théorème}
\begin{proof}
    Comme $f\in \O(U)$, $f$ est $\R$-différentiable. Donc il existe un voisinage $V$ ouvert de $U$ contenant $a$ sur lequel $f_{\mid V} : V \to f(V)$ est un difféomorphisme. Alors, $\d_{f(z)}(f^{-1}) = \left(d_{z} f\right)^{-1}$ et donc $f^{-1} \in \O(U)$.
\end{proof}
\begin{proof}[Idée des Séries Majorantes]\ref{thm:inv_loc_serie_maj}
    \begin{itemize}
        \item On suppose d'abord $a = 0, f(a) = 0, f'(a) = 1$. On a 
        \[
            f(z) = z - \sum_{n \geq 2}a_{n}z^{n}, z\in D(0, r)
        \]
        On veut résoudre $f(z) = \omega = z - \sum_{n \geq 2} a_{n}z^{n}$ i.e. $z = \omega + \sum_{n \geq 2}a_{n}z^{n}$. Mais, $\sum_{n \geq 2}a_{n}z^{n} = \O(w^{2})$ : 
        \[
            z = \omega + \sum_{n \geq 2}a_{n}\left(\omega + \O(\omega^{2})\right)^{n} = \omega + a_{2}\omega^{2} + \O(\omega^{3})
        \]
        On peut alors réinjecter : 
        \[
            z = \omega + a_{2}\omega^{2} + \left(2a_{2}^{2} + a_{3}\right)\omega^{3} + \O(\omega^{4})
        \]
        et ainsi de suite : 
        \[
            z = \omega + \sum_{n = 2}^{N}P_{n}(a_{2}, \ldots, a_{n})\omega^{n} + \O(\omega^{N+1})
        \]
        où les $P_{n} \in \N[X_{2}, \ldots, X_{n}]$.
        \item Montrons maintenant que cette série converge lorsque $N \to \infty$. On sait que la série $\sum a_{n}z^{n}$ converge sur $D(0, r)$. Pour $r' < r$, $\abs{a_{n}r'^{n}} \to 0$. Donc il existe $M > 0$ tel que $\abs{a_{n}}\leq M^{n}$. Or, \[z = \omega + \sum_{n = 2}^{+ \infty}P_{n}\left(M^{2}, \ldots, M^{n}\right)\omega^{n}\] est solution de :
        \[
            \begin{aligned}
                \omega =& z - \sum_{n \geq 2}M^{n}z^{n}\\
                =& z - \left(\frac{1}{1 - Mz} - 1 - Mz\right)
            \end{aligned}
        \]
        Donc 
        \[
            \begin{aligned}
                \left(1 - Mz\right)\omega =& z(1 - Mz) - 1 + 1 - Mz + Mz(1 - Mz)
            \end{aligned}
        \]
        C'est à dire : 
        \[
            z^{2}\left(M + M^{2}\right) + z\left(-M\omega - 1\right) + \omega = 0
        \]
        ou 
        \[
            z = \frac{\left(M\omega + 1\right) - \sqrt{\left(1 + M\omega\right)^{2} - 4\omega\left(M + M^{2}\right)}}{2(M + M^{2})}
        \]
        On prend ici pour $\sqrt{\cdot}$ la détermination holomorphe de $\left(\right)^{1/2}$ qui existe sur $D(1, 1)$ et pour laquelle $\sqrt{1} = 1$ de sorte que pour $\omega = 0$, $z = 0$.\\
        La série définissant $\sqrt{\cdot}$ converge alors sur $D(0, R)$ où $R = \frac{1}{\left(1 + \sqrt{2}\right)M + 4M^{2}}$. En effet, alors, on a 
        \[
            \abs{M^{2}\omega^{2}} \leq M^{2}\abs{\omega}R \leq \frac{M^{2}\abs{\omega}}{\left(1 + \sqrt{2}\right)M} = \left(\sqrt{2}- 1\right)M\abs{\omega}
        \]
        et donc 
        \[
            \abs{\left(2M + 4M^{2}\right)\omega - M^{2}\omega^{2}} \leq \left(2M + 4M^{2}\right)\abs{\omega} + \abs{M^{2}\omega^{2}}\leq \left(\left(1 + \sqrt{2}\right)M + 4M^{2}\right)\abs{w} < 1
        \]
        D'où la convergence de $g(\omega) = \omega + \sum_{n \geq 2} P_{n}\left(a_{2}, \ldots, a_{n}\right)\omega^{n}$ sur $D(0, R)$. et $g(D(0, R)) \subset D(0, 1/M)$.
        \item Par identification de la série entière en zéro et principe du prolongement analytique, nous avons $f \circ g(\omega) = \omega$ pour $\omega \in D(0, R)$. De plus, par construction, $g$ est injective sur $W = D(0, R)$ et l'image $\omega = f(z)$ atteint surjectivement $W$ sur $g(W) \subseteq D(0, 1/M) \cap f^{-1}(W)$. Prenons $V$ la composante connexe de $0$ dans $D(0, 1/M) \cap f^{-1}(W)$. Alors $f(V) \subset W$ et $g(W) \subset V$. $V, W$ sont ouverts et $f_{\mid V} \circ g_{\mid W} = id_{W}$. Par connexité de $V$ et prolongement analytique, $g_{\mid W} \circ f_{\mid V} = id_{V}$.
    \end{itemize}
\end{proof}
\subsection{Théorème de l'Application Ouverte}
\begin{théorème}{Pré-Application Ouverte}{pre_open_app}
    Soit $f \in \O(U)$ non constante au voisinage de $a \in U$, $f(a) = 0$ et \[m = \min\{k \in \N^{*}\mid f^{(k)}(a) \neq 0\}\]
    Il existe alors un voisinage ouvert $V$ de $a$, un voisinage ouvert $W$ de $0$ et un biholomorphisme $\phi : V\to W$ tel que $\phi$ envoie $a$ sur $0$ et $f(z) = f(a) + \phi(z)^{m}$.
\end{théorème}
\begin{proof}
    D'après le théorème \ref{thm:multiplicité} il existe $U'\subseteq U$ un voisinage de $a$ et $g\in \O(U')$ tels que pour tout $z \in U'$
    \[
        f(z) - f(a) = \alpha(z - a)^{m}g(z)
    \]
    avec $\alpha \in \C^{*}$ et $g(a) = 1$.\\
    Soit $ V = \left\{z \in U'\mid \abs{g(z) - 1} < 1\right\}$. C'est un voisinage de $a$ sur lequel $\exp \frac{1}{m}\log(g(z))$ existe.\\
    On a alors
    \[
        \forall z \in V', f(z) = f(a) + \left(\phi(z)\right)^{m}
    \]
    où 
    \[
        \phi(z) = \alpha_{m}(z - a)\exp\left(\frac{1}{m}\log(g(z))\right)
    \]
    où $\alpha_{m}^{m} = \alpha$. Alors, $\phi \in \O(V')$ avec $\phi(a) = 0$ et $\phi'(a) = 1$. Par théorème d'inversion locale \ref{thm:inv_loc}, on a un voisinage $V \subset V'$ de $a$ sur lequel $\phi$ est un biholomorphisme.
\end{proof}

\begin{corollaire}{Solutions d'une Équation}{nb_sol_eq}
    Soit $f\in \O(U)$ non constante au voisinage de $a \in U$ et \[m = \min\{k \in \N^{*}\mid f^{(k)}(a) \neq 0\}\]. Alors, $\exists r,\rho \in \R_{+}^{*}$ tels que $\forall \omega \in D(f(a), \rho) \setminus \{f(a)\}$ l'équation $f(z) = \omega$ a exactement $m$ solutions dans $D(a, r)$.
\end{corollaire}
\begin{proof}
    On écrit par le théorème \ref{thm:pre_open_app} précédent $f(z) = \omega = f(a) + \phi(z)^{m}$ où $\phi : V \to W$ est tel que $\phi(a) = 0$. On suppose $\phi(z) = \left(\omega - f(a)\right)^{1/m}$ pour une certaine détermination de l'exponentielle. 
    On prend $r$ tel que $D(a, r) \subset V$. $\phi(D(a, r))$ est un ouvert de $W$ voisinage de $0$. Il existe un $\rho'$ tel que $D(0, \rho')$ est inclus dans $\phi(D(a, r))$. 
    Alors, pour tout $\omega \in D(f(a), \rho'^{m})$, $\left(\omega - f(a)\right)^{1/m} \in D(0, \rho')$. Mézalor, $e^{2ik\pi/m}\left(w -f(a)\right)^{1/m}$ sont dans $D(0, \rho')$.
    On obtient alors \[z_{k} = \phi^{-1}\left(e^{2ik\pi/m}\left(\omega - f(a)\right)^{1/m}\right) \in D(a, r)\].\\
    Les $z_{k}$ sont solutions de $f(z) = \omega$ et donc il y en a bien exactement $m$. \\
    De même, l'équation $f(z) = f(a)$ n'a qu'une solution $z = a$ dans $D(a, r)$ de multiplicité $m$.
\end{proof}

\begin{théorème}{Application Ouverte}{open_app}
    Une fonction holomorphe non constante sur un ouvert $U$ connexe est une application ouverte.
\end{théorème}
\begin{proof}
    Par le corollaire \ref{cor:nb_sol_eq}, tout point $z_{0} \in U$ admet un voisinage $V_{z_{0}} \subset U$ tel que $f(V_{z_{0}}) = D(f(z_{0}), \rho(z_{0}))$. Ainsi, $f(U) = \cup D(f(z_{0}), \rho(z_{0}))$ est ouvert. 
\end{proof}
\begin{théorème}{Théorème d'Inversion Gloable}{global_inv}
    Soit $U$ un ouvert connexe et $f \in \O(U)$ injective. Alors : 
    \begin{enumerate}
        \item $f(U)$ est un ouvert de $\C$
        \item $f'$ ne s'annule pas sur $U$
        \item $f : U \to f(U)$ est un biholomorphisme
    \end{enumerate}
\end{théorème}
\begin{proof}
    \begin{enumerate}
        \item D'après le théorème de l'application ouverte \ref{thm:open_app}, $f$, injective donc non constante, est ouverte donc $f(U)$ est ouverte et $f$ est une bijection continue ouverte de $U$ dans $f(U)$, i.e., un homéomorphisme. 
        \item Supposons qu'il existe $z_{0}$ pour lequel $f'(z_{0}) = 0$. Dans le théorème \ref{cor:nb_sol_eq}, on a un entier $m \geq 2$ et donc $f$ n'est pas injective au voisinage de $z_{0}$ ce qui est absurde. Donc $f'$ ne s'annule pas sur $U$. 
        \item D'après les deux premiers points et le théorème \ref{thm:inv_loc} d'inversion locale, $f^{-1}$ est holomorphe sur $f(U)$ et $f: U \to f(U)$ est un biholomorphisme.
    \end{enumerate}
\end{proof}

\subsection{Lemme de Schwarz}
\begin{théorème}{Principe du Maximum}{principe_max}
    Soit $f \in \O(U)$
    \begin{enumerate}
        \item Si $\abs{f}$ admet une maximum local en un point $a \in U$, alors $f$ est constante sur la composante connexe contenant $a$. 
        \item Pour tout $K \subset U$
        \[
            \max_{K}\abs{f} = \max_{\partial K}\abs{f}, \max_{K}\Re f = \max_{\partial K}\Re f, \max_{K}\Im f = \max_{\partial K} \Im f
        \]
    \end{enumerate}
\end{théorème}
\begin{proof}
    \begin{enumerate}
        \item Supposons $f$ non constante sur la composante connexe $U_{0}$ de $U$ contenant $a$ avec $\abs{f(a)} = \sup_{U}\abs{f} = \sup_{U_{0}}\abs{f}$. D'après le théorème de l'application ouverte, $f$ est ouverte sur $U_{0}$. L'image $f(U_{0})$ est un voisinage de $f(a)$ donc contient des points de module strictement supérieur à $\abs{f(a)}$.
        \item Si $\max_{\partial K}\Re f < \max_{K}\Re f$, il existe $z_{0} \in \mathring{K}$ avec $\Re f(z_{0}) = \max_{K}\Re f$. Soit $U_{0}$ une composante connexe de $z_{0}$ dans $\mathring{K}$ et $f$ non constante dans $U_{0}$. Alors $f(U_{0})$ est un ouvert qui contient $f(z_{0})$ et qui est contenue dans le demi-plan $\left\{w\mid \Re w \leq \Re f(z_{0})\right\}$. Donc $f$ est constante sur $U_{0}$ et $\Re f_{\mid \partial U_{0}} = \Re f(z_{0})$ par continuité de $f$. Or
        \[
            \emptyset \neq \partial U_{0} \subset \partial \mathring{K} = \overline{\mathring{K}} \setminus \mathring{K} \subseteq \overline{K} \setminus \mathring{K} = \partial K
        \]
        ainsi, $\max_{K}\Re f = \Re f(z_{0})$ est atteint sur $\partial K$. Les cas $\abs{f}$ et $\Im f$ sont analogues. 
    \end{enumerate}
\end{proof}

\begin{théorème}{Lemme de Schwarz}{lemme_schwarz}
    Soit $f$ holomorphe sur $D(0, 1)$ avec $f(0) = 0$ et $\abs{f(z)} \leq 1$ sur $D(0, 1)$. Alors, on a 
    \[
        \forall z \in D(0, 1), \abs{f(z)} \leq \abs{z} \text{ et } \abs{f'(0)}\leq 1
    \]
    De plus si 
    \[
        \exists z_{0}\in D(0, 1)\setminus \{0\}, \abs{f(z_{0})} = \abs{z_{0}} \text{ ou } \abs{f'(0)} = 1
    \]
    alors $f$ est une rotation : $f(z) = az$ pour $a \in \C, \abs{a} = 1$.
\end{théorème}
\begin{proof}
    L'application $g(z) = f(z)/z$ est holomorphe sur $D(0, 1)$ et vérifie $g(0) = f'(0)$. En appliquant le principe du maximum à la fonction $g$ sur $D(0, r), r < 1$, on obtient : 
    \[
        \sup_{D(0, r)}\abs{g} = \sup_{\partial D(0, r)}\abs{g} = \sup_{\partial D(0, r)}\abs{f}/r \leq 1/r
    \]
    En faisant tendre $r$ vers $1$, nous obtenons
    \[
        \sup_{D(0, 1)}\abs{g} \leq 1 \Longleftrightarrow \begin{cases}
            \abs{f(z)} \leq \abs{z} & \forall z \in D(0, 1)\setminus \{0\}\\
            \abs{f'(0)} \leq 1 &
        \end{cases}
    \]
    De plus, si $\exists z_{0} \in D(0, 1)\setminus \{0\}$, $\abs{f(z_{0})} = \abs{z_{0}}$ ou si $\abs{g(0)} = \abs{f'(0)} = 1$, alors $\abs{g}$ admet un maximum local en $z_{0}$ ou $0$ donc $g = a$ est constante avec $\abs{a} = 1$.
\end{proof}

\begin{corollaire}{Point Fixe}{point_fixe}
    Soit $f : D(0, 1) \to D(0, 1)$ holomorphe, $f\neq {\rm Id}$. Alors $f$ a au plus un point fixe. 
\end{corollaire}
\begin{proof}
    Supposons que nous ayons $a \neq b \in D(0, 1)$ avec $f(a) = a$ et $f(b) = b$.\\
    Si $a = 0, f(0) = 0, f(b) = b$, le lemme de Schwarz montre que $f = e^{i\theta}z$ et comme $f(b) = b$, $f = \rm Id$.\\
    On peut donc supposer $a \neq 0$. Posons $\phi_{\alpha}(z)\footnote{C'est la fonction du jour !} = \frac{z - a}{1 - \bar{a}z}$ et 
    \[
        g = \phi_{\alpha} \circ f \circ \phi_{\alpha}^{-1} \text{ et } \lambda = \phi_{\alpha}(b) \neq 0
    \]
    Ainsi, $g : D(0, 1) \to D(0, 1)$ avec deux points fixes $g(0) = 0, g(\lambda) = \lambda$. Donc $g = {\rm Id} = f$. Absurde.
\end{proof}

\subsection{Disque Unité et Inversion Locale Effective}
\begin{définition}{Automorphisme}{}
    Un automorphisme de $U$ est une bijection holomorphe de $U$ dans $U$. On note $\Aut U$ l'ensemble des automorphismes de $U$. 
\end{définition}

On rappelle que le groupe unitaire de signature $(n, m)$ est le groupe des matrices préservant une forme hermitienne de signature $(n, m)$. Il est isomorphe au groupe $U(n, m)$ préservant la forme hermitienne diagonale de signature $(n, m)$ : 
\[
    U(n, m) = \left\{\begin{pmatrix}
        A & B \\ C & D
    \end{pmatrix} \in {\rm GL}_{n + m}(\C) \ \middle| \ \begin{aligned}&AA* - BB* = I_{n}\\
    &DD* - CC* = I_{m}\\ &AC* = BD* \end{aligned} \right\} 
\]

\begin{théorème}{Automorphisme du Disque Unité}{}
    \[
        \Aut D(0, 1) = \left\{\phi_{\alpha} : z \longmapsto e^{i\theta}\frac{z - a}{1 - \bar{a}z}\ \middle|\ \theta \in \R, a \in \C, \abs{a} < 1\right\} = {\rm PSU}(1, 1)
    \]
\end{théorème}
\begin{proof}
    \begin{itemize}
        \item Nous vérifions que $\phi_{\alpha} \in \Aut D$. Soit $f \in \Aut D$. On note $a = f(0)$. On a $\phi_{\alpha} \circ f(0) = 0$ et pour tout $z \in D$, $\abs{\phi_{\alpha} \circ f(z)} < 1$. Le lemme de Schwarz \ref{thm:lemme_schwarz} implique alors que $g = \phi_{\alpha} \circ f$ vérifie $\abs{g(z)} \leq \abs{z}$ et de même $\abs{g^{-1}(z)} \leq \abs{z}$ donc $\abs{g(z)} = \abs{z}$ pour $z\in D$. Donc il existe $\theta \in \R$ tel que $g(z) = e^{i\theta}z$.
        \item L'application : 
        \[
            \phi_{\alpha}\longmapsto \frac{1}{1- \abs{a}^{2}}\begin{pmatrix}
                e^{i\theta/2} & -\alpha e^{i\theta/2} \\
                -\bar{\alpha}e^{-i\theta/2} & e^{-i\theta/2}
            \end{pmatrix}
        \]
        induit un automorphisme 
        \[
            \Aut D \rightarrow {\rm PSU}(1, 1) = \left\{\begin{pmatrix}
                A & B \\ \overline{A} & \overline{B}
            \end{pmatrix}\ \middle| \ \abs{A}^{2} - \abs{B}^{2} = 1\right\}/\pm {\rm Id}
        \]
    \end{itemize}
\end{proof}
\begin{corollaire}{Schwarz-Pick}{schwarz-pick}
    Soit $f : D(0, 1) \to D(0, 1)$ holomorphe. Pour $z \in D(0, 1)$ on a :
    \[
        \frac{\abs{f'(z)}}{1 - \abs{f(z)}^{2}} \leq \frac{1}{1 - \abs{z}^{2}}
    \]
    avec égalité si et seulement si $f \in \Aut D(0, 1)$.
\end{corollaire}
\begin{proof}
    La dérivée de $\phi_{\alpha}(z) = \frac{z - \alpha}{1 - \bar{\alpha}z}\in \Aut D$ vérifie : 
    \[
        \phi_{\alpha}'(z) = \frac{1 - \abs{\alpha}^2}{\left(1 - \bar{\alpha}z\right)^{2}}, \phi_{\alpha}'(0) = 1 - \abs{\alpha}^{2}, \abs{\phi_{\alpha}'(\alpha)} = \frac{1}{1 - \abs{\alpha}^{2}}
    \]
    Soit $z_{0} \in D$ et $g = \phi_{f(z_{0})} \circ f \circ \phi_{-z_{0}}$. On a $g(0) = 0$ et $g : D\to D$. Le lemme de Schwarz \ref{thm:lemme_schwarz} implique que $\abs{g(z)} \leq \abs{z}$ et $\abs{g'(0)} \leq 1$. L'inégalité attendue résulte de 
    \[
        g'(0) = \phi_{f(z_{0})}(f(z_{0}))f'(z_{0})\phi_{-z_{0}}'(0) = \frac{1}{1 - \abs{f(z_{0})}^{2}}f'(z_{0})(1 - \abs{z_{0}})^{2}
    \]
    Le cas d'égalité correspond à $\abs{g'(0)} = 1$. Alors $g$ et donc $f$ sont des automorphismes de $D$. 
\end{proof}

\begin{théorème}{Bloch-Landau}{bloch-landau}
    Soit $f \in \O(D(z_{0}, r))$ telle que $f'(z_{0}) \neq 0$. Alors il existe $U \subset D(z_{0}, r)$ tel que $f_{\mid U}$ est un biholomorphisme de $U$ sur $f(U) = D(\omega_{0}, R)$ disque de rayon $R \geq \frac{r}{12}\abs{f'(z_{0})}$.
\end{théorème}
\begin{proof}
    Quitte à considérer la restriction de $f$ et à remplacer $f$ par $f(z_{0} + rz)$ on peut considérer définie au voisinage du disque fermé $\overline{D}(0, 1)$. On pose
    \[
        m = \sup_{z \in \overline{D}(0, 1)}\left(1 - \abs{z}^{2}\right)\abs{f'(z)} \geq \abs{f'(0)}
    \]
    La valeur $m$ est atteinte en $\alpha \in D(0, 1)$ et alors $m = \left(1 - \abs{\alpha}^{2}\right)\abs{f'(\alpha)}$. Posons $h = f\circ\phi_{-\alpha}$. On a : 
    \[
        h(0) = f(\alpha), h'(0) = f'(\alpha)\phi_{-\alpha}(0) = \left(1 - \abs{\alpha}^{2}\right)f'(\alpha), \abs{h'(0)} = m \geq \abs{f'(0)}
    \]
    De plus, pour $\abs{z} < 1$, d'après le corollaire de Schwarz-Pick \ref{cor:schwarz-pick} : 
    \[
        \left(1-\abs{z}^{2}\right)\abs{h'(z)} = \frac{\left(1 - \abs{z}^{2}\right)\abs{\phi_{-a}'(z)}}{1 - \abs{\phi_{-\alpha}(z)}^{2}}\left(1 - \abs{\phi_{-\alpha}(z)}^{2}\right)\abs{f'(\phi_{-\alpha}(z))} \leq m
    \]
    Quitte à remplacer $h$ par $\frac{1}{h'(0)}(h-h(0))$, on peut supposer $h(0) = 0$ et $m = h'(0) = 1$. Donc : 
    \[
        \abs{h'(z)} \leq \frac{1}{1 - \abs{z}^{2}}, z \in D(0, 1)
    \]
    Le rayon de convergence du développement en série entière\footnote{DONC DE TAYLOR}
    \[
        h(z) = z + \sum_{n = 2}^{+ \infty}a_{n}z^{n}
    \]
    est supérieur à $1$ et les inégalités de Cauchy appliquées à $h'$ sur le disque $D(0, \rho)$ donnent 
    \[
        h^{(n)}(\omega) = \frac{(n- 1)!}{2\pi\rho^{n-1}}\int_{0}^{2\pi}\frac{h'(\omega + re^{it})}{e^{i(n - 1)t}}\d t \text{ et } n\abs{a_{n}} \leq \frac{1}{(1 - \rho^{2})\rho^{n - 1}}
    \]
    Or, $\rho \mapsto (1 - \rho^{2})\rho^{n - 1}$ atteint son maximum en $\rho = \sqrt{\frac{n-1}{n + 1}}$. Pour ce $\rho$, on obtient : 
    \[
        \abs{a_{n}} \leq \frac{n + 1}{2n}\left(1 + \frac{2}{n - 1}\right)^{(n-1)/2}
    \]
    et pour $M = 3^{3/4}/2$, $\abs{a_{n}} \leq M^{n}$ si $n \geq 2$.\\
    En reprenant la preuve du théorème d'inversion locale par la méthode des séries majorantes \ref{thm:inv_loc}, on obtient que $h$ est un biholomorphisme d'un ouvert $U$ sur le disque $D(0, R)$ avec : 
    \[
      R = \frac{1}{\left(1 + \sqrt{2}\right)M + 4M^{2}}>\frac{1}{12}
    \]
\end{proof}

\section{Espaces de Fonctions Holomorphes}
\subsection{Convergence de Suites de Fonctions Holomorphes}



\end{document}
