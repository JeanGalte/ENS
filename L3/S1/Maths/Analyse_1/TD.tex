\documentclass{cours}
\title{TD Topo}
\date{\today}
\author{Matthieu Boyer}

\begin{document}
\part{TD3}
\section{Exercice 1 : Echauffement}
    \subsection{Question 1}
        \subsubsection{Question a.}
            Plus ou moins vrai, $f\mid_{A}$ est continue pour la topologie induite/trace
        \subsubsection{Question b.}
            Faux, il suffit de prendre la fonction $\mathrm{sign}$ qui n'est pas continue sur 
             $\R$ mais qui l'est sur $\R^{+*}$
    \subsection{Question 2}
        Vrai, les singletons sont ouverts
    
    \subsection{Question 3}
        Faux, les singletons ne sont pas ouverts.
    
    \subsection{Question 4}
        \subsubsection{Question a.}
            On a $\pi\left(\left[0, 1\right[\right) = \left[0, 1\right[$. Mais $\left[\cdot\right]^{-1} \left(\left[0, 1\right[\right) = \left]-1, 1\right[$ qui est ouvert.

        \subsubsection{Question b.}
            On a $\pi(1) = 1$ et $\pi\left(\left[0, 1\right]\right) = \left[0, 1\right]$. Or, ce segment contient un voisinage de $1$ : ${1}$. Donc c'est bien un voisinage de $1$
        
        \subsubsection{Question c.}
            On ne peut pas séparer $-1$ et $1$.
        
    \subsection{Question 5}
        \subsubsection{Question a.}
            Faux : ${0, -1}$ n'est pas ouvert. 
        \subsubsection{Question b.}
            Faux : ${0}$ est ouvert. 
    
\section{Exercice 2 : Topologie Induite}
    \subsection{Question 1}
        Oui bon ça va hein

    \subsection{Question 2}
        \subsubsection{Question a.}
            La fonction $j$ étant croissante de réciproque croissante pour $\subset$, c'est bien un homéomorphisme. 
        \subsubsection{Question b.}
            $\overline{\left\{\omega\right\}} = Y$ et donc : $\left\{\left\{\omega\right\}\right\}$ est une base finie de $Y$.
        \subsubsection{Question c.}
            On a $\omega \in U \cap V$

\section{Exercice 3 : Séparation et espaces quotients}
    \subsection{Question 1}
        Si $(x, y) \in (X \times X)/\mathcal{R}$, on a $U, V$ ouverts de $X/\mathcal{R}$ tels que : 
        \[
            \begin{aligned}
                &x \in U, y \in V\\
                &U\cap V = \emptyset
            \end{aligned}    
        \]
        Alors, $\left\{\left[t\right] \mid t \in U\right\}$ (de même pour $V$) est ouvert. Donc en particulier, $\mathcal{R}$ est fermé.

    \subsection{Question 2}
        Si $\mathcal{R}$ est fermée, alors si, $(x, y) \in (\R \times \R)/\mathcal{R}$ il existe des voisinages disjoints de $x$ et $y$. Par ouverture de $\pi$ on obtient bien la séparation de $X/\mathcal{R}$.

    \subsection{Question 3}
        \subsubsection{Question a.}
            \begin{itemize}
                \item $(i. \Rightarrow ii.)$  On considère : $\left\lVert\cdot\right\rVert : S \mapsto \sup_{x\in S} d(x, F)$. Il est clair que cette application est positive et est nulle si et seulement si $\forall x \in S, x \in \overline{F} = F$ i.e. $S \subset F$. Par les propriétés de $d$, cette fonction définit bien une norme sur $E/F$ en passant au quotient. 
                \item $(ii. \Rightarrow iii.)$ En particulier, $E/F$ est métrisable donc est séparé. 
                \item $(iii. \Rightarrow i.)$ Ceci est une conséquence de la question 1. 
            \end{itemize}

        \subsubsection{Question b.}
            \begin{itemize}
                \item $(\Leftarrow)$ Si $F = \ker f$ est fermé. En particulier si $U \subset \Im f$ est ouvert, en quotientant par $F$, puisque $E/F$ est normable, $f$ est continue. 
                \item $(\Rightarrow)$ Si $f$ est continue, il est clair que $\ker f$ est fermé. 
            \end{itemize}

\section{Exercice 4 : Lemme d'Urysohn}
    \subsection{Question 1}
        En prenant pour ouverts dans la définition d'un espace normal $f^{-1}\left(\left[0, 1/3\right[\right)$ et $f^{-1}\left(\left[2/3, 1\right[\right)$, on a bien le résultat. 

    \subsection{Question 2}
        Si $(X, d)$ est métrique, si $F_{0}, F_{1}$ sont fermés disjoints dans $X$. En particulier, en prenant un recouvrement d'ouverts le plus petit possible de $F_0$ et un de $F_{1}$, on a bien le résultat.
    
    \subsection{Question 3}
        \subsubsection{Question a.}
            Déjà, il existe une bijection $r$ de $\N$ dans $\mathcal{D}$. Ensuite, on peut définir par récurrence la famille $G$. On suppose que les $r_{k}$ pour $k < n$ sont déjà définis. \\
            On pose alors $U_{n} = F_{0} \cup \bigcup\limits_{k < n, r_{k} < r_{n}} \overline{G_{r_{k}}}$. C'est un fermé, inclus dans l'ouvert : $V_{n} = F_{1}^{\complement} \cap \bigcap\limits_{k < n, r_{k} > r_{n}} G_{r_{k}}$.\\
            Puisquee $X$ est normal, il existe donc un ouvert $G_{r_{n}}$ tel que : $U_{n} \subset G_{r_{n}}$ et $ \overline{G_{r_{n}}} \subset V_{n}$.\\
            On a bien défini une famille de fermés $\left(G_{x}\right)_{x \in \mathcal{D}}$ qui convient. 
        
        \subsubsection{Question b.}
            Il est clair que $f$ est bien définie, à valeurs dans ${0, 1}$. 
            De plus, il est aussi clair que : $f\left(F_{0}\right) = 0$ puisque $\forall x \in F_{0}, x \in G_{1}$ et $x \in F_{0}$.\\
            Ensuite, si $x \in F_{1}, x \notin G_{1} \subset F_{1}^{\complement}$. Donc $f\left(F_{1}\right) = 1$.\\
            Enfin, par densité des nombres dyadiques, il est clair que $f$ est continue. 

\section{Exercice 5 : Quelques propriétés des espaces produits}
    \subsection{Question 1}
            Bah oui. 'fin, c'est trivial quoi. 

    \subsection{Question 2}
            Faites un effort svp.

    \subsection{Question 3}
            \begin{itemize}
                \item $(\Leftarrow)$ Si $I$ est dénombrable, le résultat est direct en prenant pour métrique l'infimum des métriques
                \item $(\Rightarrow)$ Sinon, si $I$ n'est pas dénombrable, supposons qu'il y ait une métrique $d$ qui induit la topologie produit sur $X$. En particulier, si on pose se donne une famille croissante $C$ de parties finies de $I$, $O_{i, n} = \prod_{k \in C_{i}} B_{X_{k}}(x_{k}, 1/n)$, les $O_{i, n}$ sont ouverts donc sont des ouverts pour $d$. Mais alors, en faisant tendre $i$ vers l'infini, on n'obtient plus des ouverts, ce qui contredit l'existence de $d$. 
            \end{itemize}


\section{Exo 10 TD4/ TD5}

\subsection{Question 1}
On peut appliquer la propriété universelle à $K = Y$, $f = j_{x}$. On dispose d'une unique fonction continue $\hat{f}$ telle que $\hat{f} o \iota_{X} = j_{X}$. On a alors de même une fonction continue $\hat{g}$ en inversant les rôles de $\hat{X}$ et $Y$.
En appliquant la propriété universelle à $K = \hat{X}$ et $f = \iota_{X}$ on dispose d'une unique application continue telle que $\hat{h} o \iota_{X} = \iota_{X}$. Par unicité, $\hat{h} = \textmd{id}$. Comme $\hat{g} o \hat{f}$ convient aussi, on a bien le résultat.

\subsection{Question 2}
\begin{itemize}
    \item $\iota_{X}$ est continue
    \item Soit $x, y$ deux points distincts. $\left\{y\right\}$ est fermé car $X$ est séparé. Puisque $X$ est de Tychonoff, il existe $f : X \rightarrow [0, 1]$ continue qui envoie $x$ sur $0$ et $y$ sur $1$. On applique la propriété universelle à $f$, pour $K = [0, 1]$. On dispose de $\hat{f} : \hat{X} \rightarrow [0, 1]$ continue telle que $f(\iota_{X}(x)) = 0$ et $f(\iota_{X}(y)) = 1$ donc $\iota_{X}$ est injective, et donc bijective sur son image. 
    \item Soit $F \subset X$ un fermé. Soit $x \in X \setminus F$. Montrons que $\iota_{X}(x) \subset U \subset X \setminus \iota_{X}(F)$. Puisque $X$ est de Tychonoff, il existe $f$ de $X$ dans $[0, 1]$ continue telle que $f(x) = 0$ et $f^{-1}(F) = \{1\}$. Par propriété universelle, il existe $\hat{f}$ continue telle que $\hat{f} o \iota_{X} = f$. Remarquons alors que $\hat{f}^{-1}([0, 1/2[)$ est un ouvert par continuité de $\hat{f}$. De plus, $\iota_{X}(x)$ est dans cet ouvert, et $\iota_{X}(F) \cap \hat{f}^{-1}([0, 1/2[) = \emptyset$ d'où le résultat. Donc $\iota_{X}$ est ouverte et donc sa réciproque est continue. Finalement, on a bien le résultat. 
\end{itemize}

\subsection{Question 3}
Théorème de Tychonoff

\subsection{Question 4}
\subsubsection{Question a.}
Si on a $I(x) = I(y)$, en particulier, leurs images par toute application continue de $X$ dans $[0, 1]$ sont égales. Donc en particulier, $x$ et $\left\{y\right\}$ sont pas séparables et donc, puisque $X$ est de Tychonoff, on a le résultat. 

\subsubsection{Question b.}
Soit $U = \prod_{g \in C_{X}} O_{g}$ un ouvert avec $O_{g} = [0, 1]$ sauf en $g_{1}, \ldots, g_{n} \in C_{X}$ où $O_{g_{i}} = U_{i}$ ouvert de $[0, 1]$. On a alors :
\begin{equation*}
    \begin{split}
        I(x) \in U &\Leftrightarrow \forall i = 1, \ldots, n,\ I(x)(g_{i}) \in U_{i}\\
        &\Leftrightarrow\forall i = 1, \ldots, n, \ g_{i}(x) \in U_{i}
    \end{split}
\end{equation*}
Ainsi : $I^{-1}(U) = \bigcap_{i = 1}^{n} g_{i}^{-1}(U_{i})$ qui est un ouvert par continuité de $g_{i}$. 
\subsubsection{Question c.}
Si $F$ est un fermé de $X$, $x \notin F$. Puisque $X$ est de Tychonoff, il existe $f$ de $X$ dans $[0, 1]$ continue telle que $f(x) = 0$ et $f^{-1}(F) = \{1\}$. Alors, on a $I(x)(f) = 0$ et $\forall y \in F, \ I(y)(f) = \{1\}$. On a donc, en particulier, $U \cap I(F) = \emptyset$ où $U = [0, 1/2[ \times [0, 1]^{C_{X} \setminus \{f\}}$ et donc $I$ est ouverte.


\part{TD6}
\section{Exercice 1}
\subsection{Question 1}
Pas un espace métrique, le TD Man ce bâtard...

\subsection{Question 2}
Bah non, $1 - 1/n$ ne converge pas dans $]0, 1[$. On prend $[0, 1]$.

\subsection{Question 3}
Oui : fermé dans le complet $C^{0}_{B}$

\subsection{Question 4}
Non : pas un fermé

\subsection{Question 5}
Non, $c_{0}$ n'est pas fermé : $\overline{c_{0}} = l^{\inf}$ est complet

\subsection{Question 6}
Oui : preuve par cours.

\subsection{Question 7}
Non : $c_{0}$

\section{Exercice 2}
\subsection{Question 1}
La topologie induite explose sinon.

\subsection{Question 2}
Si $E = O_{0} \cup O_{1}$ :
Il n'est pas connexe par arcs car $\bigcup_{\N}(\frac{1}{n})$ n'est pas dense dans $[0, 1]$ : on ne peut pas sortir de $(0, 0)$.

\subsection{Question 3}


\end{document}