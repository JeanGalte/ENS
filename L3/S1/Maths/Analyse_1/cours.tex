\documentclass{cours}

\title{Topologie et Calcul Différentiel}
\author{Djalil Chafaï}
\date{2023 - 2024}

\begin{document}
\newpage
\section{Espaces Topologiques}

\subsection{Espaces à produit scalaire, espaces normés, espaces métriques, espaces topologiques.}

\begin{definition}
    Un produit scalaire sur un $\K$-ev est une forme linéaire, symétrique (ou hermitienne) et définie positive.
    Quand $\K = \C$, on dit que le produit scalaire est sesquilinéaire.
\end{definition}

\begin{proposition}
    \begin{itemize}
        \item Relation de Pythagore : $\norm{x + y}^{2} = \norm{x}^{2} + \norm{y}^{2} + 2\Re(\scalar{x, y})$
        \item Identité du Parallélograme : $\norm{x + y}^{2} + \norm{x - y}^{2} = 2(\norm{x}^{2} + \norm{y}^{2})$
        \item Inégalité de Chauchy-Schwarz : $\abs{\scalar{x, y}} \leq \norm{x}\norm{y}$
    \end{itemize}
\end{proposition}

\begin{definition}
    Une norme sur un $\K$-ev ($\K \in\left\{\R, \C\right\}$) est une forme positive sous-additive homogène séparée.
\end{definition}

\begin{definition}
    Une distance ou une métrique sur un ensemble est une forme positive séparée symétrique vérifiant l'inégalité triangulaire.
\end{definition}

\begin{definition}
    Une topologie $\O \in \mathcal{P}(X)$ sur un ensemble $X$ est une collection de partie de $X$ stable par réunion quelconque, intersections finies, contenant l'espace et le vide. On appelle ses éléments des ouverts
\end{definition}

\subsection{Fermés}
\begin{definition}
    \begin{itemize}
        \item Un ensemble $A$ est fermé si et seulement si $A^{\complement}$ est ouvert. 
        \item L'adhérence d'un ensemble est le plus petit fermé le contenant : \[\overline{A} = \bigcap_{A\subset F, F fermé}F  = \left\{x \in X, \forall O \in \O, x \in O \rightarrow O \cap A \neq \emptyset\right\}\]
        \item L'intérieur de $A$ est le plus grand ouvert qu'il contient : \[\mathring{A} = \bigcup_{O \subset A, O ouvert} = \left\{x \in X, \exists\ O \in \O, \ x \in O \subset A \right\}\]
        \item La frontière de $A$ est : $\partial A = \overline{A}\setminus\mathring{A}$
        \item $A$ est dense si d'adhérence égale à $X$.
    \end{itemize}  
\end{definition}

\begin{definition}
    \begin{itemize}
    \item $x$ est intérieur à $A$ si $x \in \mathring{A}$.
    \item $x$ est adhérent à $A$ lorsque $x \in \overline{A}$. On dit alors que $x$ est isolé lorsqu'il existe $O_{x}$ voisinage ouvert de $x$ d'intersection $x$ avec $A$. Sinon, $x$ esst d'accumulation.
    \end{itemize}
\end{definition}

\subsection{Voisinages, convergence et continuité.}

\begin{definition}
    Un voisinage d'un point $x$ est une partie qui contient un ouvert contenant $x$.
\end{definition}

\begin{definition}
    Une suite converge vers $x$ pour une topologie lorsque pour tout voisinage de $x$, la suite appartient à ce voisinage àpcr.
\end{definition}

\begin{proposition}
    Si $F$ fermé, $x_{n} \in F \to x$, alors $x \in F$. La réciproque est fausse en générale.
\end{proposition}

\begin{theorem}
    Dans un espace métrique, $x_{n} \to x$ ssi $d(x_{n}, x) \to 0$
\end{theorem}

\begin{definition}
    Une application $f$ est dite :
    \begin{itemize}
        \item continue en $x$ lorsque pour tout voisinage $V$ de $f(x)$, il existe un voisinage $W$ de $x$ tel que $f(W) \subset V$.
        \item séquentiellement continue en $x$ lorsque pour toute suite $x_{n} \to x$, $f(x_{n}) \to f(x)$.
    \end{itemize}
\end{definition}

\begin{proposition}
    La continuité implique la continuité séquentielle.
\end{proposition}

\begin{proposition}
    Soit $f : X \rightarrow Y$. On a équivalence entre : 
    \begin{itemize}
        \item $f$ est continue
        \item Les images réciproques par $f$ des ouverts de $Y$ sont des ouverts de $X$.
        \item Les images réciproques par $f$ des fermés de $Y$ sont des fermés de $X$.
    \end{itemize}
\end{proposition}

\begin{definition}[Propriété de Fréchet-Urysohn]
    $X$ vérifie la propriété de Fréchet-Urysohn si : 
    \[\forall A \subset X,\ x \in \overline{A}, \text{ il existe } x_{n} \in A^{\N},\ x_{n} \to x\]    
\end{definition}

\begin{theorem}
    Si $X$ vérifie la propriété de Fréchet-Urysohn, pour tout espace $Y$ et tout $f : X \rightarrow Y$, la continuité équivaut à la continuité séquentielle.
\end{theorem}

\begin{definition}
    Un homéomorphisme est une bijection continue de réciproque continue.
\end{definition}

\subsection{Bases de topologie}
\begin{definition}
    Soit $\mathcal{B} \subset \O$ une famille d'ouverts. $\mathcal{B}$ est une base de $\O$ quand : $\forall O \in \O, \exists \left(B_{i}\right)_{i}\in \mathcal{B}, O = \cup_{i} B_{i}$ ou de manière équivalente quand $\forall O \in \O, x \in O, \exists B \in \mathcal{B},\ x \in B \subset O$. 
\end{definition}

\begin{theorem}
    Soit $\mathcal{B} \subset \O$ une base. On a : 
    \begin{itemize}
        \item $X = \cup_{B \in \mathcal{B}} B$
        \item $\forall B_{1}, B_{2} \in \mathcal{B},\ x \in B_{1} \cap B_{2},\ \exists B_{3} \in \mathcal{B}, x \in B_{3} \subset B_{1} \cap B_{2}$.
    \end{itemize}
    Réciproquement, si une famille vérifie ces propriétés, alors $\O = \left\{\cup_{B \in \mathcal{A} B}\right\}_{\mathcal{A} \subset \mathcal{B}}$ est la plus petite topologie qui contient $\mathcal{B}$, appelée topologie engendrée par $\mathcal{B}$.
\end{theorem}

\begin{definition}
    Une base locale au point $x$ est une famille d'ouverts contenant $x$ et dont au moins l'un est inclus dans chaque ouvert contenant $x$.
\end{definition}

\begin{definition}
    Un espace topologique est :
    \begin{itemize}
        \item à base dénombrable de voisinages si tout point possède une base dénombrable de voisinages.
        \item à base dénombrable lorsqu'il possède une base dénombrable (c’est plus fort !)
        \item séparable lorsqu'il existe une partie dénombrable dense.
    \end{itemize}
\end{definition}

\begin{theorem}
    Un espace à base dénombrable est toujours séparable. La réciproque est vraue pour un espace métrisable.
\end{theorem}

\begin{theorem}
    Tout espace à base dénombrable de voisinages (en particulier tout espace métrisable) est un espace de Fréchet-Urysohn.
\end{theorem}

\subsection{Axiomes de Séparation}
\begin{definition}
    Axiome T2 : Tous deux points peuvent être séparés par deux ouverts distincts.
\end{definition}

\begin{theorem}
    Pour tout espace topologique métrisable : 
    \begin{itemize}
        \item Les singletons sont fermés.
        \item Pour tous fermés $F_{0}, F_{1}$, il existe $f$ continue valant $i$ sur $F_{i}$.
    \end{itemize}
\end{theorem}

\begin{lemma}
    Dans un espace métrique, $F$ est fermé si et seulement si $d(x, F) = 0 \Rightarrow x \in F$.
\end{lemma}

\subsection{Topologies}
\subsubsection{Topologie Trace}
\begin{definition}
    On appelle topologie trace la topologie induite par la topologie de $X$ sur $A \subset X$ est la topologie la moins fine sur $A$ qui rend l'inclusion canonique continue.
\end{definition}
\begin{proposition}
    \begin{itemize}
        \item La restriction de la métrique induit la topologie trace.
        \item La définition est emboîtable.
        \item La fermeture d'un ensemble pour la topologie trace est la trace de sa fermeture. Ce n'est pas vrai pour l'intérieur.
        \item Si $x_{n} \rightarrow x_{*}\in A$ ssi $x_{n} \rightarrow x_{*}$ dans $X$.
    \end{itemize}
    \begin{itemize}
        \item Si $\O$ est à base dénombrable (resp. de voisinages), $\O_{A}$ l'est aussi
        \item Si $\O$ est séparée (axiome T2), $\O_{A}$ aussi.
        \item Si $\O$ est métrisable est séparable, alors $\O_{A}$ est métrisable est séparable.
    \end{itemize}    
\end{proposition}

\subsubsection{Topologie Produit}
\begin{definition}
    On appelle topologie produit ou cylindrique sur $X = \prod_{i \in I} X_{i}$ la topologie engendrée par les $\prod_{i \in I}O_{i}$ avec $O_{i} \neq X_{i}$ sur un nombre fini de $i$. C'est la topologie la moins fine sur $X$ qui rend les projections canoniques continues.
\end{definition}
\begin{lemma}
    On a : $x_{n} \to x$ si et seulement si $x_{n, i} \to x_{i}$ pour tout $i$.
\end{lemma}

\begin{proposition}
    \begin{itemize}
        \item Si tous les $X_{i}$ vérifient $T2$, $X$ vérifie $T2$
        \item Si $I$ est au plus dénombrable, et tous les $X_{i}$ sont à base dénombrable (de voisinages), $X$ l'est aussi.
        \item Si $I$ est au plus dénombrable ou a le cardinal de $\R$, et si les $X_{i}$ sont tous séparables, $X$ aussi.
        \item Si $I$ est au plus dénombrable, et si les $X_{i}$ sont métrisables par les $d_{i}$, $X$ est métrisable par :
        \begin{itemize}
            \item $\max_{i}(d_{i})$ si $I$ est fini
            \item $\max_{i}\min(d_{i}, 2^{-i})$ si $I$ est infini dénombrable.
        \end{itemize}
    \end{itemize}

\end{proposition}

\subsubsection{Topologies Initiale et Finale}
\begin{definition}
    \begin{itemize}
        \item Soient $f_{i} : X \rightarrow X_{i}$. La topologie engendrée sur $X$ par les $f_{i}^{-1}(O_{i})$ où $O_{i}$ est ouvert dans $X_{i}$ est appelée topologie initiale. C'est la moins fine qui rend $f_{i}$ continue pour tout $i$.
        \item Soient $g_{i} : X_{i} \rightarrow X$. La topologie engendrée par les ensembles $O$ tels que $g_{i}^{-1}(O)$ est ouvert dans $X_{i}$ est appelée topologie finale. C'est la plus fine qui rend $g_{i}$ continue pour tout $i$. 
    \end{itemize}
\end{definition}

\subsubsection{Topologie Quotient}
\begin{definition}
    Soit $\sim$ une relation d'équivalence sur $X$. La topologie quotient sur $X/\sim$ est la plus fine qui rend la projection canonique continue : $O \subset X/\sim$ est ouvert ssi $\left[\cdot\right]^{-1}(O) = \left\{x \in X \mid \left[x\right] \in O\right\}$ est ouvert dans $X$. C'est la topologie finale de la projection canonique.
\end{definition}

\newpage
\section{Compacité}
\subsection{Quasi-Compéacité}
\begin{definition}
    Un espace est dit quasi-compact lorsqu'il vérifie la propriété de Borel-Lebesgue : De tout recouvrement par des ouverts on peut  extraire un sous-recouvrement fini. Un espace est dit compact lorsqu'il est quasi-compact et séparé.
\end{definition}

\begin{definition}
    Un sous ensemble est quasi-compact lorsqu'il est quasi compact pour la topologie trace.
\end{definition}

\begin{proposition}
    \item Dans $\R^{n}$, $K$ est compact si et seulement si il est fermé borné.
    \item Si $K_{1}, K_{2}$ sont quasi compacts, $K_{1} \cup K_{2}$ est quasi compact.
\end{proposition}

\begin{theorem}
    \item Si $F$ est fermé dans $K$ quasi compact, $F$ est quasi compact.
    \item Si $K$ est quasi compact dans $X$ séparé, $K$ est fermé.
\end{theorem}

\begin{definition}
    Si $X$ est séparé, $A \subset X$ est relativement compact lorsque $\overline{A}$ est compact.
\end{definition}

\begin{theorem}
    \begin{itemize}
        \item Si $f : X \rightarrow Y$ est continue, $X$ est quasi compact, alors $f(X)$ est quasi-compact.
        \item Si $f : X \rightarrow \R$ et $X \neq \emptyset$ est quasi compact, alors, $\exists\ x_{\star} \in X,\ f(x_{\star}) = \sup_{x \in X}f(x) < \infty$.
    \end{itemize}
\end{theorem}

\begin{theorem}
    Si $f : X \rightarrow Y$ est une bijection continue avec $X$ quasi compact et $Y$ séparé, $f^{-1}$ est continue.
\end{theorem}

\subsection{Théorème de Tykhonov}
\begin{theorem}\label{thm:Tykhonov}
    Tout produit de (quasi-)compacts est (quasi-)compact.
\end{theorem}
\subsection{Compacité Métrique}
\begin{definition}
    Un $\epsilon$-réseau est un ensemble $A$ fini tel que tout point est à distance au plus $\epsilon$ d'un point de $A$.
\end{definition}

\begin{lemma}
    Un espace métrique compact possède un $\epsilon$-réseau fini pour tout $\epsilon$.    
\end{lemma}

\begin{theorem}
    Pour un espace métrisable, on a équivalence entre : 
    \begin{enumerate}
        \item $X$ est compact
        \item De toute suite de $X$ on peut extraire une sous-suite convergeant dans $X$.
    \end{enumerate}
    Dans ce cas on a : 
    \item Lemme de Lebesgue : pour tout recouvrement par des ouverts $O_{i}$, il existe $r > 0$ tel que pour tout $x \in X$, il existe $i_{x}$ tel que $B(x, r) \subset O_{i_{x}}$.
\end{theorem}

\subsection{Compacité Locale}
\begin{definition}
    Un espace est localement compact lorsque tout point possède un voisinage quasi-compact.
\end{definition}

\begin{definition}
    Un espace est dénombrable à l'infini s'il admet un recouvrement dénombrable par des quasi-compacts (qu'on peut supposer croissants sans perte de généralité). 
\end{definition}

\begin{lemma}
    Un espace métrisable compact est localement compact et dénombrable à l'infini, et cela est vrai pour tout ouvert pour la topologie induite.
\end{lemma}

\begin{theorem}
    Si un espace est localement compact et dénombrable à l'infini, il existe une suite $K_{n}$ de quasi-compacts croissante d'union $X$ et tel que tout quasi-compact inclus dans $X$ est inclus dans au moins l'un des $K_{n}$. On parle de suite exhaustive de compacts.
\end{theorem}

\subsection{Compactification d'Alexandrov}
\begin{theorem}
    Soit $X$ un espace topologique et un point à l'infini $\infty \notin X$. Soit $X^{\star} = X\cup \left\{\infty\right\}$, $\O^{\star} \subset \mathcal{P}(X^{\star})$ formé par les ouverts de $X$ et les complémentaires dans $X^{\star}$ des quasi-compacts fermés de $X$. Alors : 
    \begin{enumerate}
        \item $\O^{\star}$ est une topologie sur $X^{\star}$.
        \item $X^{\star}$ est quasi-compact
        \item L'injection canonique est continue et ouverte
        \item $X^{\star}$ est séparé si et seulement si $X$ est séparé et localement compact.
        \item $X$ est dense dans $X^{\star}$ si et sseulement si $X$ n'est pas quasi-compact fermé.
    \end{enumerate}
\end{theorem}

\subsection{Théorème de Baire}
\begin{lemma}
    Pour $X$ un espace topologique, $X$ est quasi-compact si et seulement si pour toute famille de fermés $(F_{i})_{i \in I}$ telle que $\bigcap_{i\in I^{'}} \neq \emptyset$ pour tout $I^{'} \subset I$ fini, on a : $\bigcap_{i} F_{i} \neq \emptyset$.
\end{lemma}

\begin{lemma}
    Si $X$ est quasi-compact éparé alors : 
    \begin{itemize}
        \item Tout point et tout fermé ne le contenant pas sont séparables par des ouverts.
        \item Pour tout $x \in X$ et tout ouvert $O \owns x$, il existe $O^{'} \owns x$ tel que $\overline{O^{'}} \subset O$.
    \end{itemize}
\end{lemma}

\begin{theorem}\label{thm:Bairemétrique}
    Si $X$ est quasi-compact alors il est de Baire : toute intersection d'une suite d'ouverts denses est dense.    
\end{theorem}

\section{Complétude}
\subsection{Suites de Cauchy}
\begin{definition}
    Une suite $x_{n}$ est de Cauchy lorsque pour tout $\epsilon > 0$, il existe $N = N_{\epsilon}$ tel que pour tous $n, m \geq N$, $d(x_{n}, x_{m}) < \epsilon$. Un espace métrique est complet lorsque toute suite de Cauchy converge.
\end{definition}

\begin{lemma}
    Si $X$ est complet, $F \subset X$ est fermé, alors $F$ est complet. Si $A \subset X$ est complet, alors $A$ est fermé.
\end{lemma}

\begin{lemma}
    Soit $X$ complet et $X = F_{0} \supset F_{1} \supset \dots$ une suite décroissante de fermés non vide et de diamètres tendant vers $0$. Alors leur intersection est un certain point $x \in X$.
\end{lemma}

\begin{theorem}
    Un espace métrique est compact si et seulement si il est complet et admet un $\epsilon$-réseau pour tout $\epsilon$.
\end{theorem}

\begin{theorem}
    \begin{itemize}
        \item Les $R^{n}$ sont complets
        \item Les $l^{p}$ pour $p \in \left[1, \infty\right]$ sont conplet.
    \end{itemize}
\end{theorem}

\begin{theorem}
    \begin{itemize}
        \item Si $K$ est compact et $Y$ métrique complet, alors $\mathcal{C}(K, Y)$ est métrique complet.
        \item Si $X$ est localement compact à base dénombrable de voisinages et $Y$ métrique complet alors $\mathcal{C}(X, Y)$ est métrisable complet.
    \end{itemize}
\end{theorem}


\begin{definition}
    On définit la distance de Hausdorff entre deux fermés d'un espace métrique de diamètre fini par : 
    \[d_{H}(F_{1}, F_{2}) < r \Leftrightarrow \text{ pour tout } x \in F_{1, 2},\ \exists y \in  F_{2, 1},\ d(x, y) < r\]
    On note $\mathcal{F}(X)$ l'ensemble des fermés non-vides de $X$, et $\mathcal{K}(X)$ l'ensemble des compacts non-vides.
\end{definition}

\begin{theorem}
    \begin{itemize}
        \item Si $X$ complet, $\mathcal{F}(X)$ et $\mathcal{K}(X)$ sont complets.
        \item Si $X$ est compact, $\mathcal{K}(X)$ est compact.
    \end{itemize}
\end{theorem}


\subsection{Espaces Polonais, de Banach, de Hilbert}
\begin{definition}
    Un espace topologique est :
    \begin{itemize}
        \item polonais lorsqu'il est séparable et métrisable complet
        \item de Banach lorsque c'est un ev normé complet
        \item de Hilbert loesque c'est un ev à produit scalaire complet
    \end{itemize}
\end{definition}

\begin{theorem}
    Un ev normé est un espace de Banach ssi toute série absolument convergente est convergente.
\end{theorem}

\subsection{Complétion}
\begin{definition}
    Soit $X$ un espace métrique non complet. Son complété $(X^{'}, d')$ est un espace métrique complet tel que $X \subset X^{'}$ et $X$ est dense dans $X^{'}$. On le construit ainsi :
    \begin{itemize}
        \item Soit $\tilde{X}$ l'ensemble des suites de Cauchy, muni de la relation : $x_{n} \sim y_{n}$ ssi pour tout $\epsilon > 0$, il existe un rang pour lequel les suites sont à distance au plus $\epsilon$.
        \item On considère $X^{'} = \tilde{X}/\sim$. On considère la quantité $d'((x_{n}), (y_{n})) = \lim_{n \to \infty} d(x_{n}, y_{n})$. C'est bien une métrique compatible avec la topologie de $X^{'}$.
    \end{itemize}
\end{definition}

\begin{remark}
    Tous deux complétés sont isomètres.
\end{remark}

\begin{lemma}
    Si $f : X \to Y$ est uniformément continue, $Y$ est complet, il existe une unique fonction continue prolongée sur le complété de $X$ et égale à $f$ sur $X$.
\end{lemma}

\begin{theorem}\label{thm:Bairecomplet}
    Si $X$ est complet, alors il est de Baire.    
\end{theorem}

\section{Connexité}
\subsection{Connexité, connexité par arcs, composantes connexes}
\begin{definition}
    Un espace est : 
    \begin{itemize}
        \item connexe lorsqu'il n'est pas partitionnable en deux ouverts non-vides
        \item connexe par arcs lorsque les points sont reliés par des arcs
    \end{itemize}
\end{definition}

\begin{theorem}
    \begin{itemize}
        \item $X$ est connexe ssi $\emptyset$ et $X$ sont les seules parties à la fois ouvertes et fermées.
        \item $X$ est connexe ssi il n'est pas partitionnable en deux fermés non-vides.
        \item Si $f : X \to Y$ est continue et $X$ connexe (resp. par arcs), alors $f(X)$ est connexe (resp. par arcs)
        \item Si $X$ est connexe par arcs, alors il est connexe, et la réciproque est fausse.
        \item Si $\cap_{i} A_{i} \neq \emptyset$ avec les $A_{i}$ connexes (resp. par arcs), $\cup_{i} A_{i}$ est connexe (resp. par arcs)
        \item Si les $X_{i}$ sont connexes (resp. par arcs), alors $\prod_{i} X_{i}$ est connexe (resp. par arcs).
    \end{itemize}
\end{theorem}

\begin{definition}
    La composante connexe $C_{x}$ de $x \in X$ est la plus grande partie connexe de $X$ contenant $x$. Un espace est totalement discontinu si $C_{x} = \left\{x\right\}$ pour tout $x$. 
\end{definition}

\subsection{Connexité Métrique}
\begin{definition}
    Un espace métrique est bien echaîné lorsque pour tout $\epsilon >0$, et tous $x, y \in X$ il existe une suite finie $x = x_{0}, x_{1}, \ldots, x_{n} = y$ telle que $d(x_{i}, x_{i+1}) < \epsilon$ pour tout $i$.
\end{definition}

\begin{theorem}
    Si un espace est connexe alors il est bien enchaîné, et la réciproque est fausse mais devient vraie en ajoutant la compacité..
\end{theorem}

\section{Espaces de fonctions continues sur un métrique compact}
\begin{definition}
    Pour une suite $f_{n}$ dans $\mathcal{C}(K, Y)$ et $f$ dans $\mathcal{C}(K, Y)$ : \begin{itemize}
        \item $f_{n} \to f$ ponctuellement lorsque pour tout $x\in K, f_{n}(x) \to f(x)$
        \item $f_{n} \to f$ uniformément lorsque la convergence a lieu dans $\mathcal{C}(K, Y)$.
    \end{itemize}
\end{definition}

\begin{theorem}[De Dini]
    Si $Y = \R$, si la suite $f_{n}$ est croissante, et $f$ est continue, la convergence ponctuelle implique la convergence uniforme.
\end{theorem}

\begin{theorem}[De Heine]
    Toute fonction $f \in \mathcal{C}(K, Y)$ est uniformément continue.    
\end{theorem}

\begin{theorem}[de Arzelà-Ascoli]\label{thm:Ascoli}
    $A \subset \mathcal{C}(K, Y)$ a une adhérence compacte ssi les deux conditions suivantes sont réalisées : 
    \begin{itemize}
        \item Compacité Ponctuelle : $\forall x \in K, \left\{f(x)\mid f \in A\right\}$ a une adhérence compacte dans $Y$.
        \item La famille $A$ est uniformément équicontinue : pour tout $\epsilon > 0$, il existe $\eta > 0$ tel que pour tout $f \in A$, et tous $x, y \in K$, si $d_{K}(x, y) < \eta$, alors $d_{Y}(f(x), f(y)) < \epsilon$.
    \end{itemize}
\end{theorem}

\begin{theorem}[de Stone-Weierstrass]\label{thm:Stone-Weierstrass}
    Soit $\mathcal{A}$ une sous-algèbre de $\mathcal{C}(K, \R)$. vérifiant la propriété de prescription de valeurs arbitraires en deux points arbitraires : pour tous $x, y \in K$, $a, b,\in \R$, il existe $f \in \mathcal{A}$ telle que $f(x) = a$ et $f(y) = b$. Alors $\mathcal{A}$ est dense dans $\mathcal{C}(K, \R)$.
\end{theorem}

\begin{corollary}[Théorème de Weierstrass]
    Pour tout $n$, $K \subset \R^{n}$, $\R\left[x_{1}, \ldots, x_{n}\right]$ est dense dans $\mathcal{C}(K, \R)$.
\end{corollary}

\begin{corollary}[de Stone-Weierstrass Complexe]
    Si de plus la famille $\mathcal{A}$ est stable par conjugaison et à valeurs complexes, elle est dense dans $\mathcal{C}(K, \C)$.    
\end{corollary}

\begin{corollary}
    Pour tout $n$, $K\subset \C^{n}$, $\C\left[z_{1}, \ldots, z_{n}, \overline{z_{1}}, \ldots, \overline{z_{n}}\right]$ est dense. En particulier, $\C\left[e^{i\theta}, e^{-i\theta}\right]$ est dense dans $\mathcal{C}(S^{1}, \C)$.
\end{corollary}

\section{Opérateurs Linéaires Bornés}
\subsection{Définitions et Duéalité}
\begin{definition}
    Soient $X, Y$ des $\K$ ev normés avec $\K \in \left\{\R, \C\right\}$.
    \begin{itemize}
        \item $u : X \to Y$ est un opérateur linéaire borné lorsque $u$ est linéaire et qu'il est $M\in \left[0,\infty[\right.$ tel que pour tout $x \in X$, $\norm{u(x)}_{Y} \leq M\norm{x}_{X}$.
        \item On note $L(X, Y)$ l'ev des opérateurs linéaires bornés $X \to Y$.
        \item $L(X, Y)$ est normé par la norme d'opérateur, et a une structure d'algèbre.
    \end{itemize}
\end{definition}

\begin{lemma}
    Pour $u$ linéaire, on a équivalence entre : 
    \begin{enumerate}
        \item $u \in L(X, Y)$
        \item $u$ est Lipschitz
        \item $u$ est uniformément continue
        \item $u$ est continue
        \item $u$ est continue en $0$.
    \end{enumerate}
\end{lemma}

\begin{lemma}
    Si $Y$ est un Banach, $L(X, Y)$ est un Banach.
\end{lemma}

\begin{definition}
    Si $X$ est un $\K$-Banach, $L(X, \K)$ est appelé dual de $X$, noté $X^{'}$ ou $X^{\star}$.
\end{definition}

\begin{theorem}
    Si $p \in \left[1, \infty)\right.$ et $q = \frac{1}{1 - \frac{1}{p}} = \frac{p}{p-1}$ est le conjugué de Hölder de $p$, alors : 
    \[
        \Phi : \mathcal{l}^{q} \rightarrow \left(\mathcal{l}^{p}\right)^{'}, y \mapsto \left(x \mapsto \sum_{n} x_{n}y_{n}\right)
    \]
    est une bijection linéaire isométrique : $(\mathcal{l}^p)^{'}$ est isomorphe à $\mathcal{l}^{q}$.
\end{theorem}

\begin{lemma}
    Une forme linéaire est continue ssi son noyau est fermé.
\end{lemma}

\subsection{Banach-Steinhaus}
\begin{theorem}\label{thm:BanachSteinhaus}
    Si $X$ est un Banach, et $Y$ un evn, alors pour tout $A \subset L(X, Y)$, la bronitude ponctuelle est équivalente à la bornitude uniforme : 
    \[
        \forall x \in X, \sup_{u \in A} \norm{u(x)}_{Y} < \infty \Leftrightarrow \sup_{u\in A}\norm{u}_{L(X, Y)} < \infty
    \]
\end{theorem}
\begin{corollary}
    Soit $u_{n}$ dans $L(X, Y)$, où $X$ est un Banach et $Y$ un evn. La convergence ponctuelle entraîne la continuité de la limite. 
\end{corollary}

\subsection{Hahn-Banach}
\begin{theorem}\label{thm:HahnBanach}
    Soit $X \subset \tilde{X}$ un sous-espace d'un evn sur $\K \in \left\{\R, \C\right\}$. Soit $u \in L(X, \K)$ une forme linéaire. Alors il existe $\tilde{u} \in L(\tilde{X}, \K)$ telle que $\tilde{u}_{\mid X} = u$ et $\norm{\tilde{u}} = \norm{u}$.
\end{theorem}

\begin{corollary}
    Si $X$ est un Banach, et $X^{''}$ est sont bidual, l'injection canonique $\iota : X \rightarrow X^{''}$ est une isométrie linéaire : $\norm{\iota(x)} = \norm{x}$.
\end{corollary}

\begin{corollary}
    L'application $\Phi : \ml^{1} \to \left(\ml^{\infty}\right)^{'}, \ \Phi(y)(x) = \sum_{n} x_{n}y_{n}$ est une isométrie linéaire non surjective. En d'autres termes :
    \[\ml^{1} \subsetneq \left(\ml^{\infty}\right)^{'} = \left(l^{1}\right)^{''}\]
\end{corollary}

\subsection{Banach-Schauder}
\begin{theorem}[de Banach-Schauder ou de l'application ouverte]\label{thm:BanachSchauder}
    Si $X$ et $Y$ sont des Banach et si $u \in L(X, Y)$ est surjective, alors $u$ est une application ouverte.    
\end{theorem}

\begin{corollary}
    \begin{itemize}
        \item (\textbf{inverse continu}) : Si $X$ et $Y$ de Banach et $u\in L(X, Y)$ est bijective, alors $u^{-1} \in L(Y, X)$. On parle de Théorème d'Isomorphisme de Banach.
        \item (\textbf{équivalence des normes}) : Si $\norm{\cdot}, \norm{\cdot}^{'}$ sont deux normes qui font d'un même espace vectoriel normé $X$ un espace de Banach. S'il existe $c \in \left(0, \infty\right)$ telle que $\norm{\cdot} \leq c\norm{\cdot}^{'}$ alors il existe $C \in \left(0, \infty\right)$ telle que $\norm{\cdot}^{'} \leq C \norm{\cdot}$.
        \item (\textbf{théorème du graphe fermé}) : Si $X$ et $Y$ sont deux Banach et $u : X \to Y$ est linéaire, alors $u \in L(X, Y)$ si et seulement si son graphe est fermé dans $X \times Y$.
        \item (\textbf{structure des Banach séparables}) : tout Banach séparable est isomorphe à quotient de $\ml^{1}$ par un sous-espace fermé.
    \end{itemize}
\end{corollary}

\subsection{Algèbres de Banach, Rayon Spectral, Inverse}
\begin{definition}
    Si $X$ est un Banach, on définit l'espace vectoriel $L(X)$ normé par $\normop{u} = \sup_{\norm{x} = 1} \norm{u(x)}$. Le produit naturel $uv = u \circ v$ en fait une algèbre de Banach : $\normop{uv} \leq \normop{u} \normop{v}$. Le rayon spectral de $u \in L(X)$ est $\rho(u) = \lim_{n \to \infty} \normop{u^{n}}^{1/n} \leq \normop{u}$.
\end{definition}
\begin{remark}
    \begin{itemize}
        \item Lemme de Fekete : Si $a_{n}$ est sous-additive, $\lim_{n} \frac{1}{n} a_{n} = \inf_{n} \frac{1}{n} a_{n}$. La formule de $\rho$ fait sens en prenant $a_{n} = \log \normop{u^{n}}$.
        \item Le rayon spectral est inchangé avec une norme équivalente.
        \item On généralise les algèbres de matrices à la dimension infinie. 
        \item En dimension finie, $L(X)$ est isomorphe à $\mat{n}$ et le rayon spectral est égal au maximum des modules des valeurs propres par décomposition de Jordan.
        \item Lorsque $X$ est de dimension infinie, il n'y a pas vraiment d'analogue à la décomposition de Jordan. L'équation aux valeurs propres n'est pas une bonne manière de définir le spectre des opérateurs et on définit plutôt : \[\text{spec}(u) = \left\{\lambda \in \C \mid u - \lambda\Id \text{ n'est pas inversible à inverse continu} \right\}\] Alors, $\rho(u) = \sup \left\{\abs{\lambda \mid \lambda \in \text{spec}(u)}\right\}$.
    \end{itemize}
\end{remark}

\begin{theorem}
    Soit $X$ un Banach, et $u \in L(X)$.
    \begin{enumerate}
        \item Si $\rho(u) < 1$, alors $\Id - u$ est inversible dans $L(X)$ et \[\left(\Id - u\right)^{-1} = \sum_{n = 0}^{\infty} u^{n}\]
        \item Si $u$ est inversible et $\normop{v} \leq \normop{u^{-1}}^{-1}$ alors $u - v$ est inversible dans $L(X)$ et : \[\left(u - v\right)^{-1} = \left(\Id - u^{-1}v\right)^{-1}u^{-1} = \sum_{n = 0}^{\infty}\left(u^{-1}v\right)^{n}u^{-1}\]
        \item L'ensemble des $u \in L(X)$ inversibles (groupe linéaire) est un ouvert de $L(X)$.
    \end{enumerate}
\end{theorem}

\subsection{Intégrale de Riemann pour les fonctions de la variable réelle à valeurs dans un Banach}
\begin{theorem}
    Soit $X$ un Banach, $\left[a, b\right] \subset \R$. On note $\A(\left[a, b\right], X) \subset \cont\left(\left[a, b\right], X\right)$ l'ensemble des fonctions affines par morceaux. C'est un sev de $\cont(\left[a, b\right], X)$
    Il existe une unique application liénaire continue $I : \cont(\left[a, b\right]) \rightarrow X$ telle que pour tout fonction $f \in \A(\left[a, b\right], X)$ affine par morceaux associée à une subdivision $a = a_{0} < \ldots < a_{n} = b$ et à des valeurs $f_{0}, \ldots, f_{n} \in X$ :
    \[
        I(f) = \sum_{i = 0}^{n - 1}(a_{i + 1} - a_{i})\frac{f_{i} + f_{i + 1}}{2}
    \]
    On note : $\int\limits_{a}^{b} = I(f)$. De plus pour tout $f \in \cont\left(\left[a, b\right], X\right)$ :
    \[
        \norm{\int_{a}^{b} f(t) \d t} \leq \int_{a}^{b}\norm{f(t)}\d t
    \]
\end{theorem}

\section{Espaces de Hilbert}
\subsection{Projection Orthogonale sur un Convexe Fermé}
\begin{theorem}
    Si $X$ est un Hilbert, $C \subseteq X$ un convexe fermé, pour tout $x \in X$, il existe un unique $p_{C}(x) \in C$ tel que :
    \[
        \norm{x - p_{C}(x)} = d(x, C)
    \]
\end{theorem}
\begin{corollary}
    Si $X$ est un Hilbert et $F$ est un sev de $X$ fermé alors : 
    \begin{enumerate}
        \item $p_{F}$ est linéaire
        \item $p_{F}(x)$ est caractérisé par $x - p_{F}(x) \perp F$
        \item $p_{F}$ est $1$-Lipschitz donc continue
        \item Si $dim F = n < \infty$, $p_{F}(x) = \sum_{i = 1}^{n}\scalar{x, e_{i}}e_{i}$
        \item $X = F \bigoplus F^{\perp}$ et $\left(F^{\perp}\right)^{\perp} = F$.
    \end{enumerate}
\end{corollary}



\end{document}