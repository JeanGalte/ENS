\documentclass{cours}

\begin{document}
\section{TD1}

\section{Mesures Additives et $\sigma$-additives}
\subsection{Exercice 1}
On pose $\tilde{A_{0}} = A_{0}$ et pour $n \in \mathbb{N}$, on pose $\tilde{A_{n+1}} = A_{n+1} \setminus \tilde{A_{n}}$.

On a alors : $\mu \left(\bigcup\limits_{n\in\mathbb{N}}A_{n}\right) = \mu\left(\bigsqcup\limits_{n\in\mathbb{N}}\tilde{A_{n}}\right)$ car ces unions sont égales.

Mais, par $\sigma$-additivité de $\mu$ : $\mu\left(\bigsqcup\limits_{n\in\mathbb{N}}\tilde{A_{n}}\right) = \sum_{n\in \mathbb{N}}\mu\left(\tilde{A_{n}}\right)$

Par croissance/positivité : $\sum_{n\in \mathbb{N}}\mu\left(\tilde{A_{n}}\right) \leq \sum_{n\in \mathbb{N}}\mu\left(A_{n}\right)$.

Donc: $\mu \left(\bigcup\limits_{n\in\mathbb{N}}A_{n}\right)\leq \sum_{n\in \mathbb{N}}\mu\left(A_{n}\right)$

\subsection{Exercice 2}
Il est clair que $\mathcal{G}$ est stable par intersections finies et contient $\emptyset$ et $E$. Par ailleurs, si $A_{i}$ est une famille d'éléments de $\mathcal{G}$, en posant $\tilde{A_{n}}$ l'union des $n$ premiers $A_{i}$, la suite $\tilde{A_{i}}$ est une suite croissante d'éléments de $\mathcal{G}$

\subsection{Exercice 3}
\subsubsection{Question 1}
$\liminf\limits_{n\to +\infty} A_{n}$ est l'ensemble des éléments qui sont dans tous les $A_{k}$ àpcr.

$\limsup\limits_{n\to +\infty} A_{n}$ est l'ensemble des éléments qui sont apparaissent une infinité de fois dans les $A_{k}$.

\subsubsection{Question 2}
On a : $\left(\liminf\limits_{n \to + \infty} A_{n}\right)^{\complement}\left(\bigcup\limits_{n \geq 1} \bigcap\limits_{k \geq n} A_{k}\right)^{\complement} = \left(\bigcap\limits_{n \geq 1} \left(\bigcap\limits_{k \geq n} A_{k}\right)^{\complement}\right) = \left(\bigcap\limits_{n \geq 1} \bigcup\limits_{k \geq n} A_{k}^{\complement}\right) = \limsup\limits_{n\to +\infty} A_{n}^{\complement} $

\subsubsection{Question 3}
On a : $1_{\liminf A_{n}} = \liminf 1_{A_{n}}$. De même pour $\limsup$

\subsubsection{Question 4}
Par 1. on a : 
\begin{enumerate}
    \item $\liminf A_{n} = F \cup G$ et $\limsup A_{n} = F \cap G$
    \item $\liminf A_{n} = \left]0, 3\right] \cup \left[-1, 2\right] = \left[-1, 3\right]$ et $\limsup A_{n} = \left]0, 2 \right]$
\end{enumerate}

\subsubsection{Question 5}
Par continuité croissante : $\mu\left(\bigcup\limits_{n \geq 1} \bigcap\limits_{k\geq n}A_{k} \right) = \sup\limits_{n} \mu\left(\bigcap\limits_{k \geq n} A_{k}\right) \leq \liminf\limits_{n} \mu\left(A_{n}\right)$

On pose : $A_{k} = \left[k; k+1\right]$. Alors, \qedsymbol.

\subsubsection{Question 6}
Par continuité décroissante : $\mu\left(\limsup\limits_{n}A_{n}\right) = \inf\limits_{n} \mu\left(\bigcup\limits_{k \geq n} A_{k}\right)$
Par $\sigma$-additivité, on obtient bien le résultat.



\end{document}