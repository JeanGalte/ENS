\documentclass{cours}

\title{Intégration et Probabiliéts}

\begin{document}
\section{Espaces Mesurés}
    \subsection{Ensembles Mesurables}
    \subsection{Mesures Positives}
    \subsection{Fonctions Mesurables}
    \begin{theorem}
        La composition de deux applications mesurables est mesurable.
    \end{theorem}
    \begin{remark}[Composition Mesurable]
        Il faut bien que les applications $f$ et $g$ partagent un espace, avec la \emph{même} tribu (comme la chanson).
        On définit fréquemment deux tribus différentes sur $\mathbb{R}^{d}$ : la tribu borélienne et la tribu de Lebesgue, tribu complétée de la tribu borélienne pour la mesure de Lebesgue $\mathcal{M}(\lambda) = \left\{A \subset \mathbb{R}^{d}, \exists B_{1}, B_{2} \in \mathcal{B}(\mathbb{R}^{d}), B_1 \subset A \subset B_2 \text{ et } \lambda(B_2 \setminus B_1) = 0 \right\}$
        et on a : $B(\mathbb{R}^{d}) \subsetneq \mathcal{M}(\lambda)$. Dans certains livres : $f$ est mesurable si $f : \left(\mathbb{R}, \mathcal{M}(\lambda)\right) \rightarrow \left(\mathbb{R}, \mathcal{B}(\mathbb{R})\right)$ est mesurable.
    \end{remark}
    \begin{proposition}
        Pour que $f$ soit mesurable, il suffit qu'il existe une sous-classe engendrant $\mathcal{B}$ pour laquelle la propriété est vraie.
    \end{proposition}
    \begin{corollary}
        Si $f : \mathbb{R}^{d_1} \rightarrow\mathbb{R}^{d_2}$ est continue, elle est mesurable pour les boréliens.
    \end{corollary}
    \begin{corollary}
        Une application produit est mesurable.
    \end{corollary}
    \begin{proof}
        On a : $A_{1} \bigotimes A_{2} = \sigma\left(A_{1} \times A_{2}\right)$
    \end{proof}
    \begin{lemma}
        Les applications $(+) (\times) (\max) (\min)$ de deux fonctions réelles sont mesurables
    \end{lemma}
    \begin{corollary}
        Les parties positives et négatives d'une fonction sont mesurables
    \end{corollary}
    \begin{proposition}
        Si les $f_n$ sont mesurables de $E$ dans $\overline{\mathbb{R}}$ alors : $\sup_n f_n, \inf_n f_n, \liminf f_n, \limsup f_n$ sont mesurables.
        En particulier : $\lim_n f_n$ est mesurable si la suite CS.
    \end{proposition}
    \begin{proof}
        \begin{enumerate}
            \item Si $f(x) = \inf f_{n}(x)$ : $f^{-1}\left[-\infty, a\right[ = \bigcup_{n} \left\{x \mid f_{n}(x) < a\right\}$. De même pour $\sup$. On en déduit immédiatement $\liminf f_n = \sup_{n \geq 0} \inf_{k \geq n} f_{k}$.
            \item On a : $\left\{x\in E\mid \lim f_{n}(x) \text{ existe}\right\} = \left\{x \in E \mid \liminf f_{n}(x) = \limsup f_{n}(x)\right\} = \mathcal{G}^{-1}(\Delta)$ où $\mathcal{G} = (\liminf f_{n}, \limsup f_{n})$ et $\Delta$ est la diagonale de $\overline{\mathbb{R}}^{2}$.
        \end{enumerate}
    \end{proof}
    \begin{definition}[Mesure-Image]
        On appelle mesure image de $\mu$ par $f$, notée $f_{\#}\mu$ la mesure $f_{\#}\mu(B) = \mu(f^{-1}(B))$
    \end{definition}

    \subsection{Classe Monotone}
    \begin{definition}[Classe Monotone]
        $\mathcal{M} \in \mathcal{P}(E)$ est une classe monotone si :
        \begin{enumerate}
            \item $E\in \mathcal{M}$
            \item Si $A, B \in \mathcal{M}$ avec $A\subset B$, $B \setminus A \in \mathcal{M}$
            \item Si $(A_{n}) \in \mathcal{M}^{\mathbb{N}}$ croissante, $\bigcup\limits_{n\in\mathbb{N}} A_{n} \in \mathcal{M}$
        \end{enumerate}
    \end{definition}

    \begin{remark}
        Toute tribu est une classe monotone
    \end{remark}

    \begin{lemma}
        Si $\mathcal{M}$ est une classe monotone stable par intersections finies, c'est une tribu.
    \end{lemma}
    \begin{definition}
        Si $\mathcal{C} \subset \mathcal{P}(E)$ : $\mathcal{M}(\mathcal{C}) = \bigcap\limits_{\mathcal{M} \text{classe monotone, } \mathcal{C}\subset\mathcal{M}}$
    \end{definition}
    \begin{theorem}[Lemme de Classe Monotone]
        Si $\mathcal{C} \subset \mathcal{P}(E)$ est stable par intersections finies : $\mathcal{M}(\mathcal{C}) = \sigma{\mathcal{C}}$
    \end{theorem}
    \begin{remark}
        Les classes monotones sont des outils plus maniables que les tribus et se marient mieux avec les propriétés des mesures. Le théorème fait le lien entre tribus et classes monotones, ce qui facilite la vie avec les mesures. 
    \end{remark}
    \begin{proof}
        Point Méthodologique : ne pas essayer d'exprimer des éléments de $\mathcal{C}$. T'façon les preuves constructives, c'est pour les salopes. 
    \end{proof}
    \begin{remark}
        On peut en déduire l'unicité de la mesure de Lebesgue. C'est une conséquence du théorème suivant.
    \end{remark} 
    \begin{theorem}
        Soit $\mathcal{C}$ stable par intersections telle que $\sigma{\mathcal{C}} = \mathcal{A}$.
        On suppose $\mu_{1}(A) = \mu_{2}(A), \forall A \in \mathcal{C}$
        Alors : \begin{enumerate}\item $\mu p.p.$, l'application $u \mapsto f(u, x)$ est continue en $u_{0}$
            \item Si $\mu_{1}(E) = \mu_{2}(E) < +\infty$ alors $\mu_{1} = \mu_{2}$
            \item S'il existe $(E_n) \in \mathcal{A}^{\mathbb{N}}$ croissante d'union $E$ et de mesures égales et finies par $\mu_{1}$ et $\mu_{2}$ alors $\mu_{1} = \mu_{2}$
        \end{enumerate}
    \end{theorem}
    \begin{proof}
        \begin{enumerate}
            \item Cas fini : $\mathcal{M} = \left\{A \in \mathcal{A} \mid \mu_{1}(A) = \mu_{2}(A)\right\}$ est une classe monotone. Donc $\mathcal{M} = \mathcal{A}$ par Lemme de Classe Monotone
            \item Cas Infini : On applique le cas fini à $E_{n}$ en prenant la restriction. Par continuité croissante, on obtient bien le résultat. 
        \end{enumerate}
    \end{proof}
    
\section{Intégration par rapport à une mesure}
    \subsection{Intégration Positive}
    \begin{definition}
        $f$ mesurable à valeurs réelles est étagée si elle prend un nombre fini de valeurs. 
    \end{definition}
    \begin{remark}
        \begin{enumerate}
            \item Les Fonctions en escalier sur un intervalle sont étagées
            \item L'indicatrice d'un ensemble est étagée, en particulier : $1_{\mathbb{Q}}$ est étagée.
        \end{enumerate}
    \end{remark}
    \begin{definition}
        Si $f$ est étagée, et prend les valeurs : $\alpha_{1} < \ldots < \alpha_{n}$, l'écriture canonique de $f$ est, avec $A_{i} = f^{-1}(\left\{\alpha_{i}\right\})$ : 
        $f =  \sum\limits_{i}^{n} \alpha_{i}1_{A_{i}}$\\
        Pour $f$ mesurable, on pose alors : $\int f \mathrm{d}\mu = \sup\limits_{h \in \mathcal{E}, h \leq f} g \mathrm{d}\mu$
    \end{definition}
    \begin{proposition}
        L'intégrale est une forme linéaire monotone i.e. l'intégrale d'une fonction positive est positive. Ceci s'étend aux fonctions intégrables.
    \end{proposition}
    \begin{theorem}[De Convergence Monotone]\label{TCM}
        Si $f_{n}$ est croissante positive et tend vers $f$ :
        \[
            \int f \mathrm{d}\mu = \lim_{n\to \infty} \uparrow \int f_{n} \mathrm{d} \mu
        \]
    \end{theorem}
    \begin{proof}
        Par croissance : \[ \int f \mathrm{d}\mu \geq \lim_{n \to \infty}\uparrow \int f_{n} \mathrm{d}\mu.\]
        Il suffit donc d'établir l'autre inégalité : soit $h = \sum \alpha_{i}\mathds{1}_{A_{i}}$ étagée positive inférieure à $f$. Soit $a \in [0, 1[$. On pose: \[E_{n} =\left\{x \in E \mid ah(x) \leq f_{n}(x)\right\}\]
        Les $E_{n}$ sont mesurables et, puisque $a < 1$ et $f = \lim \uparrow f_{n}$, $E = \bigcup \uparrow E_{n}$.\\
        Comme $f_{n} \geq a\mathds{1}_{E_{n}}h$, \[\int f_{n} \mathrm{d}\mu \geq \int \mathds{1}_{E_{n}}h = a \sum \alpha_{i}\mu(A_{i}\cap E_{n})\]
        En passant à la limite croissante, comme $A_{i} \cap E_{n} \uparrow A_{i}$, puis en faisant tendre $a$ vers 1 on trouve le résultat. 
    \end{proof}

    \begin{theorem}
        Soit $f$ mesurable positive. Il existe une suite croissante de fonctions étagées positives de limite $f$.
    \end{theorem}
    \begin{corollary}
        Si les $f_{n}$ sont positives : \[
          \sum_{n} \int f_{n} \mathrm{d}\mu = \int f_{n} \sum_{n }\mathrm{d}\mu
        \]
    \end{corollary}
    \begin{theorem}[Lemme de Fatou]
        Si les $f_{n}$ sont mesurables positives: 
        \[
            \int \liminf f_{n} \mathrm{d}\mu \leq \liminf \int f_{n} \mathrm{d}\mu
        \]
    \end{theorem}

    \begin{proof}
        On a : \[
            \liminf f_{n} = \lim_{k \to \infty} (\inf_{n \geq k} f_{n})        \]
        Donc, par théorème de convergence monotone \ref{TCM} : \[
            \int \liminf f_{n} \mathrm{d}\mu = \lim_{k \to \infty} \int \left(\inf_{n \geq k} f_{n}\right) \mathrm{d}\mu
        \]
        Par ailleurs, si $p \geq k$, on a : $\inf\limits_{n \geq k} f_{n} \leq f_{p}$, d'où : \[
            \int \left(\inf_{n \geq k} f_{n}\right) \mathrm{d}\mu \leq \inf_{p\geq k} \int f_{p} \mathrm{d}\mu
        \]
        En passant à la limite croissante quand $k \to \infty$, on a : \[   
            \lim_{k \to \infty} \int \left(\inf_{n \geq k} f_{n}\right)\mathrm{d}\mu \leq \lim_{k \to \infty} \inf_{p \geq k} \int f_{p} \mathrm{d}\mu = \liminf \int f_{n} \mathrm{d}\mu
        \]
        ce qui conclut.
    \end{proof}

    \begin{proposition}
        Soit $f$ mesurable positive :
        \begin{itemize}
            \item $\forall a > 0, \mu \left(\left\{x \in E \mid f(x) \geq a\right\}\right) \leq \frac{1}{a}\int f \mathrm{d} \mu$ [Inégalité de Markov]
            \item $\int f \mathrm{d}\mu < \infty \Rightarrow f < \infty p.p.$
            \item $\int f \mathrm{d}\mu \Leftrightarrow f = 0 p.p.$
            \item $f = g p.p. \Rightarrow \int f \mathrm{d}\mu = \int g \mathrm{d}\mu$
        \end{itemize}
        On peut généraliser cette dernière proposition aux fonctions intégrables.
    \end{proposition}

    \subsection{Fonctions Intégrables}
    \begin{definition}
        On dit qu'une fonction est intégrable si l'intégrale de sa norme est finie. En ce cas, l'intégrale de la fonction est la somme des intégrales de ses fonctions composantes.
    \end{definition}
    \begin{theorem}
        L'intégrale vérifie l'inégalité triangulaire : Soit $f$ intégrable
        \[
            \int \abs{f} \mathrm{d}\mu \geq \abs{\int f \mathrm{d}\mu}
        \]
    \end{theorem}
    \begin{proof}
        Ecrire $\abs{z}^{2} = z\overline{z}$ pour $z$ l'intégrale de $f$. En notant $a$ le conjugué de $z$, on a le résultat.
    \end{proof}

    \begin{theorem}[De Convergence Dominée]\label{TCD}
        Soit $f_{n}$ une suite de fonctions mesurables à valeurs dans $\R$ ou $\C$. On suppose :
        \begin{enumerate}
            \item Il existe $f$ mesurable telle que $f_{n}(x) \to f(x) \mu p.p.$
            \item Il existe $g$ telle que $\abs{f_{n}(x)} \leq g(x) \mu p.p.$
        \end{enumerate}
        Alors : 
        \[
            \lim_{n\to \infty} \int \abs{f_{n}(x) - f(x)} \mu(\mathrm{d}x) = 0
        \]
        et 
        \[
            \lim_{n \to \infty} \int f_{n} \mathrm{d}\mu = \int f \mathrm{d}\mu   
        \]
    \end{theorem}

    \begin{proof}
        \begin{enumerate}
            \item On suppose dans un premier temps que les hypothèses $(1)$ et $(2)$ sont vérifiées sur tout l'ensemble. On remarque $\abs{f} \leq g$ donc $f$ est intégrable. \\
            Ensuite, puisque par inégalité triangulaire, $\abs{f - f_{n}} \leq 2g$ et $\abs{f - f_{n}} \to 0$, par lemme de Fatou : 
            \[
                \liminf \int \left(2g - \abs{f - f_{n}}\right) \mathrm{d}\mu \geq \int \liminf \left(2g - \abs{f - f_{n}}\right) \mathrm{d}\mu = 2 \int g \mathrm{d}\mu.
            \]
            Par linéarité, comme $\liminf -u_{n} = -\limsup u_{n}$ : 
            \[
                2 \int g \mathrm{d}\mu - \limsup \int \abs{f - f_{n}} \mathrm{d}\mu \geq 2 \int g \mathrm{d}\mu
            \]
            D'où, $\int \abs{f - f_{n}} \mathrm{d}\mu \to 0$. Par inégalité triangulaire: 
            \[
                \abs{\int f \mathrm{d}\mu - \int f_{n}\mathrm{d}\mu} \leq \int \abs{f - f_{n}} \mathrm{d}\mu
            \]
            \item Dans le cas général, on suppose cette fois ci $(1)$ et $(2)$. On pose alors : 
            \[
                A = \left\{x \in E \mid f_{n}(x) \to f(x) \text{ et pour tout n } \abs{f_{n}(x)}\leq g(x)\right\}.
            \]
            Par hypothèses, $\mu\left(A^{\complement}\right) = 0$ et par la première partie de la preuve appliquée à : 
            \[
                \tilde{f}_{n}(x) = \mathds{1}_{A}(x)f_{n}(x),  \tilde{f}(x) = \mathds{1}_{A}(x)f(x)
            \]
            Comme $f = \tilde{f} p.p.$ et $f_{n} = \tilde{f}_{n} p.p.$, on a bien le résultat par la première partie de la preuve. 
        \end{enumerate}
    \end{proof}
    
    \subsection{Intégrales dépendant d'un Paramètre}
    Principe : Utiliser le TCD pour montrer des propriétés de régularité.
    \begin{remark}[Exemples d'utilisation]
        \begin{itemize}
            \item Pour $f$ intégrable à variables dans $\R^{d}$, on définit la transformée de fourier $\hat{f}$ par : \[   
                \hat{f}(\xi) = \int_{\R^{d}} \exp{\left(-\xi \cdot x\right)}f(x)\mathrm{d}x\]
            \item Soit $f$ intégrable à variables dans $\R^{d}$, $g$ continue bornée à variables dans $\R^{d}$. On définit la convolée de $f$ et $g$ par : 
            \[f \star g : x \mapsto \int_{\R^{d}} f(y)g(x - y)\mathrm{d}y \]
        \end{itemize}
    \end{remark}  

    \begin{theorem}[De Continuité sous l'intégrale]
        Soit $U$ un ouvert de $\R^{d}$, $u_0 \in U$. Soit $f : U \times E \to \R$ vérifiant : \begin{enumerate}
            \item $\forall u \in U$, l'application $x \in E \mapsto f(u, x)$ est mesurable.
            \item \item $\mu p.p.$, l'application $u \mapsto f(u, x)$ est continue en $u_{0}$
            \item Il existe une application $g : E \to \left[0, +\infty\right[$, intégrable telle \[\forall u \in U, \mu p.p., \abs{f(u, x)} \leq g(x)\]
            
        \end{enumerate}
        Alors, $F(u) = \int f(u, x) \mu(\mathrm{d}x)$ est bien définie et est continue en $u_{0}$.
    \end{theorem}
    \begin{proof}
        Par l'hypothèse $(iii.)$, $F$ est bien définie.\\
        Soit $(u_{n})$ une suite de limite $u_{0}$. Par $(ii.)$, $f(u_{n}, x) \to_{n\to \infty} f(u_{0}, x), \mu p.p.$. Par $(iii.)$, on peut appliquer le théorème de convergence dominée \ref{TCD}, ce qui donne le résultat par caractérisation séquentielle de la limite. 
    \end{proof}
    \begin{corollary}
        Les fonctions définies en exemple sont continues.
    \end{corollary}

    \begin{theorem}[De Dérivation sous l'intégrale]
        Soit $U$ un ouvert de $\R$, $u_0 \in U$. Soit $f : U \times E \to \R$ vérifiant : \begin{enumerate}
            \item $\forall u \in U$, $x \mapsto f(u, x)$ est mesurable
            \item $\mu p.p.$, $u \mapsto f(u, x)$ est dérivable en $u_{0}$ 
            \item il existe $g$ positive intégrable, telle que $\mu p.p.$ : \[\forall u \in U, \abs{f(u, x) - f(u_{0}, x)} \leq g(x)\abs{u - u_{0}}\]
        \end{enumerate}
        Alors : \[F(u) = \int f(u, x) \mu(\mathrm{d}x) \text{ est dérivable et } F^{'}(u_{0}) = \int \frac{\partial f}{\partial u}(u_{0}, x)\mu(\mathrm{d}x)\]
    \end{theorem}
    \begin{proof}
        On applique le TCD \ref{TCD} à $\phi_{n}(x) = \frac{f(u_{n}, x) - f(u_{0}, x)}{u_{n} - u_{0}}$ où $u_{n} \to u_{0}$.
    \end{proof}
    \begin{corollary}
        En choisissant pour la transformée de fourier des fonctions dont les premiers moments sont intégrables, on gagne en régularité. En particulier, si une fonction est à support compact, sa transformée de fourier est indéfiniment dérivable. 
    \end{corollary}

    \subsection{Mesures Produits}
        On se donne $(E_{1}, \mathcal{A}_{1}, \mu_{1})$ et $(E_{2}, \mathcal{A}_{2}, \mu_{2})$, et on veut :
        \begin{enumerate}
            \item Définir une mesure produit sur $\left(E_{1} \times E_{2}, \mathcal{A}_{1} \bigotimes \mathcal{A}_{2}\right)$
            \item Démontrer les théorèmes de Fubini sur la mesure ainsi définie. 
        \end{enumerate}
        \subsubsection{Préliminaires}
        \begin{definition}
            Soit $C \in \mathcal{A}_{1} \bigotimes \mathcal{A}_{2}, x_{1} \in E_{1}, x_{2} \in E_{2}$. On note : 
            \[
                \begin{aligned}
                    C_{x_{1}} &= \left\{y \in E_{2}\mid (x_{1}, y) \in C\right\}\\
                    C_{x_{2}} &= \left\{y \in E_{1}\mid (y, x_{2}) \in C\right\}\\
                \end{aligned}
            \]
            On définit par ailleurs, si $f : \left(E_{1} \times E_{2}, \mathcal{A}_{1} \bigotimes \mathcal{A}_{2}\right) \to \left(\R, \mathcal{B}(\R)\right)$, les applications partielles $f_{x_{1}}$ et $f^{x_{2}}$.
        \end{definition}
        \begin{lemma}
            \begin{itemize}
                \item $\forall C \in \mathcal{A}_{1} \bigotimes \mathcal{A}_{2}, x_{1} \in E_{1}, x_{2} \in E_{2}$, $C_{x_{1}} \in A_{2}$ et $C^{x_{2}} \in A_{1}$
                \item $f_{x_{1}}$ et $f^{x_{2}}$ sont mesurables.
            \end{itemize}
        \end{lemma}
        \begin{proof}
            \begin{itemize}
                \item On introduit la classe des $C_{x_{1}}$ pour $x_{1} \in E_{1}$. Cette classe contient les pavés et est une tribu.
                \item 
            \end{itemize}
            
        \end{proof}
\end{document}
