\documentclass{cours}

\begin{document}
\section{chaipukoi}
\subsection{Extensions et Cohomologie}
Si $A$ et $G$ sont fixés, on veut classifier les suites exactes courtes :
\begin{equation}\label{Extension}
    1 \rightarrow A \xrightarrow{i} \tilde{G} \xrightarrow{\pi} G \rightarrow 1   
\end{equation}
Etant donné une telle sec est-ce que $i(A)$ admet un complément dans $\tilde{G}$?

\begin{lemma}
    Soit une extension comme ci-dessus. Il y a équivalence entre : 
    \begin{enumerate}
        \item $i(A)$ admet un complément dans $\tilde{G}$
        \item $\pi$ admet une section ensembliste qui est un morphisme de groupes. 
    \end{enumerate}
\end{lemma}

\begin{definition}
    On dit que la suite exacte courte est scindée si les conditions équivalentes du lemme précédent sont satisfaites. 
\end{definition}

\begin{theorem}[Schur-Zassenhaus]
    Toute extension de $G$ par $A$ avec $\abs{G} \wedge \abs{A} = 1$ est scindée
\end{theorem}

Dans la suite, on suppose que $A$ est abélien.

\begin{definition}
    Un $G$-module est la donnée d'un groupe abélien $(A, +)$ muni d'une action de $G$ sur $A$ vérifiant $g(a + b) = ga + gb$ ou, ce qui revient au même, telle que le morphisme $G \rightarrow S_{A}$ associé soit à valeurs dans $Aut(A)$.
\end{definition}

\begin{proposition}
    La donnée de la suite exacte courte \ref{Extension} munit le groupe abélien $A$ d'une structure de $G$-module par : 
    \[
        g.a = i^{-1}\left(\tilde{g}i(a)\tilde{g}^{-1}\right)    
    \]
    où $\tilde{g} \in \tilde{G}$ est un relevé de $g$ par $\pi$.
\end{proposition}

\begin{example}[Extensions Centrales]
    Une extension \ref{Extension} de $G$ par $A$ est dite \emph{centrale} si $i(A) \subseteq Z(\tilde{G})$.  
\end{example}

On fixe une extension \ref{Extension} de $G$ par $A$, et on considère une section ensembliste $s : G \rightarrow \tilde{G}$. Il existe un unique élément $c(g, g^{'})$ dans $A$ tel que : 
\[
    s(g)s(g^{'}) = i(c(g, g^{'}))s(gg^{'})
\]
On remarque qu'alors $s$ est un morphisme si et seulement si $c = Ob(s)$ est nulle.

\begin{lemma}
    Soient $s$ une section ensembliste de $\pi$ et $c = Ob(s)$. On a : 
    \[
        g.c(g^{'}g^{''}) - c(gg^{'}, g^{''}) + c(g, g^{'}g^{''}) - c(g, g^{'}) = 0, \ \forall g, g^{'}, g^{''} \in G
    \]
\end{lemma}

\begin{definition}
    Si $A$ est un $G$-module, on note $Z^{2}(G, A)$ l'ensemble des fonctions vérifiant l'identité du lemme précédent. Une telle fonction est appelée $2$-cocycle de $G$ à valeurs dans $A$.
\end{definition}

Une autre section de $\pi$ que $s$ est de la forme $s_{\epsilon} : g \mapsto i(\epsilon(g))s(g)$ où $\epsilon$ est une fonction arbitraire de $G$ dans $A$. Les deux $2$-cocycles $c = Ob(s)$ et $c_{\epsilon} = Ob(s_{\epsilon})$ sont alors liés par : 
\[
    c_{\epsilon}(g, g^{'}) = c(g, g^{'}) + g.\epsilon(g^{'}) - \epsilon(gg^{'}) + \epsilon(g)    
\]

\begin{definition}
    Si $A$ est un $G$-module, on note $B^{2}(G, A)$ l'ensemble des fonctions $\partial\epsilon : G \times G \rightarrow A$ de la forme $g, g^{'} \mapsto g.\epsilon(g^{'}) - \epsilon(gg^{'}) + \epsilon(g)$ avec $\epsilon : G\rightarrow A$. Une telle fonction $f$ est appelée $2$-cobord de $G$ à valeurs dans $A$.
\end{definition} 

\begin{definition}
    Pour tout $G$-module $A$, le groupe $B^{2}(G, A)$ est un sous-groupe de $Z^{2}(G, A)$ et on définit le $2$-ème groupe de cohomologie de $G$ à valeurs dans $A$ comme le groupe abélien quotient : 
    \[
        H^{2}(G, A) = Z^{2}(G, A)/B^{2}(G, A)
    \]
\end{definition}

\begin{proposition}
    Si $s$ est une section de $\pi$, la classe de $Ob(s)$ ne dépend pas du choix de la section $s$. On la note $\left[E\right]$ et on l'appelle classe de cohomologie assoicée à \ref{Extension}. La sec \ref{Extension} est scindée si, et seulement si sa classe $\left[E\right]$ est nulle. 
\end{proposition}

\begin{theorem}[Schur-Zassenhaus, Cohomologique]
    Soient $G$ un groupe et $A$ un $G$-module : 
    \begin{enumerate}
        \item Si $G$ est fini, alors $\abs{G}x = 0$ pour tout $x \in H^{2}(G, A)$
        \item Si $A$ est fini, alors $\abs{A}x = 0$ pour tout $x \in H^{2}(G, A)$
    \end{enumerate}
    En particulier, si $G$ et $A$ sont finis avec $\abs{G} \wedge \abs{A} = 1$, on a $H^{2}(G, A) = 0$.
\end{theorem}

\begin{proposition}
    Pour tout $G$-module $A$ et $x \in H^{2}(G, A)$, il existe une extension de $G$ par $A$ vérifiant $\left[E\right] = x$. 
\end{proposition}
\begin{proposition}
    Soient $A$ un $G$-module et $E_{k} = \left(\tilde{G}_{k}, i_{k}, \pi_{k}\right)$ pour $k = 1, 2$ deux extensions de $G$ par le même $G$-module $A$. On a $\left[E_{1}\right] = \left[E_{2}\right]$ si et seulement si il existe un isomorphisme $\phi : \tilde{G}_{1} \rightarrow \tilde{G}_{2}$ vérifiant $\phi \circ i_{1} = i_{2}$ et $\pi_{2} \circ \phi = \pi_{1}$
\end{proposition}
\begin{corollary}
    Soit $A$ un $G$-module, l'application $(E) \mapsto \left[E\right]$ induit une bijection entre l'ensemble $\mathcal{E}(G, A)$ des classes d'isomorphisme d'extensions de $G$ par le $G$-module $A$ et l'ensemble $H^{2}(G, A)$.
\end{corollary}

\begin{theorem}[Schur]
    Considérons $\znz{2}$ comme $A_{n}$-module trivial. On a, pour $n \geq 4$ : 
    \[
        H^{2}(A_{n, \znz{2}}) \simeq \znz{2}
    \]
\end{theorem}


\end{document}