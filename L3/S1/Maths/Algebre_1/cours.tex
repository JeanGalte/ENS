\documentclass{cours}
\title{Algèbre 1}
\author{Gaëtan Chenevier}
\date{\today}

\begin{document}
\newpage
\section{Ensembles Quotients}
\subsection{Partitions et Relations d'Equivalence}
\begin{definition}
    Une \emph{partition} d'un ensemble $X$ est un ensemble de parties non vides de $X$ de réunion disjointe $X$. 
\end{definition}
\begin{definition}
    On appelle \emph{fibre} d'une application $f : X \rightarrow Y$ en $y \in Y$ l'ensemble $f^{-1}(y) = \left\{x \in X \mid f(x) = y\right\}$. Il s'agit d'une partition de $X$ indexée par $Y$. Toute partition de $X$ s'obtient ainsi. 
\end{definition}

\begin{definition}
    Une \emph{relation d'arité} $n$ sur un ensemble $X$ est la donnée d'un ensemble $R \subseteq X^{n}$. Une relation binaire $R$ i.e. une partie de $X\times X$ est dite d'\emph{équivalence} si elle est réflexive, transitive et symétrique. On appelle \emph{classe de $R$-équivalence} de $x$ l'ensemble $\left[x\right]_{R} = \left\{y \in X \mid \left\{x, y\right\} \in R\right\}$
\end{definition}

\begin{proposition}\label{partclassdeq}
    Les classes d'équivalences d'une relation $R$ sur $X$ forment une partition de $X$. 
\end{proposition}

\begin{definition}
    Si $R$ est une relation d'équivalence sur $X$, le sous-ensemble de $P(X)$ constitué des classes de $R$-équivalence est appelé \emph{ensemble quotient} de $X$ par $R$, noté $X/R$. L'application $\pi_{R} : X \rightarrow X/R, x \mapsto \left[x\right]_{R}$ est appelée \emph{projection canonique} associée à $R$. C'est une surjection dont les fibres sont par définition les classes d'équivalences de $R$.
\end{definition}

\begin{example}
    On définit $\Z/n\Z$ l'ensemble quotient de $\Z$ pour la relation $n \mid b - a$. On note $\overline{k}$ la classe de $k$.
\end{example}

\subsection{Passage au Quotient}
\begin{theorem}[Propriété Universelle du Quotient]\label{proprietequotient}
    Soient $f : X \rightarrow Y$ une application et $R$ une relation d'équivalence sur $X$. On suppose que $f$ est constante sur chaque classe d'équivalence sur $X$. Alors, il existe une unique application $g : X/R \rightarrow Y$ telle que $g\left(\left[x\right]_{R}\right) = f(x)$ pour tout $x \in X$, i.e. vérifiant $g \circ \pi_{R} = f$.
\end{theorem}
\begin{proof}
    Par surjectivité de $\pi_{R}$, $g$ est unique. De plus, si $C$ est une classe de $R$-équivalence, il y a un sens à poser $g(C) = f(x)$ car $C$ est une classe d'équivalence sur laquelle $f$ est constante.
\end{proof}

\subsection{Sections et systèmes de représentants}
\begin{definition}
    Une \emph{section} de $f : X \rightarrow Y$ est une application $s : Y \rightarrow X$ telle que $f \circ s = \textmd{id}_{Y}$
\end{definition}
\begin{proposition}
    $f$ possède une section $\Rightarrow$ $f$ est surjective
\end{proposition}

\begin{definition}[Axiome du Choix]\label{AC}
    Pour tout ensemble $X$ il existe une application $\tau : P(X) \setminus \left\{\emptyset\right\} \rightarrow X$ telle que $\tau(E) \in E$ pour toute partie non vide $E$ de $X$. On appelle $\tau$ fonction de choix sur $X$. 
\end{definition}

\begin{proposition}
    Les propositions suivantes sont équivalentes à l'axiome du choix (donc fausses): 
    \begin{enumerate}
        \item Toute surjection admet une section.
        \item Pour toute famille d'ensembles non vides $\left\{X_{i}\right\}_{i\in I}$, $\pi_{i\in I}X_{i}$ est non vide.
    \end{enumerate}
\end{proposition}

\begin{definition}
    Un \emph{représentant} d'une classe de $R$-équivalence d'un ensemble $X$ est un élément de cette classe. Un \emph{système de réprésentants} de $\left(X, R\right)$ est la donnée d'une partie de $X$ contenant un et un seul représentant de chaque classe de $R$-équivalence. C'est l'image d'une section de $\pi_{R}$.
\end{definition}
\begin{remark}
    Ceci est également équivalent à \ref{AC}
\end{remark}

\subsection{Lemme de Zorn}
\begin{definition}
    \begin{itemize}
        \item Un \emph{relation d'ordre} sur un ensemble $X$ est une relation binaire $\leq$ réfléxive, transitive et antisymétrique. On dit alors que $X$ est ordonné.
        \item L'ordre $\leq$ est total quand tous deux éléments de $X$ sont comparables. 
        \item On appelle majorant d'une partie $Y$ de $X$, tout élément $x \in X$ tel que $y \leq x$ pour tout $y \in Y$. On parle de plus grand élément dans le cas $Y = X$.
        \item $x \in X$ est un élément maximal si le seul $y \in X$ tel que $y \leq x$ est $x$. Un plus grand élément est nécessairement maximal, et unique s'il existe.
        \item On appelle $X$ inductif si tout sous-ensemble totalement ordonné admet et majorant.
        \item On appelle bon ordre un ordre pour lequel toute partie non vide admet un plus petit élément.
    \end{itemize}
\end{definition}

\begin{theorem}[Lemme de Zorn]\label{Zorn}
    Un ensemble ordonné inductif possède au moins un élément maximal. Ceci est équivalent à l'axiome du choix \ref{AC}.
\end{theorem}
\begin{corollary}
    Tout espace vectoriel possède une base.
\end{corollary}
\begin{corollary}[Théorème de Zermelo]\label{Zermelo}
    Tout ensemble peut être muni d'un bon ordre.
\end{corollary}
\begin{proof}
    C'est équivalent à l'axiome du choix donc faux et les preuves prennent trois plombes.
\end{proof}


\newpage
\section{Généralités sur les Groupes}
\subsection{Exemples de Groupes}
\begin{definition}
    Une \emph{loi de composition interne} est une application $\star : X \times X \rightarrow X$.
\end{definition}
\begin{definition}[Groupe]
    Un groupe est un ensemble $G$ muni d'une loi de composition associative, unifère et inversible, i.e.:
    \begin{enumerate}
        \item $\forall \ (x, y, z) \in G, \ x \star (y \star z) = (x \star y) \star z$ 
        \item $\exists \ e \in G,\ \forall x \in G, \ e\star x = x \star e = x$.
        \item $\forall x \in G, \ \exists y \in G, \ x \star y = y \star x = e$
    \end{enumerate}
\end{definition}

\begin{remark}
    Le neutre est unique.
\end{remark}

\begin{example}[Groupe Symétrique]
    On note : $\mathfrak{S}_{X} = X^{X}$ le groupe muni de la loi $\circ$ de composition des applications, appelé \emph{groupe symétrique} de $X$, de neutre $\textmd{id}_{X}$. L'inverse d'une bijection $\sigma$ est sa bijection réciproque $\sigma^{-1}$. On note $\mathfrak{S}_{n} = \lvert 1, n\rvert^{\lvert 1, n\rvert}$ et alors $\abs{\mathfrak{S}_{n}} = n!$.
\end{example}

\begin{definition}
    Un groupe est dit \emph{abélien} lorsque tous deux élements commutent.
\end{definition}

\begin{definition}
    Une partie $H$ d'un groupe $G$ est un \emph{sous-groupe} de $G$ lorsque la loi induite par le produit dans $G$ fait de $H$ un groupe. On le notera ici $H \leq G$.
\end{definition}

\begin{example}[Groupes d'ordre n]
        Pour $n \geq 1$, on note $\mu_{n}$ le sous-groupe de $\C^{\times}$ composé des racines $n$-ièmes de l'unité. C'est un sous-groupe d'ordre $n$. L'application $\Z/n\Z \rightarrow \mu_{n}, \overline{k} \mapsto e^{2ik\pi/n}$ est un isomorphisme de groupe. 
\end{example}

\begin{definition}
    Un \emph{anneau} est un groupe abélien $(A, +)$ muni d'une loi associative unifère et distributive sur $+$, notée $\times$. Il est dit commutatif lorsque la loi produit est commutative.
\end{definition}

\begin{definition}
    On note $A^{\times}$ le groupe des inversibles du monoïde $(A, \cdot)$.
\end{definition}

\begin{proposition}
    La loi d'un groupe vérifie les propriété de la loi produit usuelle sur $\R$.
\end{proposition}

\begin{definition}
    On appelle groupe engendrée par une partie $X$ de $G$ le plus petit sous groupe de $G$ contenant $X$. C'est l'ensemble des produits de puissances d'éléments de $X$.
\end{definition}

\subsection{Morphismes}
\begin{definition}
    On appelle \emph{morphisme} une application entre deux groupes qui préserve le produit. On note $Hom(G, G^{'})$ l'ensemble des morphismes de $G$ dans $G^{'}$. Ce n'est à priori pas naturellement un groupe si $G^{'}$ n'est pas abélien.  \\
    On dit que $G$ et $G^{'}$ sont isomorphes lorsqu'il existe un morphisme bijectif de l'un vers l'autre. La réciproque d'un isomorphisme est un isomorphisme. On note alors $G \simeq G^{'}$.
\end{definition}

\begin{proposition}[Transport de Structure]\label{transport}
    Si $G$ est un groupe, $\phi : X \rightarrow G$ une bijection, il existe une unique loi de groupe sur $X$ telle que $\phi$ soit un isomorphisme, à savoir $x \star y = \phi^{-1}(\phi(x)\phi(y))$. On dit que la loi est \emph{déduite} de celle de $G$ par transport de structure via $\phi$.
\end{proposition}

\begin{definition}
    On appelle \emph{automorphisme} de $G$ un isomorphisme de $G$ dans $G$. L'ensemble des automorphismes $Aut(G)$ est un sous groupe de $S_{G}$. On appelle automorphisme intérieur associé à $g \in G$ l'application : $h \in G \mapsto ghh^{-1}$.
\end{definition}

\begin{definition}
    On appelle \emph{noyau} d'un morphisme $\ker(f) = f^{-1}(1) = \left\{g \in G \mid f(g) = 1\right\}$. C'est un sous-groupe de $G$.
\end{definition}

\begin{proposition}
    Si $f \in Hom(G, G^{'})$ :
    \begin{enumerate}
        \item $H \leq G \Rightarrow f(H) \leq G^{'}$
        \item $H \leq G^{'} \Rightarrow f^{-1}(H) \leq G$
    Avec $\mathcal{A}$ l'ensemble des sous-groupes de $G$ contenant $\ker f$ et $\mathcal{B}$ celui des sous-groupes de $G^{'}$ inclus dans $\textmd{Im} f$, alors :
        \item $\mathcal{A} \rightarrow\mathcal{B}, H \mapsto f(H)$ est une bijection croissante. 
    \end{enumerate}
\end{proposition}

\begin{proposition}
    Les fibres non vides de $f$ sont en bijection avec $\ker f$. En particulier : 
    \begin{itemize}
        \item $f$ injective $\Leftrightarrow$ $\ker f = \left\{1\right\}$.
        \item Si $G$ est fini, $\abs{G} = \abs{\text{Im } f}\abs{\ker f}$.
    \end{itemize}
\end{proposition}

\begin{theorem}[Cayley]
    Tout groupe d'ordre fini $n$ est isomorphe à un sous-groupe de $S_{n}$. 
\end{theorem}

\begin{lemma}
    Si $\phi : X \rightarrow Y$ est bijective, l'application : $\phi_{X, Y} : S_{X} \rightarrow S_{Y}, \sigma \mapsto \phi \circ \sigma \circ \phi^{-1}$ est un isomorphisme de groupes. 
\end{lemma}

\begin{definition}
    Un morphisme d'anneau est un morphisme des groupes additifs et des monoïdes multiplicatifs (en particulier, il envoie $1$ sur $1$).
\end{definition}

\subsection{Groupes Cycliques et Monogènes}
\begin{proposition}
    Les sous-groupes de $\Z$ sont les $n\Z$.
\end{proposition}

\begin{proposition}
    Si $g \in G$ est d'ordre fini $n$, alors $\langle g \rangle$ a exactement $n$ éléments et est isomorphe à $\Z/n\Z$. 
\end{proposition}

\begin{definition}
    Un groupe $G$ est \emph{monogène} s'il est engendré par un seul élément, appelé \emph{générateur}. Il est \emph{cyclique} s'il est fini. 
\end{definition}

\begin{corollary}
    Un groupe $G$ est monogène infini si et seulement si il est isomorphe à $\Z$. Il est cyclique d'ordre $n \geq 1$ si et seulement si isomorphe à $\Z/n\Z$.
\end{corollary}

\begin{proposition}[Générateurs d'un Groupe Cyclique]
    \begin{itemize}
        \item Les générateurs de $\Z, +$ sont les $k \in \Z$ tels que $\Z = k\Z$, i.e. $k = \pm 1$.
        \item Pour $k \in \Z$, $G = \scalar{g}$ un groupe cyclique d'ordre $n$, on a équivalence entre :
        \begin{enumerate}
            \item $\scalar{g^{k}} = G$
            \item $g \in \scalar{g^{k}}$
            \item $\exists k^{'} \in \Z,\ kk^{'} = 1 \text{ mod } n$
            \item $\overline{k} \in \left(\Z/n\Z\right)^{\times}$
            \item $k \wedge n = 1$
        \end{enumerate}
    \end{itemize}
\end{proposition}
\begin{corollary}
    Un groupe cyclique d'ordre $n$ a exactement $\phi(n)$ générateurs.
\end{corollary}
\begin{corollary}
    Si $G$ est cyclique d'ordre $n$ : $Aut(G) = \left\{g \mapsto g^{k} \mid k \in (\Z/n\Z)^{\times}\right\}$. On a alors un isomorphisme de $(\Z/n\Z)^{\times}$ dans $Aut(G)$.
\end{corollary}

\begin{remark}
    Si $g \in G$ est d'ordre fini $n$, si $d \geq 1$, $g^{d}$ est d'ordre fini $\frac{n}{n \wedge d}$.
\end{remark}

\begin{proposition}
    Si $G$ est cyclique d'ordre $n$, $d \mapsto G_{d} = \left\{g^{d} \mid g \in G\right\}$ est une bijection de l'ensemble des diviseurs de $n$ sur l'ensemble des sous-groupes de $G$. 
\end{proposition}

\begin{theorem}[Chinois]
    Soient $m, n \in \Z$ premiers entre eux. L'application $\Z \rightarrow (\Z/n\Z) \times (\Z/m\Z),\ k \mapsto \left(k \mod n, k \mod m\right)$ définit un isomorphismepar par passage au quotient de par la propriété universelle \ref{proprietequotient}.
\end{theorem}

\subsection{Théorème de Lagrange} % et pas de la ferme
\begin{definition}
    Si $A, B$ sont deux parties d'un groupe, $AB = \left\{ab \mid a \in A, b\in B\right\}$. Si $A = \left\{g\right\}$, on le note $gB$.
\end{definition}
\begin{lemma}
    $H \leq G \Leftrightarrow \left(H \neq \emptyset, HH = H, H^{-1} = H\right)$.
\end{lemma}
\begin{definition}
    On pose $g\sim_{H}g^{'}$ si $g^{'} \in gH$. C'est une relation d'équivalence. On note $G/H$ son ensemble quotient, et on appelle indice de $H$ dans $G$ son cardinal noté $[G : H]$.
\end{definition}

\begin{theorem}[Lagrange]\ref{Lagrange}
    Si $H$ est un sous-groupe de $G$, $G \sim H \times (G/H)$. En particulier, si deux des trois ensembles $G, H, G/H$ sont finis, $\abs{G} = \abs{H}\left[G : H\right]$.
\end{theorem}
\begin{corollary}
    \begin{itemize}
        \item Si $H$ est un sous-groupe du groupe fini $G$, $\abs{H} \mid \abs{G}$.
        \item Si $G$ est fini, $g\in G$, $g^{\abs{G}} = 1$.
        \item $n^{p-1} \cong 1 \text{ mod } p$ pour $n\in \Z, p \in \P$.
        \item Tout groupe d'ordre premier $p$ est isomorphe à $\Z/p\Z$.
    \end{itemize}
\end{corollary}

\begin{theorem}[Cauchy]
    Soit $G$ un groupe fini, $p$ un nombre premier divisant $\abs{G}$. $G$ possède un élément d'ordre $p$. Si $G$ est abélien, on peut généraliser immédiatement à tout $p \in \Z$.
\end{theorem}

\subsection{Sous-groupes finis de $k^{\times}$ et $\left(\Z/n\Z\right)^{\times}$}
\begin{theorem}
    Si $k$ est un corps, tout sous-groupe fini de $k^{\times}$ est cyclique.
\end{theorem}
\begin{lemma}[Cauchy]
    Soit $G$ un groupe, $x, y$ deux éléments qui commutent d'ordres $a$ et $b$ premiers entre eux. Alors, $xy$ est d'ordre $ab$. 
\end{lemma}
\begin{theorem}[Gauss]
    Pour $p$ premier, le groupe $\left(\Z/p\Z\right)^{\times}$ est cyclique.
\end{theorem}
\begin{definition}
    Un isomorphisme de groupes $\left(\Z/p\Z\right)^{times} \simeq \Z/(p-1)\Z$ est appelé un logarithme discret.
\end{definition}
\begin{definition}
    Pour un groupe, on note $G^{(n)}$ le groupe des puissances $n$-ièmes.
\end{definition}
\begin{proposition}
    Soient $p\in\P$, $n\geq 1$ et $m = (p-1)\wedge n$.
    \begin{enumerate}
        \item $\left(\Z/p\Z\right)^{\times, (n)}$ est cyclique d'ordre $\frac{p - 1}{m}$ et égal à $\left(\Z/p\Z\right)^{\times, (m)}$
        \item Pour $x \in \left(\znz{p}\right)^{\times}$, on a $x \in \left(\znz{p}\right)^{\times, (n)}$ si et seulement si $x^{\frac{p-1}{m}} = 1$, i.e. $X^{\frac{p-1}{m}}$ a au plus $\frac{p-1}{m}$ racines dans $\znz{p}$ et donc ses racines sont exactement les puissances $n$-èmes.
    \end{enumerate}
\end{proposition}

\begin{proposition}
    Si $p$ est premier impair, $m \geq 1$, alors $\left(\znz{p^{m}}\right)^{\times}$ est cyclique. Si $m\geq 2$, $\left(\znz{2^{m}}\right)^{\times} \simeq \znz{2} \times \znz{2^{n-2}}$
\end{proposition}

\subsection{Groupes Quotients}
\begin{definition}
    Un sous-groupe $H$ de $G$ est dit \emph{distingué}, noté $H \lhd G$ si l'une des conditions équivalentes suivantes est vérifiée : 
    \begin{enumerate}
        \item $gHg^{-1} \subset H,\ \forall g \in G$
        \item $gHg^{-1} = H,\ \forall g \in G$
        \item $gH = Hg, \ \forall g \in G$.
    \end{enumerate}
\end{definition}

\begin{remark}
    Tous les sous-groupes d'un groupe abélien sont distingués. Un groupe d'indice $2$ dans $G$ est distingué.
\end{remark}
\begin{definition}
    Le normalisateur de $H$ dans $G$ est le sous-groupe de $G$ défini par $N_{G}(H) = \left\{g \in G \mid gHg^{-1} = H\right\}$.
\end{definition}

\begin{theorem}
    Soit $H$ un sous groupe d'un groupe $G$. 
    \begin{enumerate}
        \item Il existe au plus une loi de groupe sur $G/H$ telle que la projection canonique $G \rightarrow G/H$ soit une loi de groupe. 
        \item Une telle loi existe si, et seulement si, on a $H \lhd G$, auquelle cas c'est la loi induite par le produit sur $P(G)$.
    \end{enumerate}
\end{theorem}

\begin{definition}
    Si $H \lhd G$, le groupe quotient $G/H$ est la donnée de l'ensemble $G/H$ muni de son unique loi de groupe telle que la projection canonique est un morphisme de groupes.
\end{definition}

\begin{definition}
    On pose $\left(\frac{x}{p}\right) = 1$ si $x$ est un carré non nul, $0$ si $x$ est nul et $-1$ sinon. $x \mapsto \left(\frac{x}{p}\right)$ est un morphisme multiplicatif. 
\end{definition}

On va étudier les groupes en chercher à étudier des groupes plus simples : étant donné un groupe $G$, on cherche $H \subsetneq G$ un groupe distingué non trivial pour étudier $H$ et $G/H$, d'ordres plus petits. 
\begin{definition}
    Un groupe $G$ est dit \emph{simple} si ses seuls groupes distingués sont $\left\{1\right\}$ et $G$.
\end{definition}

\begin{theorem}[Propriété Universelle des Groupes Quotients]\label{grpquotients}
    Si $H \lhd G$, et si $f : G \rightarrow G^{'}$ est un morphisme, $g = f \circ \pi$ est un morphisme de $G/H$ dans $G^{'}$ tel que $g(H) = {1}$.
\end{theorem}

\begin{theorem}[Premier Théorème d'Isomorphisme]
    Si $f$ est un morphisme de $G$ dans $G^{'}$, alors $f$ induit par passage au quotient un isomorphisme de groupes de $G/\ker f$ dans $\emph{Im} f$
\end{theorem}

\begin{proposition}[Troisième Théorème d'Isomorphisme]
    Soit $H \lhd G$ : 
    \begin{enumerate}
        \item $H \mapsto K/H$ induit une bijection croissante entre sous groupes de $G$ contenant $H$ et sous-groupes de $G/H$.
        \item Dans cette bijection, $K/H \lhd G/H \Leftrightarrow K \lhd G$ auquel cas le morphisme naturel $G/H \rightarrow G/K$ induit un isomorphisme $\left(G/H\right)/\left(K/H\right) \rightarrow G/K$.
    \end{enumerate}
\end{proposition}

\newpage
\section{Groupes Abéliens de Type Fini}
\subsection{Caractères}
\begin{definition}
    Un \emph{caractère} d'un groupe $G$ est un morphisme de $G$ dans $\C^{\times}$.
\end{definition}

\begin{proposition}
    Soit $G = \scalar{g}$ un groupe cyclique d'ordre $n$. Pour $\zeta \in \mu_{n}$, il existe un unique caractère $\chi_{\zeta}$ de $G$ tel que $\chi_{\zeta}(g) = \zeta$. De plus, $\zeta \mapsto \chi_{\zeta}$ est un isomorphisme de groupes. 
\end{proposition}

\subsection{Décomposition de Fourier finie}
\begin{definition}
    Si $G$ est un groupe fini, on note $L^{2}(G)$ le $\C$-espace vectoreil des fonctions $G \rightarrow \C$ muni du produit hermitien. C'est un espace de dimension finie $\abs{G}$. On note $\hat{G}$ l'ensemble des caractères de $G$. On rappelle que $\C^{\times}$ étant abélien, $\hat{G} = Hom(G, \C^{\times})$
\end{definition}

\begin{theorem}
    Soit $G$ un groupe fini. 
    \begin{enumerate}
        \item L'ensemble $\hat{G}$ est une famille livre et orthonormée de $L^{2}(G)$ (Orthogonalité des Caractères)
        \item Si $G$ est abélien, $\hat{G}$ est une base de $L^{2}(G)$.
    \end{enumerate}    
\end{theorem}

\begin{corollary}
    Soit $G$ abélien fini
    \begin{enumerate}
        \item On a $\abs{\hat{G}} = \abs{G}$
        \item Pour toute fonction $f : G \rightarrow \C$ on a $f = \sum_{\chi \in \hat{G}} \scalar{f, \chi} \chi$
    \end{enumerate}
\end{corollary}

\begin{proposition}
    Soit $G$ abélien fini, $H \subset G$ un sous-groupe. Pour tout caractère $\chi$ de $H$, il existe $\tilde{\chi}$ de $G$ tel que $\tilde{\chi_{\mid H}} = \chi$ 
\end{proposition}

\begin{definition}
    Un groupe abélien $D$ est \emph{divisible} si le morphisme de groupes $x \mapsto x^{n}$ est surjectif pour tout $n \geq 1$.
\end{definition}

\begin{proposition}[Prolongement des Morphismes]
    Soient $G, H, D$ des groupes abéliens avec $D$ divisible, $H \subset G$ et $f : H \rightarrow D$ un morphisme de groupes. Alors il existe un morphisme de groupes $\tilde{f} : G \rightarrow D$ tel que $\tilde{f_{\mid H}} = f$.
\end{proposition}

\subsection{Structure des groupes abéliens finis}
\begin{theorem}\label{structuregrpabelfinis}
    Soit $G$ abélien fini, il existe un unique entier $n\geq 0$ et des uniques entiers $a_{i} > 1$ vérifiant $a_{1} \mid a_{2} \mid \ldots \mid a_{n}$ et $G \simeq \prod_{i=1}^{n} \znz{a_{i}}$.
\end{theorem}

\begin{definition}
    L'\emph{exposant} d'un groupe fini $G$ est le plus petit entier $e \geq 1$ vérifiant $g^{e} = 1$ pour tout $g \in G$. C'est le ppcm des ordres des éléments de $G$. 
\end{definition}

\subsection{Existence}
\begin{lemma}
    Si $G$ est abélien fini, il existe un élément d'ordre l'exposant. 
\end{lemma}

\begin{proposition}
    Soit $G$ un groupe, $H\leq G, K \leq G$. On suppose $H \cap K = 1, \ G = HK$ et enfin $hk = kh$ pour tout $h \in H,\ k \in K$. L'application produit sur $H \times K$ définit un isomorphisme de groupes. 
\end{proposition}

De ces deux propositions, on peut prouver la partie existence du théorème. 

\subsection{Exemple}
\begin{definition}
    Soit $p$ un nombre premier. Un groupe abélien fini est $p$-élémentaire si on a $g^{p} = 1$ pour tout $g \in G$.
\end{definition}

\begin{definition}
    On définit $G^{\sharp}$ le $\znz{p}$-espace vectoriel dont $G$ est le groupe additif. 
\end{definition}

\begin{proposition}
    Soit $p$ premier, $G$ abélien fini. $G$ est $p$-élémentaire si et seulement si $G \simeq \left(\znz{p}\right)^{n}$ pour un certain $n \geq 1$. Le nombre minimal de générateurs de $G$ est $\dim_{\znz{p}}G^{\sharp}$.
\end{proposition}

\subsection{Unicité}
\begin{definition}
    On note $\min(G)$ le nombre minimal de générateurs de $G$. Il est fini si et seulement si $G$ est de type fini. 
\end{definition}
\begin{proposition}
    Supposons qu'on écrit une décomposition de $G$ comme dans le théorème \ref{structuregrpabelfinis}. On a $n = \min{(G)}$
\end{proposition}

\begin{definition}
    Soit $G$ abélien. Le sous-ensemble $G\left[n\right] = \left\{g^{n} = 1\right\}$ est un sous groupe de $G$ appelé $n$-torsion de $G$. 
\end{definition}

\begin{lemma}
    Soit $G$ et $H$ abéliens et $n \geq 1$.
    \begin{enumerate}
        \item On a $\left(G \times H\right)\left[n\right] = G\left[n\right] \times H\left[n\right]$
        \item Tout (iso-)morphisme $G \rightarrow H$ induit un (iso-)morphisme $G\left[n\right] \rightarrow H\left[n\right]$.
        \item Supposons $G$ cyclique d'ordre $m$ et $p$ premier. Alors $G\left[p\right] = \left\{1\right\}$ sauf si $p \mid m$ auquel cas $G\left[p\right] \simeq \znz{p}$ et $G/G\left[p\right] \simeq \znz{m/p}$.
    \end{enumerate}
\end{lemma}

\subsection{Groupes Abéliens de Type Fini}
On note ici $G$ un groupe abélien additif

\begin{definition}
    Soit $\mathcal{F} = \left\{g_{1},\ldots, g_{n}\right\}$ une famille d'éléments de $G$ et 
    \[
        f : \Z^{n} \rightarrow G, (m_{i}) \mapsto \sum_{i = 1}^{n}m_{i}g_{i}
    \]
    On dit que $\mathcal{F}$ est libre (ou $\Z$-libre) si $f$ est injectif. On dit que $\mathcal{F}$ est génératrice si $f$ est surjectif, et est une base si $f$ est bijectif.
\end{definition}

\begin{definition}
    Un groupe abélien est dit \emph{libre de rang $n$} s'il possède une $\Z$ base à $n$ éléments, i.e. s'il est isomorphe à $\Z^{n}$. Par conventions, $\left\{0\right\}$ est libre de rang $0$.
\end{definition}

\begin{lemma}
    Pour tout entier $n\geq 0$, $\min{\left(\Z^{n}\right)} = n$. En particulier, $\Z^{n} \simeq Z^{m} \Leftrightarrow n = m$.
\end{lemma}

\begin{definition}
    On appelle \emph{sous-groupe de torsion} de $G$, le sous-groupe de $G$ noté $G_{\text{tor}} =\left\{g \in G \mid \exists n \geq 1, \ ng = 0\right\}$
\end{definition}
 
\begin{theorem}[Dirichlet]
    Si $G$ est abélien de type fini, $G_{\text{tor}}$ est fini et il existe un unique $n \in \N$ tel que $G \simeq G_{\text{tor}} \times \Z^{n}$.
\end{theorem}

\begin{corollary}
    Un groupe abélien de type fini sans torsion est libre
\end{corollary}

\begin{lemma}
    Si $f:  G \rightarrow \Z$ est surjectif, $G \simeq \Z \times \ker f$.
\end{lemma}

\begin{lemma}
    Si $A, B$ sont deux groupes abéliens avec $A$ fini et $B$ libre de rang fini, alors, avec $G = A\times B$ : $G_{\text{tor}} = A \times \left\{0\right\}$ et $G/G_{\text{tor}} \simeq B$.
\end{lemma}

\newpage
\section{Groupe Symétrique et Dévissage}
\subsection{Actions de Groupes}
\begin{definition}
    Une action de $G$ sur $X$ est une application : 
    $\cdot : G \times X \rightarrow X$ vérifiant : 
    $ 1\cdot x = x$ et $g \cdot \left(h \cdot x\right) = (gh) \cdot x$.
\end{definition}

\begin{definition}
    Soit $G$ agissant sur $X$, et $x \in X$.
    \begin{itemize}
        \item $O_{x} = \left\{gx \mid g \in G\right\} \subset X$ est l'\emph{orbite} de $x$ sous $G$, aussi notée $Gx$.
        \item Le sous-groupe $G_{x} = \left\{g \in G \mid gx = x\right\}$ est appelé \emph{stabilisateur} de $x$ ou \emph{groupe d'isotropie de} $x$, noté $\text{Stab}_{G}(x)$.
    \end{itemize}
\end{definition}

\begin{lemma}
    On a : $G_{gx} = gG_{x}g^{-1}$.
\end{lemma}

\begin{proposition}
    \begin{itemize}
        \item Les orbites sous $G$ forment une partition de $X$. 
        \item Pour tout $x \in X$, on a une bijection $G/G_{x} \xrightarrow{\sim} O_{x}$ envoyant $gG_{x}$ sur $gx$. En particulier, si $G$ est fini, on a $\abs{G} = \abs{G_{x}}\abs{O_{x}}$
    \end{itemize}
\end{proposition}

\begin{corollary}
    On note $x_{i}$ des représentants des orbites de $G$ dans $X$. On a : 
    \[
        \abs{X} = \sum_{i\in I} \abs{O_{x_{i}}} = \sum_{i\in I} \abs{G}/\abs{G_{x_{i}}}
    \]
\end{corollary}

\begin{theorem}[Premier Théorème de Sylow]
    Soit $G$ fini d'ordre $p^{n}m$ avec $p$ premier et $m \wedge p = 1$. Alors $G$ possède un sous-groupe d'ordre $p^{n}$, appelé un $p$-Sylow de $G$.
\end{theorem}

\begin{definition}
    Une action de $G$ sur $X$ est \emph{transitive} si on a $X \neq \emptyset$ et si $\forall\ x, y \in X,\ \exists \ g \in G, \ y = gx$, i.e. que $X$ a une et une seule orbite sous l'action de $G$.
\end{definition}

\begin{definition}
    Le noyau d'une action est le noyau du morphisme $G \rightarrow S_{X}$ associé à l'action. C'est un sous-groupe distingué de $G$. Une action est dite \emph{fidèle} si son noyau est $\left\{1\right\}$.
\end{definition}

\begin{definition}
    Une action est \emph{libre} si on a toujours $G_{x} = \left\{1\right\}$.
\end{definition}

\begin{definition}
    Deux actions d'un même groupe sur deux ensembles $X$ et $Y$ sont \emph{isomorphes} s'il existe une bijection $f$ vérifiant $f(g \cdot x) = g \star f(x)$. 
\end{definition}


\begin{proposition}
    Une action transitive $\left(X, \cdot\right)$ est isomorphe à l'action par translations de $G$ sur $G/G_{x}$
\end{proposition}

\begin{proposition}
    Deux actions transitives sont isomorphes si et seulement si elles ont les mêmes stabilisateurs. 
\end{proposition}

\subsection{Groupes Symétriques et Alternés}
\begin{proposition}
    Toute permutation $\sigma$ de $S_{n}$ s'écrit comme un produit de cycles à supports disjoints. L'ordre de $\sigma$ est alors le ppcm des longueurs des cycles.
\end{proposition}

\begin{proposition}
    Les transpositions engendrent $S_{n}$
\end{proposition}
\begin{lemma}
    Si $\sigma \in S_{n}$, $c = \left(i_{1}, \ldots, i_{k}\right)$ est un $k$-cycle : $\sigma c\sigma^{-1} = \left(\sigma(i_{1}), \ldots, \sigma(i_{k})\right)$.
\end{lemma}

\begin{proposition}
    \begin{itemize}
        \item Les $(i, i+1)$ engendrent $S_{n}$. Ils sont appelés générateurs de Coxeter.
        \item La transposition $(1, 2)$ et le cycle $(1 2\ldots n)$ engendrent $S_{n}$.
    \end{itemize} 
    En particulier, $\min S_{n} = 2$.
\end{proposition}

\begin{definition}
    \begin{itemize}
        \item Une partition de l'entier $n$ est une suite décroissante $n_{1} \geq \ldots \geq n_{r}$ d'entiers strictement positifs de somme $n$.
        \item Le type de $\sigma \in S_{n}$ est la partition de l'entier $n$ définie par les cardinaux des orbites de $\sigma$.
    \end{itemize}    
\end{definition}

\begin{proposition}
    Deux éléments de $S_{n}$ sont conjugués si et seulement si ils ont même type.
\end{proposition}

\begin{definition}
    Pour $k\geq 1$ entier, $G$ agissant sur $X$ avec $\abs{X} \geq k$, $G$ agit $k$-transitivement sur $X$ si pour deux $k$-uplets d'éléments distincts de $X$ il existe $g \in G$ tel que $gx_{i} = y_{i}$ pour tout $i$.
\end{definition}

\begin{definition}
    La \emph{signature} de $\sigma \in S_{n}$ est : 
    \[
        \epsilon(\sigma) = \prod_{\left\{i, j\right\}} \frac{\sigma(i) - \sigma(j)}{i - j}
    \]
    C'est un morphisme de groupes $S_{n} \rightarrow \left\{\pm 1\right\}$ valant $-1$ sur les transpositions. On note $A_{n}$ son noyau. C'est un sous-groupe distingué.
\end{definition}

\begin{proposition}
    Pour $n \geq 3$, $A_{n}$ agit $\left(n - 2\right)$-transitivement sur $\lvert 1, n \rvert$. Les $k$-cycles sont conjugués sous l'action de $A_n$ pour $k \in \lvert 2, n-2 \rvert$.
\end{proposition}

\subsection{Les suites exactes}

\begin{definition}
    Une suite de $n \geq 2$ morphismes de groupes $(f_{1}, \ldots, f_{n})$ est \emph{exacte} si $\text{Im } f_{i} = \ker f_{i+1}$ pour tout $i$.
\end{definition}

\begin{definition}
    Une suite exacte de la forme $1 \rightarrow H \xrightarrow{i} G \xrightarrow{\pi} K \rightarrow 1$ est une suite exacte \emph{courte}.
\end{definition}

\begin{proposition}
    Il est équivalent de se donner : 
    \begin{itemize}
        \item Une suite exacte $1 \rightarrow H \xrightarrow{i} G \xrightarrow{\pi} K \rightarrow 1$
        \item Un sous-groupe distingué $H^{'} \subset G$ et des isomorphismes $i^{'} : H \xrightarrow{\sim} H^{'}$ et $\pi^{'} : G/H^{'} \xrightarrow{\sim} K$.
    \end{itemize}
\end{proposition}

\begin{definition}[Groupe diédral]
    Pour $n\geq 3$, on définit le groupe diédral $D_{2n}$ comme le sous groupe de $S_{n}$ engendré par $(1 2 \ldots n)$ et l'élément $\tau$ défini par $\tau(i) = n+1-i$.
\end{definition}

\begin{definition}
    Si $G, H, K$ sont des groupes donnés, $G$ est extension de $K$ par $H$ s'il existe une suite exacte courte $1\rightarrow H \rightarrow G \rightarrow K \rightarrow 1$.
\end{definition}

\subsection{Dévissage de $S_{n}$}

\begin{theorem}
    Les seuls sous-groupes distingués de $S_{n}$ sont $\left\{1\right\}, A_{n}, S_{n}$ et $K_4$ dans le cas $n = 4$. 
\end{theorem}

\begin{theorem}
    Pour $n \geq 5$, $A_{n}$ est simple non abélien.
\end{theorem}

\begin{corollary}
    \begin{itemize}
        \item Pour $n \neq 4$, toute action de $A_{n}$ est fidèle ou triviale.
        \item Une action transitive de $S_{n}$ sur un ensemble à $m > 2$ éléments est fidèle, sauf peut-être si $n=4$ et $m = 3$ ou $6$.
    \end{itemize}
\end{corollary}

\subsection{Commutateur et Groupes Dérivés}

\begin{definition}
    Le groupe dérivé d'un groupe $G$ est le sous-groupe $D(G) = \left[G, G\right]$ engendré par les $\left[x, y\right] = xyx^{-1}y^{-1}$. On a $D(G) = \left\{1\right\}$ si et seulement si $G$ est abélien.
\end{definition}

\begin{corollary}
    $D(G)$ est un sous-groupe caractéristique de $G$.
\end{corollary}

\begin{corollary}
    Soit $G$ un groupe.
    \begin{itemize}
        \item Tout morphisme $f : G \rightarrow G^{'}$ avec $G^{'}$ abélien vérifie $D(G) \subset \ker f$.
        \item Pour $H \lhd G$ alors $G/H$ est abélien si et seulement si, $D(G) \subset H$.
    \end{itemize}
\end{corollary}

\begin{proposition}
    On a :
    \begin{itemize}
        \item $D(S_{n}) = A_{n}$
        \item $D(A_{n}) = A_{n}$ pour $n \geq 5$.
        \item $D(A_{4}) = K_{4}$ et $D(A_{n}) = \left\{1\right\}$ pour $n \leq 3$
    \end{itemize}
\end{proposition}


\begin{definition}
    Un groupe $G$ est résoluble s'il existe $n$ tel que $D^{n}(G) = \left\{1\right\}$. Le plus petit $n$ est appelé classe de résolubilité de $G$.
\end{definition}

\begin{proposition}
    Si $G$ est un groupe et $H \lhd G$, $G$ est résoluble si et seulement si $H$ et $G/H$ le sont. Alors, la classe de $G$ est inférieure à la somme des classes de $H$ et de $G/H$.
\end{proposition}

\begin{proposition}
    Le groupe $T_{n}(k)$ est résoluble de classe $\leq 1 + \lceil \log_{2}(n) \rceil$.
\end{proposition}

\subsection{Dévissage en Produit Semi-Direct}
\begin{definition}
    Si $H \leq G$, un \emph{complément} de $H$ dans $G$ est $K\leq G$ tel que $G = HK$ et $H \cap K = \left\{1\right\}$
\end{definition}

\begin{remark}
    Soit $N \lhd G$, et $K$ un complément de $N$ dans $G$. Pour tout $n, n^{'} \in N$, $k, k^{'} \in K$, on a : 
    \[
        (nk)(n^{'}k^{'}) = n(kn^{'}k^{-1})kk' \text{ avec } kn^{'}k^{-1} \in N
    \]
    Autrement dit : 
    \[
        (nk)(n^{'}k^{'}) = n\text{int}_{k}(n')kk'
    \]
    La structure de groupe de $G$ se déduit de celle de $N$, $K$ et de la connaissance de l'application : $\alpha : k \in K \mapsto \text{int}_{k\mid N}$
\end{remark}

On se fixe dans la suite deux tels groupes $N$ et $K$, et un morphisme de groupe $\alpha$ de $K$ dans $\text{Aut}(N)$.

\begin{definition}
    La loi $\star_{\alpha} : (N \times K) \times (N \times K) \rightarrow N \times K, (n, k), (n^{'}, k^{'}) \mapsto (n\alpha_{k}(n^{'}), kk^{'})$ est une loi de groupe, qui munit $N \times K$ d'une structure de groupe noté $N \rtimes_{\alpha} K$ et appelé produit semi-direct (externe) de $K$ par $N$ associé à $\alpha$.
\end{definition}

\begin{proposition}
    Soit $G$ un groupe, $N \lhd G$ et $K$ un complément de $N$ dans $G$. Soit $\alpha : K \rightarrow \text{Aut}(N),\ k \mapsto \alpha_{k}$. La bijection $N \times K \rightarrow G, (n, k) \mapsto nk$ est un isomorphisme de groupes : $N \rtimes_{\alpha} K \xrightarrow{\sim} G$. On dit aussi que $G$ est \emph{produit semi-direct interne} de $K$ par $N$
\end{proposition}

\begin{proposition}[Suivi des Isomorphismes]
    Soit $G = N \rtimes_{\alpha} K$, $a : N^{'} \xrightarrow{\sim} N$ et $b : K^{'} \xrightarrow{\sim} K$ des isomorphismes. La bijection $N^{'}\times K^{'} \rightarrow G, (n^{'}, k^{'}) \mapsto a(n^{'})b(k^{'})$ est un isomorphisme de groupes de $N^{'} \rtimes_{\alpha^{'}} K^{'}$ dans $G$, où $\alpha^{'} : k^{'} \mapsto \alpha_{k^{'}} = a^{-1}\circ \alpha_{b(k^{'})} \circ a$.
\end{proposition}

\begin{proposition}
    Un groupe d'ordre $2p$ avec $p$ premier impair est soit isomorphe à $\znz{p}$ soit à $D_{2p}$.
\end{proposition}

\begin{proposition}
    Les groupes non abéliens d'ordre $\leq 8$ sont $S_{3}$, $D_{8}$ et $H_{8}$.
\end{proposition}

\newpage
\section{Groupes et Symétries}
\subsection{Sous-groupes Finis de $O(2)$ et $SO(3)$}.
Ici, $E$ est un espace euclidien de dimension $n \geq 1$
\begin{definition}
    On définit la \emph{réflexion} par rapport à $H$ un hyperplan de $E$, l'application $s_{H} \in O(E)$ définie par : $s_{H}(h + d) = h - d$ où $h, d \in H \times H^{\perp}$. Pour $v \in E$ non nul, on appelle aussi \emph{réflexion de vecteur } $v$ la réflexion $s_{v} = s_{v^{\perp}}$.
\end{definition}

\begin{theorem}[Cartan-Dieudonné] 
    Tout élément de $O(E)$ est produit d'au plus $n$ réflexions. En particulier, tout élément de $SO(E)$ est produit d'au plus $n/2$ produits de deux réflexions.   
\end{theorem}
\begin{remark}
    $SO(2)$ est isomorphe au groupe $S^{1}$ des rotations du plan. On peut également montrer que $O(2) \simeq SO(2) \rtimes_{\alpha} \znz{2}$ avec $\alpha_{\bar{1}}(g) = g^{-1}$.
\end{remark}

\begin{corollary}
    Tout élément non trivial de $SO(3)$ possède une et une seule droite fixe dans $E$.
\end{corollary}

\begin{lemma}
    Si $g\in O(E)$ préserve $F \subset E$, il préserve $F^{\perp}$.
\end{lemma}

\begin{definition}
    Pour $P \subset E$, on note $\text{Iso}(P) = \left\{g \in O(E) \mid g(P) = P\right\}$ le sous-groupe des isométries orthogonales de $P$.
\end{definition}

\begin{definition}
    On note $\mathscr{P}_{m}$ un polygone régulier du plan à $m \geq 3$ côtés centré en $0$.
\end{definition}

\begin{proposition}
    $\text{Iso}(\poly{m})$ agit sur l'ensemble $\mathscr{S}$ des sommets de $\poly{m}$, puisque ce sont les points à distance maximale de $0$. Cette action définit un morphisme $f$ qui induit un isomorphisme $\text{Iso}(\poly{m}) \xrightarrow{\sim} D_{2m}$. De plus, $\text{Iso}^{+}(\poly{m}) \simeq \znz{m}$.
\end{proposition}

En considérant les groupes d'isométries de figures planes bien choisies, on trouve trois autres classes de conjugaison.

\begin{proposition}
    Soit $G \leq O(2)$ fini. Alors, soit $G$ est isomorphe à $1$, $\znz{2}$ ou $\left(\znz{2}\right)^{2}$, soit il existe un polygone régulier $\poly{}$ du plan euclidien tel que $G = \text{Iso}(\poly{})$ ou $G = \text{Iso}^{+}(\poly)$
\end{proposition}
\begin{remark}
    Le groupe des isométries de $\left[-1, 1\right] \times \{0\}$ est isomorphe à $\left(\znz{2}\right)^{2}$, qu'on note parfois $D_{4}$.
\end{remark}

\begin{definition}
    $G \leq O(E)$ est dit irréductible, s'il n'existe aucun sous-espace non-dégénéré stable par $G$.
\end{definition}

On suppose désormais $n = 3$. 

\begin{remark}
    Un groupe est irréductible s'il stabilise un plan ou, de manière équivalente, une droite. On a donc un morphisme injectif \emph{diagonal} :
    $\applic{O(2)}{SO(3)}{g}{\begin{matrix}g & 0\\ 0 & \det g\end{matrix}}$
\end{remark}

\begin{definition}
    Soit $P \subset E$ un \emph{solide de Platon}\footnote{polyèdre régulier}. On définit ses sommets, ses arêtes et ses faces comme ses parties extrémales de dimension 0, 1 et 2 : 
    \[
        \forall x, y \in P, \ \left]x, y \right[\ \cap F \neq \emptyset \Rightarrow \left[x, y\right]\subset F
    \]
    L'action de $\text{Iso}(P)$ préserve l'ensemble $\mathscr{S}$ des sommets, celui $\mathscr{A}$ des arêtes et celui $\mathscr{F}$ des faces. On note que l'action de $\text{Iso}(P)$ sur $\mathscr{S}$ engendre $E$. Notons que dès que $-1 \in \text{Iso}(P)$, $\text{Iso}(P) = \left\{\pm 1\right\} \times \text{Iso}^{+}(P)$.
\end{definition}

\begin{itemize}
    \item \textsc{Le Tétraèdre Régulier} $T$ : En regardant l'action sur les sommets, on obtient : 
    \begin{proposition}
        $\text{Iso}(T) \simeq S_{4}$ et $\text{Iso}^{+}(T) \simeq A_{4}$.
    \end{proposition} Le déterminant sur $\text{Iso}(T)$ correspond à la signature sur $S_{4}$. En regardant de plus les paires d'arêtes orthogonales, on fournit un morphisme $\text{Iso}(T) \rightarrow S_{3}$.
    \item \textsc{Le Cube} ou \textsc{Hexaèdre Régulier} $C$ : En regardant l'action sur les paires de sommets, on obtient : 
    \begin{proposition}
        $\text{Iso}^{+}(C) \simeq S_{4}$ et $\text{Iso}(C) = \{\pm 1\} \times \text{Iso}^{+}(C)$.
    \end{proposition}
    En considérant l'action sur les paires de faces opposées, on retrouverait un morphisme $S_{4} \rightarrow S_{3}$.
    \item \textsc{L'Octaèdre Régulier} $O$ : En regardant les centres des faces, on trouve un cube $C$ appelé cube dual et dont les centres des faces sont les sommets d'un nouvel octaèdre $O^{'}$. On en déduit que : 
    \begin{proposition}
        $\text{Iso}(O) = \text{Iso}(C) = \text{Iso}(O^{'})$
    \end{proposition}
    \item \textsc{Le Dodécaèdre Régulier} $D$ : En regardant l'action sur les sommets, et en regardant les triplets de diarête\footnote{couples d'arêtes parallèles} deux à deux orthogonales, on obtient :
    \begin{proposition}
         $\text{Iso}^{+}(D) \simeq A_{5}$ et on conclut car $-1 \in \text{Iso}(P)$.
    \end{proposition}
    \item \textsc{L'Icosaèdre Régulier} $I$ : On vérifie comme pour le cube que le dual de $I$ est un dodécaèdre et que l'on a :
    \begin{proposition}
        $\text{Iso}(I) = \text{Iso}(D)$.
    \end{proposition}
    En regardant l'action sur les faces opposées, on retrouve l'action sur les pentagones mystiques.
\end{itemize}

\begin{theorem}[Klein]
    Tout sous-groupe fini irréductible de $SO(3)$ est le groupe des isométries directes d'un solide de Platon, et donc isomorphe à $A_{4}, S_{4}$ ou $A_{5}$.
\end{theorem}

\begin{lemma}[Burnside-Frobenius]
    Soit $G$ fini agissant sur un ensemble fini $X$. On note $r$ le nombre de $G$-orbites dans $X$ et pour $g \in G$ on note $\text{Fix}(g)$ l'ensemble des points fixes de $g$ dans $X$. On a alors :
    \[
        r = \frac{1}{\abs{G}} \sum_{g \in G}  \abs{\text{Fix}(g)}
    \]
\end{lemma}

\begin{lemma}
    Si $G \leq SO(3)$ est fini, soit $X \subset S^{2}$ l'ensemble des pôles\footnote{couples de points fixes} des éléments non triviaux de $G$, soient $x_{1}, \ldots, x_{r}$ des représentants des orbites de $G$ dans $X$ et $n_{i} = \abs{G_{x_{i}}}$ triés dans l'ordre croissant. Alors, on a soit $r = 2$, $\abs{X} = 2$ et $G = G_{x_{1}} = G_{x_{2}}$ soit $r = 3$ et $\abs{G}$ et les $n_{i}$ sont données par : 
    \begin{center}
        \begin{tabular}{cccccccc}
            $\abs{G}$ & $n_{1}$ & $n_{2}$ & $n_{3}$ & $\abs{O_{x_{1}}}$ & $\abs{O_{x_{2}}}$ & $\abs{O_{x_{3}}}$ & $\abs{X}$\\
            \midrule
            $2m$ & $2$ & $2$ & $m$ & $m$ & $m$ & $2$ & $2m+2$\\
            \midrule
            $12$ & $2$ & $3$ & $3$ & $6$ & $4$ & $4$ & $14$\\
            \midrule
            $24$ & $2$ & $3$ & $4$ & $12$ & $8$ & $6$ & $26$\\
            \midrule
            $60$ & $2$ & $3$ & $5$ & $30$ & $20$ & $12$ & $60$            
        \end{tabular}
    \end{center}
\end{lemma}

<<<<<<< HEAD
\subsection{Le Groupe $SP(1)$}
\subsubsection{L'algèbre des quaternions de Hamilton}
\begin{definition}
    On se place dans $\mat{2}(\C)$ et considère :
    \[
        I = \begin{matrix}
            i & 0\\ 0 & -i
        \end{matrix},
        J = \begin{matrix}
            0 & -1 \\ 1 & 0
        \end{matrix}
        \text{ et } K = IJ = \begin{matrix}
            0 & -i\\ -i & 0
        \end{matrix}
    \]
    On définit alors $\mathbb{H} = \vect[\R]{1, I, J, K} \subset \mat{2}{\C}$\\

    On définit de plus : 
    \[
        t(q) = \Tr(q), n(q) = \det q \text{ et } q^{\star} = \transpose{\bar{q}} = t(q)1 - q \in \H
    \]
\end{definition}

\begin{proposition}
    $\H$ est un corps gauche de centre $\R$.
\end{proposition}

\begin{proposition}[Cayley - Hamilton]
    Par théorème de Cayley-Hamilton sur $\mat{2}{\C}$ : $q^{2}t(q)q + n(q)1 = 0$ ce qui ici vaut : 
    \[qq^{\star} = q^{\star}q = n(q)1\]
\end{proposition}


\subsubsection{Le groupe $Sp(1)$}
\begin{definition}
    On pose $Sp(1) = \left\{q \in \H \mid n(q) = 1\right\}$. C'est un sous-groupe de $\H^{\times}$
\end{definition}

\begin{remark}
    L'application : 
    \[
        \applic{\R^{4}}{\H}{\left(t, x, y, z\right)}{t + xI + yJ + zK}
    \]
    identifie la sphère unité euclidienne $S^{3}$ à $Sp(1)$, ce qui munit $S^{3}$ d'une loi de groupe non commutative par transfert de structure. On sait que $S^{1}$ et $S^{3}$ sont les deux seules sphères euclidiennes que l'on peut munir d'une loi de groupe topologique.
\end{remark}

\begin{remark}
    $Sp(1)$ s'identifie à $\left\{\begin{matrix}\alpha & -\bar{\beta} \\ \beta & \bar{\alpha}\end{matrix}\mid \abs{\alpha}^{2} + \abs{\beta}^{2} = 1\right\}$ de $SL_{2}(\C)$.
\end{remark}

\begin{proposition}
    Par décomposition polaire : $\H^{\times} = \R_{> 0} \times Sp(1)$
\end{proposition}

\begin{proposition}
    L'élément $-1$ est l'unique élément d'ordre $2$ de $Sp(1)$
\end{proposition}

\begin{proposition}
    Un élément $q \in Sp(1)$ est d'ordre $m > 2$ si et seulement si $t(q) = 2 \cos(2k / m)$ avec $k \in \Z$ et $k \wedge m = 1$
\end{proposition}

\subsubsection{L'espace euclidien $\H$}
\begin{definition}
    On définit sur $\H$ un produit scalair réel par $\scalar(q, q^{'}) = \frac{1}{2}t(q^{\star}q^{'})$
\end{definition}

\begin{proposition}
    L'application $Sp(1) \times Sp(1) \to O(\H)$ qui à $\left(q_{1}, q_{2}\right) \mapsto L_{q_{1}}R_{q_{2}}$ est un morphisme d'image $SO(\H)$ et de noyau $\scalar{(-1, -1)} \simeq \znz{2}$ où $L_{q}$ désigne la translation à gauche par $q$ et $R_{q}$ la translation à droite. 
\end{proposition}

\begin{proposition}
    L'application $Sp(1) \rightarrow SO(\H^{0})$, $q\mapsto \text{int}_{q_{\mid\H^{0}}}$ ou $\H^{0} = 1^{\perp} = \left\{q \in \H \mid t(q) = 0\right\}$.
\end{proposition}
=======
>>>>>>> d3f8ccda5fc06e4828340c67f282e2474a630bdf



\end{document}



