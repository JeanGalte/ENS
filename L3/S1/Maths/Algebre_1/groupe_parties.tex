\documentclass{cours}

\title{Sur une Manière de Construire une Loi de Groupes sur les Parties d'un Groupe Fini}

\begin{document}
\section{Action de $\znz{n}$ sur ses Parties}
\subsection{Additive}
Dans cette sous-section on considère l'action de $\znz{n}$ sur l'ensemble $\mP$ de ses parties définie par : 
\[
	\forall P \subseteq \znz{n}, \forall k \in \znz{n}, k\oplus P = \left\{k + x \mid x \in P\right\}
\]
Cette action est fidèle mais elle n'est pas libre. En considérant $2 \in \znz{4}$ et $P = \{0, 2\}$, on a $2 \oplus P = P$. Pour la même raison, cette action n'est pas transitive.

\subsection{Multiplicative}
On considère l'action de $\znz{n}$ sur l'ensemble $\mP$ de ses parties définie par : 
\[
	\forall P \subseteq \znz{n}, \forall k \in \znz{n}, k\otimes P = \left\{k x \mid x \in P\right\}
\]
Cette action est fidèle mais elle n'est pas libre : En considérant $x \in \znz{p}^{*}$ et $P = \znz{p}$, on a $x \otimes P = P$ car $x$ est inversible. 

\section{Groupe des Parties de $\znz{n}$}
\subsection{Loi de Corps}
Comme $\abs{\mP\left(\znz{n}\right)} = 2^{n}$, on peut définir par transfert de structure grâce à une bijection ensembliste entre $\mP\left(\znz{n}\right)$ et $\mathbb{F}_{2^{n}} = \mathbb{F}_{2}[X] / P_{2, n}$ où $P_{2, n} \in \mathbb{F}_{2}_{n}[X]$ est irréductible. 
Cette loi n'est pas très naturelle au sens où elle n'a aucun rapport avec $\znz{n}$ mais avec $\znz{2}^{n}$. 

\subsection{Loi de Groupe}


\end{document}
