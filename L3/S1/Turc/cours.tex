\documentclass{cours}
\title{Cours de Turc Débutant}
\author{Marc Aymes}
\date{2123-2124}
\usepackage{longtable}
\usepackage{nicematrix}

\newcommand{\ch}{\c{s}}
\newcommand{\Ch}{\c{S}}
\newcommand{\I}{\.{I}}
\newcommand{\ug}{\u{g}}
\newcommand{\sci}{\textsc{i}}
\newcommand{\sca}{\textsc{a}}
\newcommand{\sce}{\textsc{e}}

\begin{document}
\section{Birinci Ders}
\subsection{Quelques Mots}
\begin{tabular}{>{\it}p{.5\textwidth}|>{\it}p{.5\textwidth}}
    \toprule
    taksi                                                          & okul                                       \\
    gar (\text{précédé d'un morphème qui indique le type de gare}) & tuvalet (\text{les toilettes/la toilette}) \\
    pazar (\text{marché/dimanche})                                 & tekstil                                    \\
    spor                                                           & üniversite                                 \\
    telefon                                                        & bisküvi (\text{biscuit})                   \\
    fotokopi                                                       & kuaför                                     \\
    ferforje                                                       & asansör                                    \\
    mikrop                                                         & klasör                                     \\
    deterjan                                                       & mösyö                                      \\
    banliyö                                                        & feribot                                    \\
    istatistik                                                     & fi\ch (\text{prise électrique / facture})  \\
    kontaklens                                                     & afi\ch                                     \\
    istasyon (\text{gare, presque pareil que gar})                 & ti\ch ört                                  \\
    tren                                                           & \ch ömendöfer                              \\
    pantolon                                                       & \ch ike                                    \\
    lambda                                                         & \ch oke                                    \\
    banyo (\text{salle de bain/bain})                              & maç                                        \\
    klima (\text{climatisation})                                   & ofsayt (\text{hors-jeu})                   \\
    kürdan                                                         & gofret                                     \\
    ekipman                                                        & foto\ug raf                                \\
    otomatikman                                                    & co\ug rafya (\text{géographie})            \\
    mayo                                                           & tüyo (\text{information/tip})              \\
    \bottomrule
\end{tabular}

\subsection{\textit{Sesli ve sessiz harfler} (Voyelles et consonnes)}
\begin{itemize}
    \item Accentuation: L'accent tonique porte généralement sur la dernière syllabe du mot, sauf suffixe ou particule \emph{enclitique} qui déplace l'accent sur la syllabe précédente.
    \item L'alphabet turc comporte 29 lettres dont 8 voyelles :
          \begin{center}
              abcçdefg\ug h\i ijklmnoöprs\ch tuüvyz
          \end{center}
    \item Lettres singulières : \textit{ç, \ug, \i, i, ö, s, \ch, u, ü}
\end{itemize}

\subsection{Prononciation}
\begin{itemize}
    \item Le \textit{l} est généralement \og plat\fg mais devient \og creux \fg devant ou après \textit{a, \i, o, u}
    \item On prononce le doublement consonantique
    \item Il y a un assourdissement (dévoisement)/sonorisation (voisement) des consonnes finales :
          \begin{center}
              \textit{b, d, g, c} $\leftrightarrow$ \textit{p, t, k, ç}
          \end{center}
          Une consonne finale redevient sonore lorsqu'une voyelle lui est suffixée.
    \item \textit{\ug} est \og doux \fg (\textit{yumu\ch ak}) muet, allonge la voyelle qui le précède. Entre deux voyelles, il ne se prononce pas. Après des voyelles \og fine \fg, dites aussi antérieures ou palatales (\textit{e, i, ö, ü}) il se prononce souvent en consonne \textit{y}.
    \item \textit{h} est toujours \og aspiré \fg, sauf dans \textit{Mehmet}
    \item \textit{\i} n'est ni \og i \fg ni \og ö \fg. Il se prononce en ramenant la langue en arrière
    \item \textit{y} est toujours une consonne, jamais un \og i \fg.
\end{itemize}

\subsection{\textit{Lütfen dikkat}}
\begin{itemize}
    \item Le verbe \og être \fg au présent n'a pas d'équivalent en turc. On recourt à un équivalent nominal utilisé comme prédicat. Auquel on adjoint \emph{facultativement} le suffixe \textit{-d\i r}. On parle donc de {\bf prédication nominale}. La proposition formée est donc nominale et non verbale. \textit{De\ug il} est la forme négative de cette prédication nominale.
    \item La particule interrogative \textit{m\textsc{i}} :
          \begin{itemize}
              \item s'emploie en l'absence de pronom interrogatif
              \item se place aussitôt après le mot sur lequel porte l'interrogation
              \item est enclitique (accentue la syllage qui la précède)
          \end{itemize}
    \item Le comportement des voyelles est régie par la règle de l'hamonie vocalique (en \textsl{small caps} ci-après).
\end{itemize}

\section{Ikinci Ders}
\subsection{Morphologie Générale}
\begin{itemize}
    \item Il n'y a pas de genre en turc
    \item Il y a deux catégories de mots: noms et verbes
    \item Pour les noms, il y a une différence entre emploi comme substantif, adjectif et adverbe, selon l'emploi.
    \item L'adjectif est invariable, il n'y a pas d'accord morphologique des mots entre eux (en nombre et en cas. )
\end{itemize}

\subsection{Harmonie Vocalique}
La plupart des mots sont régis par l'harmonie vocalique. Les suffixes sont vocalisés suivant deux types :
\begin{center}
    \textit{e/a} ou \textit{i/\i/ü/u}
\end{center}
On a alors les changements suivants :
\begin{tabular}{>{\it}c@{\ \ $\Rightarrow$\ \ }>{\it}c}
    a/\i & a \text{ ou } \i \\
    e/i  & e \text{ ou } i  \\
    o/u  & a \text{ ou } u  \\
    ö/ü  & e \text{ ou } ü
\end{tabular}

\subsection{\textit{Biti\ch kenlik}: Suffixes Dérivatifs}
Le turc est une langue agglutinante. Les racines lexicales et suffixes s'ajoutent les uns aux autres dans un ordre syntaxiquement réglé. La suffixation est le procédé unique de la morphologie turque. Il existe aussi des agglutinations non suffixales : \textit{günay-d\i n}. Les suffixes peuvent être nominaux ou verbaux. \\
Les suffixes de dérivation forment des noms à partir de noms ou de verbes :
\begin{center}
    \begin{tabular}{>{\bf}cc@{\ \ $\longrightarrow$\ \ }>{\it}cc}
        Suffixe  & Sens            & Exemple   & Traduction                    \\
        -l\sci   & être muni de    & renkli    & coloré                        \\
        -s\sci z & être privé de   & i\ch siz  & chômeur                       \\
        -c\sci   & métier/activité & ögrenci   & étudiant                      \\
        -da\ch   & compagnonnage   & arkada\ch & ami (litt., compagnon de dos) \\
        -l\sci k & diminutif       & deftercik & petit cahier                  \\
    \end{tabular}
\end{center}
\subsection{Pronoms Personnels}
\begin{center}
    \begin{tabular}{cccc}
                  & 1ère    & 2ème    & 3ème      \\
        Singulier & \it Ben & \it Sen & \it O     \\
        Pluriel   & \it Biz & \it Siz & \it Onlar
    \end{tabular}
\end{center}
On peut suffixer ces pronoms : \textit{sensiz} veut dire sans toi, \textit{benci} égocentrique et \textit{benlik} égo.

\subsection{Nombres}
On compte en base dix en MSB :
\begin{center}
    \begin{tabular}{c>{\it}c|c>{\it}c}
        1   & bir   & 10   & on         \\
        2   & iki   & 20   & yirmi      \\
        3   & üç    & 30   & otuz       \\
        4   & dört  & 40   & k\i rk     \\
        5   & be\ch & 50   & elli       \\
        6   & alt\i & 60   & altm\i \ch \\
        7   & yedi  & 70   & yetmi\ch   \\
        8   & sekiz & 80   & seksen     \\
        9   & dokuz & 90   & doksan     \\
        100 & yüz   & 1000 & bin
    \end{tabular}
\end{center}

\section{\textit{Üçünçü Ders}}
\subsection{\textit{Ünlü uyumu - etrafl\i ca} : Catégorisation des voyelles}
\begin{center}
    \linespread{1.5}
    \begin{NiceTabular}{p{.2\linewidth}cccc}
                                                          & \multicolumn{2}{c}{\textit{Düz} (plane) = non labiale} & \multicolumn{2}{c}{\textit{Yuvarlak} (ronde) = labiale}                                   \\
                                                          & \textit{Geni\ch} (ouverte)                             & \textit{Dar} (étroite)                                  & \textit{Geni\ch} & \textit{Dar} \\
        \textit{Kal\i n} (épaisse) = postérieure, vélaire & \it \bf a                                              & \it \bf \i                                              & \it \bf o        & \it \bf u    \\
        \textit{\I nce} (fine) = antérieure, palatale     & \it \bf e                                              & \it \bf i                                               & \it \bf ö        & \it \bf ü

        \CodeAfter
        \begin{tikzpicture}
            \draw [gray] (1-|1) -- (1-| 2) ;
            \draw [black] (1-|2) -- (1-| 6) ;
            \draw [gray] (2-|1) -- (2-| 2) ;
            \draw [black] (2-|2) -- (2-| 6) ;
            \draw [black] (3-|1) -- (3-| 6) ;
            \draw [black] (4-|1) -- (4-| 6) ;
            \draw [black] (5-|1) -- (5-| 6) ;
            \draw [gray] (1-|1) --(3-|1);
            \draw [gray] (1-|2) --(3-|2);
            \draw [black] (3-|1) -- (5-|1);
            \draw [black] (3-|2) -- (5-| 2);
            \draw [black] (2-|3) -- (5-| 3);
            \draw [black] (1-|4) -- (5-| 4);
            \draw [black] (2-|5) -- (5-| 5);
            \draw [black] (1-|6) -- (5-|6);
        \end{tikzpicture}
    \end{NiceTabular}
\end{center}
S'ensuivent, suivant \og La tendance humaine naturelle au moindre effort musculaire \fg :
\begin{itemize}
    \item si le mot commence par une voyelle antérieure, celles qui suivent le sont également ; idem pour les postérieures.
    \item si la première voyelle est non labiale, les suivantes le sont également. Ex. {\it i\ch-siz}
    \item si la première voyelle est labiale, les suivantes sont soit labiales fermées (ex. {\it yol-cu}), soit non labiales ouvertes (ex. {\it yol-da}, en chemin).
\end{itemize}

Le vocable est riche en termes qui sont en eux-mêmes des exceptions à ces règles :
\begin{itemize}
    \item \textit{anne, dahi, elma, hangi, hani, inanmak, karde\ch, selam, \ch i\ch man, tiyatro, viraj, ziyaret}
    \item les mots composés : \textit{aç\i kgöz, bilgisayar, çekyat, han\i meli}
    \item certains suffixes sont invariables {\it -da\ch, -ki} : \textit{din-da\ch, gönül-da\ch, meslek-ta\ch, ülk£u-da\ch, ak\ch amki, yar\i nki, duvardaki, yoldaki}
\end{itemize}
Dans la plupart des cas, les suffixes se règlent sur la dernière voyelle du mot, sauf pour certains mots dits \og d'emprunt \fg ou exceptions fameuses gouvernées par une forme d'harmonie consonantique ({\it saat-te}).

\subsection{\textit{Biti\ch kenlik} - Suffixes Nominaux}
Sur la base nominale se greffent principalement (outre les suffixes de dérivation dont l'apprentissage relève davantage du vocabulaire que de la grammaire) trois types de suffixes dont l'étude grammaticale est nécessaire :
\begin{enumerate}
    \item Suffixe de Nombre : Le Pluriel\\
          \begin{itemize}
              \item Emploi Nominal : \textit{iyi gün-ler, iyi ak\ch am-lar} ; ou usage idiomatique avec un prénom : \textit{Marc'lar} = Marc et ses proches
              \item Emploi Verbal : \textit{gidiyor-lar} = ils partent
          \end{itemize}
    \item Suffixe de Personne : Le Possessif
          \begin{center}
              \begin{tabular}{cccc}
                            & 1ère           & 2ème           & 3ème         \\
                  Singulier & -(\sci)m       & -(\sci)n       & -(s)\sci     \\
                  Pluriel   & -(\sci)m\sci z & -(\sci)n\sci z & -l\sca r\sci
              \end{tabular}
          \end{center}
          Exemples : \textit{okulum, kitab\i n, kalemi, yüzü\ug ümüz, defteriniz, apartmanlar\i, dairem, hocan, lisesi, Fransa'm\i z, bölgeniz, üyeleri}
    \item Suffixes de Cas :
          \begin{center}
              \begin{NiceTabular}{c>{\bf}ccc>{\it}cc}
                  \multirow{4}{.15\linewidth}{Cas Spatiaux}     & Locatif   & -d\sca     & sur/en/dans         & ben-de         & sur moi         \\
                                                                & Complété  & -ki        & \og qui est \fg     & ben-de-ki      & qui est sur moi \\
                                                                & Ablatif   & -d\sca n   & venir de, approcher & \I stanbul'dan & Stanbouliote    \\
                                                                & Directif  & -(y)\sca   & aller à, s'éloigner & ev-e           & à la maison     \\
                  \multirow{2}{.15\linewidth}{Cas Grammaticaux} & Génitif   & -(n)\sci n & ComDéfini du Nom    &                &                 \\
                                                                & Accusatif & -(y)\sci   & ComDéfini du Verbe  &                &
                  \CodeAfter
                  \begin{tikzpicture}
                      \draw [dotted] (1 -| 2) -- (7 -| 2);
                      \draw [black] (2 -| 2) -- (2 -| 7);
                      \draw [black] (3 -| 2) -- (3 -| 7);
                      \draw [black] (4 -| 2) -- (4 -| 7);
                      \draw [black] (5 -| 2) -- (5 -| 7);
                      \draw [black] (6 -| 2) -- (6 -| 7);
                  \end{tikzpicture}
              \end{NiceTabular}
          \end{center}
\end{enumerate}
La déclinaison des pronoms personnels est irrégulière à la première personne au directif et au génitif. \\
Au cas absolu, tout \emph{nom} turcc est employé sans aucun suffixe de cas, sans que cela l'empêche néanmoins de remplir les fonctions grammaticales les plus variées:
\begin{itemize}
    \item Sujet : \textit{çocuk gülüyor} = l'enfant rit
    \item Complément du Nom : \textit{çocuk arabas\i} = voiture d'enfant
    \item COD : \textit{bu kad\i n üç çocuk yeti\ch iyor} = cette femme élève trois enfants
    \item Complément Post-Positionnel : \textit{çocuk için} = pour les enfants ; \textit{çocuk gibi} = comme un enfant, \textit{çocuk ile} = avec l'enfant
\end{itemize}
Les suffixes de cas ne se cumulent pas, ils prennent place après les suffixes de nombre et de personne (dans cet ordre). Quand le suffixe de cas s'ajoute à un suffixe de troisième personne, une \emph{consonne de liaison} \og n \fg s'intercale entre les deux :
\begin{itemize}
    \item \textit{ev-im-de} = dans ma maison; \textit{ev-ler-de} = dans les maisons; \textit{ev-ler-im-de} = dans mes maison.
    \item \textit{ev-i-n-de} = dans sa maison; \textit{ev-leri-n-de} = dans leur(s) maison(s) \emph{ou} dans ses maisons (car le cumul \textit{ev-ler-leri} n'est pas possible).
    \item Au directif, la consonne de liaison \og y \fg cède la place à \og n \fg: on dit \textit{ev-i-n-e} et non pas \textit{ev-i-y-e}.
\end{itemize}
Dans le cas d'un mot se terminant par une consonne sourde, la consonne initiale de certains suffixes s'assourdit : {\it d} devient {\it t}, {\it c} devient {\it ç}, etc...
Par exemple : \textit{bu hedefte} = dans cette cible; \textit{bu amaçta} = dans ce but; \textit{bu ahbapta} = chez ce pote; \textit{bu arkada\ch ta} = chez ce camarade; \textit{bu saatte} = à cette heure-ci (exception à l'harmonie vocalique).\\
De même pour le suffixe d'activité \textit{-c\sci} : \textit{sütçü} = le laitier; \textit{topçu} = l'artilleur; \textit{tarihçi} = l'historien.

\section{Dördüncü Ders}
\subsection{Suffixes Nominaux de Personne}
\begin{center}
    \begin{tabular}{c>{\it}c>{\it}c>{\it}c}
                  & 1ère       & 2ème       & 3ème    \\
        Singulier & ad\i m     & ad\i n     & ad\i    \\
        Pluriel   & ad\i m\i z & ad\i n\i z & adlar\i
    \end{tabular}
\end{center}
Ce type de suffixes est dit \og possessif \fg, car il est notamment utilisé pour traduire l'adjectif possessif français. Cependant sa fonction plus générale est d'exprimer la relation entre un nom et une personne. Ce suffixe joue un rôle essentiel dans la formation générique du rapport d'annexion, qui s'effectue par ajout d'un suffixe de 3e personne au nom complété. Le nom complément, quant à lui, est soit au génitif soit au cas absolu:
\begin{itemize}
    \item Le génitif marque le complément défini du nom : \textit{çocu\ug un arabas\i} = la voiture de l'enfant $\neq$ \textit{çocuk arabas\i} = la voiture d'enfant
    \item Le complément indéfini du nom demeure au cas absolu : \textit{Türkiye Cumhuriyeti} (république); lieux géographiques : \textit{Van gölü}, variétés botaniques ou zoologiques (\textit{semizotu} = pourpier, \textit{hamamböce\ug i} = cafard, \textit{\ch am f\i st\i \ug \i } = \textit{antep f\i st\i \ug \i} = pistache, \textit{çamf\i st\i \ug \i} = pignon de pin.)
    \item Les Épithètes de nationalité : \textit{Frans\i z ö\ug rencisi} mais \textit{Amerika'l\i ö\ug renci}. D'une manière générale, les noms désignant des groupes nationaux, sociaux ou ethniques forment des complément du nom.
\end{itemize}
Il est impossible de superposer plusieurs suffixes possessifs; le cas échéant, le suffixe de troisième personne cède la place aux suffixes de première ou de deuxième personne. \\
Au titre des combinaisons à possessifs multiples, on applique le principe d'économie : \textit{su tesis hizmetleri} = les services d'aménagement hydraulique.

A la différence du français, le complément de matière n'est pas (sauf en cas d'association inattendue) construit comme un complément du nom, mais comme épithète au cas absolu. On dit \textit{deri ceket} et non \textit{deri ceketi} pour dire \og la veste en cuir \fg. Il en va de même pour l'expression de la quantité : \textit{üç kilo ekmek} = trois kilos de pain. Dans certaines combinaisons, le premier nom prend valeur adjective, il n'y a alors pas de suffixe de personne: \textit{ara sokak} = rue de traverse; \textit{ana fikir} = idée-force.

\subsection{\textit{Rakamlar} - Numération}
Ordinaux :
\begin{center}
    \begin{tabular}{>{\it}c@{$\longrightarrow$}>{\it}c}
        Bir   & Birinci  \\
        \I ki & \I kinci \\
        üç    & Üçünçü   \\
        \dots & \dots
    \end{tabular}
\end{center}
Distributifs :
\begin{center}
    \begin{itemize}
        \item Suffixe \textit{-(\ch)\sce r}: équivalent de \og tant par tête \fg : \textit{iki\ch er elmam\i z var} = nous avons chacun deux pommes
        \item Et redoublé : \textit{alt\i \ch ar, alt\i \ch ar} = par groupes de six, six par six.
    \end{itemize}
\end{center}
Fractions :
\begin{center}
    \begin{itemize}
        \item En général, $\frac{x}{y} = $ \textit{X-d\sca Y}. Par exemple, \textit{ikide bir} = un sur deux.
        \item En particulier, \textit{yar\i m} = un demi (adjectif); \textit{yar\i} = moitié (substantif) ; \textit{buçuk} = ... et demi
    \end{itemize}
\end{center}

\subsection{\textit{Çok mu, az m\i ?}}
Après un qualificatif de nombre ou de quantité, on n'ajoute pas de suffixe de pluriel \textit{-l\sce r} : \textit{çok po\ug aça istiyorum} ou \textit{be\ch\ litre kahve}. Il en va de même pour tout ensemble collectif d'items non individualisés (quoiqu'éventuellement dénombrables): \textit{kitap al\i yorum} = je prends des livres; \textit{elma veriyorum} = je donne des pommes. Exception à valeur emphatique : \textit{çok te\ch ekkürler} = merci beaucoup.\\
\textit{ne kadar}, littéralement \og autant que quoi ?\fg, peut être distingué de \textit{kaç} qui interroge sur le nombre plutôt que sur la quantité. Ainsi, \textit{ne kadar paran var?} appelle une réponse du type \textit{az/çok} tandis que \textit{kaç paran var?} suppose de répondre numériquement (même si {\it kaç} peut également prendre une valeur indéfinie et \textit{kaça?} être synonyme de \textit{ne kadar?} pour demander le prix.)

\section{Be\ch inci Ders}
\subsection{Suffixe Nominal Prédicatif}
\begin{center}
    \begin{tabular}{>{\it\bf}c>{\it}c>{\it}c>{\it}c>{\it}c}
        -(y)\sci m          & iyi-yim        & Frans\i z-\i m         & Türk-üm       & o\ug ul-um       \\
        -s\sci n            & iyi-sin        & Frans\i z-s\i n        & Türk-sün      & o\ug ul-sun      \\
        -(d\sci r)          & iyi-(dir)      & Frans\i z-(d\i r)      & Türk-(dür)    & o\ug ul-(dur)    \\
        -(y)\sci z          & iyi-yiz        & Frans\i z-\i z         & Türk-üz       & o\ug ul-uz       \\
        -s\sci n\sci z      & iyi-siniz      & Frans\i z-s\i n\i z    & Türk-sünüz    & o\ug ul-sunuz    \\
        -(d\sci r)(l\sce r) & iyi-(dir)(ler) & Frans\i z-(d\i r)(lar) & Türk-(dür)ler & o\ug ul-(dur)lar \\
    \end{tabular}
\end{center}

\subsection{Verbes et Suffixes Nominaux}
Le verbe est formé d'une \emph{base}. Dans les dictionnaires elle est donnée augmentée du suffixe \textit{-m\sce k} de l'infinitif.\\
La conjugaison régulière s'effectue par l'ajout du suffixe nominal prédicatif. Dans la majorité des cas donc, on peut parler de conjugaison prédicative de type nominal.

\subsection{Conjugaison Prédicative de Type Nominal}
Il faut bien marquer la différence entre suffixe nominal prédicatif et suffixe nominal de personne. L'accentuation permet de trancher en cas d'ambiguïté : les prédicatifs sont enclitiques, les suffixes de personne sont accentués.



\section{Dokuzuncu Ders}
\subsection{Orhan Gencebay | \textit{Hatas\i z kul olmaz}}
\textit{a\ch k\i na} = Pour l'amour de
\subsection{Les neuf caractères de la déclinaison verbale}
Il faut souligner l'irréductibilité du système de conjugaison à la division entre temps, aspects et modes. En turc certaines formes temporelles peuvent, selon leur emploi, relever de plusieurs modes ou aspects.

\begin{description}
    \item[Progressif:] {\it -(i)yor}
        \begin{itemize}
            \item exprime k'actuib dans son développement concret, avec une nuance d'actualisation descriptive.
            \item à l'origine, dérivé de \textit{yor\i r}, aoriste de l'ancien \textit{yor\i mak}, \og aller, marcher\fg.
            \item prêter attention aux modifications vocaliques :
                  \begin{itemize}[label = \textbullet]
                      \item dans le cas d'une base verbale consonantique, la voyelle insérée avec \textit{-yor} suit l'harmonie vocalique: \textit{geliyor, görüyor, al\i yor, ko\ch uyor}
                      \item base verbale se terminant en \textit{e/a}: \textit{bekle} $\to$ \textit{bekliyor}, \textit{anla} $\to$ \textit{anl\i yor}. De même avec le négatif : \textit{al- + -ma- + -yor} $\to$ \textit{alm\i yor}, etc...
                      \item mais si la voyelle initiale du suffixe est \og ronde \fg (= labiale, \textit{o, u, ö, ü}) \emph{et} si la première voyelle de la base verbale l'est également, le \textit{e} ou {\it a} en question le sera aussi: \textit{yolla- + -yor} $\to$ \textit{yolluyor} (envoyer); \textit{türe} devient \textit{türüyor} (dériver); \textit{gözle} $\to$ \textit{gözlüyor} (observer); \textit{topla} $\to$ \textit{topluyor} (amasser)
                  \end{itemize}
        \end{itemize}
    \item [Duratif:] {\it - m\sce kt\sce}
          \begin{itemize}
              \item décrit une action en cours
          \end{itemize}
    \item [Intentif:] {\it -(y)\sce c\sce k}
          \begin{itemize}
              \item exprime l'intention; employé pour les faits dont l'accomplissement futur est tenu pour décidé.
              \item différence avec l'éventuel ou \og temps large\fg (infra) qui exprime l'éventualité sur un plan plus objectif; l'intentif exprime une probabilité plus élevée que l'aoriste à l'affirmatif; mais plus faible au négatif.
          \end{itemize}
    \item [L'éventuel ou \og temps large\fg (geni\ch\ zaman) (aoriste):] \textit{-(\sce)r, -(i)t}
          \begin{itemize}
              \item exprime :
                    \begin{enumerate}
                        \item l'action dans sa généralité, sans actualisation ni limitation de durée;
                        \item une éventualité;
                        \item une formule de requête polie: \textit{afferdersiniz, outrur musunuz ?; olur mu ?}
                    \end{enumerate}
              \item \textit{-r} après un radical verbal terminé par une voyelle: \textit{anlar, benzer, der, korur, yer}
              \item \textit{-\sce r} après un radical verbal monosyllabique terminé par une consonne: \textit{yapar, eder, kaçar, geçer, doyar} (être rassasié)
              \item \textit{-\sci r} après une base verbale polysyllabique, ou monosyllabique dérivée: \textit{b\i rak\i r, doyurur};
              \item exceptions ! Il y a 13 verbes monosyllabiques qui, bien que leur base soit non dérivée et terminée par une consonne forment leur aoriste en \textit{-(\sci r)} : \textit{al\i r, bilir, bulur, durur, gelir, görür, kal\i r, olur, ölür, san\i r, var\i r, verir, vurur} [Ne pas confondre avec les formes irrégulières du factitif.]
              \item le négatif aoriste:
                    \begin{enumerate}
                        \item est irrégulier: \textit{-m\sce z} (réduit à \textit{-m\sce} aux 1ères personnes) accentué, suivi d'une déclinaison personnelle quasi régulière (\textit{gelmem, gelmezsin, gelmez, gelmeyiz, gelmezsiniz, gelmezler}).
                        \item avec l'interrogatif, permet de former (en langue parlée) un présent immédiat: \textit{O s\i rada villan\i z gözüme çarpmaz m\i\dots}
                    \end{enumerate}
          \end{itemize}
    \item [Constatif:] \textit{-d\sci}
          \begin{itemize}
              \item exprimer le résultat d'une action ou d'un processus constatés par le locuteur. \textit{geldim} = j'arrive tout de suite, je suis là
              \item à la différence de l'aoriste, du progressif et du médiatif, il ne peut pas être projeté dans le future, et n'entre donc jamais dans la formation d'expressions du futur antérieur français.
              \item le suffixe est accentué: on distingue ainsi \textit{vard\`i} (\textit{var} + suffixe verbal constatif = \og il est arrivé \fg) et \textit{v\`ardi} (\textit{var} + suffixe prédicatif constatatif = \og il y en avait\fg).
          \end{itemize}
    \item [Médiatif (non-constatation)/inférentiel (dubitatif):] \textit{-m\sci\ch}
          \begin{itemize}
              \item exprime le résultat acquis d'une action ou d'un processus non constatés, ainsi qu'un résultat inattendu.
              \item combiné avec \textit{olacak} il permet d'exprimer le futur antérieur français: \textit{yar\i m saat sonra bu dersi bitimi\ch \ olaca\ug \i z} aussi bien qu'un futur simple d'état: \textit{ders yar\i m saat sonra bitmi\ch \ olacak}
              \item conjugé avec \textit{-d\sci r} ou d'autres formes du verbe être, cette forme perd sa connotation inférentielle et s'assimile à un passé défini: \textit{gelmi\ch tir}, il est venu.
          \end{itemize}
    \item [Déontique:] \textit{-m\sce li}
          \begin{itemize}
              \item exprime l'obligation physique, logique, technique, sociale, morale...
          \end{itemize}
\end{description}

\appendix
\newpage
\begin{longtable}{>{\it}p{.21\textwidth}p{.21\textwidth}|>{\it}p{.21\textwidth}p{.21\textwidth}}
    \multicolumn{4}{c}{\bf \large Petit Dictionnaire en Turc}                                               \\
    \toprule
    Mot en Turc      & Traduction                          & Mot en Turc    & Traduction                    \\
    \midrule \midrule
    \endfirsthead
    \toprule
    Mot en Turc      & Traduction                          & Mot en Turc    & Traduction                    \\
    \midrule \midrule
    \endhead
    \bottomrule
    \endfoot

    kitap            & livre                               & a\ug aç        & arbre                         \\
    \midrule
    sa\ug            & sain                                & yuvarlak       & rond                          \\
    \midrule
    da\ug            & montagne                            & kibrit         & allumette                     \\
    \midrule
    i\ug ne          & aiguille                            & alçak          & bas (adj.)                    \\
    \midrule
    ya\ug mur        & pluie                               & güzel          & beau                          \\
    \midrule
    ince             & fin                                 & iyi            & bon                           \\
    \midrule
    e\ug lence       & amusement                           & çirkin         & laid                          \\
    \midrule
    y\i ld\i r\i m   & foudre                              & \ch eker       & sucre                         \\
    \midrule
    ay               & lune                                & kad\i n        & femme                         \\
    \midrule
    çay              & thé                                 & sand\i k       & caisse/coffre                 \\
    \midrule
    ay\i             & ourse                               & koltuk         & fauteuil                      \\
    \midrule
    day\i            & oncle maternel                      & tavan          & plafond                       \\
    \midrule
    bu               & ce, cette, ces (ceci, proche)       & duvar          & mur                           \\
    \midrule
    da               & aussi                               & ev             & maison                        \\
    \midrule
    de\ug il         & n'est pas                           & el             & main                          \\
    \midrule
    demek            & dire                                & gül            & rose                          \\
    \midrule
    Frans\i zca      & français (langue)                   & yüzük          & anneau/bague                  \\
    \midrule
    hay\i r          & non                                 & etek           & jupe/flancs géographiques     \\
    \midrule
    kap\i            & porte                               & gömlek         & chemise                       \\
    \midrule
    masa             & porte                               & erkek          & garçon                        \\
    \midrule
    m\sc{i}          & (particule interrogative)           & ceket          & veste                         \\
    \midrule
    ne               & quoi, que                           & çizgi          & trait/tracé                   \\
    \midrule
    pencere          & fenêtre                             & ayak           & pied                          \\
    \midrule
    sandalye         & chaise                              & kapal\i        & fermé                         \\
    \midrule
    \ch u            & ce, cette, ces (cela, moins proche) & aç\i k         & ouvert                        \\
    \midrule
    beyaz            & blanc                               & bo\ch          & vide                          \\
    \midrule
    k\i rm\i z\i     & rouge                               & dolu           & complet                       \\
    \midrule
    sar\i            & jaune                               & kirli          & sale                          \\
    \midrule
    pembe            & rose                                & temiz          & propre                        \\
    \midrule
    ye\ch il         & vert                                & pahal\i        & cher                          \\
    \midrule
    siyah            & noir                                & ucuz           & bon marché                    \\
    \midrule
    yeni             & nouveau                             & eski           & vieux                         \\
    \midrule
    a\ug \i r        & lourd                               & hafif          & léger                         \\
    \midrule
    uzun             & long                                & k\i sa         & court                         \\
    \midrule
    ad               & prénom                              & ders           & leçon                         \\
    \midrule
    ho\ch            & gai/heureux                         & ho\ch geldiniz & soyez le bienvenu             \\
    \midrule
    memnum oldum     & enchanté                            & renk           & couleur                       \\
    \midrule
    i\ch             & travail                             & e\ch           & pair                          \\
    \midrule
    ögrenmek         & étudier                             & arka           & dos                           \\
    \midrule
    din              & religion                            & adam           & homme                         \\
    \midrule
    defter           & cahier                              & kalem          & crayon                        \\
    \midrule
    var              & existant/il y a                     & yok            & absent/il n'y a pas           \\
    \midrule
    anne             & mère                                & dahi           & de plus                       \\
    \midrule
    elma             & pomme                               & hangi          & quel                          \\
    \midrule
    hani             & ah ça mais                          & inanmak        & croire                        \\
    \midrule
    karde\ch         & frère                               & selam          & salut                         \\
    \midrule
    \ch i\ch man     & enflé                               & ziyaret        & visite                        \\
    \midrule
    aç\i kgöz        & dégourdi                            & bilgisayar     & ordinateur                    \\
    \midrule
    han\i meli       & chèvrefeuille                       & çekyat         & canapé-lit                    \\
    \midrule
    dinda\ch         & coreligionnaire                     & gönülda\ch     & ami intime                    \\
    \midrule
    meslekta\ch      & collègue                            & ülküda\ch      & compagnon de cause, d'idéal   \\
    \midrule
    ak\ch amki       & du soi                              & yar\i nki      & de demain                     \\
    \midrule
    süt              & lait                                & top            & artillerie                    \\
    \midrule
    tarih            & histoire                            & saat           & heure                         \\
    \midrule
    yar\i m          & un demi (adjectif)                  & yar\i          & moitié (substantif)           \\
    \midrule
    buçuk            & ... et demi                         & usta           & artisan                       \\
    \midrule
    sa\ug            & sain                                & peynir         & fromage                       \\
    \midrule
    zeytin           & olive                               & sade           & nature (plain/regular)        \\
    \midrule
    tane             & exemplaire                          & üstüne         & au-dessus                     \\
    \midrule
    az               & peu                                 & iyi o zaman    & dans ce cas                   \\
    \midrule
    al\i yorum       & prendre                             & ba\ch ka       & autre                         \\
    \midrule
    arzu             & souhait                             & yan\i nda      & avec/à côté                   \\
    \midrule
    \ch eker         & sucre                               & kahve          & café                          \\
    \midrule
    gelmek           & venir                               & gitmek         & partir                        \\
    \midrule
    afiyet           & appétit                             & olmak          & être/se produire              \\
    \midrule
    Borcum ne kadar? & Combien vous dois-je ?              & para           & monnaie/argent                \\
    \midrule
    gün              & jour                                & haber          & nouvelle(s)                   \\
    \midrule
    nas\i ls\i n     & comment vas-tu                      & sa\ug ol       & merci                         \\
    \midrule
    ara              & intervalle                          & s\i ra         & rang/tour de passage          \\
    \midrule
    ara s\i ra       & de temps à autres                   & konu\ch mak    & parler                        \\
    \midrule
    gerçek           & réalité                             & ögretmek       & enseigner                     \\
    \midrule
    i\ch te          & ben voilà quoi                      & u\ug ra\ch mak & faire des efforts             \\
    \midrule
    art\i k          & désormais, à parti de là            & yollamak       & envoyer                       \\
    \midrule
    türemek          & dériver                             & gözlemek       & observer                      \\
    \midrule
    toplamak         & amasser                             & almak          & prendre                       \\
    \midrule
    beklemek         & attendre                            & affetmek       & pardonner                     \\
    \midrule
    doymak           & être rassasié                       & b\i rakmak     & abandonner                    \\
    \midrule
    yenmek           & vaincre                             & yemek          & manger                        \\
    \midrule
    fener            & phare                               & sanmak         & penser                        \\
    \midrule
    a\ch k           & amour, passion                      & kaybetmek      & perdre                        \\
    \midrule
    bende            & esclave, serviteur                  & kul            & esclave                       \\
    \midrule
    can              & âme, être en vie                    & olmaz          & n'est pas (ne sauraît être)   \\
    \midrule
    derman           & remède, force                       & ne olur        & je t'en prie, je t'en supplie \\
    dert             & peine, maladie, douleur, tracas     & olsa           & si c'est                      \\
    \midrule
    dil              & langue                              & ölmek          & mourir                        \\
    \midrule
    duymak           & entendre, ressentir                 & raz\i          & prêt, consentant, content     \\
    \midrule
    elde olmak       & être disponible, accesible, conquis & salmak         & lâcher, libérer, dégager      \\
    \midrule
    feryat           & cri, lamentation                    & seven          & (cellui) qui aime             \\
    \midrule
    gönül            & coeur, désir, inclination           & sitem          & reproche                      \\
    \midrule
    güç              & force, difficile                    & söz            & mot, parole                   \\
    \midrule
    haber            & nouvelle                            & tez            & prompt, sous peu              \\
    \midrule
    hal              & condition, état                     & uzak           & loin                          \\
    \midrule
    hasret           & désir, envie, nostalgie             & ümit           & loin                          \\
    \midrule
    hata             & faute                               & ya\ch amak     & vivre                         \\
    \midrule
    kendim           & soi-même                            & kalp           & coeur                         \\
    \midrule
    yormak           & fatiguer                            & yorulmak       & être fatigué
\end{longtable}


\end{document}