\documentclass{Cours}
\date{\today}
\author{Matthieu Boyer}
\title{Homework Assignment 2}

\begin{document}
    \section{Exercise 1}
        \subsection{Question 1}
            For $i, j$ in $\lvert 1, n \rvert^{2}$, we have $AB_{i, j} = A_{i*}B_{*j} = \sum_{k = 1}^{n} A_{i, k}B_{k, j}$, thus $AB$ is computed by computed, for all $i, j$ the product $A_{i,k}B_{k,j}$ and thus uses at most $\sum_{k = 1}^{n} a_{k}b_{k}$ multiplications.

        \subsection{Question 2}
            The number of multiplications and additions is two times the number of multiplications. We just need to get a majoration of the number of multiplications. Yet, since $a_{k} \leq n$ for all $k$, $\sum_{k = 1}^{n} a_{k} b_{k} \leq n \sum_{k=1}^{n}b_{k} = mn$. Then the number of multiplications and additions required is $\O(mn)$.

        \subsection{Question 3}
            Multiplying a matrix in $\mathcal{M}_{ap, bp}$ by a matrix $\mathcal{M}_{bp, cp}$ can, by seeing the matrices as block matrices, be seen as multiplying two matrices in $\mathcal{M}_{p}$ the number of times we need to compute the product of matrix in $\mathcal{M}_{a,b}$ by a matrix  in $\mathcal{M}_{b, c}$ :
            \[
                \left(\begin{matrix}
                    \alpha_{1, 1} & \ldots & \alpha_{1, bp}\\
                    \vdots & & \vdots\\
                    \alpha_{ap, 1} & \ldots & \alpha_{ap, bp}
                \end{matrix}\right)
                = \left(
                \begin{matrix}
                    A_{1, 1} & \ldots & A_{1, b}\\
                    \vdots & & \vdots\\
                    A_{a, 1} & \ldots & A_{a, b}
                \end{matrix}\right)
                \text{ where } A_{i, j} = \left(
                    \begin{matrix}
                        \alpha_{i p + 1, j p + 1} & \ldots & \alpha_{i p + 1, j (p + 1)}\\
                        \vdots & & \vdots\\
                        \alpha_{i (p + 1), j p + 1} & \ldots & \alpha_{i (p + 1), j (p + 1)}
                    \end{matrix}                    
                    \right) \in \mathcal{M}_{p}
            \]  
            Thus, we can multiply a matrix in $\mathcal{M}_{ap, bp}$ by a matrix $\mathcal{M}_{bp, cp}$ in $M(a, b, c)M(p, p, p)$ multiplications. Thus : 
            \[
                M(ap, bp, cp) \leq M(a, b, c)M(p, p, p)
            \]
            
        \subsection{Question 4}
            \begin{itemize}
                \item If $0 \leq r \leq \alpha$: $w(1, r, 1)$ is the smallest number $k$ such that $M(n, n^{r}, n) = \O(n^{k + o(1)})$. But, again by seeing $A$ an $n\times n^{\alpha}$ matrix as a $n\times n^{r}$ matrix next to a $n, n^{\alpha} - n^{r}$ matrix and same for $B$, we get $M(1, r, 1) \leq M(1, \alpha, 1)$ and thus $w(1, r, 1) \leq w(1, \alpha, 1) = 2$.
                \item If $\alpha \leq r \leq 1$ : by seeing a $n \times n^{r}$ matrix $A$ as a $n^{\frac{1 - r}{1 - \alpha}} \times n^{\frac{\left(1 - r\right)\alpha}{1 - \alpha}}$ bloc matrix with blocks of size $n^{\frac{r - \alpha}{1 - \alpha }}\times n^{\frac{r - \alpha}{1 - \alpha}}$ and applying the reasoning from 3. we get that : 
                \[
                    \begin{aligned}
                        M(n, n^{r}, n) = &\ M\left(n^{\frac{1 - r}{1 - \alpha}} \cdot n^{\frac{r - \alpha}{1 - \alpha }}, n^{\frac{\left(1 - r\right)\alpha}{1 - \alpha}}\right.\\ 
                        &\ \left.\times n^{\frac{r - \alpha}{1 - \alpha}}, n^{\frac{1 - r}{1 - \alpha}} \cdot n^{\frac{r - \alpha}{1 - \alpha }}\right)\\
                         \leq&\  M\left(n^{\frac{1 - r}{1 - \alpha}}, n^{\frac{(1 - r)\alpha}{1 - \alpha}}, n^{\frac{1 - r}{1 - \alpha}}\right)\\
                        &\ \times M\left(n^{\frac{r - \alpha}{1 - \alpha}}, n^{\frac{r - \alpha}{1 - \alpha}}, n^{\frac{r - \alpha}{1 - \alpha}}\right)\\
                        = &\ \O\left(\left(n^{\frac{1 - r}{1 - \alpha}}\right)^{{w(1, \alpha, 1)}}\left(n^{\frac{r - \alpha}{1 - \alpha}}\right)^{\omega}\right)\\
                        = &\ \O\left(n^{\frac{2 * (1 - r) + \left(r - \alpha\right) \omega}{1 - \alpha}}\right)
                    \end{aligned}      
                \]
                The first big O equality comes from a substitution in $M(n, n^{\alpha}, n) = \O(n^{w(1, \alpha, 1)})$ of $n$ by $n^{\frac{1 - r}{1 - \alpha}}$.
                We obtain : 
                \[
                    \begin{aligned}
                        w(1, r, 1)  \leq &\ \frac{2 \times \left(1 - r\right) + \left(r - \alpha\right)\omega}{1 - \alpha}\\
                        = &\ \frac{2 \times \left(1 - \alpha\right) + 2 \times \left(\alpha - r\right) + \left(r - \alpha\right) \omega}{1 - \alpha}\\
                        = &\ 2 + \frac{\omega \times \left(\alpha - r\right) - 2 \times \left(\alpha - r\right)}{1 - \alpha}\\
                        = &\ 2 + \beta(r - \alpha)
                    \end{aligned}
                \]
            \end{itemize}
            
        \subsection{Question 5}
            Let $1 \leq l \leq n$, $a_{k}, b_{k}$ such that $a_{k}b_{k}$ is decreasing.
            If $l = 1$, the result is trivial. 
        
            Consider for $i \in \left\llbracket 1, l - 1\right\rrbracket$ the quantity $a_{i}b_{j} + a_{j}b_{i}$, we get :
            \begin{itemize}
                \item If $a_{i} > a_{j}$ : $a_{i}b_{j} + a_{j}b_{i} > a_{j}b_{j}$
                \item Else : $a_{i}b_{j} + a_{j}b_{i} > a_{i}b_{i} > a_{j}b_{j}$ by hypothesis.
            \end{itemize}

            Then we get : 
            \[\sum_{j = i + 1}^{n} a_{i}b_{j} + a_{j}b_{i} > \sum_{k = l}^{n} a_{k}b_{k}\] since the $a_{k}$ and $b_{k}$ are positive.
            By summing :  
            \[\sum_{i = 1}^{l -1}\sum_{j = i + 1}^{n} a_{i}b_{j} + a_{j}b_{i} > l\sum_{k = l}^{n} a_{k}b_{k}\]
            But : 
            \[m_{1}m_{2} = \sum_{i = 1}^{n}a_{i} \sum_{j = 1}^{n}b_{j} = \sum_{i = 1}^{n}\sum_{j = i + 1}^{n}a_{i}b_{l} > \sum_{i = 1}^{l - 1}\sum_{j = i + 1}^{n}a_{i}b_{j}\]
            Thus :
            \[\sum_{k = l}^{n} a_{k}b_{k} < \frac{m_{1}m_{2}}{l}\]

        \subsection{Question 6}
            
\end{document}