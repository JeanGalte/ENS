\documentclass[12pt]{article}

\title{\textbf{Rapport de Rendu du Simulateur Netlist\\ \small SysNum 2023-2024}}
\author{\textbf{Matthieu Boyer}}
\date{\textbf{8 Novembre 2023}}
\usepackage[a4paper, margin = 3cm]{geometry}
\usepackage{lmodern}
\usepackage[T1]{fontenc} 
\usepackage[dvipsnames]{xcolor}
\usepackage{booktabs}
\usepackage{longtable}
\usepackage{array}


\begin{document}
\definecolor{ulmj}{RGB}{255, 255, 20}
\definecolor{ulmv}{RGB}{54, 1, 63}
\pagecolor{ulmv}
\color{ulmj}
\maketitle
\textbf{Les opérations à réaliser par le simulateur ont été implémentée de la manière détaillée dans le tableau ci-dessous :}\\
    \begin{longtable}{>{\bfseries\color{ulmj}}p{6cm}>{\bfseries\color{ulmj}}p{8cm}}
        \toprule
        Opération & Détails sur l'Implémentation\\
        \midrule
        Stocker les Variables & Une référence contenant une instance du module environnement, modifiée à chaque équation.
        \\
        \midrule
        Lire les Entrées & On lit l'entrée standard avec \textmd{read\_int} et on modifie le contexte courant en accord, en ajouant les valeurs au contexte de l'étape précédente \textmd{ctx}. Celui-ci est défini à la fin de l'étape de résolution des équations.
        \\
        \midrule
        Résoudre les Equations de la Netlist & Pour chaque équation, on a une méthode dédiée de résolution, détaillée plus bas. On ajoute au contexte courant \textmd{env} le résultat de l'équation en modifiant la référence associée.
        \\
        \midrule
        \textmd{Earg} & On renvoie la valeur actuelle dans \textmd{!env} ou la valeur de \textmd{v} dans \textmd{Aconst(v)}.
        \\
        \midrule
        \textmd{Ereg} & On renvoie la valeur stockée dans \textmd{ctx}, le contexte précédent. Si elle n'existe pas, on renvoie une valeur par défaut : $0$.\\
        \midrule
        \textmd{Enot} & On renvoie un \textmd{VBit} dont la valeur a été inversée.\\
        \midrule
        \textmd{Ebinop} & On effectue le calcul bit à bit de l'opérateur, en définissant un par un les comportements de chaque opérateur.\\
        \midrule
        \textmd{Emux} & On évalue la condition, puis selon le résultat, on renvoie la valeur de l'argument 1 ou de l'argument 2.\\
        \midrule
        \textmd{Erom} & On calcule la position de départ en fonction de la taille du mot et de la lecture de l'adresse, puis, selon la taille du mot, on renvoie soit un \textmd{VBit} soit une \textmd{VBitArray} contentant les bits lus.\\
        \midrule
        \textmd{Eram} & De manière similaire à \textmd{Erom}, on lit la valeur à la position définie par l'adresse de lecture et la taille des mots, selon le paramètre \textmd{write\_enable} on modifie une liste buffer d'écriture, en agrandissant si besoin le tableau référence représentant \textmd{ram}, puis on renvoie la valeur à l'index. On écrit finalement dans la ram une fois l'entièreté des équations résolues.\\
        \midrule
        \textmd{Econcat} & On concatène l'intérieur des deux \textmd{VBitArray} en traitant les \textmd{VBit} commes des arrays de taille 1.\\
        \midrule
        \textmd{Eslice} & On agit sur l'argument de la \textmd{VBitArray}\\
        \midrule
        \textmd{Eselect} & De même \\
        \midrule
        Ecriture des sorties & On se donne une fonction pour print un booléen, qu'on applique ensuite à l'ensemble des sorties du programme. 
        \\
        \bottomrule
    \end{longtable}

\end{document}options