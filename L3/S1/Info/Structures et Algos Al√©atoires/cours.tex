\documentclass{cours}
\title{Structures et Algorithmes Aléatoires}
\author{Ana Busic \\ \small{Romain Cosson}}
\date{\today}

\begin{document}
\part{Cours 1 29/09}
\section*{Introduction}
Cours de probabilités discrètes i.e. sur des espaces au plus dénombrables et applications à différents domaines.
\begin{itemize}
    \item Probabilités discrètes et applications : 
    \begin{itemize}
        \item Rappels
        \item Algorithmes aléatoires
        \item Méthode probabiliste
        \item Graphes aléatoires    
    \end{itemize}
    \item Modèles Markoviens : 
    \begin{itemize}
        \item Chaînes de Markov et Comportement Asymptotique
        \item Simulation Monte Carlo et Simulation Parfaite
        \item Extensions et applications : modèles markoviens cachés, modèles markoviens de décision, champs de Gibbs, automates cellulaires probabilistes
    \end{itemize}
\end{itemize}

\section{Algorithmes Probabilistes}
Algorithme déterministe : $\forall inputs, f(inputs)$ est unique et calculé en temps fini. On ne sait pas toujours avoir une réponse rapide et correcte en utilisant un algorithme déterministe. \\
On peut ajouter de l'aléa sous forme de bits aléatoires : on perd le caractère de sortie unique.

On peut toujours modifier l'algorithme de sorte à avoir : \begin{itemize}
    \item Toujours une réponse correcte, et rapide dans la plupart des cas
    \item Une réponse correcte dans la plupart des cas, mais tout le temps rapide/ 
\end{itemize}

Exemple : Deux polynômes de degré $d$ sont-ils égaux ? \\
De manière déterministe, on passe sous forme canonique, et on a un $\O(d^2)$.\\
De manière probabiliste, on prend $r \in {1, \ldots, 100d}$ uniformément, on calcule $F(r)$ et $G(r)$ et on teste l'égalité.\\
Si $F \neq G$, il y a erreur si et seulement si $r$ est racine de $F-G$, il y a donc une probabilité d'erreur d'au plus $\frac{1}{100}$. 

Exemple : File d'attente\\
On note $X(n)$ le nombre de clients après le départ du $n$-ième client, $A(n)$ le nombre qui arrivent pendant le service du $n$-ième client. \\
Alors : $X(n+1) = \max(X(n)-1, 0) + A(n)$.\\
$(X(n))_{n\in\N}$ forme une chaîne de Markov parfois, et est toujours un processus stochastique. 

Exemple : Indépendants d'un Graphe\\
Un indépendant de $G$ est un ensemble de sommets $I$ non reliés deux à deux entre eux. On cherche à calculer sa taille moyenne.\\
Echantillonner un sous-ensemble n'est pas une solution si jamais le nombre d'indépendants est petit devant le nombre de parties.\\
La bonne méthode est de construire une chaîne de Markov. 

\section{Révisions de Probabilités}
Franchement, tu connais. 

\section{Classification d'Algorithmes Probabilistes}
Algorithme de Monte Carlo : Résout un problème de manière approchée avec erreur contrôlée en une complexité qui est une fonction déterministe des données. \\
La probabilité d'erreur est la probabilité, pour tout input que le programme renvoie une erreur. L'erreur unilatérale si il ne se trompe que sur un des deux cas. \\

Algorithme de Las Vegas : Résout un problème exactement en minimisant la complexité moyenne finie. On peut transformer un MC en LV en le répétant tant qu'il fait une erreur.

On dit qu'un algo termine avec proba $\alpha$ si toute instance termine avec probabilité au moins $\alpha$. Il termine presque sûrement si $\alpha = 1$. Il termine s'il termine sur toute entrée. 


\end{document}