\documentclass{cours}

\title{Homework Assignment 2}
\date{\today}
\author{Matthieu Boyer}

\begin{document}
    \section{Question 1}
        In datasets $4 - 5 - 6$, the modified sentences are built on the model of datasets $1 - 2 - 3$.\\
        However, they are not natural sentences. Indeed, sentence $4.d)$ would more naturally be said as \textsl{Who did the person Mary met like ?}.\\
        Yet, it would be natural to extract from datasets $1$ to $3$ this rule for building questions, and those datasets are made of frequently said sentences describing common events/questions, while the standard interrogative sentences built from datasets $4$ to $6$ are far less common (in French and English) as they would use passive modality.\\
        Thus, this shows the difficulty for a child to extrapolate the rules for building questions (and thus learning language) from their environment.

    \section{Question 2}
        From sentences $a$ and $b$ we get that morphemes \textsl{kin} means \textsl{I} and \textsl{ku} means \textsl{he}, since those morphemes are the only that differ. Similarly, we get from sentences $d$ and $e$ that \textsl{ka} means \textsl{you}.\\
        Adding to that sentences $c$ and $g$, we find that those morphemes also indicate the present tense, and that \textsl{shu} means \textsl{he} in a past tense. We extrapolate that \textsl{shin} and \textsl{sha} thus mean \textsl{I} and \textsl{you} in a past tense. Thus \textsl{k} indicates the present tense and \textsl{sh} indicates past tense \\
        From sentences, $c, d, e$, we get that \textsl{wetam\'{a}xlek\'{e}m} means \textsl{learning the (art of) weaving} and so that \textsl{katax\'{ı}n} means \textsl{continually}.\\
        Given sentences $f$ and $g$, we can get that \textsl{wetam\'{a}x} means \textsl{to learn} and \textsl{lek\'{e}m} means \textsl{(the art of) weaving}.\\
        From sentences $a$ and $b$ we identify the morpheme \textsl{sik\'{ı}xlel\'{ı}br} to mean \textsl{to study the book}, and, adding sentence $h$ we get \textsl{w\'{ı}r} means \textsl{yesterday}.\\
        Based on our knowledge of other languages, we might want to hypothetise that in \textsl{sik\'{ı}xlel\'{ı}br}, \textsl{sik\'{ı}x} means \textsl{study}, \textsl{le} means \textsl{the} and \textsl{l\'{ı}br} means \textsl{book}. 
        In fact, based on the syntaxic order of then sentences we have fully deciphered, we may extrapolate that this language follows a \textsl{subject - verb - object} order of words, and so the hypothesis before seems reasonible, but still needs hard evidence. \\
        To sum up, we obtain this table \ref{fig1}\\
        \begin{table}
            \centering
            \begin{tabular}{ll}
                \toprule
                Morpheme &Meaning\\
                \midrule
                \textsl{k} &Present Tense\\
                \textsl{sh}&Past Tense\\
                \textsl{in} &\textsl{I} \\
                \textsl{a} &\textsl{you}\\
                \textsl{u} &\textsl{he} \\
                \textsl{katax\'{ı}n} &\textsl{continually}\\
                \textsl{wetam\'{a}x} &\textsl{to learn}\\
                \textsl{w\'{ı}r} &\textsl{yesterday}\\
                \textsl{lek\'{e}m}&\textsl{(the art of) weaving}\\
                \textsl{sik\'{ı}x} &\textsl{study}\\
                \textsl{le} &\textsl{the}\\
                \textsl{l\'{ı}br} &\textsl{book}\\
                \bottomrule
            \end{tabular}        
            \caption{Table of Morphemes in the Dataset}
            \label{fig1}
        \end{table} 
        
\end{document}