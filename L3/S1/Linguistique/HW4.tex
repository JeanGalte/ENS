\documentclass{cours}
\usepackage{qtree}
\date{\today}
\title{Homework Assignment 4}
\author{Matthieu Boyer}

\begin{document}
\begin{quote}
    In this document, we will use $\bot$ for \textmd{false} and $\top$ for \textmd{true} to indicate the answer to true-false questions. I discussed this assignment with Perrine Dupont, which may have lead to similarities in our answers. (This was written in \LaTeX)w
\end{quote}

\section{Exercise 1}
\subsection{Question 1}
\tocless\subsubsection{Question a.}
$\bot$: The word \textsl{action} is not a verb in english since it never appears with the suffix \textsl{ing} or in the construction \textsl{to + verb}. It is a noun, as it can be preceeded by a determiner such as \textsl{the} or \textsl{a}, complemented by an adjective as in \textsl{the final action} or be the object of a verb group: \textsl{to realize an action}.

\tocless\subsubsection{Question b.}
$\bot$: The word \textsl{modifier} is not an adjective in english since it cannot be added to a noun group, e.g. \textsl{the modifier house} or \textsl{a modifier color}. This word is a noun : \textsl{I added a modifier to this colour}. There might be ambiguity in the sentence \textsl{I added a colour modifier}, but here \textsl{colour} acts as a complement to the noun \textsl{modifier}.

\tocless\subsubsection{Question c.}
$\top$: The word \textsl{couper} is a verb as it bears the suffix \textsl{er} and can be conjugated, such as in \textsl{Je coupe, Tu coupes, Il coupe, Nous coupons, Vous coupez, Ils coupent}, it can bear the mark of a tense : \textsl{couperai} or \textsl{coupâmes}. 

\tocless\subsubsection{Question d.}
$\top$: The word \textsl{chaque} is a determiner since it can preceed any noun group, e.g. \textsl{chaque fille} or \textsl{chaque décomposition en facteurs premiers}, and as it cannot be 

\section{Exercise 2}
\subsection{Question 1}
\tocless\subsubsection{Question a.}
$\bot$: No, the specified order in ($\alpha$) is not observed, since $B$ should come before $C$.

\tocless\subsubsection{Question b.}
$\bot$: No, the specified order in ($\delta$) is not observed, since $z$ should come before $w$.

\tocless\subsubsection{Question c.}
$\bot$: No, there is no rule of the form $A \rightarrow C$, and, even then, the specified order in ($\delta$) is not observed, since $z$ should come before $w$.

\tocless\subsubsection{Question d.}
$\top$: From ($\alpha$), the structure $A \rightarrow B$ exists, and from ($\beta$), the structure $A \rightarrow x$ exists.

\tocless\subsubsection{Question e.}
$\bot$: There is no rule allowing $B$ to have two \og children \fg in the tree, i.e. no rule of the form $B \rightarrow \aleph\ \beth$ or $B \rightarrow \aleph\ (\beth)$. Both ($\beta$) and ($\gamma$) cannot apply here since they plan for $B$ to have a unique child, either $x$ or $y$.

\tocless\subsubsection{Question f.}
$\bot$: There is no rule allowing for $C$ to have a unique children $w$. The specified order in ($\delta$) is not observed since there is no $z$ to come before $w$.

\subsection{Question 2}
\tocless\subsubsection{Question a.}
$\bot$: The sentence here is \textsl{Joe thinks it is cold here}. It can be separated into two proposition : \textsl{Joe} and \textsl{thinks it is cold here}. The first one is a noun phrase while the second one is a verb phrase containing a verb \textsl{thinks} and a complement phrase.
\begin{center}
    \Tree [.S \qroof{Joe}.NP  [.VP [.V thinks ] \qroof{it is cold here}.CP ] ]
\end{center}

\tocless\subsubsection{Question b.}
$\top$: The sentence here is \textsl{The milk perished}. It is composed of a noun phrase \textsl{The milk} and a verb phrase \textsl{perished}. The first is composed of a determiner \textsl{the} and a noun \textsl{milk}.

\tocless\subsubsection{Question c.}
$\bot$: The sentence is \textsl{Joe met Sam}. Here we have a noun phrase constituted of a noun \textsl{Joe}, and a verb phrase made of a verb \textsl{met} and a noun complement \textsl{Sam} :
\begin{center}
    \Tree [.S \qroof{Joe}.NP [.VP [.V met ] \qroof{Sam}.NP ] ]
\end{center}

\tocless\subsubsection{Question d.}
$\bot$: The sentence here is \textsl{Joe recalled the charges against him}. It is composed of a noun phrase \textsl{Joe} and a verb phrase \textsl{recalled the charges agains him}. The verb phrase is constituted of a verb \textsl{recalled} and a noun phrase \textsl{Joe} and a verb \textsl{recalled}. The complement phrase is constituted of a noun phrase, \textsl{the charges} where the \textsl{the} is a determiner and \textsl{charges} is a noun, and a prepositional (?) phrase \textsl{against him} where \textsl{against} is a proposition and \textsl{him} is a noun phrase made of a single pronoun.
\begin{center}
    \Tree [.S \qroof{Joe}.NP [.VP [.V recalled ] [.NP \qroof{the charges}.NP [.PP [.P against ] \qroof{him}.NP ] ] ] ]
\end{center}

\tocless\subsubsection{Question e.}
$\bot$: The sentence here is \textsl{Fat people eat accumulates}, and it is composed of a noun phrase \textsl{Fat people eat} and a verb phrase \textsl{accumulates}. The noun phrase is made of a noun \textsl{fat}, and a verb phrase \textsl{people eat}, itself made of a noun phrase \textsl{people} and a verb \textsl{eat}.
\begin{center}
    \Tree [.S [\qroof{Fat}.NP \qroof{people eat}.CP? ] [.VP [.V accumulates ] ] ]
\end{center}

\tocless\subsubsection{Question f.}
$\bot$: The sentence here is \textsl{Time flies like an arrow}. It is composed of a noun phrase \textsl{Time} and a verb phrase \textsl{flies like an arrow}, made of a verb \textsl{flies} and a prepositional phrase \textsl{like an arrow} composed of a preposition \textsl{like} and a noun phrase \textsl{an arrow}.
\begin{center}
    \Tree [.S \qroof{Time}.NP [.VP  [.V flies ] [.PP [.P like ] \qroof{an arrow}.NP ] ] ]
\end{center}

\tocless\subsubsection{Question g.}
$\bot$: The sentence is \textsl{Joe hoped that you had left}. It is composed of a noun phrase \textsl{Joe} and a verb phrase \textsl{hoped that you had left} which is made of a verb \textsl{hoped} and a prepositional phrase \textsl{that you had left}. This phrase is built of a preposition \textsl{that} and a tense phrase \textsl{you had left}, in which \textsl{you} is a noun phrase, \textsl{had} is a tense marker, and \textsl{left} is a verb.
\begin{center}
    \Tree [.S \qroof{Joe}.NP [.VP [.V hoped ] [.PP [.P that ] \qroof{you had left}.VP ] ] ]
\end{center}

\tocless\subsubsection{Question h.}
$\bot$: The sentence is \textsl{The accident deprived him of his mobility}. \textsl{The accident} is a noun phrase, \textsl{deprived him of his mobility} is a verb phrase where \textsl{deprived} is a verb, \textsl{him} is a noun phrase and \textsl{of his mobility} is a prepositional phrase.
\begin{center}
    \Tree [.S \qroof{The accident}.NP [.VP [.V deprived ] \qroof{him}.NP [.PP [.P of ] \qroof{his mobility}.NP ] ] ]
\end{center}

\end{document}