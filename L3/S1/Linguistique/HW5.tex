\documentclass{cours}
\usepackage{qtree}

\title{Homework Assignment 6}
\date{\today}
\author{Matthieu Boyer}

\begin{document}
    \section{Exercise 1}
        \subsection{Question 1}
        $\top$ : If \textsl{Nobody spoke or danced}, in particular, if someone spoke somebody would have \textsl{spoke or danced} thus : 
        \[
            \textsl{Nobody spoke or danced} \vdash \textsl{Nobody spoke}
        \]  
        Indeed, \textsl{Nobody spoke or danced} does not have the same meaning as \textsl{Nobody spoke or Nobody danced}.

        \subsection{Question 2}
        $\bot$ : If \textsl{Nobody slept well}, that does not mean \textsl{Nobody slept}. The demarcation is to be seen between \textsl{Nobody} and \textsl{slept well}, thus :
        \[
            \textsl{Nobody slept well} \nvdash \textsl{Nobody slept}
        \]

        \subsection{Question 3}
        $\top$ : If \textsl{Everybody slept well}, that means that \textsl{Everybody slept} and that the sleeping they did was \textsl{well}.
        \[
            \textsl{Everybody slept well} \vdash \textsl{Everybody slept}
        \]

        \subsection{Question 4}
        $\top$ : If \textsl{Not all students laughed}, that means that the set $\mathscr{S}$ of students who laughed is not equal to the set of all students, meaning there is at least an element in its complement (in the set of all students), and thus, that \textsl{At least one student didn't laugh} :
        \[
            \textsl{Not all students laughed} \vdash \textsl{At least one student didn't laugh}
        \]

        \subsection{Question 5}
        $\bot$ : In the previous question we showed that $\mathscr{S}^{\complement} \neq \emptyset$. Yet, we cannot conclude on its cardinal, and thus cannot extrapolate on the number of students who didn't laugh : we cannot say that \emph{necessarily}, \textsl{Several students didn't laugh} as we can imagine a joke about someone who makes everyone laugh but that person :
        \[
            \textsl{Not all students laughed}  \nvdash \textsl{Several students didn't laugh}
        \]

        \subsection{Question 6}
        $\top$ : If \textsl{Someone spoke or danced}, that means there exist an individual who \textsl{spoke or danced} and thus that \textsl{Someone spoke or someone danced} :
        \[
            \textsl{Someone spoke or danced} \vdash \textsl{Someone spoke or someone danced}
        \]  

        \subsection{Question 7}
        $\bot$ : If \textsl{Everybody spoke or danced}, that means that each individual either \textsl{spoke}, \textsl{danced} or did both. We cannot ensure that everybody \textsl{Everybody spoke} or that \textsl{Everybody danced}, simply that the union of those sets covers the set of all individuals considered : 
        \[
            \textsl{Everybody spoke or danced}  \nvdash \textsl{Everybody spoke or everybody danced}
        \]

        \subsection{Question 8}
        $\top$ : If \textsl{Everybody spoke and danced}, that means that each individual did both \textsl{speak} and \textsl{dance}. We can then ensure that everybody \textsl{Everybody spoke} and that \textsl{Everybody danced}, thus the intersection of those sets covers the set of all individuals considered : 
        \[
            \textsl{Everybody spoke and danced}  \vdash \textsl{Everybody spoke and everybody danced}
        \]

    \section{Exercise 2}
        \subsection{Question 1}
            \subsubsection{Question a.}
                \begin{itemize}
                    \item \textsl{Mediocre} : From the following sentences, we see that \textsl{mediocre}, just as \textsl{fake} has a functional nature.
                    \begin{center}
                        \begin{tabular}{c}
                            \textsl{John is a mediocre lawyer but a great carpenter}\\
                            \textsl{This meal is mediocre}
                        \end{tabular}
                    \end{center}
                    Indeed, we see in the first sentence that \textsl{John is a lawyer} and that \textsl{John is not mediocre} since \textsl{John is a great carpenter}. Moreover, just as for \textsl{fake}, we see from the second sentence that \textsl{mediocre} does not necessarily need a \textit{physical} word next to it, but may apply to a trace of the word described.

                    \item \textsl{Beautiful} : From the following sentences, we see that \textsl{mediocre}, just as \textsl{fake} has a functional nature.
                    \begin{center}
                        \begin{tabular}{c}
                            \textsl{John is a beautiful person with an ugly face}\\
                            \textsl{This piece of art is beautiful}\\
                            \textsl{Mary is a beautiful dancer}
                        \end{tabular}
                    \end{center}
                    Applying the same arguments, we see in the first sentence that \textsl{John is a person} and that \textsl{John is not beautiful} since \textsl{John has an ugly face}. Moreover, just as in the first part of this question, we see from the second sentence that \textsl{beautiful} can be applied to the trace of a word.\\ 
                    In the third sentence however, there might be two interpretations : either \textsl{Mary is beautiful and Mary is a dancer}, which would be an intersective use or \textsl{Mary dances beautifully}, which would be a subsective use. 
                \end{itemize}

            \subsubsection{Question b.}
                We will study the adjective \textsl{French} :
                \begin{itemize}
                    \item It seems that \textsl{French} is an intersective adjective : if \textsl{Jean is a French firefighter}, then \textsl{Jean is French} and \textsl{Jean is a firefighter}. Moreover, if \textsl{Jean is an arsonist}, then \textsl{Jean is a French arsonist}. 
                    \item Yet, in the sentence \textsl{These yellow things are French Fries}, \textsl{French} modifies \textsl{fries} but not yellow things : there is no reason for the \textsl{yellow things} to be \textsl{French}. This is a non-intersective use of \textsl{French}
                \end{itemize}
        
                \subsection{Question 2}
                \subsubsection{Predicate 1 : \textsl{Difficult}}
                    It seems that \textsl{difficult} has a free variable of the form ${\left[for\ X\right]}$ after it indicating the person it refers to, much like \textsl{delicious}. In the sentence \textsl{skateboarding is difficult}, one can reply it is the easiest thing they've ever done and lead to \textsl{Well, skateboarding is difficult \textit{for me}}.
                    Then, the values it might take are things that can experience difficultness, such as humans, dogs or living creatures. In the following, the italic text symbolises the value of the free variable :
                    \begin{itemize}
                        \item \textsl{This is a difficult \textit{for me} exam}
                        \item \textsl{Opening nuts is difficult \textit{for birds}}
                    \end{itemize}
    
                    Sometimes, it also seems that \textsl{difficult} can take a second free variable of the form ${\left[to\ X\right]}$, representing the thing that is complemented :
                    \begin{itemize}
                        \item A : \textsl{I absolutely love linguistics\footnote{This was a sentence pronounced during the writing of this homework.} !}\\
                        B : \textsl{Me too, but it is so difficult \textit{for me} \textit{to do linguistics}...}
                        \item A : \textsl{I can't believe he was late.}\\
                        B : \textsl{He is so difficult \textit{for me} \textit{to live with him}}
                    \end{itemize}
                    This second variable can take as an input anything that can be found difficult, mainly actions such as skateboarding, cooking, or doing linguistics.
    
                \subsubsection{Predicate 2 : \textsl{Legal}}
                    It seems that \textsl{legal} has a free variable of the form ${\left[in\ X\right]}$ which indicates the location in which everything it is true : a county, a state, a country, a region and so on : 
                    \begin{itemize}
                        \item \textsl{Cannabis is legal \textit{in the Netherlands}}
                        \item \textsl{Going 300 miles an hour in a car is legal \textit{on the Highway in Germany}}
                    \end{itemize}
    
                    And, as for \textsl{difficult}, we can say that \textsl{legal} takes another free variable of the form $\left[to \  X\right]$ representing what the illegal action is. 
                    \begin{itemize}
                        \item A : \textsl{Man, I really do love to smoke cannabis ! What about you ?}\\
                        B : \textsl{Dude, it's illegal \textit{to smoke cannabis} \textit{in Britain} !}
                    \end{itemize}
    
                \subsubsection{Predicate 3 : \textsl{Of the utmost importance}}
                    Following the previous question, it seems that \textsl{Of the utmost importance} takes a free variable of the form $\left[to \ X\right]$ which indicates to whom the considered object if \textsl{of the utmost importance}. The values it may take are things that can have a link to the importance of a thing, such as people, groups of people, or something with a changing value (see examples 3 and 4) :
                    \begin{itemize}
                        \item \textsl{Linguistics are of the utmost importance \textit{to me}}
                        \item \textsl{Whether Michael slept well or not is of the utmost importance \textit{to the students he grades}}
                        \item \textsl{This exam is of the utmost importance \textit{to your scolarity}}
                        \item \textsl{The pluviometry is of the utmost importance \textit{to the crops}}
                    \end{itemize}
    
                    Again, as in the two previous examples, we can say that \textsl{of the utmost importance} takes another free variable of the form $\left[to \ X\right]$ which indicates the thing it refers to : 
                    \begin{itemize}
                        \item A : \textsl{Do you think I need to do my linguistics homework ?}\\
                        B : \textsl{Yes, it is of the utmost importance \textit{to do your homework} \textit{to your success} !}
                    \end{itemize}
    
                    Then, we may add a third free variable right after \textsl{utmost}, of the form $\left[compared\ to \ X\right]$. It indicates, relatively to the context, on what scale is the importance considered : 
                    \begin{itemize}
                        \item A : \textsl{I am not going to complete my linguistics homework...}\\
                        B : \textsl{Well, you have to, it is of the utmost \textit{compared to your other projects} importance}
                        \item A : \textsl{I need to eat today}\\
                        B : \textsl{Yes, it is of the utmost \textit{compared to anything that can be done} importance !}
                    \end{itemize}
    


\end{document}

