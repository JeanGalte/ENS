\documentclass{cours}
\usepackage{enumitem}

\title{Cours N°1}
\author{Salvador Mascarenhas \\ \small Michael Goodale (TA)}
\date{21 Septembre 2023}


\begin{document}

\section*{Introduction}
    DEC is called like that because is would have bothered
    Prof : Master at NYU, Junior Research at Oxford, here since 2016
    Using language as a window into human mind\\
    Linguistics is a broad term for serious principled study of language. 
    Many perspectives, from the cognitive point or study language by an external perspective (structuralism : Ferdinand de Saussure, or Leonard Bloemfield) looking out on structures.
    Also quite general here.
    Language looked at as a social entity.
    History of languages, typology of languages.
    Not only about cognitive studies, pretty broad look out.\\
    Teaching assistant : Michael Goodale PhD student, LRSCP, computational models of Language Acquisitions. Statistical inference and model language after formal tools.
    Practical Skills really.\\
    Assessments : Homework, graded on a qualitative schedule, due on lecture days.
    Final : Last Lecture, 30\% of grade
    10\% of grade in TA participation. 
    Website : Moodle hosted by Université de Paris : \url{https://moodle.u-paris.fr/course/view.php?id=7374}
    Syllabus : \url{https://moodle.u-paris.fr/pluginfile.php/1002151/mod_resource/content/5/intro-ling-syllabus-2023.pdf}

\section*{Schedule}
\begin{enumerate}
    \item Language as a psychological and social entity
    \item Language (non-)variation : Universals, variation within parameters
    \item \textbf{Morphology}, language typology
    \item \textbf{Syntax I} constituent structure; selection and subcategorization
    \item \textbf{Syntax II} subcategorization; X-bar theory
    \item \textbf{Semantics I} first look at meaning \textsl{Studied by Salvador}
    \item \textbf{Semantics II} philosophy of language and the case for methodological solipism.
    \item \textbf{Phonology} phonetic macro classes;
    \item Language and Reasoning
    \item Sign languages (guest lecture)
    \item Language in the brain; deficit-lesion method; functional brain imagining; psycholinguistics; parsing, reading, lexical access.
    \item Language and thoughts in minds vs. machines
\end{enumerate}

\section{Remarks and Observations about the Nature of Language}
R.Descartes \textit{"Discourse on the Method"} : Humans, everyone of them, can speak. Animals, though they have what is needed, can't express their thoughts.
Insights :
\begin{enumerate}
    \item It doesn't matter on general intelligence, social intelligence or any measure of your intellectual abilities. Happens despite any other difficulties.
    \item \emph{To Our Knowledge} Not any other animal can do what we can do. Article from the \textit{NYT}, saying animals \textbf{can} speak, though it's embarassing. A Chasm \textit{appears} between humans and animals. Yet, it is continuous of what happens in the animal kingdom.
    \item Animals are not incapable of language because they can reproduce human language, or use signs to communicate. Studies on Non-Human primate vocalizations, 3-4 words, all alarm calls $\rightarrow$ Language is independent on organs and communication systems. Yet is it a panic reaction or a real communication. Cannot conclude on A.Is then...
\end{enumerate}

\section{The Goal of Linguistics}
A complete understanding of how sound (/sign/etc) relates to meaning : 
\begin{itemize}
    \item in terms of the speaker's knowledge : the state their mind is in by virtue of having acquired a natural language (competence). Distinguished from mastery of language/way it is produced. Describing skill, not usage. 
    \item in terms of using that knowledge in linguistic tasks, like uncovering meaning from sound in real-time comprehension; executing motor commands necessary to externalize meaning in language production (performance)
\end{itemize}
Competence/performance is not really a sort of Chomsky. Chaz Firestone (Yale) published on competence/performance saying machines have been tested on performance and not competence. 
Tight connection between thinking and speaking. 
Behaviourism = school of thought that tried to figure out a way of studying humans that postulated and said absolutely about our mind. Not only linguistic behaviour : we shouldn't describe what happens in people's heads, just study what outputs comes from what inputs, without looking at the stimuli. Not what we will postulate. 
Freud sucks.
Cannot look at functions : functions = algorithms, studied based on input/output pairs. Cannot do deduction nor induction but only abduction. Yet, we have no less reasons to believe it is right than to believe black holes exist.

\section{Different levels of Study}
Example : \textsl{Mushrooms are an edible fungus}
\begin{enumerate}
    \item Sound Categories : Studying the sound signal based on the phonemes, represented in the mind.  
    \item Morphemes : first chucks of phonemes that has a meaning : Morphology. Here : \textsl{Mushroom} and \textsl{z} or \textsl{edi} and \textsl{able}. Sometimes they are not pronounced : need for a rigorous description. Sometimes, they're redundant, and appear with the same meaning in different places : compare theories. What is the probability of that happening ? And how about irregularities : Past = laugh-\textit{ed} or g\textit{a}ve ? Past in a concept that can manifest itself in different places : simple theory. FMRI theory can improve this theory. Morphemes don't always come in the same orthograph nor sound.
    \item Words : Not much to do here. 
    \item Semantics : Organizing words into phrases. Here : \textsl{edible fungus} is a phrase, but \textsl{edible} is not. Must be done formally.
\end{enumerate}
Three way of looking : Us, looking from a native's judgement - introspection \textbf{will} answer some questions. 
Exaggeration, yet : no written language, only looking at spokn language. 

\section{Language and Societies}
    \subsection{Language and Classes}
        Language display depends only on human factors, political relationships, genetic factors : distinction between the animal and the meat (names coming from French, spoke by the upper class) in english. Happened in other languages. Different from hyperonyms like \textit{poultry} for \textit{chicken}. Context can explain linguistic aberrations. 
    \subsection{Language and Dialect}
        Language and Dialect are political constructs and arbitrary decisions: \begin{quote}
            A language is a dialect with an army and a navy. (M.Weinreich)
        \end{quote} 
    \subsection{Infinity of Language}
        Sentences are of arbitrary length, and can always be augmented. Yet infinite-ish sentences are impossible to comprehend because Performance is finite, i.e. cognitive resources are finite. There is a \textit{finite} system generating infinitely many linguistic representations : recursions are of the order.
    \subsection{Description}
        Not looking for rules prescripting language (fuck l'académie française), but only for rules describing them. No better way to speak, the way you ought to speak has nothing to do with linguistics and only with politics. Yet, language are principled, even those which are \textit{proscribed} : adding \textit{fucking} in the middle of a word : \textit{Phila-fucking-delphia}. Rule here : \textit{fucking} comes before the stressed syllable and the material right before needs to be heavy. Heavy comes from phonology, rigorous theory of the weight of syllables. 
    \subsection{Phonological Differences}
        Languages have different constraints on the syllables composing their words : \textit{*pnick} works in French but not in Engl*sh.
    \subsection{Internal Structure of Sentences}
        Sentences are made of constituents that don't act up the same in every language : \textit{des} is not used in Engl*sh ($\sim$ \textit{of the}). They cannot be broken : \textit{des burgers et des frites}. It is mysterious tho ? Maybe language has something else to do for us than communicate\ldots\\
        They cannot be considered alone : \textit{Fat cats eat} and \textit{Fat cats eat accumulates}. Supposition that two words next to each other are related in written language. Also, prosody is a big help in understanding. 

\end{document}



