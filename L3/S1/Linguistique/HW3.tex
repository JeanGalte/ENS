\documentclass{Cours}
\usepackage{qtree}
\title{Homework Assignment 3}
\author{Matthieu Boyer}
\date{\today}

\begin{document}
    \section{Question 1}
    \begin{tabular}{l|cc}
        \toprule
        Case :& I. & II.\\
        \midrule Tree : & \Tree [un [lock able ] ] & \Tree [[un lock ] able ] \\
        \midrule
        This reads as : &\textsl{which cannot be locked} & \textsl{which can be opened}\\
        \bottomrule
    \end{tabular}

    \section{Question 2}
    The French word \textsl{infermable}, meaning \textsl{which cannot be locked}, can only be analysed as the following tree : 
    \begin{center}
        \Tree [in [ferm(e) able ] ]    
    \end{center}
    

    Indeed, while \textsl{fermable} exists in French, and we can form \textsl{infermable} by adding to it the prefix \textsl{it}, the word \textsl{inferme} does not exist in French. Thus, \textsl{infermable}, while close in meaning to one of the usages of \textsl{unlockable}, only has one and cannot derive from two different trees.

    \section{Question 3}
        \subsection{Question a.}
    \begin{tabular}{cccc}
        \toprule 
        Morpheme & Meaning & Type & Category of Stem\\
        \midrule

        
    \end{tabular}

\end{document}