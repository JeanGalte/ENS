\documentclass{article}
\usepackage{booktabs}
\usepackage[a4paper,margin=1.5cm]{geometry}
\usepackage{cmap}
\usepackage{lmodern}
\usepackage[utf8]{inputenc}
\usepackage[T1]{fontenc}
\usepackage[french]{babel}
\usepackage{multirow}
\usepackage{pdflscape}
\title{Guide d'Utilisation de la Borne d'Arcade}
\author{Matthieu Boyer, Elias Coppens et Nicolas ?}
\date{}

\begin{document}
\maketitle
\section*{Introduction}
Ce Manuel est destiné aux (nouveaux) utilisateurs de la Borne d'Arcade en K-Fêt, après sa restauration par HackENS.

\section*{Comment Utiliser la Borne d'Arcade}
\begin{enumerate}
    \item Allumez l'écran avec le bouton indiqué par la pastille rouge power. Si ça ne suffit pas, vérifiez si la borne est branchée.
    \item Si un jeu est déjà en cours, appuyez sur les 4 boutons au milieu en même temps pour quitter le jeu et revenir au menu de choix des jeux.
    \item Utilisez le joystick pour vous déplacer dans le menu de choix des jeux. Vous pouvez changer de catégorie en allant à gauche ou à droite. Le choix du jeu se fait avec le bouton jaune.
    \item Pour les jeux d'arcade, appuyez sur le bouton \$ pour ajouter du crédit (HackENS les offres). Une fois des crédits ajoutés, appuyez sur le bouton noir central de gauche pour débuter tout seul ou sur le bouton tout à droite représentant deux jeunes conscrits pour jouer à deux (ou pour rejoindre une partie solo). 
\end{enumerate}

\section*{Recommandations de Jeux}
Si vous ne savez pas quoi choisir, en voici quelques sympas pour vous aider : 
\begin{itemize}
    \item Puzzle Bobble 2 (Puzzle-game, catégorie Kombat)
    \item Space Invaders DX (catégorie Arcade)
    \item Metal Slug 3 (Run'n'gun, catégorie Beat'em Up)
    \item Aerofighters 2 (Shoot'em Up)
    \item Windjammers (Frisbee, catégorie Sports)
    \item Et pour remplacer EggNogg avant sa réinstallation, n'hésitez pas à essayer Mortal Kombat ou Street Fighter, vous pourriez aimer. 
\end{itemize}

\section*{Commandes}
Pour finir, voici un petit tableau des correspondances des touches dans certains jeux : 
Boutons Communs : 
\begin{itemize}
    \item Start Joueur 1 : Bouton Central Gauche
    \item Start Joueur 2 : Bouton Central Droite
    \item Crédits - Select Joueur 1 : Bouton \$
\end{itemize}
Pour le reste, les boutons sont identiques pour les deux joueurs : 
\begin{center}
    Correspondances des Commandes
    \begin{tabular}{lcccccc}
        \toprule
        & Jaune & Vert & Bleu & Rouge & Noir & Orange\\
        \midrule
        NeoGeo & A & B & C & D & & \\
        \midrule
        SNES & B & Y & L & A & X & R \\
        \midrule
        MegaDrive & A & B & C & X & Y & Z \\
        \midrule
        N64 & A &  & Z & B & L & R \\
        \midrule
        PS2 & Croix & Carré & L1 & Cercle & Triangle & R1 \\
        \midrule
        Street Fighter & Light Punch & Medium Punch & Heavy Punch & Light Kick & Medium Kick & Heavy Kick\\
        \bottomrule
    \end{tabular}

\end{center}

\end{document}
