\titre{Le retour du COFyparle}

Les plus vieilleux d'entre vous sauront sans doute qu'avant (sans toutefois préciser avant quoi), le\ptm a secrétaire du
\cof{} animait une émission de radio au sein de l'ENS~; mais je vous parle d'un temps que les moins de 20 ans ne peuvent
(très littéralement) pas connaître, et ces ondes ont disparu. 

Suivant mes illustres, admirables et admiré·es prédécesseur·ices, j'ai décidé de relancer cette tradition. Une fois par mois
(au moins~!), j'écrirai un article dans le \BOcal{}, avec (ou sans) l'un\ptm e de mes collègues, camarades et ami\ptm es
membres du \cof. 

Je profite de cet espace pour vous dire que le nouveau Bûro a bien pris ses fonctions, que mes tongs vont bien, et que
le \cof{} est en permanence autour du carré tous les midis de 12h à 14h et soirs de 18h à 20h entre le lundi et le
vendredi. D'ailleurs, je rappelle à tous\ptm{}tes, comme l'ont si bien dit les \BOcaleuxses{} lors de leurs deux derniers
numéros, que pour parler du Bûro 2024, par opposition au Burô 2023, on met l'accent circonflexe sur le u.

Bisous (si consentis), bon ski à celleux qui y vont, et bonne fin de mois à tous\ptm tes! 

\signature{Matthieu,\\ secrétaire du nouveau \cof}
