\titre{La Préparation de la Kâlloc}

Cette semaine, en Aqua, les conscrit·es du département d'informatique (malheureusement, je n'inclusive que pour la forme\dots) vous proposent de déguster crêpes véganes, cookies et galettes au sarrasin~! Dans cet article, je vous présente les recettes qu'ils ont utilisées, et quelques moments de leur fabrication. 

\soustitre{Les courses}
Avant de faire les courses il fallait faire la liste de ce dont nous avions besoin~: de la farine, beaucoup, du lait végétal, beaucoup, du lait animal, quelques oeufs, du sucre et de quoi garnir les crêpes (jambon, emmental râpé, confitures, pâte à tartiner et autres\dots).
Il ne fallait pas non plus oublier de quoi réconforter les courageux·ses cuisinier·es en bonbons et jus de fruits~! 
Les courageux·ses porteur·ses de courses ont dû porter un sac rempli à rabord de courses pour presque 20kg d'ingrédients~!

\soustitre{Les recettes}
Toutes les recettes présentées dans cet article sont particulièrement simples : il suffit de tout mélanger et d'éviter les grumeaux~!

\begin{description}
	\item[Les Crêpes Véganes] Nos vaillant·es conscrit·es ont utilisé la recette employée par la Psychédéliste et la Listeception (et donc les \cof{} 23 et 24). Pour beaucoup de crêpes (on a oublié de compter\dots)~:7

		\begin{tabular}{ll}
			\bf Ingrédient & Quantité\\
			Farine & 500 Grammes\\
			Lait Végétal & 1 Litre\\
			Sucre & 4 Cuillers à Soupe Rases\\
		\end{tabular}

	\item[Les Galettes au Sarrasin] Pour une dizaine de galettes~: 

		\begin{tabular}{ll}
			\bf Ingrédient & Quantité\\
			Farine de Sarrasin/Blé Noir & 250 Grammes\\
			Eau & 70cL\\
			Oeuf & 1 Oeuf\\
		\end{tabular}

	\item[Les Cookies] Pour une trentaine de cookies~:

		\begin{tabular}{ll}
			\bf Ingrédient & Quantité\\
			Farine & 400 Grammes\\
			Oeuf & 1\\
			Beurre & 250 Grammes\\
			Chocolat & 200 Grammes\\
			Cassonnade & 170 Grammes\\
			Vergeoise Brune & 170 Grammes\\
			Levure & 1 Sachet\\
		\end{tabular}
	Pour bien cuire les cookies, il faut placer des boules de pâte sur une plaque au four à 140°, et les écraser un peu. Les cookies cuisent une quinzaine de minutes puis reposent à nouveau une quinzaine de minutes.
\end{description}

\soustitre{La cuisson}
Vous trouverez en dernière page des images d'illustration de la fabrication des gourmandises.

\soustitre{La vente~!}
Nos conscrit·es ont décidé de se placer en Aqua tous les midis de cette semaine, de 11h45 à 14h~!
Il vous suffit de passer devant le stand et d'accepter nos cookies~! N'hésitez pas à passer~! Tous les prix sont libres avec un prix minimum.

\signature{Pandada, pour les conscrit·es du DI}
