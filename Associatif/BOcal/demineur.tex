\titre{Guide du Parfait Démineur}
\signature{Pandada}

\paragraph{Les règles !}
Le démineur est un jeu se jouant sur un ordinateur, à partir d'une grille à deux dimensions. 
Chaque case contient soit une mine, soit le nombre de mines dans les cases adjacentes, en comptant les voisins en diagonale. 
Ce nombre est donc compris entre $0$ et $8$. On connaît le nombre de mines. \\
L'objectif est de trouver l'emplacement de toutes les mines. 
Pour cela, on dispose de deux outils,
\begin{itemize}
	\item le \emph{drapeau} qui permet de marquer un emplacement où on suppose/sait qu'il y a une mine, utilisable avec le clic droit ou la barre espace
	\item le \emph{révéleur} qui permet de révéler le nombre de mines adjacentes à une case, utilisable avec le clic gauche ou la touche d'entrée. Attention, son utilisation sur une mine la fera exploser et vous fera perdre la partie.
\end{itemize}


\paragraph{Installation}
Le bon démineur, c'est un jeu sans hasard, avec des grilles générées le plus aléatoirement possible.\\
En particulier, un bon démineur ne doit jamais vous laisser bloqué dans une situation, ou même en étudiant toutes les possibilités dans la grille, vous ne pouvez pas faire de coup sûr. \\
Votre serviteur vous recommande d'installer le démineur de la collection de puzzles portables de Simon Tatham. 

\paragraph{Techniques de Jeu}
La section qui suit va se concentrer sur un certain nombre de petits conseils pour bien jouer au démineur, et maximiser votre taux de victoire. Les conseils sont rangés par ordre d'évidence.
\begin{itemize}
	\item Soyez attentifs au nombre de mines autour de vos cases, n'en marquez pas trop. 
	\item Restez concentré sur une zone de la grille, ne vous dispersez pas tant que vous pouvez avancer. 
	\item Faites attention à la fatigue, une partie sur une grille experte peut durer plus d'un quart d'heure au début.
	\item Vous pouvez utiliser le révéleur sur une case révélée pour révéler les cases adjacentes sans drapeau. 
	\item Tant que possible, recoupez les informations de vos cases. En particulier, limitez les positions possibles pour les mines adjacentes à une case dès que vous le pouvez en révélant les cases sur lesquelles il ne peut pas y avoir de mines : voyez les informations de vos cases comme un nombre de mines dans un ensemble de cases qui diminuent chaque fois que vous ajoutez un drapeau adjacent.
	\item Essayez de reconnaître des paternes que vous avez déjà résolus. Attention, si vous ne voyez pas un paterne à un endroit, il peut apparaître une fois des drapeaux ajoutés. 
	\item Entraînez vous, la plupart des conseils ci-dessus vous deviendront naturels avec l'expérience.
	\item Enfin, ne vous énervez pas, il est normal de perdre et de faire des erreurs. Essayer d'aller vite vous fera faire des erreurs, il faut prendre l'habitude.
\end{itemize}

\paragraph{Quelques Variantes}
Il existe des variantes du démineur où les cases peuvent contenir plusieurs mines, jouées sur un tore (c'est à dire que les bords supérieurs/inférieurs et gauche/droite sont reliés) ou sur la surface d'autres objets.
Il y a aussi des variantes en 3D, sur un plateau avec des cases hexagonales, des démineurs infinis (attention, ils reposent souvent sur la chance), mixées avec un nonogramme, ou presque n'importe quelle autre jeu. 
Ma variante préférée étant sans doute le \emph{Knightsweeper}, où l'objectif est de retrouver tous les \textit{chevaliers} sur le plateau, et on nous donne le nombre de chevaliers qui attaquent une case, selon les règles des échecs.




