\documentclass{article}
\title{Compte-Rendu de la Réunion Liste}
\date{23 Novembre 2023}
\author{}
\usepackage[french]{babel}
\usepackage[a4paper, margin=3cm]{geometry}


\begin{document}
\maketitle
\section*{Ordre du Jour}
Présents : 
\begin{itemize}
    \item Anaëlle
    \item Ruben
    \item Matthieu
    \item Antoine 
    \item Judith
    \item Tikinas
    \item Zachary
    \item Marion
    \item Alfred 
    \item Sophie
    \item Chiara
    \item Raphaël
\end{itemize}
\tableofcontents

\section{Bilan de la Dernière Réunion}
\begin{itemize}
    \item Anaëlle l'a très mal vécue : Brainstorming sans aucun cadre et c'était horrible à présider. Les réunions devraient bien se passer et chacun doit respecter le temps de parole, en particulier il faut que chacun puisse donner ses idées à tous, d'où l'ordre du jour. 
    \item Nouveau système de réunions : Un modérateur qui gère tous les tours de table. On respecte l'ordre du jour. En cas de réaction rapide faire des pistolets avec les mains. Pour approuver secouer les mains. 
    \item Au moins, beaucoup de progrès ont été faits. Le travail fourni n'a pas été \textsl{efficace} mais au moins productif. 
\end{itemize}

\section{Vidéos de Campagne}
\begin{itemize}
    \item Qui monte ? Antoine pour les parties COF. Potentiellement plus simple pour le BDA, au moins trier les rushs. Blague montrant à quel point le consentement est important.
    \item Filmés : Intro type film d'horreur (montés) + une partie des scènes de The Offliste. 
    \item Planning pas complètement finis pour tout le monde, surtout côté BDA. Une partie des scènes a été écrite. 
    \item Rappels : Script + Comptes-Rendus sur le Drive. 
    \item Script BDA basé sur \textit{Head Will Roam}(?) Clip construit dessus. Lieu de tournage ok
    \item Rendu de la vidéo \textbf{FINIE} le 29 Novembre. Tout doit donc être fini de tourner lundi soir. 
\end{itemize}

\section{Infos de la Semaine}
\begin{itemize}
    \item Pas de stickers => mais des affiches.
    \item Faire des crêpes et des gateaux.
    \item Besoin de au moins 500€ sachant qu'on est financés à auteur de 260€ par le COF. 
    \item Perm pour la nuit : le faire en général, mais si possible perm caisse de 00h à 01h (principalement) et de 01h à 02h. Envoyer un message au COF/Anaëlle/Ruben si besoin. 
    \item Réu pour les listes COF/BDS avec les deux assos et toutes les listes sûrement la semaine prochaine.
\end{itemize}

\section{Soirée bbCOF}
\begin{itemize}
    \item Soirée à organiser le 8 Décembre.
    \item 6 personnes sont là (manque les med-sciences + Zachary et Matthieu)
    \item Possibilité potentiellement de demander au BDS. 
    \item Planning Prévisionnel fait par Ruben : ce qui doit être fait, et jusqu'à quel dates : 
    \begin{enumerate}
        \item Administratif : Ruben
        \item Affiches : Judith 
        \item Coller les affiches : tout le monde, deux par deux
        \item Liste des Courses + Recettes de Cocktails (25cL = max 1 dose bar)\footnote{Judith est danger public - Ruben}: 
        \item Faire les courses : Matthieu, Tikinas, Sophie (et Alfred potentiellement) 
        \item Le Jour même -> Installer : tout ceux qui sont là (avec aide possible des k-fêteux), Faire les boissons : Pareil si possible
        \item Mixage : BOUM ? Lumière : PLS ? 
    \end{enumerate}
    \item Rappel plus généraux : Un exté reste toujours sous la responsabilité de celui qui l'invite. 
\end{itemize}

\section{Tournage}
PAUSE TOURNAGE OUIIIIIIIIIIII
\begin{quotation}
    C'est donc ça, le tournage partiel - Alfred
\end{quotation}

\section{Rappels COF}

\section{AG Liste}






\end{document}
