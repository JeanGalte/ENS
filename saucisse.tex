\documentclass{cours}
\title{Démonstration de la Saucisse et Applications}
\author{Matthieu Boyer}
\date{31 Novembre 2023}

\begin{document}
\begin{abstract}
    Dans cet article, nous nous attelons à une démonstration du fait que nous vivons dans une saucisse. Nous en déduirons plusieurs applications importantes dans des domaines variés. Suivant \cite{uvc}, nous en déduirons la véritable et fondamentale nature du monde.
\end{abstract}

\section*{Introduction}
Il est depuis longtemps conjecturé que nous, être humains vivons dans une saucisse. Voici aujourd'hui une preuve de ce fait, en partant de plusieurs points physiques importants. Commençons par quelques définitions:  
\begin{definition}
    Dans l'article, \emph{nous} désigne l'ensemble des personnes en capacité de lire cette article, sans distinction de genre, de couleur de peau, de richesse, d'origine ou de discipline scientifique\footnote{L'auteur aimerait toutefois rappeler que les chimistes sont des dégénérés.}. 
\end{definition}

Dans cet article, nous acceptons l'utilisation de tout ce qui est évident. 


\section{Quelques Lemmes}
\subsection{Du \texttt{Miracle Sort}}
On rappelle l'algorithme de tri \texttt{Miracle Sort} :
\begin{algorithm}\label{msort}
    \caption{Miracle Sort}
    \begin{algorithmic}
        \Input \ $L$
        \EndInput
        \While {$L$ is not sorted}
            \State Wait
        \EndWhile\\
        \Return $L$
    \end{algorithmic}
\end{algorithm}

\begin{proposition}
    Cet algorithme est correct.
\end{proposition}
\begin{proof}
    En effet, il est clair que la valeur renvoyée par cet algorithme est la valeur de $L$ triée. 
\end{proof}

\begin{lemma}\label{GS-MSort}
    Cet algorithme termine si et seulement si la génération spontanée de la vie est possible. 
\end{lemma}
\begin{proof}
    En effet, il est connu en physique que, si des rayons cosmiques peuvent atteindre la Terre, causant une fois des ordinateurs touchés, des \textit{Bitflips}, c'est à dire des modifications spontanées de la valeur de ce bit. On montre de même en Biologie, que ces rayons cosmiques, s'ils existent, permettent la génération spontanée d'organismes vivants.
\end{proof}

\subsection{Existence et Existence de la Réalité}
\begin{definition}
    Dans la suite de cet article, la notion d'\emph{existence} revêt une couleur philosophique, puisqu'on dira que quelque chose existe si et seulement si elle a une réalité physique.
\end{definition}
\begin{lemma}\label{gsexist}
    La génération spontanée de la vie équivaut à notre existence.
\end{lemma}
\begin{proof}
    En remontant l'arbre généalogique d'\textsl{Homo Erectus}, nous nous rappelons que nous n'existons que parce que la vie existe, et donc que la génération spontanée a eu lieu.
\end{proof}

\begin{lemma}\label{phyuseless}
    La physique est inutile si et seulement si nous existons.
\end{lemma}
\begin{proof}
    On rappelle que la physique est utile si, et seulement si, nous existons, puisqu'alors il est nécessaire de vérifier si nous existons. 
\end{proof}
\begin{remark}
    Ce Théorème, bien que souvent renié par les physiciens, les rappelle à la dure réalité de leur métier : étudier des choses qui ne servent à rien, tout en inventant le fait qu'elles n'existent pas. 
\end{remark}

\begin{lemma}\label{novr}
    La physique est useless si et seulement si nous ne vivons pas dans une réalité virtuelle.
\end{lemma}
\begin{proof}
    En effet, si on vit dans une réalité virtuelle, il est nécessaire d'étudier au maximum sa physique, afin de pouvoir la dominer, et réciproquement. 
\end{proof}

\section{Le Théorème}
\subsection{Un Dernier Lemme}
\begin{lemma}\label{msort-saucisse}
    Le \texttt{Miracle Sort} termine si et seulement si nous vivons dans une Saucisse.
\end{lemma}
\begin{proof}
    Ceci découle immédiatement des lemmes \ref*{GS-MSort}, \ref*{gsexist}, \ref*{phyuseless}, \ref*{novr}.
\end{proof}

\subsection{Enfin !}
\begin{theorem}
    Nous vivons dans une saucisse !
\end{theorem}
\begin{proof}
    Il est clair que l'exécution de \texttt{Miracle Sort} sur la liste $[x]$ termine pour tout $x$. En particulier, par le lemme \ref*{msort-saucisse}, on obtient le résultat.
\end{proof}
\begin{remark}
    Suivant \cite{uvc}, il est bien évidemment entendu que la saucisse dont on parle est un Chien-Saucisse Géant. 
\end{remark}

\section{Des Corollaires}
Nous allons maintenant donner quelques corollaires de ce théorème, sans démonstrations, celles-ci étant laissées en exercice au lecteur. 
\subsection{Scientifiques}
\begin{proposition}
    La physique est inutile.
\end{proposition}

\begin{proposition}\label{acuseless}
    L'axiome du choix est inutile.
\end{proposition}
\begin{remark}
    On savait déjà que celui-ci était faux. 
\end{remark}


\begin{proposition}\label{btp}
    Toute preuve a besoin d'être constructiviste.
\end{proposition}
\begin{remark}
    Ceci est un complément à \ref*{acuseless}
\end{remark}

\subsection{Réalistes}
\begin{proposition}
    Betterave.
\end{proposition}

\begin{proposition}
    Le béton, c'est la vie.
\end{proposition}
\begin{proof}
    C'est une conséquence de \ref{btp}
\end{proof}

\subsection{Théorème du Sandwich au Jambon}
\begin{theorem}[Du Sandwich au Jambon]
    Dans un espace euclidien de dimension $n$, étant données $n$ parties Lebesgue-Mesurables de mesures finies, il existe un hyperplan séparant les susnommées parties en deux sous-parties de mesures égales. 
\end{theorem}


\begin{thebibliography}{9}
    \bibitem{uvc} Nous vivons tous dans le ventre d'un chien géant, \textit{Ultra Vomit}
\end{thebibliography}


\end{document}