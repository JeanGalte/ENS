\documentclass[french]{mpbmail}
\usepackage{cmap}
\usepackage[T1]{fontenc}
\usepackage[utf8]{inputenc}
\usepackage[french]{babel}
\usepackage{microtype}
\usepackage{lmodern}
\usepackage{enumitem}
\usepackage{hyperref}
\usepackage{amsmath, amsthm, amssymb}
\usepackage{booktabs,multirow}
\usepackage{graphicx}
\usepackage{array}

\newtheorem{theorem}{Théorème}
\newtheorem{corollary}{Corollaire}[theorem]
\newtheorem{lemma}[theorem]{Lemme}
\renewcommand\qedsymbol{$\blacksquare$}
\DeclareMathOperator{\sgn}{sgn}

\begin{document}
    \begin{letter}{%
        Idil Albayrak\\
        7 Avenue Albert Einstein\\
        69100 Villeurbanne
        }
        \opening{Mon coeur,}
        Voici un petit fait mathématique que j'apprécie énormément.

        On se place ici dans une situation typique: Plusieurs personnes cherchent à partager équitablement un paquet de nouilles. 
        On supposera ici que le paquet de nouilles est homogène, et que $n$ personnes cherchent à se le partager en fonction de leur faim, que l'on représentera par des mesures de probabilités $\mu_{i}$ sur l'intervalle $X = \left[0 ; 1\right]$ avec une densité $f_{i}$. 
        Le lecteur attentif aura alors remarqué qu'on cherche une partition $X_{1}, \ldots, X_{n}$ de $X$ où la i-ème personne reçoit la part représentée par $X_{i}$ des nouilles.

        On cherche à traduire mathématiquement. On introduit ainsi le système d'équations que l'on cherche à résoudre: 
        \begin{equation*}\tag{$\star$}
            \int_{0}^{x_{1}}f_{\sigma(1)}(x) \mathrm{d}x = \int_{x_{1}}^{x_{2}}f_{\sigma(2)}(x)\mathrm{d}x = \ldots = \int_{x_{n}}^{1}f_{\sigma(n)}(x) \mathrm{d}x 
        \end{equation*}                        

        \begin{theorem}[De Partage des Nouilles]\label{thm:nouille-split}
            Pour toutes fonctions de densités $f_{i}$ et permutations $\sigma \in \mathfrak{S}_{n}$ de l'ensemble des mangeurs de nouilles le système ($\star$) a une solution. 
        \end{theorem}
        
        Pour cela, on rappelle le théorème de Borsuk-Ulam: 
        \begin{theorem}[Borsuk-Ulam]\label{thm:borsuk-ulam}
            Si $f$ est une fonction continue sur une sphère de dimension $n$, i.e. sur la frontière de la boule euclidienne de $\mathbb{R}^{n+1}$, à valeurs dans un espace euclidien de dimension $n$, il existe deux points antipodaux sur cette sphère de même image par $f$.
            Autrement dit: 
            \[\forall f: S^{n} \rightarrow \mathbb{R}^{n},\text{ continue}, \exists x_{0} \in S^{n}, f(x_{0}) = f(-x_{0})   \]
        \end{theorem}

        \begin{proof}
            On ne démontre ici que le cas en dimension $2$, celui-ci se généralisant aisément par le lecteur curieux en dimension $n$. On raisonne par l'absurde, avec les notations ci-dessus.
            On rappelle que $S^{2}$ est simplement connexe et que son groupe fondamental est trivial.
            On définit sur $S^{2}$: 
            \[g: x \in S^{2} \mapsto \frac{f(x)-f(-x)}{\left\lVert f(x) - f(-x)\right\rVert}\]
            $g$ est bien définie puisqu'on suppose qu'il n'existe pas de point $x_{0}$ convenable. On considère le lacet $\alpha$ de $S^{2}$ défini par: $\alpha(t) = (\cos{(2\pi t)}, \sin{(2\pi t)})$.
            Comme $g$ est impaire, on a, en notant $g_{*}$ le morphisme du groupe fondamental de $S^{2}$ dans celui de $S^{1}$ induit par $g$:
            \begin{equation}
                \label{9}
                \forall t \in \left[0, \frac{1}{2}\right], g_{*}\alpha(t + 1/2) = -g_{*}\alpha(t)
            \end{equation}
            Par ailleurs, il existe une homotopie $\rho$ de $\left[0, 1\right]$ dans $\mathbb{R}$ telle que le lacet $g_{*}\alpha$ de $S^{1}$ s'écrive:
            \begin{equation*}
                \forall t \in \left[0, 1\right], g_{*}\alpha(t) = (\cos{(2\pi \rho(t))}, \sin{(2\pi \rho(t))}), \text{avec } \rho(0) = 0
            \end{equation*}
            De \eqref{9} on déduit alors: 
            \begin{equation*}
                \forall t \in \left[0, 1/2\right] \ 2\nu(t) = 2(\rho(t + 1/2) - \rho(t)) \in \mathbb{Z}
            \end{equation*}
            D'où, par continuité de la fonction $\nu$, celle-ci étant définie sur un connexe et à valeurs dans un ensemble discret, elle est constante et s'écrit sous la forme $c/2$ où $c\mod 2 = 1$. On en déduit que: 
            \begin{equation*}
                \rho(1) = \rho \left(\frac{1}{2} \right) + \frac{c}{2} = \left(\rho(0) + \frac{c}{2} \right) + \frac{c}{2} = c
            \end{equation*}
            Donc $\rho(1)$ est impair, différent de $0$. En particulier, $g_{*}\alpha$ n'est pas homotope à un point, et fait $c$ tours autour du cercle. Ainsi, l'image de $g_{*}$ est différente de l'élément neutre, mais $g_{*}$ est un morphisme du groupe trivial dans un groupe isomorphe à $\mathbb{Z}$, ce qui conclut le raisonnement par l'absurde.
        \end{proof}
        
        \begin{proof}(Du théorème de Partage des Nouilles \ref*{thm:nouille-split})
            On introduit les fonctions $F_{i}$ définies sur $S^{n-1} = \left\{e = (e_{0}, \ldots, e_{n-1}) \in \mathbb{R}^{n}\ \vline \left\lVert e\right\rVert _{2} \right\}$ par:
            \begin{equation*}
                F_{i}(e) = \sgn(e_{i+1})\times \int_{e_{1}^{2} + \dots + e_{i}^{2}}^{e_{1}^{2} + \dots + e_{i}^{2} + e_{i+1}^{2}} f_{\sigma(i+1)}(x) \mathrm{d}x - \sgn(e_{1})\int_{0}^{e_{1}^{2}}f_{\sigma(1)}(x) \mathrm{d}x
            \end{equation*}
            On définit également $f$ par $f(e) = (F_{1}(e), \ldots, F_{n-1}(e))$.
            $f$ étant clairement continue et antipodale, par le Théorème de Borsuk-Ulam \ref*{thm:borsuk-ulam}, il existe $e \in S^{n-1}$ tel que $f(\tilde{e} ) = 0$
            Alors, avec $x_{0} = 0$ puis par récurrence en définissant $x_{i} = x_{i-1} + \tilde{e}_{i}^{2}$, le lecteur pourra vérifier qu'on obtient bien une solution au système ($\star$).
        \end{proof}
 
        Par ailleurs, on aurait pû donner une autre démonstration du Théorème de Partage des Nouilles \ref*{thm:nouille-split} se basant sur le théorème du Sandwich au Jambon, corollaire du Théorème de Borsuk-Ulam \ref*{thm:borsuk-ulam}                      
        
        \begin{theorem}[Du Sandwich au Jambon]
            Étant données $n$ parties \textit{Lebesgue}-mesurables d'un espace euclidien de dimension $n$, il existe au moins un hyperplan affine divisant chaque parties en deux sous-ensembles de mesure égale.
        \end{theorem}

        La preuve de ce théorème est laissée en exercice au lecteur, puisqu'il ne s'agit que d'une application du Théorème de Borsuk-Ulam \ref*{thm:borsuk-ulam}
        Je te laisse un lien vers un article qui présente ce problème : \\
        \url{https://www.math.univ-toulouse.fr/~cheze/equitable-simple-cake-cutting-cheze.pdf}

        \closing{En espérant te revoir \textsl{au plus vite}}
        \ps PS : Je t'aime. Plus que ce théorème.
    \end{letter}
\end{document}
