\documentclass{beamercours}
\title{Cours TalENS 2023-2024}
\subtitle{Détermination, Crayons, Angles Droits, Glissières}
\date{27 Janvier 2024}

\begin{document}
\maketitle

\section*{Introduction}
\begin{frame}
    \frametitle{Introduction Historique}
    \only<1>{}

\end{frame}

\section{Formalisme !}
\subsection{Polynômes sur un Corps}
\begin{frame}
    \frametitle{Le Corps}
    \begin{définition}{Corps}{}
        Un corps est un ensemble muni : 
        \begin{itemize}
            \visible<2->{\item D'une addition avec un neutre $0$ notée $+ : (x, y) \mapsto x + y$}
            \visible<3->{\item D'une multiplication avec un neutre $1$ notée $\times : (x, y) \mapsto xy$ distributive sur l'addition}
        \end{itemize}
        \visible<4->{Pour laquelle tout élément (sauf $0$) est inversible pour la multiplication et la loi de produit nul est vérifiée.}
    \end{définition}
    \visible<5>{On notera $\K$ un tel ensemble. $\R$, $\Q$, $\Z/p\Z = \mathbb{F}_{p}$ sont des corps.}
\end{frame}

\begin{frame}
    \frametitle{Polynômes à une Indéterminée}
    \visible<1->{\begin{définition}{Polynôme sur $\K$}{}
        Un polynôme à coefficients dans $\K$ est une suite finie d'éléments de $\K$.
    \end{définition}}
    \only<2>{On les note sous la forme : 
    \[
        \sum_{i = 0}^{d} a_{i}X^{i}
    \]}
    \only<3>{On appelle le symbole $X$ l'indéterminée. Ce n'est pas un nombre. On note $\K[X]$ l'ensemble des polynômes à coefficients dans $\K$. On appelle $d$ le degré de $P$.}
\end{frame}

\begin{frame}
    \frametitle{Calcul sur les Polynômes}
    \only<1-3>{\begin{propositionfr}{Opérations}{}
        Si $P = \sum_{i = 0}^{d_{1}} a_{i}X^{i}$ et $Q = \sum_{j = 0}^{d_{2}} b_{j}X^{j}$ sont deux polynômes :
       \begin{itemize}[<+->]
        \item $P + Q = \sum_{i = 0}^{\max(d_{1}, d_{2})} (a_{i} + b_{i})X^{i}$ est un polynôme de degré $\leq \max(\deg P, \deg Q)$.
        \item $X^{k}P = \sum_{i = 0}^{d}a_{i}X^{i + k}$ est un polynôme. 
        \item En particulier, $PQ$ est un polynôme de degré $\deg P + \deg Q$ et si $k \in \N$, $P^{k}$ est un polynôme.
       \end{itemize} 
    \end{propositionfr}}
    \only<4->{\begin{définition}{Composition}{}
        Pour $\alpha \in \K$, on note $P(\alpha) \in \K$ le nombre : $\sum_{i = 0}^{d_{1}}a_{i}\alpha^{i}$. On note de plus $P \circ Q$ le polynôme
        \[
            P \circ Q = \sum_{i = 0}^{d_{1}} a_{i}Q(X)^{i}
        \]
        On a $\deg P\circ Q = \deg P \times \deg Q$
    \end{définition}}
\end{frame}

\begin{frame}
    \frametitle{Polynômes à Plusieurs Indéterminées}
    \begin{définition}{Polynômes à Plusieurs Indéterminées}{}
        Un polynôme à $k + 1$ indéterminées est un polynôme à coefficients dans $\K[X_{1}, \ldots, X_{k}]$
    \end{définition}
    \visible<2>{\begin{remarque}{Intégrité}{}En réalité, $\K[X]$ n'est pas un corps, mais seulement un anneau intègre.\end{remarque}}
    \visible<3>{$P$ se met sous la forme 
    \[
        P(X) = \sum_{i_{1} = 0}^{d_{1}}\sum_{i_{2}=0}^{d_{2}}\ldots\sum_{i_{k} = 0} \alpha_{i_{1}, \ldots, i_{k}}X_{1}^{i_{1}}X_{2}^{i_{2}}\ldots X_{k}^{i_{k}}
    \]
    }

\end{frame}

\subsection{Equations Polynômiales et Applications}
\begin{frame}
    \frametitle{Equation Polynômiale}
    \begin{définition}{Equation Polynômiale}{}
        Une équation polynômiale est une équation de la forme 
        \[
            P(x) = \sum_{i = 0}^{d} a_{i}x^{i} = b
        \]
        \only<2-3>{\visible<2-3>{On peut se restreindre au cas $b = 0$ en enlevant $b$ à $P$.\\}
        \visible<3>{On appelle \emph{racines} de l'équation les éléments de $\left\{\alpha \mid P(\alpha) = b \right\}$. On dit que $d = \deg P$ est le degré de l'équation. }}
        \only<4->{Pour $k$ indéterminées, on remplace $x$ par un $k$-uplets $x_{1}, \ldots, x_{k}$}
    \end{définition}
\end{frame}

\begin{frame}
    \frametitle{Solutions à une Équation Polynômiale}
    \begin{propositionfr}{Nombres de Solution}{}
        Une équation définie par $P$ a au plus $\deg P$ solutions
    \end{propositionfr}
    \visible<2->{\begin{théorème}{D'Alembert Gauss}{}
        Une équation polynômiale définie par $P$ a toujours exactement $\deg P$ solutions sur un corps algébriquement clos. $\C$ est algébriquement clos.
    \end{théorème}}
\end{frame}

\begin{frame}
    \frametitle{Applications I}
    \begin{définition}{Droite}{}
        Une droite est un ensemble de la forme $D(a, b) = \{ax + b \mid x \in \R\}$
    \end{définition}

    \visible<2->{En particulier, si on a deux droites $D(a, b), D(a', b')$, leur intersection est définie par l'ensemble
    \[
        \left\{ax + b = a'x + b'\right\} = \left\{(a - a')x + (b - b') = 0\right\}
    \]}
\end{frame}

\begin{frame}
    \frametitle{Applications II}
    \begin{définition}{Cercle}{}
        Un cercle est un ensemble de la forme $C((x_{0}, y_{0}), r) = \left\{(x - x_{0})^{2} + (y - y_{0})^{2} = r^{2}\right\}$
    \end{définition}

    \visible<2->{En particulier, si on a deux cercles $C((x_{1}, y_{1}), r)_{1}, C((x_{2}, y_{2}), r_{2})$, leur intersection est définie par l'ensemble
    \[
        \left\{(a - x_{1})^{2} + (b - y_{1})^{2} = r_{1}^{2}\right\}
    \]}
\end{frame}

\end{document}
