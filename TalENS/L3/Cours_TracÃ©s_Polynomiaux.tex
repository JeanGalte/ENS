\documentclass{beamercours}
\title{Cours TalENS 2023-2024}
\subtitle{Détermination, Crayons, Angles Droits, Glissières}
\date{27 Janvier 2024}


\begin{document}
\maketitle

\section*{Introduction}
\begin{frame}
    \frametitle{Introduction Historique}
    \only<1>{}

\end{frame}

\section{Formalisme !}
\subsection*{La Notion d'Equation Polynômiale sur un Corps}
\begin{frame}
    \frametitle{Le Corps}
    \begin{définition}{Corps}{}
        Un corps est un ensemble muni : 
        \begin{itemize}
            \visible<2->{\item D'une addition avec un neutre $0$ notée $+ : (x, y) \mapsto x + y$}
            \visible<3->{\item D'une multiplication avec un neutre $1$ notée $\times : (x, y) \mapsto xy$ distributive sur l'addition}
        \end{itemize}
        \visible<4->{Pour laquelle tout élément (sauf $0$) est inversible pour la multiplication et la loi de produit nul est vérifiée.}
    \end{définition}
    \visible<5>{On notera $\K$ un tel ensemble}
\end{frame}

\begin{frame}
    \frametitle{Polynômes}
    \visible<1->{\begin{définition}{Polynôme sur $\K$}{}
        Un polynôme à coefficients dans $\K$ est une suite finie d'éléments de $\K$.
    \end{définition}}
    \visible<2->{On les note sous la forme : 
    \[
        \sum_{i = 0}^{d} a_{i}X^{i}
    \]}
    \visible<3->{On appelle le symbole $X$ l'indéterminée. Ce n'est pas un nombre. On note $\K[X]$ l'ensemble des polynômes à coefficients dans $\K$. On appelle $d$ le degré de $P$.}
\end{frame}

\begin{frame}
    \frametitle{Calcul sur les Polynômes}
    \only<1-3>{\begin{propositionfr}{Opérations}{}
        Si $P = \sum_{i = 0}^{d_{1}} a_{i}X^{i}$ et $Q = \sum_{j = 0}^{d_{2}} b_{j}X^{j}$ sont deux polynômes :
       \begin{itemize}[<+->]
        \item $P + Q = \sum_{i = 0}^{\max(d_{1}, d_{2})} (a_{i} + b_{i})X^{i}$ est un polynôme.
        \item $X^{k}P = \sum_{i = 0}^{d}a_{i}X^{i + k}$ est un polynôme. 
        \item En particulier, $PQ$ est un polynôme et si $k \in \N$, $P^{k}$ est un polynôme.
       \end{itemize} 
    \end{propositionfr}}
    \only<4->{\begin{définition}{Composition}{}
        Pour $\alpha \in \K$, on note $P(\alpha) \in \K$ le nombre : $\sum_{i = 0}^{d_{1}}a_{i}\alpha^{i}$. On note de plus $P \circ Q$ le polynôme
        \[
            P \circ Q = \sum_{i = 0}^{d_{1}d_{2}}
        \]
    \end{définition}
        }

\end{frame}

\end{document}
