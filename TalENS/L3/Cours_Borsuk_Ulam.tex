\documentclass{beamercours}

\title{Cours TalENS 2023-2024}
\subtitle{Goûters, Socialisme, Chaleur et PB\&J} 
\date{16 Décembre 2023}


\begin{document}
\maketitle
\section{Mise en Situation}
\subsection{Le Problème}
\begin{frame}
    \frametitle{Vie en Communauté}
    \only<1-5>{\visible<1-5>{Placez vous dans la situation suivante : \\}
    \begin{itemize}
        \visible<2-5> {\item Avec plusieurs de vos amis, vous avez décidé de faire un gâteau.}
              \visible<3-5> {\item Une fois celui-ci cuit, vient le moment de le découper.}
              \visible<4-5> {\item Cependant, vous n'avez pas tous aussi faim les uns que les autres.}
              \visible<5-5> {\item Comment faire pour le découper sans que personne ne soit lésé et que le découpeur ne soit assailli pour ses préférences dans le groupe d'amis.}
    \end{itemize}}
    \only<6>{Vous apprendrez plus tard dans votre vie, qu'en réalité, l'application la plus utile de ce que nous allons voir est en réalité une application au découpage de nouilles.}
\end{frame}

\subsection{Mathématiquement}
\begin{frame}
    \frametitle{Modélisation}
    \visible<1-> {On modélise l'ensemble des mangeurs de gâteau par le segment d'entier $\onen{n}$\\}
    \visible<2-> {On note $\mathfrak{S}_{n}$ l'ensemble des permutations de ce segment d'entier. \\}
    \visible<3-> {On représente la faim du mangeur $i$ par une mesure de probabilité $\mu_{i}$ de densité $f_{i}$. \\}
    \visible<4-> {On cherche une partition $X_{1}, \ldots, X_{n}$ du gâteau, que l'on modélise par l'ensemble $\left[0, 1\right]$.}
\end{frame}

\begin{frame}
    \frametitle{Equitabilité du Partage}
    On a plusieurs manières de juger de la manière de couper le gâteau :
    \begin{itemize}
        \visible<2->{\item Proportionellement: $\forall i \in \onen{n}, \mu_{i}(X_{i}) \geq \frac{1}{n}$}
              \visible<3->{\item Exactement: $\forall i \forall j, \mu_{i}(X_{j}) = \frac{1}{n}$}
              \visible<4->{\item Sans Avarice: $\forall i, \forall j, \mu_{i}(X_{i}) \geq \mu_{i}(X_{j})$}
              \visible<5->{\item Equitablement: $\forall i, \forall j, \mu_{i}(X_{i}) = \mu_{j}(X_{j})$}
    \end{itemize}
    \visible<6>{On choisit ici de s'intéresser au partage Équitable du gâteau, mais des résultats existent sur les autres types de partage.}
\end{frame}

\begin{frame}
    \frametitle{Théorème de Partage de Nouilles}
    \visible<1->{\begin{theorem}[De Partage de Nouilles]
            Pour toutes fonctions de densités $f_{i}$ et permutations $\sigma \in \mathfrak{S}_{n}$, on considère le système $(\star)$ suivant :
            \begin{equation*}\tag{$\star$}
                \int_{0}^{x_{1}}f_{\sigma(1)}(x) \mathrm{d}x = \int_{x_{1}}^{x_{2}}f_{\sigma(2)}(x)\mathrm{d}x = \ldots = \int_{x_{n}}^{1}f_{\sigma(n)}(x) \mathrm{d}x
            \end{equation*}
            Celui-ci a une solution.
        \end{theorem}}
    \visible<2>{On cherche donc à démontrer ce Théorème.}
\end{frame}

\section{Formalisations}
\subsection{Combinatoire}
\begin{frame}
    \frametitle{Partitions}
    \only<1> {On appelle partition d'un ensemble toute sous division de cet ensemble en plusieurs parties sans recouvrement.\\}
    \only<2-> {Formellement, pour un ensemble $X$, il s'agit d'un ensemble $(X_{i})_{i\in I}$ tel que :
    \[
        \bigcup_{i\in I}X_{i} = X \text{ et } X_{i} \cap X_{j} = \emptyset
    \]}
    \only<3> {Par exemple, $\left\{\left\{1, 3\right\}, \left\{2, 4\right\}\right\}$ est une partition de $\onen{4}$}
    \only<4> {Dans le cas de $\left[0, 1\right]$ on dénote une partition comme une suite strictement croissante de réels $x_{1}, \ldots, x_{n}$ de sorte qu'on \textit{découpe} l'intervalle en plus petits intervalles.}
\end{frame}

\begin{frame}
    \frametitle{Permutations}
    \only<1> {On appelle permutation d'un ensemble, toute manière de le réordonner.\\}
    \visible<2-> {Formellement, il s'agit d'une bijection d'un ensemble et dans lui-même, puisqu'il s'agit juste de renommer chaque élément.\\}
    \visible<3-> {On note souvent les permutations entre parenthèses : $\left(1 \ 2 \ 3\right)$ est une permutation de l'ensemble $\onen{3}$ mais aussi de $\onen{4}$.\\}
    \visible<4-> {Dans le cas de l'ensemble $\onen{n}$, on note l'ensemble de ses permutations $\mathfrak{S}_{n}$ et on l'appelle Groupe Symétrique d'ordre $n$. Il est de cardinal $n!$ (c'est beaucoup)}
\end{frame}

\begin{frame}
    \frametitle{Modélisation}
    \visible<1-> {Ici, on représente le gâteau de manière continue par le segment $\left[0, 1\right]$. On dit que c'est une description continue du problème puisque l'ensemble considéré n'est pas dénombrable.\\}
    \visible<2-> {Puisque la faim de chacun ne dépend pas de la position autour de la table, on doit pouvoir résoudre le problème, quel que soit l'ordre des mangeurs, i.e. quelle que soit la permutation de l'ensemble $\onen{n}$ des mangeurs. \\}
    \visible<3-> {On cherche alors une partition du gâteau, i.e. du segment $\left[0, 1\right]$, dépendant de la permutation $\sigma \in \mathfrak{S}_{n}$ des mangeurs.}
\end{frame}

\subsection{Probabilités}
\begin{frame}
    \frametitle{Théorie de la Mesure}
    \only<1-3>{\visible<1-3>{On appelle mesure $\mu$ sur un ensemble $E$ une application d'une tribu $\A$ de l'ensemble de ses parties à valeurs positives qui vérifie certaines propriétés:}

    \begin{itemize}
        \visible<2-3>{\item $\mu(A \sqcup B) = \mu(A) + \mu(B)$}
              \visible<3-3>{\item $\mu(\emptyset) = 0$}

    \end{itemize}}
    \only<4->{
        \visible<4->{De ces propriétés de base on en déduit quelques propriétés :}
        \begin{itemize}
            \visible<5->{\item $\mu(A \cup B) = \mu(A) + \mu(B) - \mu(A \cap B)$}
                  \visible<6->{\item Si $A \subseteq B$ : $\mu(A) \leq \mu(B)$ et $\mu(B \setminus A) = \mu(B) - \mu(A)$}
        \end{itemize}}
\end{frame}

\begin{frame}
    \frametitle{Quelques Mesures}
    Voici quelques mesures déjà connues :
    \begin{itemize}
        \visible<1->{\item La mesure de comptage sur $\left(\N, \P(\N)\right)$ : $\mu(A) = \abs{A}$ }
              \visible<2->{\item La mesure de Dirac en $x$ sur $\left(E, \A\right)$ quelconque : $\delta_{x}(A) = \begin{cases} 1 &\text{ si } x\in A\\ 0  &\text{ sinon} \end{cases}$}
              \visible<3->{\item La mesure de Lebesgue sur $\left(\R, \B(\R)\right)$ : $\lambda\left(\left[a, b\right]\right) = b - a$}
    \end{itemize}

\end{frame}
\begin{frame}
    \frametitle{Mesure de Probabilité}
    \visible<1->{Une mesure de probabilité $p$ sur un univers (un ensemble $\Omega$) est une mesure sur les parties de cet univers (les évènements) de masse totale $1$.\\}
    \visible<2->{L'application de cette fonction à un évènement représente la probabilité de celui-ci. On peut généraliser la propriété sur l'union ci-dessus à un nombre dénombrable d'évènements. }
\end{frame}

\begin{frame}
    \frametitle{Densité}
    \visible<1->{Dans le cas d'une mesure de probabilité $p$ sur un ensemble continu, on dit qu'elle est non-atomique lorsque $p(\left\{x\right\}) = 0$.\\}
    \visible<2->{On dit de plus qu'elle est à densité lorsqu'il existe une fonction $\mu_{p}$ telle que :
        \[
            \forall A \subseteq \Omega, p(A) = \int_{A} \mu_{p} \d\lambda
        \]
        où $\lambda$ correspond \textit{intuitivement} à la taille de l'ensemble $A$. En particulier $\lambda(\left[a, b\right]) = b - a$.}
\end{frame}

\begin{frame}
    \frametitle{Modélisation}
    \only<1-2>{\visible<1->{Supposons notre gâteau hétérogène (mettons, un gâteau marbré). Chacun de nos mangeurs ayant sa préférence, on représente leur appétance pour les parties du gâteau par une mesure de probabilité à densité sur ce gâteau.\\}
    \visible<2->{La part que l'un de nos mangeurs va manger est alors, par définition de l'intégrale par rapport à une mesure :
    \[
        \mu_{i}(X_{i}) = \int_{X_{i}}f_{i} \d\mu_{i}
    \]}}
    \only<3>{Intuitivement, c'est une somme (C'est une vision très physicienne/historique de l'intégrale) sur chaque petite portion du gâteau de l'appétance du mangeur $i$ pour cette portion.}
\end{frame}


\section{Une Démonstration du Théorème}
\subsection{Quelques Lemmes Utiles}
\begin{frame}
    \frametitle{Borsuk-Ulam ?}
    \begin{theorem}[Borsuk-Ulam]
        Si $f$ est une fonction continue sur une sphère de dimension $n$ à valeurs dans un espace euclidien de dimension $n$, il existe deux points antipodaux sur cette sphère de même image par $f$. Autrement dit, à isomorphisme près :
        \[
            \forall f \in \cont^{0}\left(S^{n}, \R^{n}\right), \ \exists x_{0} \in S^{n}, \ f(x_{0}) = f(-x_{0})
        \]
    \end{theorem}
\end{frame}

\begin{frame}
    \frametitle{Catégorie de Lusternick-Schnirelmann}
    \only<1>{\begin{definition}
            On définit la Catégorie de Lusternick-Schnirelmann d'un espace $X$ noté $\text{cat}(X)$ comme le nombre minimal d'ensembles ouverts (moins 1) suffisant à le recouvrir.
        \end{definition}}
    \only<2>{
        \begin{theorem}[Catégorie des Espaces Projectifs]
            On a : $\text{cat}(\R P^{n}) = n$
        \end{theorem}
        \begin{proof}
            On a $H^{\star}(\R P^{n}; \Z_{2}) = \Z_{2}\left[X_{1}\right]/\left(x_{1}^{n + 1}\right)$. Donc $\text{cup}(\R P^{n}) = n$ et comme on a toujours :
            $\text{cup}(X) \leq \text{cat}(X)\leq \dim(X)$ d'où : $n = \text{cup}(\R P^{n}) \leq \text{cat}(\R P^{n}) \leq \dim(\R P^{n}) = n$
        \end{proof}
    }
\end{frame}

\begin{frame}
    \frametitle{Lusternick-Schnirelmann et Borsuk-Ulam}
    \only<1-2>{
        \visible<1-2>{
            \begin{theorem}[Lusternick-Schnirelmann]
                Si $S^{n}$ est recouvert par des ensembles ouverts $C_{0}, \ldots, C_{n - 1}$, alors au moins l'un d'entre eux contient des points antipodaux.
            \end{theorem}
        }
        \visible<2>{On en déduit assez immédiatement le théorème de Borsuk-Ulam.}}
    \only<3>{
        \begin{proof}
            Supposons qu'aucun ne contiennent de points antipodaux. Soit $A_{i} \subseteq B^{n + 1}$ l'ensemble fermé des rayons aux points de $C_{i}$. En notant la relation $\sim$ d'indentification de $S^{n}$ aux antipodes, on a : $\R P^{n + 1} = B^{n + 1}/\sim$. Par hypothèse comme $C_{i}$ n'a pas de points antipodaux, $A_{i} \hookrightarrow \R P^{n+1}$ est injective et donc, $A_{i}$ se contractant en un point, $\R P^{n + 1}$ est recouvert par $A_{0}, \ldots, A_{n - 1}$, ce qui contredit le théorème précédent.
        \end{proof}}
\end{frame}

\begin{frame}
    \frametitle{Une conséquence de Borsuk-Ulam}
    En physique, on considère que toutes les fonctions de l'espace (pression, température, humidité\dots) sont continues, dérivables et même à peu près tout ce dont on a envie.\\
    \visible<2->{Mais, puisque toutes les sphères s'obtiennent par homothétie et translation les unes par rapport aux autres, le théorème de Borsuk-Ulam est valide sur toute sphère.\\}
    \visible<3->{Et, en physique, on connaît quelque chose qui ressemble très fortement à une sphère.\\}
    \visible<4->{Ainsi, à tout instant, il existe sur Terre deux points antipodaux qui ont même pression, même température et même taux d'humidité.}
\end{frame}



\subsection{La Démonstration à proprement dit}
\begin{frame}
    \frametitle{Démonstration}
    \only<1>{On ne va pas rentrer dans les détails d'une démonstration horriblement calculatoire.}
    \only<2->{\only<2-3>{\visible<2-3>{On introduit les fonctions $\left(F_{i}\right)_{i\in \onen{n - 1}}$ suivantes sur $S^{n - 1} = \left\{e = (e_{1}, \ldots, e_{n}) \in \R^{n} \mid \sum_{i = 1}^{n} e_{i}^{2} = 1\right\}$ :
    \[
        \begin{split}
            F_{i}(e) = \sgn(e_{i + 1}) &\times \int_{e_{1}^{2} + \cdots + e_{i}^{2}}^{e_{1}^{2} + \cdots + e_{i}^{2} + e_{i + 1}^{2}} f_{\sigma(i + 1)}(x)\d x \\ &- \sgn{e_{1}}\int_{0}^{e_{1}^{2}}f_{\sigma(1)}(x)\d x
        \end{split}\]}}
    \visible<3-4>{On définit alors $f(e) = \left(F_{1}(e), \ldots, F_{n - 1}(e)\right)$. Il est clair que $f$ est continue et antipodale, donc par le Théorème de Borsuk-Ulam, elle est nulle en un certain point $\tilde{e}$.}
    \only<4->{On pose alors $x_{0} = 0$ puis $x_{i} = x_{i - 1} + \tilde{e}_{i}^{2}$}
    }
\end{frame}


\section{Extensions}
\subsection{De Borsuk-Ulam}
\begin{frame}
    \frametitle{D'autres démonstrations}
    Il existe de nombreuses démonstrations du théorème de Borsuk-Ulam se basant sur de nombreux domaines :
    \begin{itemize}
        \visible<2->{\item Par la topologie algébrique : En réduisant le groupe fondamental (trivial) de la sphère (simplement connexe) à un groupe isomorphe à $\Z$ en étudiant l'image d'un lacet de $S^{2}$ par $g(x) = \frac{f(x) - f(-x)}{\norm{f(x) - f(-x)}}$}
              \visible<3>{\item Par le Lemme de Tucker : Une triangulation antipodalement symétrique sur la boule contient une arête complémentaire. L'avantage de cette méthode étant qu'on connaît une démonstration algorithmique de ce lemme.}
    \end{itemize}
\end{frame}

\begin{frame}
    \frametitle{Des Corollaires}
    Le Théorème de Borsuk-Ulam est très, très, très utile, dans de nombreux domaines, notamment du fait de ses multiples démonstrations :
    \begin{itemize}
        \visible<2->{\item Théorème de Point Fixe de Brouwer : Toute fonction continue de $D^{n}$ dans $D^{n}$ a un point fixe.}
              \visible<3->{\item Théorème du Sandwich au Jambon : Étant données $n$ parties Lebesgue-Mesurables d'un espace de dimension $n$, il existe au moins un hyperplan affine divisant chaque parties en deux sous-ensembles de mesure égales.}
              \visible<4->{\item Théorème de Lovász : Si le complexe de voisinage d'un graphe est $k$-connecté, alors $\chi(G) \geq k + 3$.}
    \end{itemize}
\end{frame}
\begin{frame}
    \frametitle{D'autres Applications}
    De parts ses nombreux corollaires, il a aussi de nombreuses applications :
    \begin{itemize}
        \visible<2-3>{\item La partie nulle du jeu de Hex n'existe pas. C'est une application du Théorème de Point Fixe de Brouwer.}
              \visible<3-3>{\item Le problème du collier dérobé a une solution en $n$ coupes : Deux voleurs ont dérobé un collier fait de $n$ types de perles et souhaitent le partager également en deux.}
    \end{itemize}
\end{frame}

\begin{frame}
    \frametitle{Coloration de Graphes}
    On parlera sans doute de ce sujet plus en détail dans un prochain cours, mais voici tout de même un résultat qui découle du théorème de Borsuk-Ulam : \\
    \begin{theorem}
        On peut partitionner les $\binom{n}{k}$ parties à $k$ éléments d'un ensemble de $n$ éléments en $n - 2k + 2$ classes de sorte que dans chacune des classes, on ne peut choisir deux sous-ensembles disjoints. Autrement dit, si on note $KN(n, k)$ le graphe de ces parties, $\chi\left(KN(n, k)\right) = n - 2k + 2$.
    \end{theorem}

\end{frame}
\end{document}

